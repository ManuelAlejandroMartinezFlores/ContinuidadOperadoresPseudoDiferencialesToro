% ================================================================================
% El estudiante debe llenar sus datos en esta sección para que la plantilla los 
% auto-importe y genere automáticamente las páginas de portada y de firmas 
% autorizadas.
% ================================================================================
% Datos del estudiante:
% --------------------------------------------------------------------------------
% Nombre completo
\def \nombreestudiante {Manuel Alejandro Mart\'inez Flores}
% Carné
\def \uvgcarne {21403}
% Facultad
\def \uvgfacultad {Ciencias y Humanidades}
% Carrera
\def \uvgcarrera {Matemática Aplicada}

% Datos del trabajo:
% --------------------------------------------------------------------------------
% Título completo
\def \titulotesis {Continuidad de operadores pseudo-diferenciales en el toro}
% Mes de entrega
\def \mesentrega {TODO }
% Año de entrega
\def \anoentrega {2025}
% Asesor
\def \nombreasesor {Dr. Duv\'an Cardona}

% Datos del tribunal examinador:
% --------------------------------------------------------------------------------
% Nombre del primer examinador
\def \nombreprimerex {TODO}
% Nombre del segundo examinador
\def \nombresegundoex {TODO}
% Fecha de aprobación
\def \fechaaprobacion {TODO}

% Capítulos pre-definidos
% --------------------------------------------------------------------------------
% Comentar las líneas de las secciones que desean omitirse, por defecto se 
% se incluyen todas.
\def \CAPprefacio {Prefacio}
% \def \CAPagradecimientos {Agradecimientos}
\def \CAPantecedentes {Antecedentes}
% \def \CAPalcance {Alcance}
%\def \CAPanexos {Anexos}
%\def \CAPglosario {Glosario}
%\def \CAPsimbolos {Listado de símbolos}

% Formato y estilo de la plantilla
% --------------------------------------------------------------------------------
% Portada: Puede cambiarse la imagen en la portada al cambiar el nombre del 
% archivo siguiente. NOTA: debe tener la suficiente resolución para cubrir el área
% designada
\def \imagenportada {plantilla/portadatesis2.png}
% Referencias: Puede des-comentar la siguiente línea para utilizar el formato de referencias APA
% \def \usarAPA {Usar formato APA}
% Párrafo: Puede comentar la siguiente línea si desea emplear un formato de 
% párrafo distinto al establecido por defecto
% \def \parpordefecto {Formato de párrafo por defecto}
% Capítulos y secciones: Puede des-comentar la siguiente línea para establecer el 
% formato de los capítulos y secciones bajo el estándar original de UVG para
% trabajos de graduación. Este incluye: capítulos con numeración romana, secciones
% con letras mayúsculas, sub-secciones con números y sub-sub-secciones con letras
% minúsculas
% \def \capsecuvg {Formato UVG para capítulos y secciones}