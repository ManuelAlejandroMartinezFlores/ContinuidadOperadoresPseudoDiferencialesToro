
Este trabajo surge del interés por el análisis armónico y su aplicación en el estudio de operadores pseudo-diferenciales, particularmente en contextos no euclidianos. La elección del toro $\mathbb{T}^n$ como dominio de estudio responde a su riqueza estructural como variedad compacta y grupo abeliano, lo que permite abordar problemas tanto locales como globales desde una perspectiva unificada. El texto está dirigido principalmente a estudiantes e investigadores en análisis matemático que deseen adentrarse en la teoría de operadores pseudo-diferenciales en variedades compactas ejemplificada en este trabajo sobre el toro. Se asumen conocimientos básicos de análisis funcional, teoría de la medida y análisis de Fourier, aunque se incluyen capítulos preliminares para facilitar la comprensión de los conceptos fundamentales. La obra se divide en tres partes claramente diferenciadas: los capítulos iniciales establecen el marco teórico necesario; la parte central desarrolla el cálculo pseudo-diferencial toroidal; y los capítulos finales presentan los principales resultados de acotación. Cada capítulo incluye la mayoría de demostraciones pertinentes, de manera que este trabajo sea lo más autocontenido posible.


La bibliografía incluye tanto referencias clásicas como contribuciones recientes, reflejando el desarrollo histórico de la teoría. Se ha puesto especial cuidado en citar trabajos fundamentales y en destacar las conexiones entre diferentes enfoques. Como resultado de este trabajo, se produjo una serie de tres artículos cientificos originales titulados \textit{Estimates for pseudo-differential operators on the torus revisited. I, II, III}, de los cuales el primero aparecerá en el \textit{Journal of Mathematical Analysis and Applications} y los otros dos se encuentran en evaluación en otras revistas, y dos notas cortas tituladas \textit{Boundedness of pseudo-differential operators on the torus via kernel estimates} y \textit{Boundedness of toroidal pseudo-differential operators on Hardy spaces}, que apareceran en \textit{Trends in Mathematics} de la editorial \textit{Springer}, todos ellos en colaboración con el Dr. Duván Cardona. Invitamos al lector a abordar este texto como una guía para explorar un área fascinante del análisis armónico moderno. Las demostraciones seleccionados buscan no solo transmitir resultados, sino también desarrollar la intuición matemática necesaria para futuras investigaciones en el campo.


De manera muy especial, deseo agradecer al Dr. Duván Cardona, quien fue mi asesor para este trabajo, por darme la oportunidad de trabajar con él. Me ha acompañado en mi desarrollo como profesional y mis primeros pasos en la investigación, apoyandome con todos los detalles técnicos necesarios para enriquecer mi trabajo. Más importante aún, me ha aconsejado y apoyado enriqueciendome también como persona. Además, me abrió las puerta para participar en la Comunidad Internacional de Matemáticos de Latinoamerica (ICMAM Latin America), una hermosa comunidad con quienes espero poder compartir y colaborar durante muchos años más. 

Guatemala, Octubre de 2025

Manuel Alejandro Martínez Flores