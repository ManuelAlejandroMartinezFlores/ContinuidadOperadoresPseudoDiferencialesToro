El estudio de los operadores pseudo-diferenciales constituye una de las piedras angulares del análisis moderno, con aplicaciones profundas en la teoría de ecuaciones diferenciales parciales, el análisis armónico y la teoría espectral. Estos operadores generalizan tanto a los operadores diferenciales como a los multiplicadores de Fourier, permitiendo un tratamiento unificado de problemas que involucran no solo la regularidad de soluciones, sino también la acotación en diversos espacios funcionales. En el caso euclidiano, la teoría está bien establecida gracias a los trabajos fundacionales de Calderón, Zygmund, Hörmander y Fefferman, entre otros. Sin embargo, el estudio de estos operadores en variedades compactas, como el toro $\mathbb{T}^n := \mathbb{R}^n / \mathbb{Z}^n$, presenta desafíos particulares debido a la estructura discreta de su espacio de frecuencias y a la necesidad de desarrollar herramientas adaptadas a este contexto.

En el toro, el cálculo pseudo-diferencial puede abordarse de dos maneras: mediante una formulación local, tratando al toro como una variedad y utilizando particiones de la unidad, o mediante una definición global, aprovechando la estructura de grupo subyacente. Este trabajo se centra en este último enfoque, siguiendo el marco desarrollado por Ruzhansky, Turunen y Vainikko, que permite definir operadores pseudo-diferenciales toroidales a través de series de Fourier discretas. Esta perspectiva no solo es natural para el toro, sino que también facilita el estudio de propiedades de acotación en espacios de Lebesgue $L^p(\mathbb{T}^n)$ y de Sobolev $W^s_p(\mathbb{T}^n)$, entre otros.

Uno de los problemas centrales en la teoría es determinar bajo qué condiciones un operador pseudo-diferencial $T_a$, definido por un símbolo $a(x, \xi)$ en una clase de Hörmander $S^m_{\rho,\delta}(\mathbb{T}^n \times \mathbb{Z}^n)$, se extiende a un operador acotado entre espacios de funciones. Resultados clásicos, como los de Calderón-Vaillancourt para $L^2(\mathbb{R}^n)$ o los de Fefferman para $L^p(\mathbb{R}^n)$, han establecido cotas que dependen críticamente de los parámetros $m, \rho, \delta$ del símbolo. En el caso toroidal, aunque muchos de estos resultados tienen análogos, las demostraciones requieren ajustes sustanciales debido a la naturaleza discreta del espacio de frecuencias $\mathbb{Z}^n$ y a la falta de invarianza bajo cambios de coordenadas cuando $\rho > 1 - \delta$.

Este trabajo tiene como objetivo revisar y exponer de manera sistemática los resultados de continuidad de operadores pseudo-diferenciales en el toro, haciendo especial hincapié en las técnicas de demostración y en las diferencias con el caso euclidiano. Se abordarán tanto resultados clásicos como contribuciones recientes, incluyendo los teoremas de Delgado para $L^p(\mathbb{T}^n)$ con $p \geq 2$ y las extensiones de Álvarez-Hounie realizadas con Cardona para el rango completo $1 < p < \infty$. 

La exposición se estructura en tres capítulos principales. Inicialmente, se presentan los preliminares necesarios sobre espacios de funciones, transformadas de Fourier y distribuciones en $\mathbb{R}^n$ y $\mathbb{T}^n$. Además, se incluyen resultados clásicos de análisis armónico, como el hecho que el espacio $\mathrm{BMO}$ es el dual del espacio de Hardy $H^1$ y técnicas de interpolación compleja que permiten extender propiedades de continuidad a espacios $L^p$ con $1 < p < \infty$. Luego, se introduce la definición y propiedades básicas de los operadores pseudo-diferenciales en ambos contextos, destacando las particularidades del cálculo toroidal, como el uso de operadores de diferencia discreta en lugar de derivadas convencionales. Finalmente, se dedican dos capítulos a demostrar los principales resultados de continuidad en espacios de Lebesgue y de Sobolev, utilizando técnicas que incluyen interpolación, descomposiciones atómicas y estimaciones de núcleos integrales.

Cabe destacar que, a diferencia del caso euclidiano, las clases $H^1$ y $\mathrm{BMO}$ no son estables bajo la multiplicación de funciones test en el toro, lo que imposibilita tratar este espacio simplemente como una variedad mediante particiones de la unidad. Esta limitación justifica el estudio independiente del caso toroidal y la necesidad de desarrollar herramientas específicas para este contexto. Asimismo, los operadores pseudo-diferenciales con símbolos en las clases de Hörmander no son estables bajo cambios de coordenadas cuando $\rho > 1 - \delta$, lo que refuerza la relevancia de un tratamiento global mediante la transformada de Fourier discreta.

Con este trabajo, se espera proporcionar una referencia accesible y rigurosa que contribuya a la divulgación de estos temas en español y fomente futuras investigaciones en el área. La escasez de literatura en español sobre operadores pseudo-diferenciales representa una barrera significativa para estudiantes e investigadores hispanohablantes, limitando su acceso a herramientas avanzadas y reduciendo las oportunidades de formación especializada. Esta exposición busca reducir esta brecha, permitiendo el acceso a conceptos avanzados y contribuyendo a fortalecer la comunidad matemática en español.