En el caso euclidiano, Calderón y Vaillancourt demostraron que los operadores pseudo-diferenciales con símbolos en la clase $S^{0}_{\rho,\rho}(\mathbb{R}^n\times\mathbb{R}^n)$ son acotados en $L^2(\mathbb{R}^n)$ para algún $0\leq\rho<1$, véase \cite{calderon-vaillancourt-1, calderon-vaillancourt-2}. Este resultado no puede extenderse cuando $\rho=1$, es decir, existen símbolos en $S^{0}_{1,1}(\mathbb{R}^n\times\mathbb{R}^n)$ cuyos operadores pseudo-diferenciales asociados no son acotados en $L^2$; para un argumento clásico de este hecho debido a Hörmander, consúltese \cite{duoandikoetxea}. Además, Fefferman \cite{fefferman-Lp} probó la acotación $L^\infty(\mathbb{R}^n)$-$\mathrm{BMO}(\mathbb{R}^n)$ para operadores pseudo-diferenciales con símbolos en la clase $S^{m}_{\rho,\delta}(\mathbb{R}^n\times\mathbb{R}^n)$, con $m=-n(1-\rho)/2$ donde $0\leq \delta<\rho\leq 1$. Fefferman también obtuvo la acotación en $L^p(\mathbb{R}^n)$ para estas clases cuando $m\leq -n(1-\rho)|1/p - 1/2|$ y $1<p<\infty$. En vista de ejemplos clásicos debidos a Wainger y Hirschman, el resultado de Fefferman es óptimo para multiplicadores de Fourier. Cabe destacar que el desarrollo histórico del problema de la acotación en $L^p$ de operadores pseudo-diferenciales ha sido discutido en $\mathbb{R}^n$, por ejemplo, en \cite{nagase, wang}.  

Los operadores pseudo-diferenciales con símbolos en las clases de Hörmander pueden definirse en variedades $C^\infty$ mediante cartas locales. Por ello, se considera el toro $\mathbb{T}^n:=\mathbb{R}^n/\mathbb{Z}^n$ como un grupo aditivo cociente y una $n$-variedad, con el atlas preferido de sistemas de coordenadas dado por la aplicación de restricción $x\mapsto x + \mathbb{Z}^n$ en conjuntos abiertos $\Omega \subset \mathbb{R}^n$, véase McLane \cite{mclane}. Se nota que en \cite{agranovich}, Agranovich proporciona una definición global de operadores pseudo-diferenciales en el círculo $\mathbb{S}^1=\mathbb{T}^1$, en lugar de la formulación local que trata al círculo como una variedad. Mediante la transformada de Fourier, esta definición se extendió al toro $\mathbb{T}^n$. Además, se ha demostrado que las clases $(\rho,\delta)$ de Agranovich y Hörmander son equivalentes, gracias al teorema de equivalencia de McLane \cite{mclane}. En este trabajo, se consideran operadores pseudo-diferenciales toroidales en el contexto del cálculo pseudo-diferencial en el toro desarrollado por Ruzhansky, Turunen y Vainniko \cite{ruzhansky-turunen2, ruzhansky-turunen}. Asimismo, cotas $L^p$ en el círculo que pueden extenderse al toro se encuentran en \cite{wong}, en el marco clásico de la teoría de Calderón-Zygmund. Por otro lado, el análogo toroidal del resultado de Fefferman fue probado por Delgado en \cite{delgado} para el toro, aunque aún se requiere que $\delta<\rho$.   
Se nota que, para $0\leq \delta<1$ y $0<\rho\leq 1$, Álvarez y Hounie \cite{alvarez-hounie} probaron la acotación $L^p(\mathbb{R}^n)$-$L^q(\mathbb{R}^n)$ de operadores pseudo-diferenciales cuando $p\leq q$, incluso con $\delta \geq \rho$. Este resultado fue extendido al caso toroidal en \cite{Cardona:Martinez}. 
Para otros trabajos sobre acotación $L^p$ de operadores pseudo-diferenciales, se remite al lector a \cite{cardona, molahajloo-wong, ruzhansky-turunen-quant}.