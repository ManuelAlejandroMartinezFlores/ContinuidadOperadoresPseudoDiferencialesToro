En el caso euclidiano, Calderón y Vaillancourt demostraron que los operadores pseudo-diferenciales con símbolos en la clase $S^{0}_{\rho,\rho}(\mathbb{R}^n\times\mathbb{R}^n)$ son acotados en $L^2(\mathbb{R}^n)$ para algún $0\leq\rho<1$, véase \cite{calderon-vaillancourt-1, calderon-vaillancourt-2}. Este resultado no puede extenderse cuando $\rho=1$, es decir, existen símbolos en $S^{0}_{1,1}(\mathbb{R}^n\times\mathbb{R}^n)$ cuyos operadores pseudo-diferenciales asociados no son acotados en $L^2$; para un argumento clásico de este hecho debido a Hörmander, consúltese \cite{duoandikoetxea}. Además, Fefferman \cite{fefferman-Lp} probó la acotación $L^\infty(\mathbb{R}^n)$-$\mathrm{BMO}(\mathbb{R}^n)$ para operadores pseudo-diferenciales con símbolos en la clase $S^{m}_{\rho,\delta}(\mathbb{R}^n\times\mathbb{R}^n)$, con $m=-n(1-\rho)/2$ donde $0\leq \delta<\rho\leq 1$. Fefferman también obtuvo la acotación en $L^p(\mathbb{R}^n)$ para estas clases cuando $m\leq -n(1-\rho)|1/p - 1/2|$ y $1<p<\infty$. En vista de ejemplos clásicos debidos a Wainger y Hirschman, el resultado de Fefferman es óptimo para multiplicadores de Fourier. Posteriormente, Álvarez y Hounie demostraron continuidad $L^p$-$L^q$ incluso cuando $\delta>\rho$, vea \cite{alvarez-hounie}. Cabe destacar que el desarrollo histórico del problema de la acotación en $L^p$ de operadores pseudo-diferenciales ha sido discutido en $\mathbb{R}^n$, por ejemplo, en \cite{nagase, wang}.  

En $\mathbb{R}^n$, la teoría de los espacios de Hardy en varias variables fue tratada exhaustivamente por Fefferman y Stein en \cite{fefferman-stein}. Estos autores demostraron que es posible aplicar el método de interpolación compleja entre $H^1$ y $L^2$, y entre $L^2$ y $\mathrm{BMO}$, para obtener propiedades de continuidad en los espacios de Lebesgue $L^p$. Además, Fefferman descubrió de manera significativa que el dual del espacio de Hardy $H^1$ es el espacio de funciones con oscilación media acotada $\mathrm{BMO}$, véase \cite{fefferman-BMO}. Estos hechos permitieron a Fefferman probar la acotación en $L^p$, $1<p<\infty$, de operadores pseudo-diferenciales con símbolos en la clase de Hörmander $S^m_{\rho, \delta} (\mathbb{R}^n \times \mathbb{R}^n), $ donde $0\leq\delta<\rho\leq1$ y $m\leq-n(1-\rho)|1/p-1/2|$. Además, Álvarez y Hounie \cite{alvarez-hounie} demostraron continuidad $h^p$-$L^p$ y $H^p$, con $p\leq1$, para operadores pseudo-diferenciales euclideanos utilizando propiedades del kernel que explotan resultados obtenidos en el caso vectorial estudiado por Álvarez y Milman \cite{alvarez-milman}.

 Por otra parte, Fefferman y Stein introdujeron la función maximal aguda $\mathcal{M}^\#$ en \cite{fefferman-stein}, la cual sirve para caracterizar la norma del espacio de funciones con oscilación media acotada $\mathrm{BMO}(\mathbb{R}^n)$. Además, probaron que satisface una cota superior con respecto a la norma $L^p$ de funciones integrables, es decir, se cumple que $\|f\|_{L^p} \lesssim \|\mathcal{M}^\#f\|_{L^p}$. Aquí, y en lo que sigue, $A\lesssim B$ significa que existe una constante $C>0$ tal que $A\leq CB$. Además, se ha demostrado que si $T$ es un operador de Calderón-Zygmund, entonces se tiene la desigualdad puntual $\mathcal{M}^\#(Tf)(x) \lesssim \mathrm{M}_rf(x)$, donde $\mathrm{M}_r$ es la versión $L^r$ de la función maximal de Hardy-Littlewood. Combinando estas dos estimaciones se obtienen cotas de continuidad de $T$ de $L^p$ en sí mismo. Es decir, que
\begin{equation}
	\|Tf\|_{L^p} \lesssim\|\mathcal{M}^\#(Tf)\|_{L^p} \lesssim\| \mathrm{M}_rf\|_{L^p} \lesssim \|f\|_{L^p}.
	\label{eq:technique}
\end{equation}
Esta técnica ha sido ampliamente empleada en una variedad de trabajos de análisis armónico, véase \cite{fefferman-stein}. Por otro lado, Muckenhoupt probó que los pesos $w$ en la clase $A_p$ satisfacen la siguiente estimación $\|\mathrm{M}f\|_{L^p(w)} \lesssim \| f\|_{L^p(w)}$, para $1<p<\infty$, véase \cite{muckenhoupt}. Combinando estos dos hechos, Park y Tomita \cite{park-tomita} probaron la continuidad para espacios de Lebesgue pesados $L^p(w)$ para operadores pseudo-diferenciales euclidianos con símbolos en las clases de Hörmander $S^{m}_{\rho,\delta} (\mathbb{R}^n\times \mathbb{R}^n)$.

Los operadores pseudo-diferenciales con símbolos en las clases de Hörmander pueden definirse en variedades $C^\infty$ mediante cartas locales. Por ello, se considera el toro $\mathbb{T}^n:=\mathbb{R}^n/\mathbb{Z}^n$ como un grupo aditivo cociente y una $n$-variedad, con el atlas preferido de sistemas de coordenadas dado por la aplicación de restricción $x\mapsto x + \mathbb{Z}^n$ en conjuntos abiertos $\Omega \subset \mathbb{R}^n$, véase McLane \cite{mclane}. Se nota que en \cite{agranovich}, Agranovich proporciona una definición global de operadores pseudo-diferenciales en el círculo $\mathbb{S}^1=\mathbb{T}^1$, en lugar de la formulación local que trata al círculo como una variedad. Mediante la transformada de Fourier, esta definición se extendió al toro $\mathbb{T}^n$. Además, se ha demostrado que las clases $(\rho,\delta)$ de Agranovich y Hörmander son equivalentes, gracias al teorema de equivalencia de McLane \cite{mclane}. En este trabajo, se consideran operadores pseudo-diferenciales toroidales en el contexto del cálculo pseudo-diferencial en el toro desarrollado por Ruzhansky, Turunen y Vainniko \cite{ruzhansky-turunen2, ruzhansky-turunen}. Asimismo, cotas $L^p$ en el círculo que pueden extenderse al toro se encuentran en \cite{wong}, en el marco clásico de la teoría de Calderón-Zygmund. Por otro lado, el análogo toroidal del resultado de Fefferman fue probado por Delgado en \cite{delgado} para el toro, aunque aún se requiere que $\delta<\rho$. Este resultado fue extendido posteriormente a grupos de Lie compactos por Delgado y Ruzhansky \cite{delgado-ruzhansky} y a variedades con geometría acotada por Gómez Cobos y Ruzhansky \cite{cobos-ruzhansky}. También se ha extendido para clases de Hörmander subelípticas en grupos de Lie compactos en \cite{cardona-ruzhansky-subelliptic}. El resultado  de Álvarez y Hounie de continuidad $L^p$ fue extendido al caso toroidal con Cardona en \cite{Cardona:Martinez}, y para continuidad $H^p$-$L^p$ y $H^p$ con Cardona en \cite{cardona-martinez-II}. El resultado de Park y Tomita de continuidad $L^p(w)$ fue demostrado para operadores pseudo-diferenciales toroidales con Cardona en \cite{cardona-martinez-III}. Para otros trabajos sobre acotación $L^p$ de operadores pseudo-diferenciales, se remite al lector a \cite{cardona, molahajloo-wong, ruzhansky-turunen-quant}.
