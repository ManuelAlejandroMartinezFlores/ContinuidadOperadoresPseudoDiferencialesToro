
	Este trabajo presenta un estudio sistemático de la continuidad de operadores pseudo-diferenciales en el toro $\mathbb{T}^n$, abordando el problema de acotación en espacios de Lebesgue $L^p(\mathbb{T}^n)$ y de Sobolev $W^s_p(\mathbb{T}^n)$. Siguiendo el enfoque global desarrollado por Ruzhansky, Turunen y Vainikko, se analizan operadores con símbolos en clases de Hörmander $S^m_{\rho,\delta}(\mathbb{T}^n \times \mathbb{Z}^n)$, estableciendo condiciones óptimas sobre los parámetros $m$, $\rho$ y $\delta$ que garantizan la continuidad de estos operadores. Se demuestran resultados clave que incluyen extensiones toroidales de teoremas clásicos de Calderón-Vaillancourt, Fefferman, y resultados recientes de Delgado y los obtenidos con Cardona, utilizando técnicas de análisis armónico, interpolación compleja y estimaciones de núcleos integrales. El trabajo proporciona una referencia completa en español sobre el tema, destacando las diferencias fundamentales entre el caso toroidal y el euclidiano, particularmente en lo concerniente a la estructura discreta del espacio de frecuencias $\mathbb{Z}^n$.
	
	\chapter*{Abstract}
	% --- Abstract ---
		This work presents a systematic study of the continuity of pseudo-differential operators on the torus $\mathbb{T}^n$, addressing the boundedness problem in Lebesgue spaces $L^p(\mathbb{T}^n)$ and Sobolev spaces $W^s_p(\mathbb{T}^n)$. Following the global approach developed by Ruzhansky, Turunen, and Vainikko, we analyze operators with symbols in Hörmander classes $S^m_{\rho,\delta}(\mathbb{T}^n \times \mathbb{Z}^n)$, establishing optimal conditions on the parameters $m$, $\rho$, and $\delta$ that guarantee the continuity of these operators. We prove key results including toroidal extensions of classical theorems by Calderón-Vaillancourt and Fefferman, as well as recent results by Delgado and the ones obtained with Cardona, using techniques from harmonic analysis, complex interpolation, and integral kernel estimates. The work provides a comprehensive reference in Spanish on the subject, highlighting fundamental differences between the toroidal and Euclidean cases, particularly concerning the discrete structure of the frequency space $\mathbb{Z}^n$.
