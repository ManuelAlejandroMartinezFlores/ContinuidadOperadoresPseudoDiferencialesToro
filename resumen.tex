
		El presente trabajo ofrece una revisión sistemática de los resultados de continuidad para operadores pseudo-diferenciales en el toro $\mathbb{T}^n$, con especial énfasis en las diferencias técnicas respecto al caso euclidiano. Se examina la acotación de estos operadores en espacios de Lebesgue $L^p$, Sobolev $W^s_p$, espacios pesados $L^p(w)$ y espacios de Hardy $H^p$, para símbolos pertenecientes a las clases de Hörmander $S^m_{\rho,\delta}(\mathbb{T}^n \times \mathbb{Z}^n)$. La exposición se estructura en tres partes: preliminares sobre análisis de Fourier y espacios funcionales, fundamentos del cálculo pseudo-diferencial toroidal usando operadores de diferencia discreta, y demostraciones detalladas de teoremas de continuidad mediante interpolación compleja, descomposiciones atómicas y técnicas de análisis armónico. Este trabajo busca suplir la escasez de literatura en español sobre el tema, proporcionando una referencia rigurosa y accesible para la comunidad matemática hispanohablante. Como resultado de este trabajo, se produjo una serie de tres artículos cientificos originales \textit{Estimates for pseudo-differential operators on the torus revisited. I, II, III}, de los cuales el primero aparecerá en el \textit{Journal of Mathematical Analysis and Applications} y los otros dos se encuentran en evaluación en otras revistas, y dos notas cortas \textit{Boundedness of pseudo-differential operators on the torus via kernel estimates} y \textit{Boundedness of toroidal pseudo-differential operators on Hardy spaces}, que apareceran en \textit{Trends in Mathematics} de la editorial \textit{Springer}, todos ellos en colaboración con Duván Cardona.
		
		\textbf{Palabras clave:} Operadores pseudo-diferenciales, toro, análisis de Fourier, análisis armónico

\vspace{0.5in}
		This work provides a systematic review of continuity results for pseudo-differential operators on the torus $\mathbb{T}^n$, with special emphasis on the technical differences compared to the Euclidean case. We examine the boundedness of these operators on Lebesgue spaces $L^p$, Sobolev spaces $W^s_p$, weighted spaces $L^p(w)$ and Hardy spaces $H^p$, for symbols belonging to Hörmander classes $S^m_{\rho,\delta}(\mathbb{T}^n \times \mathbb{Z}^n)$. The exposition is structured in three parts: preliminaries on Fourier analysis and function spaces, foundations of toroidal pseudo-differential calculus using discrete difference operators, and detailed proofs of continuity theorems through complex interpolation, atomic decompositions, and harmonic analysis techniques. This work aims to address the scarcity of literature in Spanish on the topic, providing a rigorous and accessible reference for the Spanish-speaking mathematical community. As a result of this research, a series of three original scientific articles were produced: \textit{Estimates for pseudo-differential operators on the torus revisited. I, II, III}, of which the first will appear in the \textit{Journal of Mathematical Analysis and Applications} and the other two are under evaluation in other journals, and two short notes: \textit{Boundedness of pseudo-differential operators on the torus via kernel estimates} and \textit{Boundedness of toroidal pseudo-differential operators on Hardy spaces}, which will appear in \textit{Trends in Mathematics} by \textit{Springer}, all of which in collaboration with Duván Cardona.
		
		\textbf{Keywords:} Pseudo-differential operators, torus, Fourier analysis, harmonic analysis

