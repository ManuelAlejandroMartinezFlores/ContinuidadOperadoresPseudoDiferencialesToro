\chapter{Preliminares}

En este capítulo se revisarán aspectos básicos del análisis armónico en 
$\R^n$ y $\T^n$. 
Se recuerda que $\R^n$ es un grupo aditivo respecto a la suma usual de vectores
con subgrupo aditivo $\Z^n$. Entonces, se define al toro $n$-dimensional
como el grupo cociente $\T^n := \R^n/\Z^n = (\R/\Z)^n$. Además, 
el toro puede ser identificado con el conjunto $[0, 1)^n$ y se le puede considerar
con la topología cociente. A lo largo de este trabajo, se fijará la
medida de Lebesgue en $\R^n$. Para cualquier punto 
$x := (x_1, \ldots, x_n) \in \R^n$, se denotará la norma
euclideana como 
\begin{equation*}
    |x| := \sqrt{x_1^2 + \cdot + x_n^2}.
\end{equation*}
Sin embargo, podría ser problemático considerar potencias negativas
de la norma euclideana, debido a que se desvanece en cero. Por lo que
se considerará una función que se comporta asintóticamente similar, 
pero no presenta el mismo problema
\begin{equation*}
    \angles{x} := \sqrt{1 + |x|^2}.
\end{equation*}
Si se tiene que existe una constante $C>0$ tal que $A\leq CB$, se dice que $A\lesssim B$. Si además, $C$ depende de algún parámetro $\alpha$, se denota $A\lesssim_\alpha B$.
\section{Espacios de Lebesgue en $\R^n$ y $\T^n$ }

Sea $\Omega$ un subconjunto medible de $\R^n$. Por simplicidad, se supondrá
que $\Omega$ es abierto o cerrado.

\begin{definition}
    Dado $1\leq p < \infty$. Se dice que una función medible $f:\Omega\subset\R^n\rightarrow\C$ se encuentra en $L^p(\Omega)$ si su norma 
    \begin{equation*}
        \|f\|_{L^p(\Omega)} := \left( \int_\Omega |f(x)|^p \diff x
        \right)^{1/p}
    \end{equation*}
    es finita. Para el caso $p=\infty$, se dice que $f\in L^\infty(\Omega)$
    si es esencialmente acotada. Es decir, si
    \begin{equation*}
        \|f\|_{L^\infty(\Omega)} := \esssup_{x\in\Omega}|f(x)| < \infty,
    \end{equation*}
    donde $\esssup_{x\in\Omega}|f(x)|$ se define como el menor número real $M$
    tal que es mayor que $|f(x)|$ casi para todo $x\in\Omega$, i.e. excepto fuera de
    un conjunto de medida cero.
\end{definition}
Cabe destacar que en realidad los elementos de los espacios $L^p(\Omega)$
son clases de equivalencias de funciones iguales casi en todo $x\in\Omega$.
Sin embargo, es un detalle técnico menor y se acostumbra a tratarles como funciones.
Además, cuando $\Omega$ sea claro por el contexto, simplemente se denotará
$\|\cdot\|_{L^p(\Omega)}$ como $\|\cdot\|_{L^p}$. Ahora, se discutirán propiedades 
importantes de los espacios de Lebesgue.

\begin{proposition}[Desigualdad de Young]
    Sean $1< p, q< \infty$, tales que $\frac{1}{p} + \frac{1}{q} = 1$.
    Entonces para todos $a, b > 0$, se tiene que 
    \begin{equation*}
        ab \leq \frac{a^p}{p} + \frac{b^q}{q}.
    \end{equation*}
    Como consecuencia, para $f\in L^p(\Omega)$ y $g \in L^q(\Omega)$,
    se tiene que $fg \in L^1(\Omega)$ y 
    \begin{equation*}
        \|fg\|_{L^1} \leq \frac{1}{p}\|f\|^p_{L^p} + \frac{1}{q}
        \|g\|_{L^q}^q.
    \end{equation*}
\end{proposition}
\begin{proof}
    Esto es consecuencia del hecho que $x\mapsto e^x$ es una función
    convexa. Entonces
    \begin{equation*}
        ab = e^{\ln a + \ln b} = e^{\frac{1}{p}\ln a^p + 
        \frac{1}{q}\ln b^q} \leq \frac{1}{p}e^{\ln a^p} +
        \frac{1}{q}e^{\ln b^q} = \frac{a^p}{p} + \frac{b^q}{q}.
    \end{equation*}
    Completando así la prueba.
\end{proof}
Particularmente cuando $p=2=q$, se tiene la conocida como desigualdad de
Cauchy. 
\begin{proposition}[Desigualdad de H\"older]
    Sean $1\leq p, q\leq \infty$, tales que $\frac{1}{p} + \frac{1}{q} = 1$.
    Entonces, para $f\in L^p(\Omega)$ y $g\in L^p(\Omega)$, se tiene
    que $fg \in L^1(\Omega)$ y 
    \begin{equation*}
        \|fg\|_{L^1} \leq \|f\|_{L^p}\|g\|_
        {L^q}.
    \end{equation*}
\end{proposition}
\begin{proof}
    Para el caso $p=1$ o $p=\infty$, el resultado es trivial. Así que 
    se considerará el caso $1<p<\infty$, que es una aplicación 
    de la proposición anterior. Primero, se supone
    que $\|f\|_{L^p} = \|g\|_{L^q} = 1$. Entonces, 
    se tiene que
    \begin{equation*}
        \|fg\|_{L^1} \leq \frac{1}{p}\|f\|_{L^p} + \frac{1}{q}\|g\|_{L^q} 
        = 1.
    \end{equation*}
    Ahora, se nota que si $\|f\|_{L^p}$ o $\|g\|_{L^q}$ se anulan, 
    entonces se trivializa la desigualdad. 
    Por lo que se puede considerar el caso más general en el que ninguna de las
    normas se anula de la siguiente manera
    \begin{equation*}
        \left\| \frac{f}{\|f\|_{L^p}} \frac{g}{\|g\|_{L^q}}
        \right\|_{L^1} \leq 1.
    \end{equation*}
    El resultado sigue de la linealidad de la norma $L^1$.
\end{proof}
En el caso $p=2=q$ se obtiene la desigualdad de Cauchy-Schwarz.

\begin{proposition}[Desigualdad de Minkowski]
    Dado $1\leq p\leq\infty$, sean $f,g\in L^p(\Omega)$. Entonces se tiene que
    \begin{equation*}
        \|f+g\|_{L^p} \leq \|f\|_{L^p} +
        \|g\|_{L^p}.
    \end{equation*}
    Particularmente, $\|\cdot\|_{L^p}$ satisface la desigualdad triangular
    y $L^p(\Omega)$ es un espacio normado.
\end{proposition}
\begin{proof}
    Para $p=1$ o $p=\infty$ el resultado se obtiene gracias a la desigualdad
    triangular del módulo en los números complejos. Ahora, para $1<p<\infty$ 
    se tiene que
    \begin{align*}
        \|f+g\|_{L^p}^p & \leq \int_\Omega |f+g|^{p-1}(|f|+|g|)\diff x\\
        & = \int_\Omega |f+g|^{p-1}|f| \diff x + 
        \int_\Omega |f+g|^{p-1}|g| \diff x \\
        & \leq \left( 
            \int_\Omega |f+g|^{(p-1) \frac{p}{p-1}} \diff x 
        \right)^{\frac{p-1}{p}} \left[
            \left(\int_\Omega |f|^p\diff x\right)^{1/p} +
            \left(\int_\Omega |g|^p\diff x\right)^{1/p}
        \right] \\ 
        & = \|f+g\|_{L^p}^{p-1}(\|f\|_{L^p} + \|g\|_{L^p}).
    \end{align*}
    Aquí, la primera desigualdad es la desigualdad triangular de los números
    complejos y la segunda es la desigualdad de H\"older, por lo que se concluye
    lo deseado.
\end{proof}

Ahora se introducen dos resultados importantes y de bastante utilidad. Sin embargo,
sus demostraciones requieren de herramientas de teoría de la medida o del 
análisis complejo que se encuentran fuera del alcance de este trabajo. Por lo que
simplemente se enuncian y se recomienda al lector investigar los detalles.

\begin{proposition}[Monotonía de la norma $L^p$]\label{prop:monotonia-Lp}
    Sea $f:\Omega_1 \times \Omega_2 \subset\R^n\times\R^n\rightarrow\C$ y 
    sea $1\leq p\leq\infty$.
    Se supone que $f(\cdot,y)\in L^p(\Omega_1)$ para casi
    todo $y$, y que $y\mapsto \|f(\cdot, y)\|_{L^p}$ se encuentra en
    $L^1(\Omega_2)$. Entonces $f(x, \cdot)\in L^1(\Omega_2)$ para casi
    todo $x$, la función $x\mapsto \int_{\Omega_2} f(x, y)\diff y$ se encuentra
    en $L^p(\Omega_1)$, y
    \begin{equation*}
        \left\| \int_{\Omega_2} f(\cdot, y)\diff y
        \right\|_{L^p(\Omega_1)} \leq \int_{\Omega_2} 
        \|f(\cdot, y)\|_{L^p(\Omega_1)} \diff y.
    \end{equation*}
\end{proposition}
A continuación se presenta un resultado clásico de la interpolación de 
operadores y espacios de
funciones. Para una discusión más profunda de estas técnicas, se recomienda 
revisar Bergh y L\"ofstrom \cite{bergh-lofstrom}.
\begin{theorem}[Interpolación de Riesz-Thorin]\label{theo:riesz-thorin} 
    Sea $T:L^{p_0}(\Omega)+L^{p_1}(\Omega) \rightarrow 
    L^{q_0}(\Omega)+L^{q_1}(\Omega)$ un operador lineal tal que 
    \begin{equation*}
        \|Tf\|_{L^{q_0}} \leq M_0 \|f\|_{L^{p_0}}, \quad
        \|Tf\|_{L^{q_1}} \leq M_1 \|f\|_{L^{p_1}}.
    \end{equation*}
    Para cualquier $0<\theta<1$, se definen 
    \begin{equation*}
        \frac{1}{p_\theta} = \frac{1-\theta}{p_0} + \frac{\theta}{p_1}, 
        \quad 
        \frac{1}{q_\theta} = \frac{1-\theta}{q_0} + \frac{\theta}{q_1}.
    \end{equation*}
    Entonces, $T$ extiende a un operador continuo de $L^{p_\theta}(\Omega)$
    en $L^{q_\theta}(\Omega)$. Además, 
    \begin{equation*}
        \|Tf\|_{L^{q_\theta}} \leq M_0^{1-\theta}M_1^\theta \|f\|_{L^{p_\theta}}.
    \end{equation*}
\end{theorem}
Se continúa con el programa de definiciones y propiedades en los espacios 
de Lebesgue.

\begin{definition}[Convoluciones]
    Para funciones $f,g\in L^1(\R^n)$ se define su convolución como 
    \begin{equation*}
        (f*g)(x) := \int_{\R^n} f(x-y)g(y)\diff y.
    \end{equation*}
    Se puede notar que el cambio de variable $y\mapsto x-u$ implica la
    conmutatividad, es decir $f*g=g*f$.
\end{definition}
\begin{remark}
    En la definición anterior existe la pregunta sobre la convergencia de la
    integral. Para definir la convolución de forma rigurosa, se podría definir 
    primero para funciones que cumplan condiciones de regularidad más 
    fuertes, como las del espacio de Schwartz que se definirá en la siguiente
    sección, para luego definir el operador $*:L^1\times L^1 \rightarrow L^1$
    que estaría bien definido gracias a la siguiente propiedad.
\end{remark}

\begin{proposition}[Desigualdad de Young para convoluciones]
    Sean $1\leq p,q,r\leq \infty$ tales que $\frac{1}{p} + \frac{1}{q} = 1 + 
    \frac{1}{r}$, y sean $f \in L^p(\R^n)$, $g\in L^q(\R^n)$. Entonces
    se tiene que
    \begin{equation*}
        \|f*g\|_{L^r} \leq \|f\|_{L^p} \|g\|_{L^q}.
    \end{equation*}
\end{proposition}
\begin{proof}
    Se nota que gracias al teorema~\ref{theo:riesz-thorin} es suficiente 
    demostrar 
    \begin{equation}\label{eq:young-convolution-estimates}
        \|f*g\|_{L^p} \leq \|f\|_{L^p} \|g\|_{L^1}, \quad 
        \|f*g\|_{L^\infty} \leq \|f\|_{L^p} \|g\|_{L^t},
    \end{equation}
    para $\frac{1}{t} + \frac{1}{p} = 1$. En efecto, bastaría con considerar 
    el operador $f * \cdot$, los parametros
    \begin{equation*}
        p_0 = 1, \quad p_1 = t, \quad q_0 = p, \quad q_1 = \infty.
    \end{equation*}
    y $\|f\|_{L^p}$ como ambas constantes de estimación.
    Al aplicar la interpolación
    \begin{equation*}
        \frac{1}{r} = \frac{1-\theta}{p} + \frac{\theta}{\infty}, \quad 
        \frac{1}{q} = \frac{1-\theta}{1} + \frac{\theta}{t},
    \end{equation*}
    se obtiene la condición indicada para los parametros $p, q, r$. Ahora, 
    se procede a demostrar el primer estimativo de
    ~(\ref{eq:young-convolution-estimates}). Este se obtiene como resultado de
    la monotonia de la norma $L^p$, véase la proposición~\ref{prop:monotonia-Lp}.
    En efecto,
    \begin{align*}
        \|f*g\|_{L^p} & = \left\| \int_{\R^n} f(\cdot-y)g(y)\diff y 
        \right\|_{L^p} \\
        & \leq \int_{\R^n} \|f(\cdot-y)\|_{L^p} |g(y)|\diff y \\
        & \leq \|f\|_{L^p}\|g\|_{L^1}.
    \end{align*}
    Por otra parte, el segundo estimativo es resultado de la desigualdad de 
    H\"older 
    \begin{align*}
        \|f*g\|_{L^\infty} & \leq \int_{\R^n} |f(x-y)||g(y)| \diff y \\
        & \leq \|f\|_{L^p}\|g\|_{L^t}.
    \end{align*}
    Concluyendo con el resultado deseado.
\end{proof}
Se continua con un resultado importante sobre convergencia en 
espacios $L^p$
\begin{theorem}[Convergencia dominanda de Lebesgue]
    Sea $(f_k)_{k=1}^\infty$ una secuencia de funciones medibles en $\Omega$ 
    tales que convergen puntualmente a $f$ para casi todo $x\in\Omega$. Se supone
    que existe $g\in L^1(\Omega)$ tal que $|f_k|\leq g$ para todo $k$. Entonces
    $f$ es integrable y 
    \begin{equation*}
        \int_\Omega f \diff x = \lim_{k\rightarrow\infty} \int_\Omega 
        f_k \diff x.
    \end{equation*}
\end{theorem}

\begin{remark}
    Una implicación del resultado anterior es el hecho que los espacios $L^p$ son 
    completos, y por consecuencia son espacios de Banach. Particularmente, el 
    espacio $L^2$ es un espacio de Hilbert con producto interno dado por
    \begin{equation*}
        \angles{f, g}_{L^2} := \int_\Omega f\overline{g}\diff x,
    \end{equation*}
    donde $\overline{z}$ es el conjugado en los números complejos.
\end{remark}
Se concluye esta sección con la definición de la versión local de los espacios de Lebesgue. Para ello se necesita el siguiente espacio de funciones.
\begin{definition}[Funciones suaves de soporte compacto]
		Se dice que $\varphi:\Omega\subset\R^n\rightarrow\C$ es suave si es de clase $C^\infty$, o infinitamente diferenciable.  Se define su soporte como 
		\begin{equation*}
			\supp \varphi = \overline{\{x\in\Omega:\varphi(x)\neq 0\}}.
		\end{equation*}
		Si $\supp \varphi$ es compacto, se dice que $f\in\C^\infty_0(\Omega)$.
\end{definition}
\begin{definition}[Localización de espacios de Lebesgue]
	Se dice que una función medible $f:\Omega\subset\R^n\rightarrow\C$ es localmente integrable o pertenece a $L^{p}_{loc}(\Omega)$, con $1\leq p\leq\infty$ si 
	\begin{equation*}
		\|f\varphi\|_{L^p} < \infty,
	\end{equation*} 
	para todo $\varphi\in C^\infty_0(\Omega)$.
\end{definition}
\section{Transformada de Fourier en $\R^n$}
Ahora, se procede a definir y mostrar propiedades importantes de la
transformada de Fourier, una herramienta fundamental para el estudio
de las ecuaciones diferenciales en general y los operadores
pseudo-diferenciales en particular. 

\begin{definition}[Transformada de Fourier en $\R^n$]
    Dada $f\in L^1(\R^n)$, se define su transformada de Fourier como 
    \begin{equation*}
        (\mathcal{F}_{\R^n}f)(\xi) = \widehat{f}(\xi) := \int_{\R^n}
        e^{-2\pi i x \cdot \xi} f(x) \diff x,
    \end{equation*}
    para cualquier $\xi \in \R^n$.
\end{definition}

\begin{proposition}
    La transformada de Fourier es un operador continuo 
    $\mathcal{F}_{\R^n} : L^1(\R^n) \rightarrow L^\infty(\R^n)$ con norma uno:
    \begin{equation*}
        \|\widehat{f}\|_{L^\infty} \leq \|f\|_{L^1}.
    \end{equation*}
    Además, $\widehat{f}$ es continua en todas partes. 
\end{proposition}
\begin{proof}
    El estimativo es resultado de la desigualdad de Minkowski para integrales
    clásica 
    \begin{align*}
        \left|
            \int_{\R^n} e^{-2\pi i x \cdot \xi} f(x) \diff x
        \right| & \leq 
        \int_{\R^n} \left|e^{-2\pi i x \cdot \xi}\right|| f(x)| \diff x\\
        & \leq \|f\|_{L^1}.
    \end{align*}
    Ahora, la continuidad es consecuencia del teorema de convergencia dominada
    de Lebesgue. Para cualquier $\xi_k \rightarrow \xi$ se define 
    \begin{equation*}
        h_k(x) :=  e^{-2\pi i x \cdot \xi_k} f(x)
    \end{equation*}
    Entonces, se tiene que $|h_k| \leq |f|$ y se obtiene que 
    \begin{equation*}
        \int_{\R^n} e^{-2\pi i x \cdot \xi} f(x) \diff x = 
        \lim_{k\rightarrow\infty}\int_{\R^n} e^{-2\pi i x \cdot \xi_k} f(x) \diff x.
    \end{equation*} 
    Que es exactamente $\widehat{f}(\xi) = \lim \widehat{f}(\xi_k)$, el 
    resultado deseado.
\end{proof}
A pesar de que la transformada de Fourier está bien definida en el espacio
$L^1(\R^n)$, este presenta ciertas limitaciones técnicas debido a los pocos
requerimientos de regularidad para las funciones en este espacio. Es muy útil
tener acceso a otras herramientas resultantes de continuidad, diferenciabilidad,
y decaimiento.
Por lo tanto, se introduce notación que será importante a lo largo de este trabajo. 
\begin{definition}[Notación de multi-índice]
    Para $\alpha:=(\alpha_1, \ldots, \alpha_n), \beta:=(\beta_1, \ldots, \beta_n) 
    \in \N_0^n$, se define 
    \begin{equation*}
        \partial^\alpha := \frac{\partial^{\alpha_1}}{\partial x_1^{\alpha_1}}
        \cdots \frac{\partial^{\alpha_n}}{\partial x_n^{\alpha_n}}.
    \end{equation*}
    De forma similar, $x^\beta := x_1^{\beta_1} \cdots x_n^{\beta_n}$. Se dice 
    que $\alpha \leq \beta$ si $\alpha_i \leq \beta_i$ para todo $i$. Además, 
    se denota la longitud del multi-índice como
    $|\alpha| := \alpha_1 + \cdots \alpha_n$ y su factorial como
    $\alpha! := \alpha_1! \cdots \alpha_n!$.
\end{definition}

\begin{definition}[Espacio de Schwartz $\mathcal{S}(\R^n)$]
    Se dice que una función suave (infinitamente diferenciable) 
    $\varphi:\R^n\rightarrow\C$ se encuentra en $\mathcal{S}(\R^n)$ si se 
    cumple que
    \begin{equation*}
        \sup_{x\in\R^n}|x^\beta \partial^\alpha \varphi(x)| < \infty, 
    \end{equation*}
    para cualesquiera multi-índices $\alpha, \beta \in \N_0^n$. Ahora,
    se dice que $\varphi_j\rightarrow\varphi$ en $\mathcal{S}(\R^n)$ si
    \begin{equation*}
        \sup_{x\in\R^n}|x^\beta \partial^\alpha (\varphi_j-
        \varphi)(x)| \rightarrow 0, 
    \end{equation*}
    cuando $j\rightarrow\infty$ para cualesquiera multi-índices 
    $\alpha, \beta \in \N_0^n$.
\end{definition}
\begin{proposition}
    Para cualquier $1\leq p\leq\infty$ se tiene que 
    $\mathcal{S}(\R^n)\subset L^p(\R^n)$ con encaje continuo.
\end{proposition}
\begin{proof}
    El caso $p=\infty$ es trivial, pues las funciones en el espacio de Schwartz
    son acotadas por definición.
    Sea $\varphi_j\rightarrow0$ en $\mathcal{S}(\R^n)$, entonces
    \begin{align*}
        \int_{\R^n} |\varphi_j(x)|^p \diff x & = 
        \int_{\R^n} \angles{x}^{pN} |\varphi_j(x)|^p \angles{x}^{-pN} \diff x \\ 
        & \lesssim \max_{|\beta|\leq N} \sup_{x\in\R^n} 
        |x^\beta\varphi_j(x)|^p \int_{\R^n}\angles{x}^{-pN} \diff x \\
        & \rightarrow 0,
    \end{align*}
    donde $N\in\N$ se escoge de manera que la última integral converga. 
\end{proof}

\begin{theorem}
    Sea $\varphi\in\mathcal{S}(\R^n)$. Entonces $2\pi i\xi_j\widehat{\varphi}(\xi)
    = \widehat{\partial_j\varphi}(\xi)$ y 
    $ 2\pi i \widehat{x_j\varphi}(\xi) = - \partial_j\widehat{\varphi}(\xi)$
\end{theorem}
\begin{proof}
    Para la primera expresión se procede por integración por partes
    \begin{align*}
        \widehat{\partial_j\varphi}(\xi) & = \int_{\R^n} e^{-2\pi i x\cdot \xi} 
        \partial_{x_j}\varphi(x) \diff x \\
        & = - \int_{\R^n} (\partial_{\xi_j} e^{-2\pi i x\cdot \xi}) 
        \varphi(x) \diff x\\
        & = 2\pi i \xi_j \int_{\R^n} e^{-2\pi i x\cdot \xi} 
        \varphi(x) \diff x.
    \end{align*}
    Se nota que no aparece el término con la frontera debido a que $\varphi$ 
    se desvanece en el infinito. Ahora, para la segunda expresión 
    \begin{equation*}
        \partial_{\xi_j} \widehat{\varphi}(\xi) = 
        \int_{\R^n}  e^{-2\pi i x\cdot \xi} (-2\pi i x_j)
        \varphi(x) \diff x.
    \end{equation*}
    Concluyendo la prueba.
\end{proof}
Por lo que se puede concluir lo siguiente 
\begin{corollary}
    Sea $\varphi\in\mathcal{S}(\R^n)$. Entonces, 
    \begin{equation*}
        \xi^\beta \partial^\alpha \widehat{\varphi}(\xi) = 
        (2\pi i)^{|\alpha|-|\beta|} (-1)^{|\alpha|} \widehat{\partial^\beta 
        [x^\alpha \varphi]}(\xi).
    \end{equation*}
    Por lo que 
    \begin{align*}
        |\xi^\beta \partial^\alpha \widehat{\varphi}(\xi)| & \leq 
        |2\pi i|^{|\alpha|-|\beta|} \int_{\R^n} |\partial^\beta 
        [x^\alpha \varphi(x)]| \diff x \\
        & \leq |2\pi i|^{|\alpha|-|\beta|} \sup_{x\in\R^n} 
        \left|(1+|x|)^{n+1} \partial^\beta 
        [x^\alpha \varphi(x)]\right| \int_{\R^n} (1+|x|)^{-n-1}\diff x \\
        & = C \sup_{x\in\R^n} \left|(1+|x|)^{n+1} \partial^\beta 
        [x^\alpha \varphi(x)]\right| \\
        & < \infty.
    \end{align*}
    Particularmente, $\mathcal{F}_{\R^n}$ mapea $\mathcal{S}(\R^n)$ en sí 
    mismo. Además, por el teorema de convergencia dominada de Lebesgue, 
    la transformada de Fourier es un operador continuo.
\end{corollary}
En realidad, es un isomorfismo en $\mathcal{S}(\R^n)$. Para ello se demostrarán 
algunos lemas útiles.

\begin{lemma}[Fórmula de multiplicación para la transformada de Fourier]
    Sean $f, g \in L^1(\R^n)$. Entonces, $\int_{\R^n}\widehat{f} g \diff x 
    = \int_{\R^n} f \widehat{g} \diff x$.
\end{lemma}
\begin{proof}
    Aplicando el teorema de Fubini 
    \begin{align*}
        \int_{\R^n}\widehat{f} g \diff x & = \int_{\R^n} \left[
            \int_{\R^n} e^{-2\pi i x\cdot y} f(y) \diff y
        \right] g(x) \diff x \\
        & = \int_{\R^n} \left[
            \int_{\R^n} e^{-2\pi i x\cdot y} g(x) \diff x
        \right] f(y) \diff y \\ 
        & = \int_{\R^n} f \widehat{g} \diff y
    \end{align*}
    Concluyendo la prueba.
\end{proof}
\begin{lemma}[Transformada de Fourier para Gaussiana]
    Se tiene que
    \begin{equation*}
       \int_{\R^n} e^{-2\pi i x \cdot \xi} e^{-\varepsilon\pi^2|x|^2}
        \diff x = (\pi\varepsilon)^{-n/2} e^{-|\xi|^2/\varepsilon},  
    \end{equation*}
    para todo $\varepsilon > 0$. Gracias al cambio de variable 
    $x \mapsto 2\pi x$ y $\varepsilon \mapsto 2\varepsilon$, esto equivale a
    \begin{equation*}
       \int_{\R^n} e^{- i x \cdot \xi} e^{-\varepsilon^2|x|^2/2}
        \diff x = (2\pi/\varepsilon)^{-n/2} e^{-|\xi|^2/(2\varepsilon)}. 
    \end{equation*}
\end{lemma}
\begin{proof}
    La segunda expresión sigue del caso unidimensional
    \begin{align*}
        \int_{-\infty}^{\infty} e^{-it\tau}e^{-t^2/2} \diff t & = 
        e^{-\tau^2/2} \int_{-\infty}^{\infty} e^{-(t+i\tau)^2/2} \diff t \\
        & = e^{-\tau^2/2} \int_{-\infty}^{\infty} e^{-t^2/2} \diff t \\
        & = \sqrt{2\pi} e^{-\tau^2/2}.
    \end{align*}
    Con el cambio de variable $t\mapsto\sqrt{\varepsilon}t$ y 
    $\tau\mapsto\tau/\sqrt{\tau}$ se tiene que 
    \begin{equation*}
        \sqrt{\varepsilon}\int_{-\infty}^{\infty} e^{-it\tau}e^{-\varepsilon t^2/2} 
        \diff t = 
        \sqrt{2\pi} e^{-\tau^2/(2\varepsilon)}.
    \end{equation*}
    El caso multidimensional sigue del producto de las integrales unidimensionales.
\end{proof}
\begin{theorem}[Fórmula de inversión de Fourier]
    La transformada de Fourier es un isomorfismo de $\mathcal{S}(\R^n)$ en si 
    mismo con inverso dado por 
    \begin{equation*}
        (\mathcal{F}^{-1}_{\R^n}f)(x) := \int_{\R^n} e^{2\pi i x \cdot \xi} f(\xi)
        \diff \xi.
    \end{equation*}
\end{theorem}
\begin{proof}
    El teorema de convergencia dominada de Lebesgue permite realizar la 
    sustitución
    \begin{align*}
        (\mathcal{F}^{-1}_{\R^n}\widehat{\varphi})(x) &= 
        \int_{\R^n}e^{2\pi ix \cdot \xi} \widehat{\varphi}(\xi)\diff \xi
        = \lim_{\varepsilon\rightarrow0} 
        \int_{\R^n}e^{2\pi ix \cdot \xi} \widehat{\varphi}(\xi)
        e^{-2\varepsilon\pi^2|\xi|^2} \diff \xi \\
        &= \lim_{\varepsilon\rightarrow0} \int_{\R^n}
        \int_{\R^n}e^{2\pi i(x-y) \cdot \xi} \varphi(y)
        e^{-2\varepsilon\pi^2|\xi|^2} \diff y \diff \xi.
    \end{align*}
    Con el cambio de variable $y \mapsto y+x$ se obtiene que
    \begin{equation*}
        (\mathcal{F}^{-1}_{\R^n}\widehat{\varphi})(x) = 
        \lim_{\varepsilon\rightarrow0} \int_{\R^n}
        \int_{\R^n}e^{-2\pi iy \cdot \xi} \varphi(y+x)
        e^{-2\varepsilon\pi^2|\xi|^2} \diff y \diff \xi.
    \end{equation*}
    Por el teorema de Fubini y la transformada de Fourier para 
    Gaussianas se tiene que
    \begin{align*}
        (\mathcal{F}^{-1}_{\R^n}\widehat{\varphi})(x) &= 
        \lim_{\varepsilon\rightarrow0} \int_{\R^n} \varphi(y+x) 
        \int_{\R^n}e^{-2\pi iy \cdot \xi} 
        e^{-2\varepsilon\pi^2|\xi|^2}  \diff \xi \diff y \\
        & = \lim_{\varepsilon\rightarrow0} \int_{\R^n} \varphi(y+x) 
        (2\pi\varepsilon)^{-n/2} e^{-|y|^2/(2\varepsilon)} \diff y
    \end{align*}
    Con un último cambio de variable $y\mapsto\sqrt{\varepsilon}z$
    se concluye 
    \begin{align*}
        (\mathcal{F}^{-1}_{\R^n}\widehat{\varphi})(x) &= 
        \lim_{\varepsilon\rightarrow0} \int_{\R^n} \varphi(\sqrt{\varepsilon}z+x) 
        (2\pi)^{-n/2} e^{-|z|^2/2} \diff z \\
        & = (2\pi)^{-n/2} \varphi(x) \int_{\R^n}e^{-|z|^2/2} \diff z = \varphi(x).
    \end{align*}
    Finalizando con la prueba.
\end{proof}
El siguiente teorema relaciona la transformada de Fourier con las 
convoluciones 
\begin{theorem}
    Sean $\varphi,\psi\in\mathcal{S}(\R^n)$ entonces se cumple que 
    $\widehat{\varphi*\psi}(\xi) = \widehat{\varphi}(\xi) 
    \widehat{\psi}(\xi)$ y que 
    $\widehat{\varphi\psi}(\xi) = (\widehat{\varphi} * 
    \widehat{\psi})(\xi)$.
\end{theorem}
\begin{proof}
    Para la primera expresión se tiene que 
    \begin{align*}
        \widehat{\varphi*\psi}(\xi) &= 
        \int_{\R^n} e^{-2\pi i x\cdot\xi} (\varphi*\psi)(x)\diff x \\
        & = \int_{\R^n} \int_{\R^n} e^{-2\pi i (x-y)\cdot\xi} 
        \varphi(x-y)e^{-2\pi i y\cdot\xi} \psi(y) \diff y \diff x \\
        & = \int_{\R^n} \int_{\R^n} e^{-2\pi i z\cdot\xi} 
        \varphi(z)e^{-2\pi i y\cdot\xi} \psi(y) \diff y \diff z \\
        & = \widehat{\varphi}(\xi)\widehat{\psi}(\xi).
    \end{align*}
    Ahora, para la segunda expresión 
    \begin{align*}
        (\widehat{\varphi} * \widehat{\psi})(\xi) & = 
        \int_{\R^n} \widehat{\varphi}(\xi - y) \widehat{\psi}(y) \diff y \\
        & = \int_{\R^n} \int_{\R^n} e^{-2\pi i x\cdot(\xi - y)} \varphi(x)
        \widehat{\psi}(y) \diff x \diff y \\ 
        & = \int_{\R^n} \left[\int_{\R^n}  e^{2\pi i x\cdot y}\widehat{\psi}(y) 
        \diff y\right] e^{-2\pi i x\cdot\xi} \varphi(x) \diff x \\
        &= \int_{\R^n} e^{-2\pi i x\cdot\xi} \varphi(x) \psi(x) \diff x = 
        \widehat{\varphi\psi}(\xi).
    \end{align*}
    Completando la prueba.
\end{proof}

\section{Transformada de Fourier en $\T^n$}
TODO TODO

\section{Distribuciones y espacios de Sobolev en $\R^n$}
En esta sección se inicia introduciendo el espacio de distribuciones templadas
que permite extender la transformada de Fourier a un espacio más general 
que $L^1(\R^n)$.
\begin{definition}[Distribuciones templadas $\mathcal{S}'(\R^n)$]
    Se define el \textit{espacio de distribuciones templadas} como 
    el espacio de funcionales lineales continuos 
    $u:\mathcal{S}(\R^n)\rightarrow\C$. 
    En este caso, se entiende la continuidad en el sentido que si 
    $\varphi_j\rightarrow\varphi$ en $\mathcal{S}(\R^n)$, entonces se tiene
    que $u(\varphi_j) \rightarrow u(\varphi)$ en $\C$. Además, se dice que 
    $u_j\rightarrow u$ en $\mathcal{S}'(\R^n)$ si 
    $u_j(\varphi)\rightarrow u(\varphi)$ para todo $\varphi\in \mathcal{S}(\R^n)$.
\end{definition}
Las funciones en $\mathcal{S}(\R^n)$ se les conoce como las funciones de prueba
del espacio de distribuciones templadas. Otra notación usual para $u(\varphi)$
es $\angles{u,\varphi}$.
\begin{remark}[Funciones como distribuciones]
    Se puede considerar a $f\in L^p(\R^n)$ como una distribución templada. Se 
    define el funcional $u_f$ de la siguiente manera
    \begin{equation*}
        \angles{u_f, \varphi} := \int_{\R^n} f\varphi \diff x.
    \end{equation*}
    Claramente es un funcional lineal. La continuidad es resultado de la
    desigualdad de H\"older y el encaje continuo de las funciones de prueba
    en el espacio $L^q$. En efecto, para $\varphi_j\rightarrow\varphi$ en 
    $\mathcal{S}(\R^n)$ se tiene que
    \begin{equation*}
        |\angles{u_f, \varphi_j} - \angles{u_f, \varphi}| \leq 
        \|f\|_{L^p}\|\varphi_j - \varphi\|_{L^q}.
    \end{equation*}
    Por simplicidad se denota $\angles{u_f, \varphi} = \angles{f, \varphi}$.
\end{remark}
Particularmente, para $\varphi\in\mathcal{S}(\R^n)$, se puede motivar la 
definición de distintas propiedades de distribuciones mediante la manipulación 
del funcional $u_\varphi$ mencionado anteriormente. Por ejemplo, en vista de
la integración por partes tenemos que 
\begin{equation*}
    \angles{\partial_j\varphi, \psi} = \int_{\R^n} (\partial_j\varphi) \psi\diff x
    = - \int_{\R^n} \varphi(\partial_j \psi)\diff x = - \angles{\varphi,\partial_j
    \psi}.
\end{equation*}
Por lo que definimos la derivada en el sentido de distribuciones de la 
siguiente manera
\begin{definition}[Derivada distribucional]
    Para $u\in\mathcal{S}'(\R^n)$ se define
    \begin{equation*}
    	\angles{\partial^\alpha u, \varphi} := (-1)^{|\alpha|} \angles{u, \partial^\alpha\varphi},
    \end{equation*}
    para cualquier función de prueba $\varphi$ y cualquier multi-índice $\alpha\in\N_0^n$. 
\end{definition}
\begin{example}
	Considere la función Heaviside, o escalón, dada por
	\begin{equation*}
		H(x) := \begin{cases} 0, &  x <0 \\
			1,  &x \geq 0\end{cases}.
	\end{equation*}
	Es claro que representa una distribución templada, así que se calcula su derivada distribucional
	\begin{align*}
		\angles{\partial H, \varphi} = - \int_\R H\partial\varphi\diff x  = -\int_0^\infty \partial\varphi\diff x \ = -\varphi|_0^\infty  = \varphi(0) =: \angles{\delta, \varphi}.
	\end{align*}
	Donde $\delta$ es el funcional conocido como la delta de Dirac. Por lo que se tiene que en el sentido de distribuciónes que $\partial H = \delta$.
\end{example}
Por otra parte, la fórmula de multiplicación de Fourier motiva la definición de la transformada de Fourier para distribuciones.
\begin{definition}[Transformada de Fourier para distribuciones]
	Para $u\in\mathcal{S}'(\R^n)$ se define 
	\begin{equation*}
		\angles{\mathcal{F}u, \varphi} := \angles{u, \mathcal{F}\varphi}, \quad 
		\angles{\mathcal{F}^{-1}u, \varphi} := \angles{u, \mathcal{F}^{-1}\varphi},
	\end{equation*}
	para cualquier función de prueba $\varphi$.
\end{definition}
\begin{example}
	Considere la distribución de la delta de Dirac dada por $\angles{\delta, \varphi} := \varphi(0)$.  Se calcula su transformada de Fourier de la siguiente manera
	\begin{equation*}
		\angles{\mathcal{F}\delta, \varphi} = \angles{\delta, \widehat{\varphi}} = \widehat{\varphi}(0) = \int_\R \varphi\diff x = \angles{1, \varphi}.
	\end{equation*}
	Por lo que en el sentido de distribuiciones se tiene que $\widehat{\delta} = 1$ la función constante, que es acotada y por tanto una distribución. También se puede demostrar que $\widehat{1}  = \delta$. En efecto 
	\begin{equation*}
		\angles{\mathcal{F}(1), \varphi} = \int_\R \widehat{\varphi} \diff x = 
		\mathcal{F}^{-1}(\widehat{\varphi})(0) = \varphi(0)  = \angles{\delta, \varphi}.
	\end{equation*}
\end{example}
\begin{definition}[Espacios de Sobolev]
	Sea $1\leq p \leq\infty$ y sea $k\in\N_0$. El \textit{espacio de Sobolev} $W^k_p(\Omega)$ consiste de todas las funciones $f\in L^p(\Omega)$ tales que para cualquier multi-índice $|\alpha|\leq k$ se tiene que $\partial^\alpha f$ existe (en el sentido de distribuciones) y pertenece a $L^p(\Omega)$. Para tales funciones se define 
	\begin{equation*}
		\|f\|_{W^k_p(\Omega)} := \left( \sum_{|\alpha|\leq k} \|\partial^\alpha f\|_{L^p}^p
		\right)^{1/p},
	\end{equation*}
	para $1\leq p <\infty$. Para $p=\infty$ se define como 
	\begin{equation*}
		\|f\|_{W^k_\infty(\Omega)} := \max_{|\alpha|\leq k} \|\partial^\alpha f\|_{L^\infty}.
	\end{equation*}
\end{definition}
\begin{remark}
	Se advierte al lector que existen otras notaciones disponibles en la literatura.  Por ejemplo $L^p_k$, o $W^{p,k}$. Además, cuando $p=2$, se suele denotar como $H^k$.
\end{remark}
\begin{theorem} 
	Sea $f, g \in W^k_p(\Omega)$ y sea $\alpha$ un multi-índice con $|\alpha|\leq k$, entonces se tiene que 
	\begin{enumerate}
		\item $\partial^\alpha f \in W^{k-|\alpha|}_p$ y que $\partial^{\alpha}(\partial^{\beta}f) = \partial^{\alpha+\beta} f = \partial^{\beta}(\partial^{\alpha}f)$, para todos multi-índices que satisfacen  $|\alpha|+|\beta| \leq k$,
		\item $\lambda f + \mu g \in W^k_p$ y $\partial^\alpha(\lambda f + \mu g) = \lambda\partial^\alpha f + \mu\partial^\alpha g$, para cualesquiera $\lambda, \mu\in \C$,
		\item $\|\cdot\|_{W^k_p}$ es una norma,
		\item $W^k_p(\Omega)$ es un espacio de Banach.
	\end{enumerate}  
\end{theorem}
\begin{proof}
	Los primeros dos incisos son resultado de la definición de derivada en el sentido de distribuciones. En efecto, para $(1)$ se tiene que 
	\begin{equation*}
		\angles{\partial^\alpha(\partial^\beta f), \varphi} = 
		(-1)^{|\alpha|} = \angles{\partial^\beta f, \partial^\alpha\varphi} =
		(-1)^{|\alpha| + |\beta|} \angles{f, \partial^{\alpha+\beta}\varphi}. 
	\end{equation*}
	El otro caso es análogo. El inciso $(2)$ es resultado de la linealidad de $\angles{\cdot, \varphi}$.  Para el inciso $(3)$ es claro que $\|\lambda f\|_{W^k_p} = |\lambda|\|f\|_{W^k_p}$ por lo anterior, y que $|f\|_{W^k_p} = 0$ si y solo si $f$ se anula en casi todas partes. La desigualdad triangular para $p=\infty$ es trivial, para el caso $1\leq p<\infty$ se tiene que 
	\begin{align*}
		\|f+g\|_{W^k_p} &= \left(\sum_{|\alpha|\leq k} \|\partial^\alpha f + \partial^\alpha g\|_{L^p}^p
		\right)^{1/p} \\ 
		& \leq \left(\sum_{|\alpha|\leq k} (\|\partial^\alpha f\|_{L^p} +  \|\partial^\alpha g\|_{L^p})^p
		\right)^{1/p} \\
		& \leq \left(\sum_{|\alpha|\leq k} \|\partial^\alpha f\|_{L^p}^p
		\right)^{1/p} + \left(\sum_{|\alpha|\leq k} \|\partial^\alpha g\|_{L^p}^p
		\right)^{1/p} \\
		& = \|f\|_{W^k_p} + \|g\|_{W^k_p}.
	\end{align*}
	Para el inciso $(4)$ se toma una sucesión de Cauchy $f_j$ en $W^k_p$. Entonces, $\partial^\alpha f_j$ es una sucesión de Cauchy en $L^p$ para todo $|\alpha|\leq k$. Como $L^p$ es completo, se tiene que $\partial^\alpha f_j$ converge a algún $g_\alpha$ en $L^p$. Entonces, se tiene que
	\begin{align*}
		\angles{\partial^\alpha g_0, \varphi}& = (-1)^{|\alpha|}  \angles{g_0, \partial^\alpha \varphi} \\
		& = \lim_{j\rightarrow\infty}(-1)^{|\alpha|} \angles{f_j, \partial^\alpha \varphi} \\
		& =  \lim_{j\rightarrow\infty}\angles{\partial^\alpha f_j, \varphi} \\
		& = \angles{g_\alpha, \varphi}.
	\end{align*} 
	Por lo que $\partial^\alpha g_0 = g_\alpha$ y $f_j \rightarrow g_0$ en $W^k_p$.
\end{proof}

\section{Distribuciones y espacios de Sobolev en $\T^n$}
TODO TODO

\section{Espacios de Hardy en $\R^n$ y $\T^n$}
TODO TODO


\chapter{Operadores pseudo-diferenciales}

\section{Definición y propiedades básicas en $\R^n$}
\begin{definition}[Clases de símbolos de Hörmander $S^m_{\rho,\delta}(\R^n\times\R^n)$]
	Sean $0\leq\delta,\rho\leq1$. Se dice que $a \in S^m_{\rho,\delta}(\R^n\times\R^n)$ si $a:=(x,\xi)$ es suave en $\R^n\times\R^n$ y cumple que
	\begin{equation*}
		|\partial^\beta_x\partial^\alpha_\xi a(x, \xi)| \lesssim_{\alpha\beta}\angles{\xi}^{m-\rho|\alpha|+\delta|\beta|},
	\end{equation*}
	para cualesquiera multi-índices $\alpha,\beta$. Se dice que estos símbolos tienen orden $m\in\R$.
\end{definition}
\begin{definition}
	Sean $0\leq\delta,\rho\leq1$ y sea $a \in S^m_{\rho,\delta}(\R^n\times\R^n)$. El operador pseudo-diferencial con símbolo $a:=a(x,\xi)$ se define como
	\begin{equation*}
		T_af(x):= \int_{\R^n} e^{2\pi ix\cdot\xi} a(x, \xi)\widehat{f}(\xi)\diff \xi,
	\end{equation*}
	donde $f\in\mathcal{S}(\R^n)$. La clase de operadores pseudo-diferenciales con símbolos en $S^m_{\rho,\delta}(\R^n\times\R^n)$ se denotan por $\Psi^m_{\rho,\delta}(\R^n\times\R^n)$.
\end{definition}
\begin{proposition}
	Para $a \in S^m_{\rho,\delta}(\R^n\times\R^n)$ y $f\in\mathcal{S}(\R^n)$ se tiene que $T_af\in\mathcal{S}(\R^n)$.
\end{proposition}
\begin{proof}
	Note que como $\widehat{f}\in\mathcal{S}(\R^n)$, se tiene que 
	\[|\partial^\beta_xa(x, \xi)\widehat{f}(\xi)|\lesssim \angles{\xi}^{m+\delta|\beta|}\angles{\xi}^{-N} ,\]
	para algún $N>0$ apropiado, por lo que todas sus derivadas respecto a $x$ son absolutamente convergentes y se tiene que $T_af\in C^\infty(\R^n)$. Ahora, se define el operador 
	\begin{equation*}
		L_\xi := (1 + 4\pi^2|x|^2)^{-1}(I-\mathcal{L}_\xi),
	\end{equation*}
	donde $\mathcal{L}_\xi$ es el laplaciano. Note que $ L_\xi(e^{2\pi ix\cdot\xi}) = e^{2\pi ix\cdot \xi} $ y por integracion por partes se tiene que
	\begin{equation*}
		T_af(x) = \int_{\R^n} e^{2\pi ix\cdot \xi}L_\xi^N [a(x,\xi)\widehat{f}(\xi)] \diff \xi.
	\end{equation*}
	Por lo que $|T_af(x)| \lesssim_N \angles{x}^{-2N}$ para cualquier $N$ y se concluye que $T_af$ decae rapidamente. Este argumento se puede aplicar para cualquiera de sus derivadas y se obtiene que $T_af\in\mathcal{S}(\R^n)$.
\end{proof}
\begin{example}[Operadores diferenciales]
	Sea $P:=\sum_{|\alpha|\leq m} a_\alpha(x)\partial^\alpha_x$ un operador de derivadas parciales. Entonces, al considerarle como un operador pseudo-diferencial se tiene que su símbolo es simplemente su polinomio característico $p(x,\xi) = \sum_{|\alpha|\leq m} a_\alpha(x)(2\pi i\xi)^\alpha$. Si las funciones coeficientes $a_\alpha$ son continuas, este símbolo pertenece a la clase de Hörmander de orden $m$.
\end{example}
\begin{remark}[Kernel de un operador pseudo-diferencial]
		Se puede reescribir la definición de operador pseudo-diferencial de la siguiente manera
		\begin{align*}
			T_af(x) & = \int_{\R^n} e^{2\pi ix\cdot\xi} a(x, \xi)\widehat{f}(\xi)\diff \xi\\
			& = \int_{\R^n} \int_{\R^n} e^{2\pi i (x-y)\cdot\xi}a(x, \xi)f(y) \diff y \diff \xi\\
			& =\int_{\R^n} k(x, y) f(y)\diff y,
		\end{align*}
		donde se define en el sentido de distribuciones al kernel de Schwartz del operador pseudo-diferencial como
		\begin{equation*}
			k(x, y) := \int_{\R^n} e^{2\pi i (x-y)\cdot\xi}a(x, \xi) \diff \xi 
		\end{equation*}
\end{remark}
\begin{theorem}[Composición de operadores pseudo-diferenciales]
	Sea $0\leq\delta<1$, sea $0<\rho\leq1$, sea $a \in S^{m_1}_{\rho,\delta}(\R^n\times\R^n) $ y sea $ b\in S^{m_2}_{\rho,\delta}(\R^n\times\R^n) $. Entonces, existe un símbolo $ c\in S^{m_1+m_2}_{\rho,\delta}(\R^n\times\R^n)$ tal que 
	$T_c = T_a \circ T_b$. Además, se tiene la fórmula asimptótica 
	\begin{equation*}
		c \sim \sum_\alpha \frac{(2\pi i)^{-|\alpha|}}{\alpha!}(\partial^\alpha_\xi a)(\partial^\alpha_x b).
	\end{equation*}
	Es decir, para cualquier $N>0$, se tiene que 
	\begin{equation*}
		c - \sum_{|\alpha|<N} \frac{(2\pi i)^{-|\alpha|}}{\alpha!}(\partial^\alpha_\xi a)(\partial^\alpha_x b) \in S^{m_1+m_2-\rho N}_{\rho,\delta}(\R^n\times\R^n).
	\end{equation*}
\end{theorem}
\begin{proof}
	Fije un $x_0\in\R^n$ y sea $\chi\in C^\infty_0(\R^n)$ tal que $\supp \chi\subset\{x\in\R^n:|x-x_0|\leq 2\}$ y tal que $\chi(x) = 1$ para $|x-x_0|\leq 1$. Realice la descomposición 
	\begin{equation*}
		b = \chi b + (1-\chi)b := b_1 + b_2.
	\end{equation*}
	Entonces, se tiene que 
	\begin{align*}
		(T_a\circ T_{b_1})f(x) &= \int_{\R^n}\int_{\R^n} e^{2\pi i (x-y)\cdot\eta} a(x, \eta)
		\int_{\R^n}\int_{\R^n} e^{2\pi i(y-z)\cdot\xi}b_1(y, \xi)f(z)\diff z \diff \xi \diff y \diff \eta\\
		& = \int_{\R^n}\int_{\R^n}  e^{2\pi i (x-z)\cdot\xi} \int_{\R^n}\int_{\R^n} 
		e^{2\pi i(x-y)\cdot(\eta-\xi)} a(x, \eta)b_1(y, \xi)\diff y \diff \eta f(z) \diff z \diff \xi,
	\end{align*}
	donde se aprovecho que $(x-y)\cdot(\eta-\xi) + (x-z)\cdot \xi= (x-y)\cdot\eta + (y-z)\cdot\xi$. Por lo que se define 
	\begin{align*}
		c(x, \xi) & :=  \int_{\R^n}\int_{\R^n} 
		e^{2\pi i(x-y)\cdot(\eta-\xi)} a(x, \eta)b_1(y, \xi)\diff y \diff \eta \\ 
		&=  \int_{\R^n}\
		e^{2\pi ix\cdot(\eta-\xi)} a(x, \eta)\widehat{b_1}(\eta-\xi, \xi) \diff \eta \\
		&=  \int_{\R^n}\
		e^{2\pi ix\cdot\eta} a(x, \eta+\xi)\widehat{b_1}(\eta, \xi) \diff \eta.
	\end{align*}
	Como $b_1$ tiene soporte compacto en $x$, se tiene que $\widehat{b_1}$ es de decaimiento rápido uniformemente en $\xi$ y que 
	\begin{equation*}
		|\widehat{b_1}(\eta,\xi)| \lesssim_M \angles{\eta}^{-M}\angles{\xi}^{m_2},
	\end{equation*}
	para todo $M\leq0$. La expansión de Taylor en la segunda variable de $a(x,\xi+\eta)$ resulta en 
	\begin{equation*}
		a(x, \xi + \eta) = \sum_{|\alpha|<N} \frac{1}{\alpha!} \partial^\alpha_\xi a(x, \xi)\eta^\alpha + R_N(x, \xi, \eta),
	\end{equation*}
	donde $R_N$ es un residuo que se discutirá más adelante. Al sustituir esta expresión en la fórmula para $c(x, \xi)$ se obtiene que 
	\begin{equation*}
		\int_{\R^n} e^{2\pi i x\cdot\eta}[\partial^\alpha_\xi a(x, \xi)\eta^\alpha]\widehat{b_1}(\eta, \xi) \diff \eta = (2\pi i)^{-|\alpha|} 
		\partial^\alpha_\xi a(x, \xi) \partial^\alpha_x b_1(x, \xi),
	\end{equation*}
	que corresponden a los términos de la expansión asimptótica. Ahora, el símbolo resultante del residuo es 
	\begin{equation*}
		\int_{\R^n} e^{2\pi i x\cdot\eta} R_N(x, \xi, \eta) \widehat{b_1}(\eta, \xi) \diff \eta.
	\end{equation*}
	Pero, se puede estimar mediante 
	\begin{equation*}
		|R_N(x, \xi, \eta)| \lesssim_N |\eta|^N \max\{|\partial^\alpha_\xi a(x, \zeta)| : |\alpha|=N,\, \zeta \text{ interpolación de } \eta \text{ y }\eta+\xi\}.
	\end{equation*}
	Note que si $|\eta|\leq|\xi|/2$, entonces cualquier $\zeta$ de la expresión anterior es proporcional a $\xi$, por lo que para este caso se puede estimar
	\begin{equation*}
		|R_N(x, \xi, \eta)| \lesssim_N |\eta|^N\angles{\xi}^{m_1-\rho N}.
	\end{equation*}
	Por otra parte, si $\rho N\geq m_1$, se tiene la siguiente cota para cualquier caso
	\begin{equation*}
		|R_N(x, \xi, \eta)| \lesssim_N |\eta|^N.
	\end{equation*}
	Combinando los estimativos y la expresión del residuo del símbolo se obtiene que 
	\begin{align*}
		&\left| \int_{\R^n} e^{2\pi i x\cdot\eta} R_N(x, \xi, \eta) \widehat{b_1}(\eta, \xi) \diff \eta \right| \\
		&\lesssim_{M,N} \angles{\xi}^{m_1+m_2-\rho N} \int_{|\eta|<|\xi|/2} \angles{\eta}^{-M}|\eta|^N\diff\eta + \angles{\xi}^{m_2} \int_{|\eta|\geq|\xi|/2}\angles{\eta}^{-M}|\eta|^N\diff\eta.
	\end{align*}
	Al escoger $M$ lo suficientemente grande, se puede estimar el residuo simbólico por $\angles{\xi}^{m_1+m_2-\rho N}$. Ahora, note que $\partial^\alpha_\xi\partial^\beta_x R_N(x, \xi, \eta)$ es el residuo de la expansión de $\partial^\alpha_\xi\partial^\beta_x a(x, \xi+\eta)$. Por lo que un argumento similar resulta en 
	\begin{equation*}
		\left|\int_{\R^n} e^{2\pi i x\cdot\eta} [\partial^\alpha_\xi\partial^\beta_x R_N(x, \xi, \eta)]\widehat{b_1}(\eta, \xi) \diff \eta\right| \lesssim_{\alpha\beta} \angles{\xi}^{m_1+m_2- \rho N -\rho|\alpha| + \delta|\beta|}.
	\end{equation*}
	Ahora, solo queda demostrar que $T_a\circ T_{b_2}$ tiene símbolo de orden $-\infty$ y no afecta la fórmula asimptótica. Para ello, se utiliza integración por partes para obtener propiedades de regularidad del símbolo restante. Considere el Laplaciano en $\eta$ 
	\begin{equation*}
		\Delta_\eta^{N_1} e^{2\pi i(x-y)\cdot(\eta-\xi)} = 
		(- 4\pi^2|x-y|^2)^{N_1}  e^{2\pi i(x-y)\cdot(\eta-\xi)}.
	\end{equation*}
	Y el Laplaciano en $y$,
	\begin{equation*}
		(1-\Delta_y)^{N_2} e^{2\pi i(x-y)\cdot(\eta-\xi)}  = 
		(1+4\pi^2|\xi-\eta|^2)^{N_2} e^{2\pi i(x-y)\cdot(\eta-\xi)}.
	\end{equation*}
	Además, se tiene que 
	\begin{equation*}
		\angles{\xi-\eta}\angles{\eta} = \sqrt{1 + |\xi-\eta|^2 + |\eta|^2 + |\xi-\eta|^2|\eta|^2} \geq \sqrt{1 + |\xi|^2} = \angles{\xi}.
	\end{equation*}
	Combinando estos estimativos y las desigualdades simbólicas se obtiene que
	\begin{align*}
		|c_2(x, \xi)| & = \left| \int_{\R^n}\int_{\R^n} 
		e^{2\pi i(x-y)\cdot(\eta-\xi)} a(x, \eta)b_2(y, \xi)\diff y \diff \eta \right| \\
		&  = \left| \int_{\R^n}\int_{\R^n} 
		e^{2\pi i(x-y)\cdot(\eta-\xi)} \frac{\Delta_\xi^{N_1}a(x, \eta)}{(-4\pi^2|x-y|^2)^{N_1}} \frac{(1-\Delta_y)^{N_2}b_2(y, \xi)}{(1+4\pi^2|\xi-\eta|^2)^{N_2}}\diff y \diff \eta \right|  \\
		& \lesssim \int_{\R^n}\int_{\R^n}  \frac{\angles{\eta}^{m_1-2\rho N_1}}{\angles{x-y}^{2N_1}} \frac{\angles{\xi}^{m_2 + 2\delta N_2}}{\angles{\xi-\eta}^{2N_2}} \diff y \diff \eta \\
		& \lesssim \int_{\R^n} \angles{\eta}^{m_1-2\rho N_1 + 2N_2} \angles{\xi}^{m_2-2(1-\delta) N_2 } \diff \eta \\
		& \lesssim \angles{\xi}^{m_2-2(1-\delta) N_2 }, \\
	\end{align*}
	donde se escoge $N_1$ tal que $-2\rho N_1 + 2N_2 + m_1 < -n$. Por lo que se puede escoger $N_2$ libremente para obtener la cota deseada. Un argumento análogo funciona para las derivadas de $c_2$, por lo que este pertenece a $S^{-\infty}(\R^n\times\R^n)$.
\end{proof}
\begin{definition}[Potencial Bessel]
	Se define al potencial de Bessel de orden $m\in\R$ al operador pseudo-diferencial con símbolo $\angles{\xi}^m$. Este se denota por $J^m$.
\end{definition}
\begin{remark}[Espacios de Sobolev revisitados]
	Para $s\in\R$ se define el espacio de Sobolev $W^s_p(\Omega)$ como el conjunto de funciones integrables $f$ sobre $\Omega$, tales que 
	\begin{equation*}
		\|f\|_{W^s_p} := \|J^s f\|_{L^p} < \infty.
	\end{equation*}
	Se tiene que es
\end{remark}
\section{Definición y propiedades básicas en $\T^n$}
TODO TODO

\chapter{Continuidad de operadores pseudo-diferenciales}

\section{Continuidad en espacios de Lebesgue}
TODO TODO

\section{Continuidad en espacios de Sobolev}
TODO TODO
