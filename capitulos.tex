\chapter{Preliminares}

En este capítulo se revisarán aspectos básicos del análisis armónico en 
$\R^n$ y $\T^n$. 
Se recuerda que $\R^n$ es un grupo aditivo respecto a la suma usual de vectores
con subgrupo aditivo $\Z^n$. Entonces, se define al toro $n$-dimensional
como el grupo cociente $\T^n := \R^n/\Z^n = (\R/\Z)^n$. Además, 
el toro puede ser identificado con el conjunto $[0, 1)^n$ y se le puede considerar
con la topología cociente. A lo largo de este trabajo, se fijará la
medida de Lebesgue en $\R^n$. Para cualquier punto 
$x := (x_1, \ldots, x_n) \in \R^n$, se denotará la norma
euclideana como 
\begin{equation*}
    |x| := \sqrt{x_1^2 + \cdots + x_n^2}.
\end{equation*}
Sin embargo, podría ser problemático considerar potencias negativas
de la norma euclideana, debido a que se desvanece en cero. Por lo que
se considerará una función que se comporta asintóticamente similar, 
pero no presenta el mismo problema
\begin{equation*}
    \angles{x} := \sqrt{1 + |x|^2}.
\end{equation*}
Si se tiene que existe una constante $C>0$ tal que $A\leq CB$, se dice que $A\lesssim B$. Si además, $C$ depende de algún parámetro $\alpha$, se denota $A\lesssim_\alpha B$.
\section{Espacios de Lebesgue en $\R^n$ y $\T^n$ }

Sea $\Omega$ un subconjunto medible de $\R^n$. Por simplicidad, se supondrá
que $\Omega$ es abierto o cerrado.

\begin{definition}[Espacios de Bochner-Lebesgue]
    Sea $1\leq p < \infty$, y $X$ un espacio de Banach. Se dice que una función (fuertemente) medible $f:\Omega\subset\R^n\rightarrow X$ se encuentra en $L^p(\Omega;X)$ si su norma 
    \begin{equation*}
        \|f\|_{L^p(\Omega;X)} := \left( \int_\Omega \|f(x)\|_X^p \diff x
        \right)^{1/p}
    \end{equation*}
    es finita. Para el caso $p=\infty$, se dice que $f\in L^\infty(\Omega;X)$
    si es esencialmente acotada. Es decir, si
    \begin{equation*}
        \|f\|_{L^\infty(\Omega;X)} := \esssup_{x\in\Omega}\|f(x)\|_X < \infty,
    \end{equation*}
    donde $\esssup_{x\in\Omega}|f(x)|$ se define como el menor número real $M$
    tal que es mayor que $\|f(x)\|_X$ casi para todo $x\in\Omega$, i.e. excepto fuera de
    un conjunto de medida cero.
\end{definition}
Cabe destacar que en realidad los elementos de los espacios $L^p(\Omega;X)$
son clases de equivalencias de funciones iguales casi en todo $x\in\Omega$.
Sin embargo, es un detalle técnico menor y se acostumbra a tratarles como funciones. Particularmente, es de interés cuando $X=\C$, entonces se denotará $L^p(\Omega)$. Además, cuando $\Omega$ y $X$ sean claros por el contexto, simplemente se denotará
$\|\cdot\|_{L^p(\Omega;X)}$ como $\|\cdot\|_{L^p}$. Ahora, se discutirán propiedades 
importantes de los espacios de Lebesgue.
\begin{proposition}[Desigualdad de Young]
    Sean $1< p, q< \infty$, tales que $\frac{1}{p} + \frac{1}{q} = 1$.
    Entonces para todos $a, b > 0$, se tiene que 
    \begin{equation*}
        ab \leq \frac{a^p}{p} + \frac{b^q}{q}.
    \end{equation*}
    Como consecuencia, para $g\in L^q(\Omega;X)$ y $f\in L^p(\Omega;\mathcal{B}(X, Y))$, donde $\mathcal{B}(X, Y)$ es el espacio de operadores acotados $X\to Y$, 
    se tiene que $fg \in L^1(\Omega;Y)$ y 
    \begin{equation*}
        \|fg\|_{L^1} \leq \frac{1}{p}\|f\|^p_{L^p} + \frac{1}{q}
        \|g\|_{L^q}^q.
    \end{equation*}
\end{proposition}
\begin{proof}
    Esto es consecuencia del hecho que $x\mapsto e^x$ es una función
    convexa. Entonces
    \begin{equation*}
        ab = e^{\ln a + \ln b} = e^{\frac{1}{p}\ln a^p + 
        \frac{1}{q}\ln b^q} \leq \frac{1}{p}e^{\ln a^p} +
        \frac{1}{q}e^{\ln b^q} = \frac{a^p}{p} + \frac{b^q}{q}.
    \end{equation*}
    Completando así la prueba.
\end{proof}
Particularmente cuando $p=2=q$, se tiene la conocida como desigualdad de
Cauchy. 
\begin{proposition}[Desigualdad de H\"older]
    Sean $1\leq p, q\leq \infty$, tales que $\frac{1}{p} + \frac{1}{q} = 1$.
    Entonces, para $g\in L^q(\Omega;X)$ y $f\in L^p(\Omega;\mathcal{B}(X, Y))$, se tiene
    que $fg \in L^1(\Omega;Y)$ y 
    \begin{equation*}
        \|fg\|_{L^1} \leq \|f\|_{L^p}\|g\|_
        {L^q}.
    \end{equation*}
\end{proposition}
\begin{proof}
    Para el caso $p=1$, o $p=\infty$, el resultado es trivial. Así que 
    se considerará el caso $1<p<\infty$, que es una aplicación 
    de la proposición anterior. Primero, se supone
    que $\|f\|_{L^p} = \|g\|_{L^q} = 1$. Entonces, 
    se tiene que
    \begin{equation*}
        \|fg\|_{L^1} \leq \frac{1}{p}\|f\|_{L^p} + \frac{1}{q}\|g\|_{L^q} 
        = 1.
    \end{equation*}
    Ahora, se nota que si $\|f\|_{L^p}$ o $\|g\|_{L^q}$ se anulan, 
    entonces se trivializa la desigualdad. 
    Por lo que se puede considerar el caso más general en el que ninguna de las
    normas se anula de la siguiente manera
    \begin{equation*}
        \left\| \frac{f}{\|f\|_{L^p}} \frac{g}{\|g\|_{L^q}}
        \right\|_{L^1} \leq 1.
    \end{equation*}
    El resultado sigue de la linealidad de la norma $L^1$.
\end{proof}
En el caso $p=2=q$ se obtiene la desigualdad de Cauchy-Schwarz.

\begin{proposition}[Desigualdad de Minkowski]
    Dado $1\leq p\leq\infty$, sean $f,g\in L^p(\Omega;X)$. Entonces se tiene que
    \begin{equation*}
        \|f+g\|_{L^p} \leq \|f\|_{L^p} +
        \|g\|_{L^p}.
    \end{equation*}
    Particularmente, $\|\cdot\|_{L^p}$ satisface la desigualdad triangular
    y $L^p(\Omega;X)$ es un espacio normado.
\end{proposition}
\begin{proof}
    Para $p=1$, o $p=\infty$ el resultado se obtiene gracias a la desigualdad
    triangular de la norma en $X$. Ahora, para $1<p<\infty$ 
    se tiene que
    \begin{align*}
        \|f+g\|_{L^p}^p & \leq \int_\Omega \|f+g\|_X^{p-1}(\|f\|_X+\|g\|_X)\diff x\\
        & = \int_\Omega \|f+g\|_X^{p-1}\|f\|_X \diff x + 
        \int_\Omega \|f+g\|_X^{p-1}\|g\|_X \diff x \\
        & \leq \left( 
            \int_\Omega \|f+g\|_X^{(p-1) \frac{p}{p-1}} \diff x 
        \right)^{\frac{p-1}{p}} \left[
            \left(\int_\Omega \|f\|_X^p\diff x\right)^{1/p} +
            \left(\int_\Omega \|g\|_X^p\diff x\right)^{1/p}
        \right] \\ 
        & = \|f+g\|_{L^p}^{p-1}(\|f\|_{L^p} + \|g\|_{L^p}).
    \end{align*}
    Aquí, la primera desigualdad es la desigualdad triangular de la norma de $X$, y la segunda es la desigualdad de H\"older, por lo que se concluye
    lo deseado.
\end{proof}
\begin{theorem}[Desigualdad de Jensen]
	Sea $\mu$ una medida unitaria en $\Omega$, y sea $\varphi:I\subset\R\to\R$ una función convexa, donde $I$ es un intervalo. Entonces para funciones integrables  $f:\Omega\to I$, se tiene que 
	\begin{equation*}
		\varphi \left(\int_\Omega f\diff\mu  \right) \leq \int_\Omega \varphi\circ f\diff\mu.
	\end{equation*}
\end{theorem}
\begin{proof}
	Note que la convexidad equivale a para cualquier $y_0\in\Sigma$ existe $\lambda$ tal que $\varphi(y) \geq \varphi(y_0) + \lambda(y-y_0)$. Entonces, tome en particular
	\begin{equation*}
		y_0 := \int_\Omega f\diff\mu.
	\end{equation*}
	Se tiene que $y_0\in I$ dado que la medida $\mu$ es unitaria. Por lo que 
	\begin{equation*}
		\int_\Omega \varphi\circ f\diff \mu \geq \varphi(y_0) + \lambda \int_\Omega (f-y_0)\diff \mu = \varphi(y_0).
	\end{equation*}
	Lo que completa la prueba.
\end{proof}
Ahora se introducen dos resultados importantes y de bastante utilidad. Sin embargo,
sus demostraciones requieren de herramientas de teoría de la medida o del 
análisis complejo que se encuentran fuera del alcance de este trabajo. Por lo que
simplemente se enuncian y se recomienda al lector investigar los detalles.

\begin{proposition}[Monotonía de la norma $L^p$]\label{prop:monotonia-Lp}
    Sea $f:\Omega_1 \times \Omega_2 \subset\R^n\times\R^n\rightarrow X$ y 
    sea $1\leq p\leq\infty$.
    Se supone que $f(\cdot,y)\in L^p(\Omega_1;X)$ para casi
    todo $y$, y que $y\mapsto \|f(\cdot, y)\|_{L^p}$ se encuentra en
    $L^1(\Omega_2;X)$. Entonces $f(x, \cdot)\in L^1(\Omega_2;X)$ para casi
    todo $x$, la función $x\mapsto \int_{\Omega_2} f(x, y)\diff y$ se encuentra
    en $L^p(\Omega_1)$, y
    \begin{equation*}
        \left\| \int_{\Omega_2} f(\cdot, y)\diff y
        \right\|_{L^p(\Omega_1)} \leq \int_{\Omega_2} 
        \|f(\cdot, y)\|_{L^p(\Omega_1)} \diff y.
    \end{equation*}
\end{proposition}
A continuación se presenta un resultado clásico de la interpolación de 
operadores y espacios de funciones. Para una discusión más profunda de estas técnicas, se recomienda revisar Bergh y L\"ofstrom \cite{bergh-lofstrom}.
\begin{theorem}[Interpolación de Riesz-Thorin]\label{theo:riesz-thorin} 
    Sea $T:L^{p_0}(\Omega;X)+L^{p_1}(\Omega;X) \rightarrow 
    L^{q_0}(\Omega;Y)+L^{q_1}(\Omega;Y)$ un operador lineal tal que 
    \begin{equation*}
        \|Tf\|_{L^{q_0}} \leq M_0 \|f\|_{L^{p_0}}, \quad
        \|Tf\|_{L^{q_1}} \leq M_1 \|f\|_{L^{p_1}}.
    \end{equation*}
    Para cualquier $0<\theta<1$, se definen 
    \begin{equation*}
        \frac{1}{p_\theta} = \frac{1-\theta}{p_0} + \frac{\theta}{p_1}, 
        \quad 
        \frac{1}{q_\theta} = \frac{1-\theta}{q_0} + \frac{\theta}{q_1}.
    \end{equation*}
    Entonces, $T$ extiende a un operador continuo de $L^{p_\theta}(\Omega)$
    en $L^{q_\theta}(\Omega)$. Además, 
    \begin{equation*}
        \|Tf\|_{L^{q_\theta}} \leq M_0^{1-\theta}M_1^\theta \|f\|_{L^{p_\theta}}.
    \end{equation*}
\end{theorem}
Se continúa con el programa de definiciones y propiedades en los espacios 
de Lebesgue.

\begin{definition}[Convoluciones]
    Para funciones $g\in L^1(\Omega;X)$, y $f\in L^1(\Omega; \mathcal{B}(X, Y))$, se define su \textit{convolución} como 	
    \begin{equation*}
        (f*g)(x) := \int_{\Omega} f(x-y)g(y)\diff y.
    \end{equation*}
    Se puede notar que el cambio de variable $y\mapsto x-u$ implica la
    conmutatividad cuando $\Omega$ es invariante bajo traslaciones ($\R^n$ o $\T^n$ por ejemplo), es decir $f*g=g*f$.
\end{definition}
\begin{remark}
    En la definición anterior existe la pregunta sobre la convergencia de la
    integral. Para definir la convolución de forma rigurosa, se podría definir 
    primero para funciones que cumplan condiciones de regularidad más 
    fuertes, como las del espacio de Schwartz que se definirá en la siguiente
    sección, para luego definir el operador $*:L^1\times L^1 \rightarrow L^1$
    que estaría bien definido gracias a la siguiente propiedad.
\end{remark}
\begin{proposition}[Desigualdad de Young para convoluciones]
    Sean $1\leq p,q,r\leq \infty$ tales que $\frac{1}{p} + \frac{1}{q} = 1 + 
    \frac{1}{r}$, y sean $g\in L^q(\Omega;X)$, y $f\in L^p(\Omega; \mathcal{B}(X, Y))$. Entonces se tiene que
    \begin{equation*}
        \|f*g\|_{L^r} \leq \|f\|_{L^p} \|g\|_{L^q}.
    \end{equation*}
\end{proposition}
\begin{proof}
    Se nota que gracias al Teorema~\ref{theo:riesz-thorin} es suficiente 
    demostrar 
    \begin{equation}\label{eq:young-convolution-estimates}
        \|f*g\|_{L^p} \leq \|f\|_{L^p} \|g\|_{L^1}, \quad 
        \|f*g\|_{L^\infty} \leq \|f\|_{L^p} \|g\|_{L^t},
    \end{equation}
    para $\frac{1}{t} + \frac{1}{p} = 1$. En efecto, bastaría con considerar 
    el operador $f * \cdot$, los parametros
    \begin{equation*}
        p_0 = 1, \quad p_1 = t, \quad q_0 = p, \quad q_1 = \infty.
    \end{equation*}
    y $\|f\|_{L^p}$ como ambas constantes de estimación.
    Al aplicar la interpolación
    \begin{equation*}
        \frac{1}{r} = \frac{1-\theta}{p} + \frac{\theta}{\infty}, \quad 
        \frac{1}{q} = \frac{1-\theta}{1} + \frac{\theta}{t},
    \end{equation*}
    se obtiene la condición indicada para los parametros $p, q, r$. Ahora, 
    se procede a demostrar el primer estimativo de
    (\ref{eq:young-convolution-estimates}). Este se obtiene como resultado de
    la monotonia de la norma $L^p$, véase la Proposición \ref{prop:monotonia-Lp}.
    En efecto,
    \begin{align*}
        \|f*g\|_{L^p} & = \left\| \int_{\Omega} f(\cdot-y)g(y)\diff y 
        \right\|_{L^p} \\
        & \leq \int_{\Omega} \|f(\cdot-y)\|_{L^p} \|g(y)\|_X\diff y \\
        & \leq \|f\|_{L^p}\|g\|_{L^1}.
    \end{align*}
    Por otra parte, el segundo estimativo es resultado de la desigualdad de 
    H\"older 
    \begin{align*}
        \|f*g\|_{L^\infty} & \leq \int_{\Omega} \|f(x-y)\|_\mathcal{B}\|g(y)\|_X \diff y \\
        & \leq \|f\|_{L^p}\|g\|_{L^t}.
    \end{align*}
    Concluyendo con el resultado deseado.
\end{proof}
Las convoluciones son casos especiales de operadores con kernel valuado en operadores.
\begin{definition}[Operador con kernel valuado en operadores]
	Decimos que un operador $T:C^\infty(\Omega;X)\rightarrow C^\infty(\Omega;Y)$ tiene un \textit{kernel valuado en operadores} si se puede escribir como
	\begin{equation*}
		Tf(x) = \int_{\Omega}k(x, y)f(y)\diff y,
	\end{equation*}
	donde $k:\Omega\times\Omega \rightarrow \mathcal{B}(X,Y)$, denominado el kernel, es tal que $\|k(x, \cdot)\|_{\mathcal{B}(X,Y)}$ es integrable lejos de $x \in \Omega $.
	\label{def:operator-kernel}
\end{definition}
Se continua con un resultado importante sobre convergencia en 
espacios $L^p$
\begin{theorem}[Convergencia dominanda de Lebesgue]
    Sea $(f_k)_{k=1}^\infty$ una secuencia de funciones medibles en $\Omega$ 
    tales que convergen puntualmente a $f$ para casi todo $x\in\Omega$. Se supone
    que existe $g\in L^1(\Omega;X)$ tal que $|f_k|\leq g$ para todo $k$. Entonces
    $f$ es integrable y 
    \begin{equation*}
        \int_\Omega f \diff x = \lim_{k\rightarrow\infty} \int_\Omega 
        f_k \diff x.
    \end{equation*}
\end{theorem}
\begin{remark}
    Una implicación del resultado anterior es el hecho que los espacios $L^p$ son 
    completos, y por consecuencia son espacios de Banach. Particularmente, si $\mathcal{H}$ es un espacio de Hilbert, entonces el 
    espacio $L^2(\Omega,\mathcal{H})$ es un espacio de Hilbert con producto interno dado por
    \begin{equation*}
        (f, g)_{L^2} := \int_\Omega (f(x), g(x))_{\mathcal{H}}\diff x.
    \end{equation*}
\end{remark}
\begin{theorem}
	Las funciones continuas son densas en el espacio $L^p(\Omega;X)$.
\end{theorem}
\begin{proof}
	Basta con demostrarlo para una función simple $\chi_E$, donde $E$ es un conjunto medible. Por definición de medida de Lebesgue 
	$|E| = \inf\{\sum |I_k| : E \subset \bigcup I_k\}$, donde los $I_k$ son cubos abiertos. Entonces, siembre existe un abierto $U$, tal que $|U\setminus E|<\varepsilon$. Ahora, se toma una sucesión de funciones suaves $\phi_n$, tal que $\phi_n(0)=1$, que $\phi_n(t)=0$, para $t>1/n$. Entonces, se define $\Phi_n(x) := \phi_n(d(x, U))$, donde se toma la distancia al abierto $U$. Por lo que 
	\begin{equation*}
		\|\Phi_n - \chi_E\|_{L^p} \leq \|\chi_{U_n} - \chi_E\|_{L^p} = \|\chi_{U_n\setminus U} - \chi_{U\setminus E}\|_{L^p} \leq \|\chi_{U_n\setminus U} \|_{L^p} +\varepsilon^{1/p},
	\end{equation*}
	donde $U_n$ es el soporte de $\Phi_n$, que se puede ver decrece a $U$. La prueba se completa por convergencia dominada.
\end{proof}
Se presenta la versión local de los espacios de Lebesgue. Para ello se necesita el siguiente espacio de funciones.
\begin{definition}[Funciones suaves de soporte compacto]
		Se dice que $\varphi:\Omega\subset\R^n\rightarrow X$ es suave si es de clase $C^\infty$, o infinitamente diferenciable.  Se define su \textit{soporte} como 
		\begin{equation*}
			\supp \varphi = \overline{\{x\in\Omega:\varphi(x)\neq 0\}}.
		\end{equation*}
		Si $\supp \varphi$ es compacto, se dice que $f\in C^\infty_0(\Omega;X)$.
\end{definition}
\begin{definition}[Localización de espacios de Lebesgue]
	Se dice que una función medible $f:\Omega\subset\R^n\rightarrow X$ es localmente integrable o pertenece a $L^{p}_{loc}(\Omega; X)$, con $1\leq p\leq\infty$ si 
	\begin{equation*}
		\|f\varphi\|_{L^p} < \infty,
	\end{equation*} 
	para todo $\varphi\in C^\infty_0(\Omega)$.
\end{definition}
Ahora, se introducen los espacios de Lebesgue pesados.
\begin{definition}[Espacios de Lebesgue pesados]
	Sea $w:\Omega\subset\R^n\to\C$ una función localmente integrable no-negativa. Entonces, se puede definir la medida
	\begin{equation*}
		w(E) = \int_E w(x)\diff x.
	\end{equation*}
	Por lo que se dice que una función (fuertemente) medible $f:\Omega\to X$, pertenece al \textit{espacio de Lebesgue pesado} $L^p(\Omega;X;w)$ cuando 
	\begin{equation*}
		\|f\|_{L^p(w)} :=\left( \int_\Omega \|f(x)\|_X^p \diff w(x)\right)^{1/p} < \infty,
	\end{equation*}
	para $1\leq p<\infty$. Cuando $p=\infty$, cuando es acotada excepto en un conjunto de $w$-medida cero.
\end{definition}
\begin{proposition}[Definición alternativa de la norma $L^p$]
	Para $f\in L^p(\Omega;X;w)$, se tiene que para $0< p<\infty$,
	\begin{equation*}
		p\int_0^\infty t^{p-1} a_f(t) \diff t = \int_\Omega \|f(x)\|_X^p \diff w(x).
	\end{equation*}
	con $a_f(t) = w\{x\in\Omega: \|f(x)\|_X>t\}$.
\end{proposition}
\begin{proof}
	Basta con notar que para cualquier función diferenciable $\phi(t)$ con $\phi(0)=0$, se tiene que
	\begin{align*}
		\int_\Omega \phi(\|f(x)\|_X) \diff w(x) & = \int_\Omega \int_0^{\|f(x)\|_X} \phi'(t) \diff t \diff w(x) \\
		& = \int_0^\infty \phi'(t) \int_{\|f(x)\|_X>t}  \diff w(x) \diff t \\
		& = \int_0^\infty \phi'(t) w\{x\in\Omega: \|f(x)\|_X>t\} \diff t.
	\end{align*}
	Ahora, el resultado se obtiene cuando $\phi(t) = t^p$.
\end{proof}
A continuación, se presenta un resultado de interpolación bastante útil. 
\begin{definition}
	Sea $T$ un operador desde $L^p(\Omega;X;w)$ al espacio de funciones medibles desde $\Sigma$ hacia $Y$. Entonces, se dice que $T$ es de \textit{tipo $(p,q)$ débil} respecto a los pesos $(u,w)$, con $q<\infty$, si se tiene que 
	\begin{equation*}
		u\{x\in\Sigma:\|Tf(x)\|_Y>\lambda\}\lesssim \left(\frac{\|f\|_{L^p(w)}}{\lambda}\right)^q.
	\end{equation*}
	Además, se dice que es de tipo débil $(p, q=\infty)$ o tipo fuerte $(p, q)$, si es acotado desde $L^p(\Omega;X;w)$ hacia $L^q(\Sigma;Y;u)$.
\end{definition}
\begin{theorem}\label{theo:weak-closed}
	Sea $\{T_t\}$ una familia de operadores en $L^p(\Omega;X;w)$, y se define su operador maximal asociado 
	\begin{equation*}
		T_*f(x) = \sup_t \|T_tf(x)\|_X.
	\end{equation*}
	Si $T_*$ es de tipo $(p,q)$ débil respecto a $(w, w)$, entonces se tiene que los conjuntos
	\begin{equation*}
		\left\{f\in L^p(w): \lim_{t\to t_0} T_tf(x) = f(x)\, \text{ casi en todas partes} \right\},
	\end{equation*}
	\begin{equation*}
		\left\{f\in L^p(w): \lim_{t\to t_0} T_tf(x) = 0\, \text{ casi en todas partes} \right\},
	\end{equation*}
	son cerrados.
\end{theorem}
\begin{proof}
	Sea $(f_n)$ una suseción de funciones que converge a $f\in L^p(w)$. Entonces, se tiene que 
	\begin{align*}
		w\{\limsup_{t\to t_0} \|T_tf(x) - f(x)\|_X > \lambda \}  &\leq w\{\limsup_{t\to t_0} \|T_t(f-f_n)(x) - (f-f_n)(x)\|_X >\lambda \}\\
		& \leq w\{T_*(f-f_n)(x)>\lambda/2 \} +  w\{\|(f-f_n)(x)\|_X >\lambda/2 \}\\
		& \leq \left(\frac{C2}{\lambda}\|f-f_n\|_{L^p(w)}\right)^q + \left(\frac{2}{\lambda}\|f-f_n\|_{L^p(w)}\right)^q \to 0,
	\end{align*}
	cuando $n\to\infty$. Lo que completa la prueba del primer conjunto, para el segundo, se utiliza un argumento similar con $\{\limsup_{t\to t_0}\|T_tf(x)\|_X>\lambda\}$.
\end{proof}
\begin{theorem}[Interpolación de Marcinkiewicz]
	Sean $1\leq p_0 < p_1 \leq \infty$, y $1\leq q_0 < q_1 \leq \infty$, tales que $p_j<q_j$. Y sea $T$ un operador sublineal desde $L^{p_0}(\Omega;X; w)+L^{p_1}(\Omega;X; w)$ hacia las funciones medibles de $\Sigma$ en $Y$, es decir 
	\begin{align*}
		\|T(f+g)(x)\|_Y & \leq \|Tf(x)\|_Y + \|Tg(x)\|_Y,\\
		\|T(\lambda f)(x)\|_Y &= |\lambda|\|Tf(x)\|_Y.
	\end{align*}
	Además, suponga que $T$ es de tipo débil $(p_0, q_0)$ y $(p_1, q_1)$, respecto a las medidas $(u, w)$. Entonces, se tiene que $T$ es de tipo fuerte $(p, q)$ respecto a $(u, w)$ para $p_0<p<p_1$, $q_0<q<q_1$, y $p\leq q$, con la forma 
	\begin{equation*}
		\frac{1}{p} = \frac{1-\theta}{p_0} + \frac{\theta}{p_1}, \quad \frac{1}{q} = \frac{1-\theta}{q_0} + \frac{\theta}{q_1}.
	\end{equation*}
\end{theorem}
\begin{proof}
	Sea $f\in L^p$, y sea $\chi$ la función indicadora para puntos que satisfacen $\|f\|_X>\sigma\lambda^\gamma$, donde $\sigma$ es una constante que se escoge luego, y \begin{equation*}
		\gamma = \frac{(q_1-q_0)p_0p_1}{(p_1-p_2)q_0q_1}.
	\end{equation*}
	
	Entonces, se descompone
	\begin{equation*}
		f = f\chi + f(1-\chi) =: f_0 + f_1,
	\end{equation*}
	donde se puede ver que $f_j \in L^{p_j}$. Además, 
	\begin{equation*}
		\|Tf(x)\|_Y  \leq \|Tf_0(x)\|_Y +\|Tf_1(x)\|_Y, 
	\end{equation*}
	\begin{equation*}
		a_{Tf}(\lambda) \leq a_{Tf_0}(\lambda/2) + a_{Tf_1}(\lambda/2).
	\end{equation*}
	(Caso: $p_1<\infty$)  Entonces se tienen las desigualdades 
	\begin{equation*}
		a_{Tf_j}(\lambda/2) \leq \left(\frac{2A_j}{\lambda}\|f\|_{L^{p_j}(w)}\right)^{q_j}.
	\end{equation*}
	Además, se define 
	Entonces, se tiene que 
	\begin{align*}
		\|Tf\|_{L^q(u)}^q & = q\int_0^\infty\lambda^{q-1}a_{Tf}(\lambda) \diff \lambda \\
		& \leq \sum_j q\int_0^\infty\lambda^{q-1}a_{Tf_j}(\lambda/2) \diff \lambda \\
		& \leq \sum_j q\int_0^\infty\lambda^{q-1-q_j} (2A_j)^{q_j}\left( \int_\Omega \|f_j(x)\|_X^{p_j}\diff u(x) \right)^{q_j/p_j}\diff \lambda\\ 
		& \leq \sum_j q\int_0^\infty\lambda^{q-1-q_j} (2A_j)^{q_j}\int_\Omega \|f_j(x)\|_X^{q_j}\diff u(x) \diff \lambda\\
		& \leq q\int_0^\infty\lambda^{q-1-q_0} (2A_0)^{q_0}\int_{\|f\|_X>\sigma\lambda^\gamma} \|f(x)\|_X^{q_0}\diff u(x) \diff \lambda\\
		& \quad + q\int_0^\infty\lambda^{q-1-q_1} (2A_1)^{q_1} \int_{\|f\|_X\leq\sigma\lambda^\gamma} \|f(x)\|_X^{q_1}\diff u(x) \diff \lambda\\
		& = q(2A_0)^{q_0} \int_\Omega \|f(x)\|_X^{q_0} \int_0^{(|f|/\sigma)^{1/\gamma}} \lambda^{q-1-q_0} \diff\lambda \diff u(x) \\
		& \quad + q(2A_1)^{q_1} \int_\Omega \|f(x)\|_X^{q_1} \int_{(\|f\|_X/\sigma)^{1/\gamma}}^{\infty} \lambda^{q-1-q_1} \diff\lambda \diff u(x) \\
		& = \frac{q(2A_0)^{q_0}\sigma^{q_0-p}}{q-q_0} \int_\Omega \|f(x)\|_X^p \diff u(x) + \frac{q(2A_1)^{q_1}\sigma^{q_1-p}}{q_1-q} \int_\Omega \|f(x)\|_X^p \diff u(x) \\
		& = \left(\frac{q(2A_0)^{q_0}\sigma^{q_0-p}}{q-q_0} +\frac{q(2A_1)^{q_1}\sigma^{q_1-p}}{q_1-q} \right) \|f\|_{L^p(w)}^p.
	\end{align*}
	Entonces, si se utiliza $\|f\|_{L^p(w)}^{-1} f$ en lugar de $f$, se obtiene por homogenidad de $T$ 
	\begin{equation*}
	\|f\|_{L^p(w)}^{-q}	\|Tf\|_{L^q(u)}^q \leq C_\sigma.
	\end{equation*}
	Ahora, se escoge $\sigma=\frac{1}{2}A_0^{\frac{q_0}{q_1-q_0}}A_1^{\frac{q_1}{q_0-q_1}}$, para obtener una cota que dependa solo de $A_0,A_0$ y $p,q$. Particularmente, cuando $p_j =q_j$
	\begin{equation}\label{eq:marcinkiewicz-final}
		\|Tf\|_{L^p(u)} \leq 2p^{1/p}\left(\frac{1}{p-p_0}+\frac{1}{p_1-p}\right)^{1/p}A_0^{1-\theta}A_1^\theta\|f\|_{L^p(w)}, \quad \text{ con }\quad \frac{1}{p} = \frac{1-\theta}{p_0} + \frac{\theta}{p_1}.
	\end{equation}
	(Caso $q_1=\infty$) Es similar al caso anterior, pero solo con una desigualdad débil. En efecto 
	\begin{equation*}
		\|Tf\|_{L^q(u)}^q \leq \frac{q(2A_0)^{q_0}\sigma^{q_0-p}}{q-q_0} \|f\|_{L^p(w)}^p,
	\end{equation*}
	particularmente, si $p_j=q_j$, se escoge $\sigma = (2A_1)^{-1}$, donde $\|Tg\|_{L^\infty}\leq A_1\|g\|_{L^\infty}$, se obtiene (\ref{eq:marcinkiewicz-final}). Lo que completa la prueba.
\end{proof}
\begin{remark}
	Note que la definición de continuidad débil y el resultado de interpolación de Marcinkiewicz vale para cualquier espacio de medida. No necesariamente para la medida de Lebesgue euclideana.
\end{remark}
Ahora, se presenta un resultado de descomposición que fue demostrado por Calderón y Zigmund \cite{calderon-zygmund}.
\begin{definition}[Cubos diádicos]
	En $\R^n$, se define a $\mathcal{Q}_k$ como la colección de cubos abiertos por la derecha, cuyos vértices son puntos adyacentes del retículo $(2^{-k}\Z)^n$. Entonces, se les llama \textit{cubos diádicos} a los elementos de $\bigcup_k \mathcal{Q}_k$.
\end{definition}
\begin{remark}
	Esta definición también puede aplicar para $\Omega\subset\R^n$, al tomar la colección $\mathcal{Q}_k' := \{Q\cap\Omega: Q\in\mathcal{Q}_k\}$. Además, se tiene que estos cubos cumplen con:
	\begin{enumerate}
		\item Dado $x\in\Omega$, existe un único $Q\in\mathcal{Q}_k$, tal que $x\in Q$, para cada $k$. 
		\item Para cualesquiera dos cubos diádicos, se tiene que son disjuntos o uno esta contenido en el otro. 
		\item Cada cubo en $\mathcal{Q}_k$, está contenido en exactamente un cubo de $\mathcal{Q}_j$, para $j<k$. Además, contiene exactamente $2^n$ cubos en $\mathcal{Q}_{k+1}$.
	\end{enumerate}
\end{remark}
\begin{definition}[Operador maximal diádico]
	Para $f\in L^1_{loc}(\Omega;X)$, se define el \textit{operador maximal diádico} como 
	\begin{equation*}
		M_df(x) := \sup_k \|E_kf(x)\|_X, \quad \text{ con } \quad E_kf(x) := \sum_{Q\in\mathcal{Q}_k} \frac{\chi_Q(x)}{|Q|} \int_Q f(y) \diff y.
	\end{equation*}
\end{definition}
\begin{theorem}
	El operador maximal diádico es de tipo débil $(1, 1)$. Además, se tiene $\lim_k E_k f(x) = f(x) $, para $f\in L^1_{loc}(\Omega, X)$.
\end{theorem}
\begin{proof}
	Sin pérdida de generalidad, suponga que $f$ es no-negativo. Entonces se descompone 
	\begin{equation*}
		\{x\in\Omega: M_df(x)>\lambda\} = \bigcup_k \Omega_k,
	\end{equation*}
	donde $x\in\Omega_k$, si $k = \min\{j: E_jf(x) > \lambda\}$. Este $k$ existe porque $E_kf(x)\to0$ cuando $k\to-\infty$, para $f\in L^1$. Estos $\Omega_k$ son disjuntos y por construcción pueden escribirse como union de cubos en $\mathcal{Q}_k$. Entonces, se tiene que 
	\begin{align*}
		| \{x\in\Omega: M_df(x)>\lambda\}  | & = \sum_k |\Omega_k| \\
		& \leq \sum_k \frac{1}{\lambda} \int_{\Omega_k} E_k f(x)\diff x \\
		& = \frac{1}{\lambda} \sum_k \sum_{\substack{Q\in\mathcal{Q}_k\\ Q\subset\Omega_k}} \int_Q \frac{\chi_Q(x)}{|Q|}\int_Q f(y) \diff y \diff x \\
		& = \frac{1}{\lambda} \sum_k \sum_{\substack{Q\in\mathcal{Q}_k\\ Q\subset\Omega_k}} \int_Q f(y) \diff y\\
		& = \frac{1}{\lambda} \sum_k \int_{\Omega_k} f(y)\diff y
		\leq \frac{1}{\lambda} \|f\|_{L^1}.
	\end{align*}
	Ahora, para la segunda parte, es claro que es válido para una función continua. Entonces, en vista del Teorema \ref{theo:weak-closed}, se puede verificar para cualquier $f\in L^1_{loc}$, por un argumento de densidad.
\end{proof}
\begin{theorem}[Calderón-Zygmund]\label{theo:Calderon-Zygmund}
	Para cualquier $f$ integrable y no-negativa, existe una secuencia $\{Q_j\}$ de cubos diádicos disjuntos, tales que 
	\begin{enumerate}
		\item $f(x)\geq \lambda$ casi en todas partes para $x\notin \bigcup_j Q_j$,
		\item $\left|\bigcup_jQ_j\right| \leq \frac{1}{\lambda}\|f\|_{L^1}$,
		\item $\lambda < \frac{1}{Q_j}\int_{Q_j}f \leq 2^n\lambda$.
	\end{enumerate}
\end{theorem}
\begin{proof}
	La prueba del segundo inciso se encuentra en la demostración anterior. El primer inciso es consecuencia del hecho que si $x\notin\bigcup_jQ_j$, entonces $E_kf(x) \leq \lambda$ para todo $k$, y solo basta con tomar el límite en vista del resultado anterior. Para el tercer inciso, se ve que  por construcción de los $Q_j$, se tiene que el promedio de $f$ sobre ellos es mayor que $\lambda$, y ningún cubo que lo contenga también lo cumple. Ahora, sea $\tilde{Q}_j$ el cubo diádico más pequeño que contiene estrictamente a $Q_j$. Entonces, se tiene que 
	\begin{equation*}
		\frac{1}{|Q_j|}\int_{Q_j} f \leq \frac{|\tilde{Q}_j|}{|Q_j|}\frac{1}{|Q_j|} \int_{\tilde{Q}_j} f \leq 2^n\lambda.
	\end{equation*}
	Lo que concluye la prueba.
\end{proof}
A continuación se define un operador maximal bastante importante. El operador maximal de Hardy-Littlewood.
\begin{definition}[Operador maximal de Hardy-Littlewood]
	Para $f\in L^p_{loc}(\Omega;X)$, se define al \textit{operador $p$-maximal de Hardy-Littlewood} como 
	\begin{equation*}
		\mathrm{M}_pf(x) := \sup_{Q\ni x} \left(\frac{1}{|Q|}\int_Q \|f(y)\|_X^p\diff y\right)^{1/p},
	\end{equation*}
	donde $Q$ son cubos con lados paralelos a los ejes. Cuando $p=1$, se considera el operador de Hardy-Littlewood y se denota $\mathrm{M}f$.
\end{definition}
Ahora, se presenta un resultado que evidencia la utilidad de este operador maximal
\begin{theorem}\label{theo:Hardy-Littlewood}
	El operador maximal de Hardy-Littlewood es continuo desde $L^p(\Omega;X;\mathrm{M}w)$ hacia $L^p(\Omega;\C;w)$, para $1<p\leq\infty$.
\end{theorem}
\begin{proof}
	Note que si $\mathrm{M}w(x) = 0$ para algún $x$, entonces se tiene que $w(x)=0$ casi en todas partes, lo que trivializa el resultado. Entonces, sea $t > \|f\|_{L^\infty(\mathrm{M}w)}$, entonces 
	\begin{equation*}
		\int_{\|f\|_X > t} \mathrm{M}w(x) \diff x = 0,
	\end{equation*}
	y como $\mathrm{M}w(x) > 0$, se tiene que $|\{\|f\|_X > \lambda\}| = 0$. Esto implica que $\mathrm{M}f(x) < \lambda$ casi en todas partes. Llevando $\lambda$ al límite se puede concluir que $\|\mathrm{M}f\|_{L^\infty(w)} \leq \|f\|_{L^\infty(\mathrm{M}w)}$. Entonces solo queda demostrar la desigualdad débil de tipo $(1,1)$, para poder aplicar el Teorema de interpolación de Marcinkiewicz. Para esto, se aplica la descomposición de Calderón-Zygmund al operador maximal diádico
	\begin{equation*}
		\{x\in\Omega:M_df(x)>\lambda\} = \bigcup_jQ_j.
	\end{equation*}
	Para cada $Q_j$, se toma $3Q_j$, un cubo concéntrico tal que $\ell(3Q_j) = 3\ell(Q_j)$. Además, se fija $x \notin \bigcup_j 3Q_j$ y un cubo cualquiera $Q$ que lo contenga. Se escoge $k\in\Z$, tal que $2^{k-1}\leq\ell(Q)<2^k$, entonces existen $m\leq 2^n$ cubos diádicos en $\mathcal{Q}_k$ que se intersectan con $Q$, que se denotan $R_j$. Note que ninguno está contenido en los $Q_j$, de lo contrario $x\in \bigcup_j 3Q_j$. Entonces, se tiene que 
	\begin{align*}
		\frac{1}{|Q|} \int_Q f \diff x &= \frac{1}{|Q|}\sum_{j=1}^m \int_{Q\cap R_j} f \diff x \\
		& \leq \sum_{j=1}^m \frac{2^{kn}}{|Q|}\frac{1}{|R_j|}\int_{R_j} f\diff x \\
		& \leq 2^nm\lambda \leq 4^n\lambda.
	\end{align*}
	En conclusión, se tiene que 
	\begin{equation*}
		\{x\in\Omega:\mathrm{M}f(x)>4^n\lambda\} \subset \bigcup_j3Q_j.
	\end{equation*}
	Por lo que, aprovechando la tercera propiedad de la descomposición,
	\begin{align*}
		w\{\mathrm{M}f>\lambda\} & \leq \sum_j \int_{3Q_j} \diff w(x)\\
		& \leq \sum_j \frac{4^n}{\lambda |Q_j|} \int_{Q_j} \|f(y)\|_X \diff y\frac{3^n|Q_j|}{|3Q_j|}
		\int_{3Q_j} \diff w(x) \\
		& \leq \frac{3^n4^n}{\lambda} \sum_j \int_{Q_j} \|f(y)\|_X \mathrm{M}w(y) \diff y\\
		& \leq \frac{3^n4^n}{\lambda}  \|f\|_{L^1(\mathrm{M}w)}.
	\end{align*}
	Lo que concluye la prueba
\end{proof}
En vista de la continuidad débil $(1, 1)$, y el hecho que vale para funciones continuas, se obtiene que
\begin{corollary}[Teorema de diferenciación de Lebesgue]
	Para $f\in L^1_{loc}(\Omega;X)$, se tiene que 
	\begin{equation*}
		\lim_{r\to0}\frac{1}{|B(x, r)|} \int_{B(x, r)} f(y) \diff y = f(x),
	\end{equation*}
	casi para todo $x\in\Omega$.  En particular, $\|f\|_X\leq \mathrm{M}f$ casi en todas partes, y
	\begin{equation*}
		\lim_{r\to0}\frac{1}{|B(x, r)|} \int_{B(x, r)} \|f(y)-f(x)\|_X \diff y = 0,
	\end{equation*}
	casi en todas partes. A los puntos que lo satisfacen se les llama puntos de Lebesgue.
\end{corollary}
Ahora, se presenta una caracterización de pesos para los cuales el operador maximal de Hardy-Littlewood es continuo.
\begin{definition}[Clases de pesos de Muckenhoupt]
	Para un par de  funciones localmente integrables no-negativas $u,w:\Omega\subset\R^n\to\C$, se dice que pertenece a la \textit{clase de pesos de Muckenhoupt} $A_p$, si
	\begin{equation*}
		\mathrm{M}u(x) \lesssim w(x), \quad \text{casi para todo }x,\quad p=1;
	\end{equation*}
	\begin{equation*}
		\sup_Q \left(\frac{1}{|Q|}\int_Q u(x)\diff x \right)\left(\frac{1}{|Q|}\int_Q w(x)^{-1/(p-1)}\diff x\right)^{p-1} < \infty, \quad 1<p<\infty.
	\end{equation*}
\end{definition}
\begin{theorem}
	Se tiene que el operador maximal de Hardy-Littlewood es de tipo débil $(p,p)$, respecto a las medidas $(u, w)$ si y solo si $(u, w) \in A_p$.
\end{theorem}
\begin{proof}
	($\Rightarrow$) Primero, suponga que $f$ es no-negativa, note que si $f_Q := \frac{1}{|Q|} \int_Q f\diff x$, entonces $f_Q\leq \mathrm{M}(f\chi_Q)(x)$ para cualquier $x\in Q$. Por lo que para cualquier $\lambda < f_Q$, se tiene que $Q\subset E_{\lambda, Q} := \{x\in\Omega: \mathrm{M}(f\chi_Q)(x)>\lambda\}$, y 
	\begin{equation*}
		u(Q) \leq u(E_{\lambda,Q}) \lesssim \lambda^{-p} \int_Q |f(x)|^p\diff w(x),
	\end{equation*}
	\begin{equation*}
		(f_Q)^pu(Q) \lesssim \int_Q |f(x)|^p\diff w(x).
	\end{equation*}
	Particularmente, para cualquier conjunto medible $S\subset Q$, se puede cambiar $f$ por $f\chi_S$ y obtener 
	\begin{equation}\label{eq:key-muckenhoupt}
		\left(\frac{1}{|Q|}\int_S |f(x)|\diff x \right)^p u(Q) \lesssim \int_S |f(x)|^p\diff w(x).
	\end{equation}
	Ahora, se puede ignorar el caso trivial $u(x) = 0$, casi en todas partes, y se fija $f=\chi_S$, para obtener 
	\begin{equation*}
		|S|^p u(Q) \lesssim|Q|^p w(S),
	\end{equation*}
	y concluir que $w(x)>0$ casi en todas partes. Cuando $p=1$, fije $t > \essinf_Q w$ ,y para $S_t := \{x\in\Omega: w(x)<t\}$, se tiene que $|S_t|>0$, que $u(Q)/|Q| \lesssim t$, y que 
	\begin{equation*}
		\frac{1}{|Q|}\int_Q u(x)\diff x \lesssim \essinf_Q w \leq w(x),
	\end{equation*}
	para casi todo $x\in Q$. Es fácil ver que esto implica la condición $A_1$. Cuando $1<p<\infty$, se considera la función $f(x) := w(x)^{-1/(p-1)}$, de tal manera que $f(x) = f(x)^pw(x)$, y con $S_k:=\{x\in Q: w(x) > 1/k\}$ en (\ref{eq:key-muckenhoupt}), se  tiene que 
	\begin{equation*}
		\left(\frac{1}{|Q|}\int_{S_k} w(x)^{-1/(p-1)}\diff x \right)^p u(Q) \lesssim \int_{S_k} w(x)^{-1/(p-1)}\diff x.
	\end{equation*}
	Además, como $f$ está acotado en $S_k$, se puede manipular para obtener
	\begin{equation*}
		\left(\frac{1}{|Q|}\int_{S_k} w(x)^{-1/(p-1)}\diff x \right)^{p-1} \left(\frac{1}{|Q|}\int_Q\diff u(x)\right) \leq C.
	\end{equation*}
	Como $w(x)>0$ casi en todas partes, se tiene que $Q\setminus\bigcup_kS_k$ tiene medida cero y se puede concluir la condición $A_p$ al hacer $k\to\infty$
	\begin{equation*}
			\left(\frac{1}{|Q|}\int_{Q} w(x)^{-1/(p-1)}\diff x \right)^{p-1} \left(\frac{1}{|Q|}\int_Q\diff u(x)\right) \leq C.
	\end{equation*}
	($\Leftarrow$) Para $p=1$, por la continuidad débil del operador de Hardy-Littlewood, se tiene que 
	\begin{equation*}
		w\{\mathrm{M}f>\lambda\} \lesssim \lambda^{-1} \int_\Omega |f(x)| \mathrm{M}w(x) \diff x \lesssim \lambda^{-1} \int_\Omega |f(x)| w(x) \diff x = \|f\|_{L^1(w)}.
	\end{equation*}
	Cuando $1<p<\infty$, se utiliza la desualdad de H\"older para tener que 
	\begin{equation*}
		f_Q = \frac{1}{|Q|} \int_Q |f(x)|w(x)^{1/p}w(x)^{-1/p} \diff x 
	\end{equation*}
	\begin{equation*}
		\leq \left(\frac{1}{|Q|}\int_Q  |f(x)|^pw(x)\diff x \right)^{1/p} \left(\frac{1}{|Q|}\int_Q w(x)^{-1/(p-1)}\diff x \right)^{(p-1)/p}.
	\end{equation*}
	Entonces, se tiene que 
	\begin{align*}
		(f_Q)^pu(Q) & \leq \frac{u(Q)}{|Q|}  \left(\int_Q  |f(x)|^pw(x)\diff x \right) \left(\frac{1}{|Q|}\int_Q w(x)^{-1/(p-1)}\diff x \right)^{p-1}\\ 
		&  \lesssim \int_Q  |f(x)|^pw(x)\diff x .
	\end{align*}
	Entonces, se obtiene (\ref{eq:key-muckenhoupt}), que implica la desigualdad débil.
\end{proof}
\begin{corollary}\label{cor:muckenhoupt}
	Sea $(u, w)\in A_p$, entonces el operador $r$-maximal de Hardy Littlewood es continuo desde $L^q(\Omega; X;w)$ hacia $L^q(\Omega;\C;u)$, para $p<q/r\leq\infty$.
\end{corollary}
\begin{proof}
	Primero, considere el caso $r=1$. En vista del Teorema de Interpolacieón de Marcinkiewicz, solo queda demostrar la continuidad para $q=\infty$. Primero, como $u(E)>0$ implica $|E|>0$, se tiene que 
	\begin{equation*}
		\|\mathrm{M}f\|_{L^\infty(u)} \leq \|\mathrm{M}f\|_{L^\infty}.
	\end{equation*}
	Por otra parte, como $w(x)>0$ casi en todas partes, se tiene que $|E|>0$, implica $w(E)>0$ y que 
	\begin{equation*}
		\|f\|_{L^\infty} \leq \|f\|_{L^\infty(w)}.
	\end{equation*}
	Como además, $\|\mathrm{M}f\|_{L^\infty} \leq \|f\|_{L^\infty}$, se concluye la prueba para este caso. Ahora, como $\mathrm{M}_rf=\mathrm{M}(\|f\|_X^r)^{1/r}$, se tiene que
	\begin{equation*}
		\int_\Omega (\mathrm{M}_rf)^q \diff u(x) = \int_\Omega \mathrm{M}(\|f\|_X^r)^{q/r} \diff u(x) \lesssim \int_\Omega \|f\|_X^q \diff w(x),
	\end{equation*}
\end{proof}
\begin{remark}
	Note que este caso incluye el caso particular para las medidas $(w,\mathrm{M}w)$.
\end{remark}
Además, (\ref{eq:key-muckenhoupt}) brinda otro corolario acerca de las medidas de conjuntos
\begin{corollary}
	Sean $w\in A_p$, y sea $0<\alpha<1$. Sea un conjunto medible $S\subset Q$, donde $Q$ es un cubo, tal que $|S|\leq \alpha|Q|$, entonces existe $0<\beta<1$, tal que $w(S)\leq \beta w(Q)$.
\end{corollary}
\begin{proof}
	Considere $\chi_{Q\setminus S}$ en lugar de $S$, entonces
	\begin{equation*}
		(|Q|-|S|)^p w(Q) \leq C |Q|^p [w(Q) - w(S)],
	\end{equation*}
	\begin{equation*}
		w(Q)(1-\alpha)^p \leq C [w(Q) - w(S)],
	\end{equation*}
	\begin{equation*}
		w(S) \leq \frac{C-(1-\alpha)^p}{C}w(Q), 
	\end{equation*}
	lo que concluye la prueba.
\end{proof}
\begin{theorem}[Desigualdad inversa de Hölder]
	Sea $w\in A_p$, entonces existe $\varepsilon>0$, tal que se tiene que para cualquier cubo 
	\begin{equation*}
		\left(\frac{1}{|Q|}\int_Q w^{1+\varepsilon}\diff x \right)^{1/(1+\varepsilon)} \lesssim \frac{1}{|Q|} \int_Q w\diff x.
	\end{equation*}
\end{theorem}
\begin{proof}
	Tome un cubo $Q$, y un $\alpha$ fijo, tal que $0<\alpha<1$,  defina  entonces la sucesión  $\lambda_k := (2^n\alpha^{-1})^kw(Q)/|Q|$, de tal manera que $2^n\lambda_k/\lambda_{k+1} = \alpha$. Entonces tome la descomposición de Calderón-Zygmund de $w$ en el cubo $Q$ para todo $\lambda_k$, para así obtener sucesiones de cubos diádicos $\{Q_j^k\}$ disjuntos tales que $w(x) >\lambda_k$ para casi todo $x\in \Omega_k := \bigcup_jQ_j^k$, y
	\begin{equation*}
		\lambda_k < \frac{1}{|Q_j^k|}\int_{Q_j^k} w\diff x \leq 2^n\lambda_k.
	\end{equation*}
	Ahora, fije un cubo $Q_j^k$, entonces $Q_j^k \cap \Omega_{k+1}$ es la union de cubos $Q_i^{k+1}$ y se tiene que 
	\begin{align*}
		|Q_j^k \cap \Omega_{k+1}| & = \sum_i |Q_i^{k+1}| \\
		& \leq \frac{1}{\lambda_{k+1}} \sum_i \int_{Q_i^{k+1}} w\diff x\\
		& \leq \frac{1}{\lambda_{k+1}}  \int_{Q_j^k} w\diff x \\ 
		& \leq \frac{2^n\lambda_k}{\lambda_{k+1}}|Q_j^k| = \alpha |Q_j^k|.
	\end{align*}
	Entonces, como $w\in A_p$, existe $0<\beta<1$, tal que $w(Q_j^k \cap \Omega_{k+1}) \leq \beta w(Q_j^k)$. Al sumar respecto a todos loss cubos $Q_j^k$ se obtiene que 
	\begin{equation*}
		w(\Omega_{k+1}) \leq \beta w(\Omega_k), \quad \text{ y que } w(\Omega_k) \leq \beta^k w(\Omega_0).
	\end{equation*}
	Entonces, se tiene que 
	\begin{align*}
		\frac{1}{|Q|}\int_Q w^{1+\varepsilon} \diff x & = \frac{1}{|Q|}\int_{Q\setminus \Omega_0} w^{1+\varepsilon} \diff x + \frac{1}{|Q|} \sum_k \int_{\Omega_k\setminus \Omega_{k+1}} w^{1+\varepsilon} \diff x \\
		& \leq \lambda_0^\varepsilon \frac{w(Q)}{|Q|} + \frac{1}{|Q|}\sum_k \lambda_{k+1}^\varepsilon w(\Omega_k)  \\
		& \leq \lambda_0^\varepsilon \frac{w(Q)}{|Q|}  \frac{1}{|Q|}\sum_k (2^n\alpha^{-1})^{(k+1)\varepsilon}\lambda_0^\varepsilon  \beta^kw(\Omega_0).
	\end{align*}
	Entonces basta escoger $\varepsilon$, para que $(2^n\alpha^{-1})^\varepsilon\beta<1$, para poder estimar mediante 
	\begin{equation*}
		\lambda_0^\varepsilon \frac{w(Q)}{|Q|} + \frac{C}{|Q|} \lambda_0^\varepsilon w(\Omega_0) \lesssim \lambda_0^\varepsilon \frac{w(Q)}{|Q|}  = \left(\frac{w(Q)}{|Q|}\right)^{1+\varepsilon}.
	\end{equation*}
	Lo que completa la prueba.
\end{proof}
Una consecuencia inmediata es que 
\begin{corollary}\label{cor:inverse-Holder}
	Se tiene que 
	\begin{enumerate}
		\item Si $w\in A_p$, entonces $w^{1-p'}\in A_{p'}$,
		\item $A_p = \bigcup_{q<p}A_q$, para $1<p<\infty$,
		\item sea $w\in A_p$, entonces, existe $\varepsilon>0$, tal que $w^{1+\varepsilon}\in A_p$.
	\end{enumerate}
\end{corollary}
\begin{proof}
	Para el primer inciso, se tiene que 
	\begin{equation*}
		\sup_Q\left(\frac{1}{|Q|}\int_Q w^{1-p'}\diff x \right)\left(\frac{1}{|Q|}\int_Q w(x)^{-(1-p')/(p'-1)}\diff x\right)^{p'-1} < \infty,
	\end{equation*}
	es la condición $A_p$ elevada a $p'-1$, como $1-p' = -1/(p-1)$, y $-(1-p')/(p'-1) = 1$. Para el segundo inciso, se utiliza la desigualdad inversa de Hölder con $w^{1-p'}$, para tener que existe algún $\varepsilon>0$ tal que
	\begin{equation*}
		\left(\frac{1}{|Q|}\int_Q w^{(1-p')(1+\varepsilon)}\diff x\right)^{1/(1+\varepsilon)} \lesssim \frac{1}{|Q|}\int_Q w^{1-p'}\diff x.
	\end{equation*}
	Entonces, tome $q$, tal que $1-q' = (1-p')(1+\varepsilon)$. De esta manera, $q<p$, y
	\begin{equation*}
		\left(\frac{1}{|Q|}\int_Q w \diff x \right)\left(\frac{1}{|Q|}\int_Q w(x)^{1-q'}\diff x\right)^{q-1}  \lesssim \left(\frac{1}{|Q|}\int_Q w \diff x \right)\left(\frac{1}{|Q|}\int_Q w(x)^{1-p'}\diff x\right)^{p-1},
	\end{equation*}
	como $(q-1)(1+\varepsilon) = p-1$. Entonces, se cumple que $w\in A_q$. Para el tercer inciso. Si $p=1$, entonces se tiene que para cada cubo $Q$ y para casi todo $x\in Q$, se cumple que 
	\begin{equation*}
		\frac{1}{|Q|}\int_Q w^{1+\varepsilon}\diff y \lesssim w(x)^{1+\varepsilon}.
	\end{equation*}
	Cuando $p>1$, tome $\varepsilon$ tal que la desigualdad inversa de Hölder funciona para $w$ y  $w^{1-p'}$.  Entonces la condición $A_p$ para $w^{1+\varepsilon}$ se vuelve la condición $A_p$ de $w$ elevada a la $1/(1+\varepsilon)$. Lo que concluye la prueba. 
\end{proof}
Ahora, se presenta el concepto de dualidad entre espacios de Lebesgue valuados en espacios de Banach
\begin{definition}[Propiedad de Radon-Nikodym]
	Se dice que un espacio de Banach $X$ satisface la \textit{propiedad de Radon-Nikodym} si para cualquier medida valuada en vectores, se puede encontrar una función (fuertemente) medible que sea su densidad.
\end{definition}
\begin{theorem}
	Suponga $X'$ satisface la propiedad de Radon-Nikodym, entonces $L^p(\Omega; X)' = L^{p'}(\Omega;X')$.
\end{theorem}
\begin{proof}
	Es claro que si $f\in L^{p'}(\Omega;X')$, entonces 
	\begin{equation*}
		\int_\Omega |\angles{f, g}|\diff x \leq \|f\|_{L^{p'}}\|g\|_{L^p},
	\end{equation*}
	para cualquier $g\in L^p(\Omega;X)$. Ahora, sea un funcional continuo $T:L^p(\Omega; X)\to\C$. Entonces, se define la medida valuada en $X'$, como $\angles{\nu(E), x} := T(\chi_E\cdot x)$. Por la propiedad de Radon-Nikodym, se puede encontrar $f:\Omega\to X'$ fuertemente medible tal que 
	\begin{equation*}
		\int_\Omega \angles{f, \chi_E\cdot x} \diff y = \left\langle\int_E f\diff y, x\right\rangle = \angles{\nu(E), x} = T(\chi_E \cdot x),
	\end{equation*}
	para todo $x\in X$ y $E$ medible. Como esto funciona para toda las funciones simples, entonces vale para todas las funciones fuertemente medibles. Además, como $T$ es continuo, se tiene que $\|f\|_{L^{p'}} = \|T\|$.
\end{proof}
Para concluir, se incluye una desigualdad bastante útil.
\begin{theorem}[Hardy-Littlewood-Sobolev]
	Sea $0<\alpha<n$, y sean $1<p<q<\infty$ tales que 
	\begin{equation*}
		\frac{1}{q} = \frac{1}{p} - \frac{\alpha}{n}.
	\end{equation*}
	Además, si se define el potencial de Riesz como 
	\begin{equation*}
		I_\alpha f(x) := \int_{\Omega} \frac{f(x)\diff x}{|x-y|^{n-\alpha}},
	\end{equation*}
	entonces $\|I_\alpha f\|_{L^q} \lesssim \|f\|_{L^p}$.
\end{theorem}
\begin{proof}
	Note que 
	\begin{equation*}
		\int_{\R^n} \frac{f(y)\diff y}{|x-y|^{n-\alpha}} = \int_0^\infty \frac{r^{n-1}}{r^{n-\alpha}}  \int_{S^{n-1}} f(x+rz) \diff \sigma(z) \diff r = \int_0^\infty r^{\alpha-1} f_x(r)\diff r,
	\end{equation*}
	donde $f_xr = \int_{S^{n-1}} f(x+rz) \diff \sigma(z) $. Entonces, $\|f_x(r)\|_X \leq \mathrm{M}f(x)$, y por la desigualdad de Hölder
	\begin{equation*}
		\|f_x(r)\|_X \lesssim r^{-n}\int_{B(x, r)}| f(y)|\diff y \lesssim r^{-n}\|f\|_{L^p} r^{n(p-1)/p}  = r^{-n/p}.
	\end{equation*}
	Por lo que, para algún $r_x$,
	\begin{equation*}
		\|I_\alpha f(x) \|_X\leq \int_0^{r_x} r^{\alpha-1} \mathrm{M}f(x)\diff r + \int_{r_x}^\infty r^{\alpha-1}r^{-n/p} \|f\|_{L^p}\diff r \lesssim \mathrm{M}f(x)r_x^{\alpha} + \|f\|_{L^p}r_x^{\alpha-n/p}.
	\end{equation*}
	Al tomar $r_x := [\mathrm{M}f(x)]^{-p/n}\|f\|_{L^p}^{p/n}$, se tiene que 
	\begin{equation*}
		\|I_\alpha f(x) \|_X\lesssim [\mathrm{M}f(x)]^{1 -\alpha p/n} \|f\|_{L^p}^{\alpha p/n}.
	\end{equation*}
	Además, por la continuidad del operador maximal de Hardy-Littlewood
	\begin{equation*}
		\|I_\alpha f(x) \|_{L^q}^q \lesssim \|f\|_{L^p}^{\alpha pq/n} \int_{\R^n}\mathrm{M}f(x)^{(1 -\alpha p/n)q}\diff x \lesssim  \|f\|_{L^p}^{\alpha pq/n} \|f\|_{L^{(1 -\alpha p/n)q}}^{(1 -\alpha p/n)q}.
	\end{equation*}
	Entonces, el resultado se obtiene cuando $(1 -\alpha p/n)q = p$.
\end{proof}
\section{Transformada de Fourier en $\R^n$}
Ahora, se procede a definir y mostrar propiedades importantes de la
transformada de Fourier, una herramienta fundamental para el estudio
de las ecuaciones diferenciales en general y los operadores
pseudo-diferenciales en particular. 

\begin{definition}[Transformada de Fourier en $\R^n$]
    Dada $f\in L^1(\R^n;X)$, se define su \textit{transformada de Fourier }como 
    \begin{equation*}
        (\mathcal{F}_{\R^n}f)(\xi) = \widehat{f}(\xi) := \int_{\R^n}
        e^{-2\pi i x \cdot \xi} f(x) \diff x,
    \end{equation*}
    para cualquier $\xi \in \R^n$.
\end{definition}

\begin{proposition}
    La transformada de Fourier es un operador continuo 
    $\mathcal{F}_{\R^n} : L^1(\R^n;X) \rightarrow L^\infty(\R^n;X)$ con norma uno:
    \begin{equation*}
        \|\widehat{f}\|_{L^\infty} \leq \|f\|_{L^1}.
    \end{equation*}
    Además, $\widehat{f}$ es continua en todas partes. 
\end{proposition}
\begin{proof}
    El estimativo es resultado de la desigualdad de Minkowski para integrales
    clásica 
    \begin{align*}
        \left\|
            \int_{\R^n} e^{-2\pi i x \cdot \xi} f(x) \diff x
        \right\|_X & \leq 
        \int_{\R^n} \left|e^{-2\pi i x \cdot \xi}\right|\| f(x)\|_X \diff x\\
        & \leq \|f\|_{L^1}.
    \end{align*}
    Ahora, la continuidad es consecuencia del teorema de convergencia dominada
    de Lebesgue. Para cualquier $\xi_k \rightarrow \xi$ se define 
    \begin{equation*}
        h_k(x) :=  e^{-2\pi i x \cdot \xi_k} f(x)
    \end{equation*}
    Entonces, se tiene que $|h_k| \leq |f|$ y se obtiene que 
    \begin{equation*}
        \int_{\R^n} e^{-2\pi i x \cdot \xi} f(x) \diff x = 
        \lim_{k\rightarrow\infty}\int_{\R^n} e^{-2\pi i x \cdot \xi_k} f(x) \diff x.
    \end{equation*} 
    Que es exactamente $\widehat{f}(\xi) = \lim \widehat{f}(\xi_k)$, el 
    resultado deseado.
\end{proof}
A pesar de que la transformada de Fourier está bien definida en el espacio
$L^1(\R^n;X)$, este presenta ciertas limitaciones técnicas debido a los pocos
requerimientos de regularidad para las funciones en este espacio. Es muy útil
tener acceso a otras herramientas resultantes de continuidad, diferenciabilidad,
y decaimiento.
Por lo tanto, se introduce notación que será importante a lo largo de este trabajo. 
\begin{definition}[Notación de multi-índice]
    Para $\alpha:=(\alpha_1, \ldots, \alpha_n), \beta:=(\beta_1, \ldots, \beta_n) 
    \in \N_0^n$, se define 
    \begin{equation*}
        \partial^\alpha := \frac{\partial^{\alpha_1}}{\partial x_1^{\alpha_1}}
        \cdots \frac{\partial^{\alpha_n}}{\partial x_n^{\alpha_n}}.
    \end{equation*}
    De forma similar, $x^\beta := x_1^{\beta_1} \cdots x_n^{\beta_n}$. Se dice 
    que $\alpha \leq \beta$ si $\alpha_i \leq \beta_i$ para todo $i$. Además, 
    se denota la longitud del multi-índice como
    $|\alpha| := \alpha_1 + \cdots \alpha_n$ y su factorial como
    $\alpha! := \alpha_1! \cdots \alpha_n!$.
\end{definition}

\begin{definition}[Espacio de Schwartz $\mathcal{S}(\R^n)$]
    Se dice que una función suave (infinitamente diferenciable) 
    $\varphi:\R^n\rightarrow X$ \textit{decae rápidamente}, y se encuentra en $\mathcal{S}(\R^n;X)$ si se 
    cumple que
    \begin{equation*}
        \sup_{x\in\R^n}\|x^\beta \partial^\alpha \varphi(x)\|_X < \infty, 
    \end{equation*}
    para cualesquiera multi-índices $\alpha, \beta \in \N_0^n$. Ahora,
    se dice que $\varphi_j\rightarrow\varphi$ en $\mathcal{S}(\R^n;X)$ si
    \begin{equation*}
        \sup_{x\in\R^n}\|x^\beta \partial^\alpha (\varphi_j-
        \varphi)(x)\|_X \rightarrow 0, 
    \end{equation*}
    cuando $j\rightarrow\infty$ para cualesquiera multi-índices 
    $\alpha, \beta \in \N_0^n$.
\end{definition}
\begin{proposition}
    Para cualquier $1\leq p\leq\infty$ se tiene que 
    $\mathcal{S}(\R^n;X)\subset L^p(\R^n;X)$ con encaje continuo.
\end{proposition}
\begin{proof}
    El caso $p=\infty$ es trivial, pues las funciones en el espacio de Schwartz
    son acotadas por definición.
    Sea $\varphi_j\rightarrow0$ en $\mathcal{S}(\R^n;X)$, entonces
    \begin{align*}
        \int_{\R^n} \|\varphi_j(x)\|_X^p \diff x & = 
        \int_{\R^n} \angles{x}^{pN} \|\varphi_j(x)\|_X^p \angles{x}^{-pN} \diff x \\ 
        & \lesssim \max_{|\beta|\leq N} \sup_{x\in\R^n} 
        \|x^\beta\varphi_j(x)\|_X^p \int_{\R^n}\angles{x}^{-pN} \diff x \to 0  ,
    \end{align*}
    donde $N\in\N$ se escoge de manera que la última integral converga. 
\end{proof}
Aún más, es densa en estos espacios.
\begin{theorem}
	El espacio $\mathcal{S}(\R^n;X)$ es secuencialmente denso en $L^p(\R^n;X)$, para $1\leq p\leq\infty$.
\end{theorem}
\begin{proof}
	Primero, supónga $1\leq p<\infty$. Tome una función $\psi\in C^\infty_0(\R^n)$, tal que es igual a uno en una vecindad del origen, no-negativa, y tal que $\int\psi = 1$. Entonces tome $\psi_k(x) := k^{-n}\psi(kx)$, y defina $f_k := \psi_k * f$. Por lo que 
	\begin{equation*}
		\|f_k - f\|_{L^p} \leq \int_{\R^n} \|f(\cdot - y) - f(\cdot)\|_{L^p}k^{-n}\psi(ky) \diff y =  \int_{\R^n} \|f(\cdot - y/k) - f(\cdot)\|_{L^p}\psi(y)\diff y \to 0,
	\end{equation*}
	por el Teorema de diferenciación de Lebesgue. Ahora, si $p=\infty$, se tiene que la clase de Schwartz es secuencialmente densa en el espacio de funciones continuas, por un argumento similar al anterior, que a su vez son densas en $L^\infty$.
\end{proof}
\begin{theorem}
    Sea $\varphi\in\mathcal{S}(\R^n;X)$. Entonces $2\pi i\xi_j\widehat{\varphi}(\xi)
    = \widehat{\partial_j\varphi}(\xi)$ y 
    $ 2\pi i \widehat{x_j\varphi}(\xi) = - \partial_j\widehat{\varphi}(\xi)$
\end{theorem}
\begin{proof}
    Para la primera expresión se procede por integración por partes
    \begin{align*}
        \widehat{\partial_j\varphi}(\xi) & = \int_{\R^n} e^{-2\pi i x\cdot \xi} 
        \partial_{x_j}\varphi(x) \diff x \\
        & = - \int_{\R^n} (\partial_{\xi_j} e^{-2\pi i x\cdot \xi}) 
        \varphi(x) \diff x\\
        & = 2\pi i \xi_j \int_{\R^n} e^{-2\pi i x\cdot \xi} 
        \varphi(x) \diff x.
    \end{align*}
    Se nota que no aparece el término con la frontera debido a que $\varphi$ 
    se desvanece en el infinito. Ahora, para la segunda expresión 
    \begin{equation*}
        \partial_{\xi_j} \widehat{\varphi}(\xi) = 
        \int_{\R^n}  e^{-2\pi i x\cdot \xi} (-2\pi i x_j)
        \varphi(x) \diff x.
    \end{equation*}
    Concluyendo la prueba.
\end{proof}
Por lo que se puede concluir lo siguiente 
\begin{corollary}
    Sea $\varphi\in\mathcal{S}(\R^n;X)$. Entonces, 
    \begin{equation*}
        \xi^\beta \partial^\alpha \widehat{\varphi}(\xi) = 
        (2\pi i)^{|\alpha|-|\beta|} (-1)^{|\alpha|} \widehat{\partial^\beta 
        [x^\alpha \varphi]}(\xi).
    \end{equation*}
    Por lo que 
    \begin{align*}
        \|\xi^\beta \partial^\alpha \widehat{\varphi}(\xi)\|_X & \leq 
        |2\pi i|^{|\alpha|-|\beta|} \int_{\R^n} \|\partial^\beta 
        [x^\alpha \varphi(x)]\|_X \diff x \\
        & \leq |2\pi i|^{|\alpha|-|\beta|} \sup_{x\in\R^n} 
        \left\|(1+|x|)^{n+1} \partial^\beta 
        [x^\alpha \varphi(x)]\right\|_X \int_{\R^n} (1+|x|)^{-n-1}\diff x \\
        & = C \sup_{x\in\R^n} \left\|(1+|x|)^{n+1} \partial^\beta 
        [x^\alpha \varphi(x)]\right\|_X \\
        & < \infty.
    \end{align*}
    Particularmente, $\mathcal{F}_{\R^n}$ mapea $\mathcal{S}(\R^n)$ en sí 
    mismo. Además, por el teorema de convergencia dominada de Lebesgue, 
    la transformada de Fourier es un operador continuo.
\end{corollary}
En realidad, es un isomorfismo en $\mathcal{S}(\R^n;X)$. Para ello se demostrarán 
algunos lemas útiles.

\begin{lemma}[Fórmula de multiplicación para la transformada de Fourier]
    Sean $f\in L^1(\R^n;\mathcal{B}(X, Y))$, y $g \in L^1(\R^n;X)$. Entonces, $\int_{\R^n}\widehat{f} g \diff x 
    = \int_{\R^n} f \widehat{g} \diff x$.
\end{lemma}
\begin{proof}
    Aplicando el teorema de Fubini 
    \begin{align*}
        \int_{\R^n}\widehat{f} g \diff x & = \int_{\R^n} \left[
            \int_{\R^n} e^{-2\pi i x\cdot y} f(y) \diff y
        \right] g(x) \diff x \\
        & = \int_{\R^n} f(y)\left[
            \int_{\R^n} e^{-2\pi i x\cdot y} g(x) \diff x
        \right] \diff y \\ 
        & = \int_{\R^n} f \widehat{g} \diff y.
    \end{align*}
    Concluyendo la prueba.
\end{proof}
\begin{lemma}[Transformada de Fourier para Gaussiana]
    Se tiene que
    \begin{equation*}
       \int_{\R^n} e^{-2\pi i x \cdot \xi} e^{-\varepsilon\pi^2|x|^2}
        \diff x = (\pi\varepsilon)^{-n/2} e^{-|\xi|^2/\varepsilon},  
    \end{equation*}
    para todo $\varepsilon > 0$. Gracias al cambio de variable 
    $x \mapsto 2\pi x$ y $\varepsilon \mapsto 2\varepsilon$, esto equivale a
    \begin{equation*}
       \int_{\R^n} e^{- i x \cdot \xi} e^{-\varepsilon^2|x|^2/2}
        \diff x = (2\pi/\varepsilon)^{-n/2} e^{-|\xi|^2/(2\varepsilon)}. 
    \end{equation*}
\end{lemma}
\begin{proof}
    La segunda expresión sigue del caso unidimensional
    \begin{align*}
        \int_{-\infty}^{\infty} e^{-it\tau}e^{-t^2/2} \diff t & = 
        e^{-\tau^2/2} \int_{-\infty}^{\infty} e^{-(t+i\tau)^2/2} \diff t \\
        & = e^{-\tau^2/2} \int_{-\infty}^{\infty} e^{-t^2/2} \diff t \\
        & = \sqrt{2\pi} e^{-\tau^2/2}.
    \end{align*}
    Con el cambio de variable $t\mapsto\sqrt{\varepsilon}t$ y 
    $\tau\mapsto\tau/\sqrt{\tau}$ se tiene que 
    \begin{equation*}
        \sqrt{\varepsilon}\int_{-\infty}^{\infty} e^{-it\tau}e^{-\varepsilon t^2/2} 
        \diff t = 
        \sqrt{2\pi} e^{-\tau^2/(2\varepsilon)}.
    \end{equation*}
    El caso multidimensional sigue del producto de las integrales unidimensionales.
\end{proof}
\begin{theorem}[Fórmula de inversión de Fourier]
    La transformada de Fourier es un isomorfismo de $\mathcal{S}(\R^n;X)$ en si 
    mismo con inverso dado por 
    \begin{equation*}
        (\mathcal{F}^{-1}_{\R^n}f)(x) := \int_{\R^n} e^{2\pi i x \cdot \xi} f(\xi)
        \diff \xi.
    \end{equation*}
\end{theorem}
\begin{proof}
    El teorema de convergencia dominada de Lebesgue permite realizar la 
    sustitución
    \begin{align*}
        (\mathcal{F}^{-1}_{\R^n}\widehat{\varphi})(x) &= 
        \int_{\R^n}e^{2\pi ix \cdot \xi} \widehat{\varphi}(\xi)\diff \xi
        = \lim_{\varepsilon\rightarrow0} 
        \int_{\R^n}e^{2\pi ix \cdot \xi} \widehat{\varphi}(\xi)
        e^{-2\varepsilon\pi^2|\xi|^2} \diff \xi \\
        &= \lim_{\varepsilon\rightarrow0} \int_{\R^n}
        \int_{\R^n}e^{2\pi i(x-y) \cdot \xi} \varphi(y)
        e^{-2\varepsilon\pi^2|\xi|^2} \diff y \diff \xi.
    \end{align*}
    Con el cambio de variable $y \mapsto y+x$ se obtiene que
    \begin{equation*}
        (\mathcal{F}^{-1}_{\R^n}\widehat{\varphi})(x) = 
        \lim_{\varepsilon\rightarrow0} \int_{\R^n}
        \int_{\R^n}e^{-2\pi iy \cdot \xi} \varphi(y+x)
        e^{-2\varepsilon\pi^2|\xi|^2} \diff y \diff \xi.
    \end{equation*}
    Por el teorema de Fubini y la transformada de Fourier para 
    Gaussianas se tiene que
    \begin{align*}
        (\mathcal{F}^{-1}_{\R^n}\widehat{\varphi})(x) &= 
        \lim_{\varepsilon\rightarrow0} \int_{\R^n} \varphi(y+x) 
        \int_{\R^n}e^{-2\pi iy \cdot \xi} 
        e^{-2\varepsilon\pi^2|\xi|^2}  \diff \xi \diff y \\
        & = \lim_{\varepsilon\rightarrow0} \int_{\R^n} \varphi(y+x) 
        (2\pi\varepsilon)^{-n/2} e^{-|y|^2/(2\varepsilon)} \diff y
    \end{align*}
    Con un último cambio de variable $y\mapsto\sqrt{\varepsilon}z$
    se concluye 
    \begin{align*}
        (\mathcal{F}^{-1}_{\R^n}\widehat{\varphi})(x) &= 
        \lim_{\varepsilon\rightarrow0} \int_{\R^n} \varphi(\sqrt{\varepsilon}z+x) 
        (2\pi)^{-n/2} e^{-|z|^2/2} \diff z \\
        & = (2\pi)^{-n/2} \varphi(x) \int_{\R^n}e^{-|z|^2/2} \diff z = \varphi(x).
    \end{align*}
    Finalizando con la prueba.
\end{proof}
El siguiente teorema relaciona la transformada de Fourier con las 
convoluciones 
\begin{theorem}
    Sean $\psi\in\mathcal{S}(\R^n;X)$, y $\varphi\in\mathcal{S}(\R^n;\mathcal{B}(X, Y))$, entonces se cumple que 
    $\widehat{\varphi*\psi}(\xi) = \widehat{\varphi}(\xi) 
    \widehat{\psi}(\xi)$, que 
    $\widehat{\varphi\psi}(\xi) = (\widehat{\varphi} * 
    \widehat{\psi})(\xi)$. En particular, si $\psi,\varphi\in\mathcal{S}(\R^n;\mathcal{H})$, entonces  $\int (\varphi,\psi)_\mathcal{H} = \int (\widehat{\varphi},{\widehat{\psi}})_\mathcal{H}$.
    \label{theo:operations-fourier}
\end{theorem}
\begin{proof}
    Para la primera expresión se tiene que 
    \begin{align*}
        \widehat{\varphi*\psi}(\xi) &= 
        \int_{\R^n} e^{-2\pi i x\cdot\xi} (\varphi*\psi)(x)\diff x \\
        & = \int_{\R^n} \int_{\R^n} e^{-2\pi i (x-y)\cdot\xi} 
        \varphi(x-y)e^{-2\pi i y\cdot\xi} \psi(y) \diff y \diff x \\
        & = \int_{\R^n} \int_{\R^n} e^{-2\pi i z\cdot\xi} 
        \varphi(z)e^{-2\pi i y\cdot\xi} \psi(y) \diff y \diff z \\
        & = \widehat{\varphi}(\xi)\widehat{\psi}(\xi).
    \end{align*}
    Ahora, para la segunda expresión 
    \begin{align*}
        (\widehat{\varphi} * \widehat{\psi})(\xi) & = 
        \int_{\R^n} \widehat{\varphi}(\xi - y) \widehat{\psi}(y) \diff y \\
        & = \int_{\R^n} \int_{\R^n} e^{-2\pi i x\cdot(\xi - y)} \varphi(x)
        \widehat{\psi}(y) \diff x \diff y \\ 
        & = \int_{\R^n}e^{-2\pi i x\cdot\xi} \varphi(x)  \left[\int_{\R^n}  e^{2\pi i x\cdot y}\widehat{\psi}(y) 
        \diff y\right] \diff x \\
        &= \int_{\R^n} e^{-2\pi i x\cdot\xi} \varphi(x) \psi(x) \diff x = 
        \widehat{\varphi\psi}(\xi).
    \end{align*}
    Para la tercera expresión se define $\chi := (\cdot, \widehat{\psi})_\mathcal{H}$, entonces
    \begin{equation*}
    	\int_{\R^n} (\varphi, \psi)_\mathcal{H} = \int_{\R^n} \widehat{\chi}\varphi  = \int_{\R^n} \chi \widehat{\varphi} = \int_{\R^n} (\widehat{\varphi}, \widehat{\psi})_\mathcal{H}.
    \end{equation*}
    Completando la prueba.
\end{proof}
\begin{remark}
	Cuando $\varphi=\psi$, se tiene la indentidad de Plancherel $\int \|\varphi\|_\mathcal{H}^2 = \int \|\widehat{\varphi}\|_\mathcal{H}^2$. Esta puede extenderse para funciones en $L^2$ como se sabe que la clase de Schwartz es densa. Además,  se puede extender al caso general $X$.
\end{remark}
\begin{theorem}[Identidad de Plancherel]
	Para $f\in L^2(\R^n;X)$, se tiene que $\|f\|_{L^2} = \|\widehat{f}\|_{L^2}$.
\end{theorem}
\begin{proof}
	Tome una función simple $f=\sum_j^m x_j \chi_{E_j}$, entonces para cualquier $x^*\in X'$, se tiene que $\angles{x^*, f} \in L^2(\R^n)$. Por la identidad de Plancherel para espacios de Hilbert, se tiene que
	\begin{equation*}
		\int_{\R^n} |\angles{x^*, f}|^2= \int_{\R^n} \sum_j^m |\angles{x^*, x_j}|^2\chi_{E_j}  = \sum_j^m |\angles{x^*, x_j}|^2|E_j| = \sum_j^m  |\angles{x^*, x_j}|^2 \int_{\R^n} |\widehat{\chi_{E_j}}|^2  = \int_{\R^n} |\angles{x^*, \widehat{f}}|^2  ,
	\end{equation*}
	como  $\int |\widehat{\chi_{E_j}}|^2 = \int \chi_{E_j }= |E_j|$. Ahora, note que la imagen de $f$ está contenida en un espacio finito dimensional $Y$ generado por los $x_j$. Por lo que existen funcionales $x^*_1, \ldots,x^*_k$ tales que 
	\begin{equation*}
		\|x\|_X^2 = \sum_\ell^k |\angles{x^*_\ell, x}|^2.
	\end{equation*}
	La prueba se concluye al tomar la suma respecto a estos funcionales.
\end{proof}
\section{Transformada de Fourier en $\T^n$}
Se fija la notación del toro $n$-dimensional como $\T^n:=(\R/\Z)^n = \R^n/\Z^n$. Esta identificación se realiza tanto en el contexto de grupo aditivo cociente como en el de topología cociente. Se suele identificar a $\T^n$ con $[0, 1)^n$ como subconjunto de $\R^n$, y se fija su medida como la restricción de la medida euclideana. Se puede entender una función definida en el toro como una función 1-periodica. Es decir, si $g:\R^n\rightarrow\C$ cumple que $g(x)=g(x+k)$ para cualesquiera $x\in\R^n$ y $k\in\Z^n$, entonces puede identificarse con una función $f:\T^n\rightarrow\C$ definida como $f([x]) = g(x)$, donde $[x]$ es la clase de equivalencia de $x\in\R^n$ en el cociente. No obstante, no es necesario realizar la distinción entre punto y clase de equivalencia y se denotará $x\in\T^n$, de manera similar, se dirá que $f=g$ para los fines de este trabajo. 

\begin{definition}[Espacio de Schwartz $\mathcal{S}(\Z^n)$]
	Sea $\mathcal{S}(\Z^n;X)$ el espacio de \textit{funciones de decaimiento rápido} $\varphi:\Z^n\rightarrow X$ que satisfacen  
	\begin{equation*}
		\|\varphi(\xi)\|_X \lesssim_M \angles{\xi}^{-M},
	\end{equation*}
	en todo $\xi\in\Z^n$, para cualquier $M>0$. La convergencia de este espacio está dada por las seminormas $p_k(\varphi) := \sup_{\xi\in\Z^n}\angles{\xi}^k|\varphi(x)|$.
\end{definition}  
\begin{definition}[Transformada de Fourier periodica]\label{def:Fourier-Tn}
	Sea $\mathcal{F}_{\T^n}:C^\infty(\T^n;X) \rightarrow\mathcal{S}(\Z^n;X)$ la \textit{transformada de Fourier periodica} definida por 
	\begin{equation*}
		(\mathcal{F}_{\T^n}f)(\xi) = \widehat{f}(\xi) := \int_{\T^n} e^{-i2\pi x\cdot\xi} f(x) \diff x.
	\end{equation*}
	Además, se define la transformada de Fourier periodica inversa $\mathcal{F}_{\T^n}^{-1}:\mathcal{S}(\Z^n;X)\rightarrow C^\infty(\T^n;X)$, como
	\begin{equation*}
		(\mathcal{F}_{\T^n}^{-1}\varphi)(x)  := \sum_{\xi\in\Z^n} e^{i2\pi x\cdot\xi} \varphi(\xi).
	\end{equation*}
\end{definition}
\begin{theorem}
	La definición \ref{def:Fourier-Tn} es válida. Es decir, 
	\begin{enumerate}
		\item $\mathcal{F}_{\T^n}C^\infty(\T^n;X) \subset \mathcal{S}(\Z^n;X)$,
		\item $\mathcal{F}_{\T^n}^{-1} \mathcal{S}(\Z^n;X) \subset C^\infty(\T^n;X)$,
		\item $\mathcal{F}_{\T^n} \circ \mathcal{F}_{\T^n}^{-1}$ y $\mathcal{F}_{\T^n}^{-1} \circ \mathcal{F}_{\T^n}$ son la función identidad en $\mathcal{S}(\Z^n;X)$ y $C^\infty(\T^n;X)$ respectivamente.
	\end{enumerate}
\end{theorem}
\begin{proof}
	Para la primera parte, se toma $f\in C^\infty(\T^n;X)$ y $\alpha\in\N_0^n$. Entonces 
	\begin{align*}
		(-i2\pi\xi)^\alpha \widehat{f}(\xi) & =  \int_{\T^n}(-i2\pi\xi)^\alpha e^{-i2\pi x\cdot\xi} f(x) \diff x \\ 
		& =  \int_{\T^n} [\partial^\alpha_xe^{-i2\pi x\cdot\xi} ]f(x) \diff x\\
		& = (-1)^{|\alpha|}  \int_{\T^n} e^{-i2\pi x\cdot\xi} [\partial^\alpha_xf(x)] \diff x < \infty.
	\end{align*}
	Por lo que $\|\angles{\xi}^M\widehat{f}(\xi)\|_X < \infty$ para cualquier $M<\infty$, y $\widehat{f}\in\mathcal{S}(\Z^n;X)$. Para la segunda parte, se toma $\varphi\in\mathcal{S}(\Z^n;X)$ y se tiene que 
	\begin{equation*}
		\|\partial^\alpha_x[\mathcal{F}_{\T^n}^{-1}\varphi(x)]\|_X \leq \sum_{\xi\in\Z^n}	\left\|   	\partial^\alpha_xe^{i2\pi x\cdot\xi} \varphi(\xi)
		\right\|_X \lesssim_M \sum_{\xi\in\Z^n} |\xi^\alpha| \angles{\xi}^{-M} < \infty,
	\end{equation*}
	 para $M$ lo suficientemente grande. Por lo que $\mathcal{F}_{\T^n}^{-1}\varphi$ es suave. Para la tercera parte, primero se aprovecha la convergencia uniforme para tener que 
	\begin{align*}
		[\mathcal{F}_{\T^n} (\mathcal{F}_{\T^n}^{-1}\varphi)](\xi) &=  \int_{\T^n} e^{-i2\pi x\cdot\xi} \sum_{\eta\in\Z^n} e^{i2\pi x\cdot\eta}\varphi(\eta) \diff x \\
		& =  \sum_{\eta\in\Z^n}\varphi(\eta) \int_{\T^n} e^{i2\pi x\cdot(\eta-\xi)} \diff x \\
		& = \sum_{\eta\in\Z^n}\varphi(\eta) \delta_{\eta,\xi} = \varphi(\xi).
	\end{align*}
	Por otra parte, se tiene que 
	\begin{align*}
		[\mathcal{F}_{\T^n}^{-1} (\mathcal{F}_{\T^n}f)](x)  & = \sum_{\xi\in\Z^n} e^{i2\pi x\cdot\xi} \int_{\T^n} e^{-i2\pi y\cdot\xi} f(y)\diff y \\
		& =  \int_{\T^n} \sum_{\xi\in\Z^n}e^{i2\pi (x-y)\cdot\xi} f(y)\diff y \\
		& = \lim_{\alpha\to\infty}  \int_{\T^n} \sum_{\xi\leq\alpha} e^{i2\pi y\cdot\xi} f(x-y)\diff y\\
		& =: \lim_{\alpha\to\infty}  \int_{\T^n} D_\alpha(y)f(x-y)\diff y \\
		&=: \lim_{\alpha\to\infty}  S_\alpha f(x).
	\end{align*}
	A $D_\alpha$ se le conoce como el kernel de Dirichlet y a $S_\alpha$ como el operador de sumas parciales. Se puede trabajar el kernel de Dirichlet para obtener que 
	\begin{align*}
		D_\alpha(y) & =  \sum_{\xi\leq\alpha} e^{i2\pi y\cdot\xi}  \\
		& = \prod_{j=1}^n \sum_{\xi_j=- \alpha_j}^{\alpha_j} e^{i2\pi y_j\xi_j} \\
		& = \prod_{j=1}^n e^{-i2\pi \alpha_jy_j}\left(\frac{e^{i2\pi(2\alpha_j+1)y_j} - 1}{e^{i2\pi y_j}-1}
		\right) \\
		& = \prod_{j=1}^n \frac{\sin(\pi(2\alpha_j+1)y_j)}{\sin(\pi y_j)}.
	\end{align*}
	Ahora, se utiliza el método de sumabilidad de Césaro, que indica que el promedio de una sucesión converge al mismo límite que el límite de la sucesión, para definir el kernel de Fejér como
	 \begin{align*}
	 	\sigma_Nf(x) & := \frac{1}{(N+1)^n}\sum_{\|\alpha\|_\infty\leq N} S_\alpha f(x)\\
	 	& = \int_{\T^n} \frac{1}{(N+1)^n}\sum_{\|\alpha\|_\infty\leq N}D_\alpha(y) f(x-y)\diff y \\
	 	& =: \int_{\T^n} F_N(y) f(x-y)\diff y.
	 \end{align*}
	 El kernel de Fejér se puede reescribir notando que $e^{i2\pi y_j\xi_j}$ aparece exactamente $N+1-|\xi_j|$ veces en la sumatoria, obteniendo que
	 \begin{align*}
	 	F_N(y) &= \frac{1}{(N+1)^n} \prod_{j=1}^n \sum_{\xi_j=-N}^N (N+1-|\xi_j|)e^{i2\pi y_j\xi_j}\\
	 	& =  \frac{1}{(N+1)^n} \prod_{j=1}^n \sum_{k=0}^N \sum_{\xi_j=-k}^k e^{i2\pi y_j\xi_j}.
	 \end{align*}
	 Ahora, note que tiene una forma similar al kernel de Dirichlet para obtener que
	 \begin{align*}
	 	F_N(y) 
	 	& =\frac{1}{(N+1)^n}  \prod_{j=1}^n \sum_{k=0}^N \frac{\sin(\pi(2k+1)y_j)}{\sin(\pi y_j)} \\
	 	& = \frac{1}{(N+1)^n}  \prod_{j=1}^n \sum_{k=0}^N \frac{\sin(\pi(2k+1)y_j)\sin(\pi y_j)}{\sin^2(\pi y_j)} \\ 
	 	& = \frac{1}{2(N+1)^n}  \prod_{j=1}^n\sum_{k=0}^N \frac{\cos(2k\pi y_j) - \cos(2(k+1)\pi y_j)}{\sin^2(\pi y_j)} \\
	 	& = \frac{1}{2(N+1)^n}  \prod_{j=1}^n \frac{1-\cos(\pi(N+1)y_j)}{\sin^2(\pi y_j)} \\
	 	& = \frac{1}{(N+1)^n}  \prod_{j=1}^n \left[ \frac{\sin(\pi(N+1)y_j)}{\sin(\pi y_j)}
	 	\right]^2.
	 \end{align*}
	 Este kernel tiene las propiedades que $F_N \geq 0$, que $\int_{\T^n}F_N(y)\diff y=1$, y que para $\delta>0$ se tiene que
	 \begin{equation*}
	 	\lim_{N\to\infty}\int_{\delta<|y|} F_N(y) \diff y \leq \lim_{N\to\infty} \frac{1}{(N+1)^n\sin^{2n}(\pi \delta)} =0.
	 \end{equation*}
	 Entonces, se puede concluir que 
	 \begin{align*}
	 	\|\sigma_Nf(x) - f(x)\|_X & \leq \int_{\T^n} \|f(x-y) - f(x)\|_X F_N(y) \diff y \\
	 	& \leq \int_{|y|<\delta} \|f(x-y) - f(x)\|_XF_N(y) \diff y +2\|f\|_{L^\infty} \int_{\T^n} F_N(y) \diff y \to 0,
	 \end{align*}
	 debido a que el primer término puede ser controlado escogiendo $\delta$ lo suficientemente pequeño gracias a la continuidad de $f$, y el segundo termino puede controlarse al escoger $N$ una vez fijado $\delta$. Esta convergencia es uniforme respecto a $x$ debido a la compacidad de $\T^n$. Por lo que se justifica la definición de la transformada de Fourier y su inversa.
\end{proof}
\begin{remark}
	En la demostración anterior se utilizan técnicas de sumabilidad que son frecuentes en el análisis de Fourier. Para mayor detalle sobre estas técnicas y el análisis de Fourier se recomienda al lector revisar Duoandikoetxea \cite{duoandikoetxea}. Por otra parte, estas técnicas pueden ser utilizadas para demostrar la convergencia de estas series en la norma $L^p$.
\end{remark}
\begin{theorem}
	La serie de Fourier de $f$ converge a $f$ en la norma $L^p(\T^n;X)$.
\end{theorem}
\begin{proof}
	Para el operador de Césaro $\sigma_N$ y el kernel de Fejér se tiene que 
	\begin{align*}
		\|\sigma_Nf -f\|_{L^p}  & \leq \int_{\T^n} \|f(\cdot-y) - f(\cdot)\|_{L^p} F_N(y) \diff y \\
		& \leq \int_{|y|<\delta} \|f(\cdot-y) - f(\cdot)\|_{L^p}F_N(y) \diff y +2\|f\|_{L^p} \int_{\T^n} F_N(y) \diff y \to 0,
	\end{align*}
	donde, de nuevo, se puede controlar el primer término con $\delta$ y el segundo escogiendo un $N$ apropiado. 
\end{proof}
\begin{definition}[Espacios de sucesiones $\ell^p(\mathcal{C})$]
	El \textit{espacio de sucesiones} $\ell^p(\mathcal{C};X)$, para $\mathcal{C}$ un conjunto enumerable, consiste de las funciones $a:\mathcal{C}:\to X$, tales que 
	\begin{equation*}
		\|a\|_{\ell^p} := \left( \sum_{k\in\mathcal{C}}\|a_k\|_X^p \right)^{1/p} <\infty.
	\end{equation*}
	Generalmente, $\mathcal{C}=\Z^n$.
\end{definition}
\begin{theorem}[Identidad de Plancherel]
	Si $u\in L^2(\T^n;X)$, entonces $\widehat{u}\in\ell^2(\Z^n;X)$, y se cumple que 
	\begin{equation*}
		\|\widehat{u}\|_{\ell^2} = \|u\|_{L^2}.
	\end{equation*}
\end{theorem}
\begin{proof}
	Primero, suponga que $X=\mathcal{H}$ un espacio de Hilbert, entonces
	\begin{align*}
		(u, S_Nu)_{L^2} &= \int_{\T^n} \left(u(x), \sum_{|\xi|\leq N}\widehat{u}(\xi)e^{i2\pi x\cdot\xi}\right)_\mathcal{H}\diff x \\
		& = \sum_{|\xi|\leq N} \left(\widehat{u}(\xi) ,\int_{\T^n} u(x)e^{-i2\pi x\cdot\xi} \right)_\mathcal{H}\diff x \\
		& = \sum_{|\xi|\leq N} \left(\widehat{u}(\xi), \widehat{u}(\xi)\right))_\mathcal{H} = \|\widehat{u}_N\|_{\ell^2}^2.
	\end{align*}
	Entonces, se tiene que 
	\begin{align*}
		\|u-S_Nu\|_{L^2}^2 & = \|u\|_{L^2}^2 - (u, S_Nu)_{L^2} - (S_Nu, u)_{L^2} + \|S_Nu\|_{L^2}^2 \\
		& = 2\|u\|_{L^2}^2 - 2\|\widehat{u}_N\|_{\ell^2}^2.
	\end{align*}
	Por continuidad de las normas, se obtiene la identidad deseada al hacer $N\to\infty$. Cuando $X$ es un espacio de Banach general, tome una función simple $f=\sum_j^m x_j\chi_{E_j}$. Entonces, para cualquier $x^*\in X'$, se cumple que 
	\begin{equation*}
		\int_{\T^n} |\angles{x^*, f}|^2= \int_{\T^n} \sum_j^m |\angles{x^*, x_j}|^2\chi_{E_j}  = \sum_j^m |\angles{x^*, x_j}|^2|E_j| = \sum_j^m  |\angles{x^*, x_j}|^2 \sum_{\Z^n} |\widehat{\chi_{E_j}}|^2  = \sum_{\Z^n} |\angles{x^*, \widehat{f}}|^2,
	\end{equation*}
	como  $\sum |\widehat{\chi_{E_j}}|^2 = \int \chi_{E_j }= |E_j|$. Ahora, note que la imagen de $f$ está contenida en un espacio finito dimensional $Y$ generado por los $x_j$. Por lo que existen funcionales $x^*_1, \ldots,x^*_k$ tales que 
	\begin{equation*}
		\|x\|_X^2 = \sum_\ell^k |\angles{x^*_\ell, x}|^2.
	\end{equation*}
	La prueba se concluye al tomar la suma respecto a estos funcionales.
\end{proof}

\section{Distribuciones y espacios de Sobolev en $\R^n$}
En esta sección se inicia introduciendo el espacio de distribuciones templadas
que permite extender la transformada de Fourier a un espacio más general 
que $L^1(\R^n;X)$.
\begin{definition}[Distribuciones templadas $\mathcal{S}'(\R^n;X)$]
    Se define el \textit{espacio de distribuciones templadas} como 
    el espacio de operadores lineales continuos 
    $u:\mathcal{S}(\R^n)\to X$. 
    En este caso, se entiende la continuidad en el sentido que si 
    $\varphi_j\rightarrow\varphi$ en $\mathcal{S}(\R^n)$, entonces se tiene
    que $u(\varphi_j) \rightarrow u(\varphi)$ en $X$. Además, se dice que 
    $u_j\rightarrow u$ en $\mathcal{S}'(\R^n;X)$ si 
    $u_j(\varphi)\rightarrow u(\varphi)$ para todo $\varphi\in \mathcal{S}(\R^n)$.
\end{definition}
Las funciones en $\mathcal{S}(\R^n)$ se les conoce como las funciones de prueba
del espacio de distribuciones templadas. Otra notación usual para $u(\varphi)$
es $\angles{u,\varphi}$.
\begin{remark}[Funciones como distribuciones]
    Se puede considerar a $f\in L^p(\R^n;X)$ como una distribución templada. Se 
    define el funcional $u_f$ de la siguiente manera
    \begin{equation*}
        \angles{u_f, \varphi} := \int_{\R^n} f\varphi \diff x.
    \end{equation*}
    Claramente es un funcional lineal. La continuidad es resultado de la
    desigualdad de H\"older y el encaje continuo de las funciones de prueba
    en el espacio $L^{p'}$. En efecto, para $\varphi_j\rightarrow\varphi$ en 
    $\mathcal{S}(\R^n)$ se tiene que
    \begin{equation*}
        |\angles{u_f, \varphi_j} - \angles{u_f, \varphi}| \leq 
        \|f\|_{L^p}\|\varphi_j - \varphi\|_{L^{p'}} \rightarrow 0.
    \end{equation*}
    Por simplicidad se denota $\angles{u_f, \varphi} = \angles{f, \varphi}$. Además, este encaje es continuo, pues si $f_j\rightarrow f$ en $L^p(\R^n;X)$, se tiene que 
    \begin{equation*}
    	|\angles{f_j, \varphi} - \angles{f, \varphi}| \leq \|f_j-f\|_{L^p}\|\varphi\|_{L^{p'}}.
    \end{equation*}
\end{remark}
Particularmente, para $\varphi\in\mathcal{S}(\R^n;X)$, se puede motivar la 
definición de distintas propiedades de distribuciones mediante la manipulación 
del funcional $u_\varphi$ mencionado anteriormente. Por ejemplo, en vista de
la integración por partes tenemos que 
\begin{equation*}
    \angles{\partial_j\varphi, \psi} = \int_{\R^n} (\partial_j\varphi) \psi\diff x
    = - \int_{\R^n} \varphi(\partial_j \psi)\diff x = - \angles{\varphi,\partial_j
    \psi}.
\end{equation*}
Por lo que definimos la derivada en el sentido de distribuciones de la 
siguiente manera
\begin{definition}[Derivada distribucional]
    Para $u\in\mathcal{S}'(\R^n;X)$ se define
    \begin{equation*}
    	\angles{\partial^\alpha u, \varphi} := (-1)^{|\alpha|} \angles{u, \partial^\alpha\varphi},
    \end{equation*}
    para cualquier función de prueba $\varphi$ y cualquier multi-índice $\alpha\in\N_0^n$. 
\end{definition}
\begin{example}
	Considere la función Heaviside, o escalón, dada por
	\begin{equation*}
		H(x) := \begin{cases} 0, &  x <0 \\
			1,  &x \geq 0\end{cases}.
	\end{equation*}
	Es claro que representa una distribución templada, así que se calcula su derivada distribucional
	\begin{align*}
		\angles{ H', \varphi} = - \int_\R H\varphi'\diff x  = -\int_0^\infty \varphi'\diff x \ = -\varphi\vert_0^\infty  = \varphi(0) =: \angles{\delta, \varphi}.
	\end{align*}
	Donde $\delta$ es el funcional conocido como la delta de Dirac. Por lo que se tiene que en el sentido de distribuciónes que $ H' = \delta$.
\end{example}
Por otra parte, la fórmula de multiplicación de Fourier motiva la definición de la transformada de Fourier para distribuciones.
\begin{definition}[Transformada de Fourier para distribuciones]
	Para $u\in\mathcal{S}'(\R^n;X)$ se define 
	\begin{equation*}
		\angles{\mathcal{F}u, \varphi} := \angles{u, \mathcal{F}\varphi}, \quad 
		\angles{\mathcal{F}^{-1}u, \varphi} := \angles{u, \mathcal{F}^{-1}\varphi},
	\end{equation*}
	para cualquier función de prueba $\varphi$.
\end{definition}
\begin{example}
	Considere la distribución de la delta de Dirac dada por $\angles{\delta, \varphi} := \varphi(0)$.  Se calcula su transformada de Fourier de la siguiente manera
	\begin{equation*}
		\angles{\mathcal{F}\delta, \varphi} = \angles{\delta, \widehat{\varphi}} = \widehat{\varphi}(0) = \int_\R \varphi\diff x = \angles{1, \varphi}.
	\end{equation*}
	Por lo que en el sentido de distribuiciones se tiene que $\widehat{\delta} = 1$ la función constante, que es acotada y por tanto una distribución. También se puede demostrar que $\widehat{1}  = \delta$. En efecto 
	\begin{equation*}
		\angles{\mathcal{F}(1), \varphi} = \int_\R \widehat{\varphi} \diff x = 
		\mathcal{F}^{-1}(\widehat{\varphi})(0) = \varphi(0)  = \angles{\delta, \varphi}.
	\end{equation*}
\end{example}
\begin{theorem}
	La transformada de Fourier $\mathcal{F}$ es continua en $\mathcal{S}'(\R^n;X)$.
\end{theorem}
\begin{proof}
	Sea $u_j\rightarrow u$ en $\mathcal{S}'(\R^n;X)$, entonces
	\begin{equation*}
		\widehat{u_j}(\varphi) = u_j(\widehat{\varphi}) \rightarrow u(\widehat{\varphi}) = \widehat{u}(\varphi).
	\end{equation*}
	Por lo que es un operador continuo.
\end{proof}
\begin{lemma}
	$C^\infty_0(\R^n;X)$ es secuencialmente denso en $\mathcal{S}(\R^n;X)$.
\end{lemma}
\begin{proof}
	Sea $\psi\in C^\infty_0(\R^n)$ igual a uno en una vecindad del origen. Entonces se define $\psi_k(x) := \psi(x/k)$ y se puede verificar que $\psi_k\varphi\rightarrow\varphi$ en $\mathcal{S}(\R^n;X)$.
\end{proof}
\begin{theorem}
	$C^\infty_0(\R^n;X)$ es secuencialmente denso en $\mathcal{S}'(\R^n;X)$.
\end{theorem}
\begin{proof}
	Sea $u\in\mathcal{S}'(\R^n;X)$ y sean $\psi,\psi_k$ como en la demostración anterior. Entonces se define $\angles{\psi u, \varphi} := \angles{u, \psi\varphi}$, y se tiene que $\psi_ku\rightarrow u$ en $\mathcal{S}'(\R^n)$. En efecto, por el lema anterior se tiene que
	\begin{equation*}
		\angles{\psi_k u, \varphi} = \angles{u, \psi_k\varphi} \rightarrow \angles{u, \varphi}.
	\end{equation*}
	Similarmente, $\psi_k\widehat{u}\rightarrow\widehat{u}$ en $\mathcal{S}'(\R^n)$, lo que implica que $\mathcal{F}^{-1}(\psi_k\widehat{u})\rightarrow u$ en $\mathcal{S}'(\R^n)$ debido a la continuidad de la transformada de Fourier. Entonces, se tiene que 
	\begin{equation*}
		u_k := \psi_k [\mathcal{F}^{-1}(\psi_k\widehat{u}) ]\rightarrow u
	\end{equation*}
	en $\mathcal{S}'(\R^n)$. Solo queda demostrar $u_k\in C^\infty_0(\R^n;X)$, en el caso general, para cualquier $\chi\in C^\infty_0(\R^n)$, se tiene que 
	\begin{equation*}
		\angles{\mathcal{F}^{-1}(\chi \widehat{u}), \varphi}  = 
		\angles{\widehat{u}, \chi\mathcal{F}^{-1}\varphi} 
		 = \int_{\R^n} \angles{\widehat{u},\chi(\xi) e^{2\pi i x\cdot\xi}}\varphi(x) \diff x.
	\end{equation*}
	Por lo que se puede identificar $\mathcal{F}^{-1}(\chi \widehat{u})(x) = \widehat{u}(\chi(\xi)e^{2\pi ix\cdot\xi})$, que es continua respecto $x$ y que sus derivadas respecto a $x$ tienen soporte compacto respecto a $\xi$, por lo que las derivadas de $\mathcal{F}^{-1}(\chi \widehat{u})(x)$ tienen soporte compacto.
\end{proof}
\begin{definition}[Espacios de Sobolev]
	Sea $1\leq p \leq\infty$ y sea $k\in\N_0$. El \textit{espacio de Sobolev} $W^k_p(\R^n;X)$ consiste de todas las funciones $f\in L^p(\R^n;X)$ tales que para cualquier multi-índice $|\alpha|\leq k$ se tiene que $\partial^\alpha f$ existe (en el sentido de distribuciones) y pertenece a $L^p(\R^n;X)$. Para tales funciones se define 
	\begin{equation*}
		\|f\|_{W^k_p} := \left( \sum_{|\alpha|\leq k} \|\partial^\alpha f\|_{L^p}^p
		\right)^{1/p},
	\end{equation*}
	para $1\leq p <\infty$. Para $p=\infty$ se define como 
	\begin{equation*}
		\|f\|_{W^k_\infty} := \max_{|\alpha|\leq k} \|\partial^\alpha f\|_{L^\infty}.
	\end{equation*}
\end{definition}
\begin{remark}
	Se advierte al lector que existen otras notaciones disponibles en la literatura.  Por ejemplo $L^p_k$, o $W^{p,k}$. Además, cuando $p=2$, se suele denotar como $H^k$.
\end{remark}
\begin{theorem} \label{theo:Sobolev-Banach}
	Sea $f, g \in W^k_p(\R^n;X)$ y sea $\alpha$ un multi-índice con $|\alpha|\leq k$, entonces se tiene que 
	\begin{enumerate}
		\item $\partial^\alpha f \in W^{k-|\alpha|}_p$ y que $\partial^{\alpha}(\partial^{\beta}f) = \partial^{\alpha+\beta} f = \partial^{\beta}(\partial^{\alpha}f)$, para todos multi-índices que satisfacen  $|\alpha|+|\beta| \leq k$,
		\item $\lambda f + \mu g \in W^k_p$ y $\partial^\alpha(\lambda f + \mu g) = \lambda\partial^\alpha f + \mu\partial^\alpha g$, para cualesquiera $\lambda, \mu\in \C$,
		\item $\|\cdot\|_{W^k_p}$ es una norma,
		\item $W^k_p(\R^n;X)$ es un espacio de Banach.
	\end{enumerate}  
\end{theorem}
\begin{proof}
	Los primeros dos incisos son resultado de la definición de derivada en el sentido de distribuciones. En efecto, para $(1)$ se tiene que 
	\begin{equation*}
		\angles{\partial^\alpha(\partial^\beta f), \varphi} = 
		(-1)^{|\alpha|} \angles{\partial^\beta f, \partial^\alpha\varphi} =
		(-1)^{|\alpha| + |\beta|} \angles{f, \partial^{\alpha+\beta}\varphi}. 
	\end{equation*}
	El otro caso es análogo. El inciso $(2)$ es resultado de la linealidad de $\angles{\cdot, \varphi}$.  Para el inciso $(3)$ es claro que $\|\lambda f\|_{W^k_p} = |\lambda|\|f\|_{W^k_p}$ por lo anterior, y que $|f\|_{W^k_p} = 0$ si y solo si $f$ se anula en casi todas partes. La desigualdad triangular para $p=\infty$ es trivial, para el caso $1\leq p<\infty$ se tiene que 
	\begin{align*}
		\|f+g\|_{W^k_p} &= \left(\sum_{|\alpha|\leq k} \|\partial^\alpha f + \partial^\alpha g\|_{L^p}^p
		\right)^{1/p} \\ 
		& \leq \left(\sum_{|\alpha|\leq k} (\|\partial^\alpha f\|_{L^p} +  \|\partial^\alpha g\|_{L^p})^p
		\right)^{1/p} \\
		& \leq \left(\sum_{|\alpha|\leq k} \|\partial^\alpha f\|_{L^p}^p
		\right)^{1/p} + \left(\sum_{|\alpha|\leq k} \|\partial^\alpha g\|_{L^p}^p
		\right)^{1/p} \\
		& = \|f\|_{W^k_p} + \|g\|_{W^k_p}.
	\end{align*}
	Para el inciso $(4)$ se toma una sucesión de Cauchy $f_j$ en $W^k_p$. Entonces, $\partial^\alpha f_j$ es una sucesión de Cauchy en $L^p$ para todo $|\alpha|\leq k$. Como $L^p$ es completo, se tiene que $\partial^\alpha f_j$ converge a algún $g_\alpha$ en $L^p$. Entonces, se tiene que
	\begin{align*}
		\angles{\partial^\alpha g_0, \varphi}& = (-1)^{|\alpha|}  \angles{g_0, \partial^\alpha \varphi} \\
		& = \lim_{j\rightarrow\infty}(-1)^{|\alpha|} \angles{f_j, \partial^\alpha \varphi} \\
		& =  \lim_{j\rightarrow\infty}\angles{\partial^\alpha f_j, \varphi} \\
		& = \angles{g_\alpha, \varphi}.
	\end{align*} 
	Por lo que $\partial^\alpha g_0 = g_\alpha$ y $f_j \rightarrow g_0$ en $W^k_p$.
\end{proof}
\begin{theorem}[Encaje de Sobolev]\label{theo:Sobolev-embedding-R}
	Sea $s\in\N$, tal que $s>k+n/2$. Entonces se tiene que el espacio de Sobolev $W^s_2(\R^n;X)$ está contenido en el espacio de H\"older $C^k(\R^n;X)$ y la inclusión es continua.
\end{theorem}
\begin{proof}
	Sea $u\in W^s_2(\R^n;X)$, entonces por la identidad de Plancherel se tiene que 
	\begin{equation*}
		\|u\|_{W^s_2}^2 = \sum_{|\alpha|\leq s} \|\partial^\alpha u\|_{L^2}^2  \sim  \sum_{|\alpha|\leq s} \int_{\R^n} |\xi^\alpha|^2\|\widehat{u}(\xi)\|_X^2 \diff \xi \sim  \int_{\R^n} (1+|\xi|^2)^s \|\widehat{u}(\xi)\|_X^2 \diff \xi.
	\end{equation*}
	Ahora, basta demostrar que $\widehat{\partial^\alpha u} \in L^1(\R^n;X)$ para $|\alpha|\leq k$, porque esto implicaría que su transformada inversa es continua y acotada. Utilizando Cauchy-Schwarz, se obtiene que 
	\begin{align*}
		 \int_{\R^n} |\xi^\alpha|\|\widehat{u}(\xi)\|_X \diff \xi & \leq 
		 \left( \int_{\R^n} (1+|\xi|^2)^s \|\widehat{u}(\xi)\|_X^2 \diff \xi \right)^{1/2}
		 \left( \int_{\R^n} \frac{|\xi|^{2|\alpha|}}{(1+|\xi|^2)^s}  \diff \xi \right)^{1/2} \\
		 & \lesssim \|u\|_{W^s_2}.
	\end{align*}
	Donde la segunda integral es finita dado que $2|\alpha| - 2s \leq 2k -2s < -n$. Además, se tiene que 
	\begin{equation*}
		\sup\|\partial^\alpha u\|_X \leq \|\widehat{\partial^\alpha u} \|_{L^1} \lesssim  \|u\|_{W^s_2},
	\end{equation*}
	por lo que la inclusión es continua y se completa la prueba.
\end{proof}
\section{Distribuciones y espacios de Sobolev en $\T^n$}
\begin{theorem}[Distribuciones templadas $\mathcal{S}'(\Z^n;X)$]
	Los elementos $u$ del espacio de distribuciones templadas $\mathcal{S}'(\Z^n;X)$, que consiste de operadores lineales continuos de $\mathcal{S}(\Z^n)$ en $X$,  tienen la forma 
	\begin{equation*}
		\varphi \mapsto \angles{u, \varphi} := \sum_{\xi\in\Z^n} u(\xi)\varphi(\xi).
	\end{equation*}
\end{theorem} 
\begin{proof}
	Note que se puede definir 
	\begin{equation*}
		u(\eta) := \angles{u, \delta_\eta},
	\end{equation*}
	y como cada $\varphi\in\mathcal{S}'(\Z^n)$ puede ser escrito como 
	\begin{equation*}
		\varphi(\xi) = \sum_{\eta\in\Z^n} \varphi(\eta)\delta_\eta(\xi),
	\end{equation*}
	entonces se concluye que 
	\begin{equation*}
		\angles{u, \varphi} = \left\langle u, \sum_{\eta\in\Z^n}  \varphi(\eta)\delta_\eta \right\rangle = \sum_{\eta\in\Z^n}  \varphi(\eta)\angles{u, \delta_\eta} = \sum_{\eta\in\Z^n}  \varphi(\eta)u(\eta).
	\end{equation*}
\end{proof}
\begin{definition}
	El espacio de \textit{distribuciones periódicas} $\mathcal{D}'(\T^n;X)$ consiste de los operadores lineales continuos definidos de $C^\infty(\T^n)$ en $X$. Note que esto incluye a las funciones $f\in L^p(\T^n;X)$ con $1\leq p\leq\infty$ definiendo su operador correspondiente de la siguiente manera
	\begin{equation*}
		\angles{f, \varphi} := \int_{\T^n} f\varphi\diff x.
	\end{equation*}
	Además, similarmente al caso euclideano, se define la derivada distribucional como a continuación 
	\begin{equation*}
		\angles{\partial^\alpha u, \varphi} := (-1)^{|\alpha|}\angles{u, \partial^\alpha\varphi}.
	\end{equation*}
\end{definition}
\begin{definition}
	Se define la \textit{transformada de Fourier periodica} en el sentido de distribuciones como un operador $\mathcal{F}_{\T^n}:\mathcal{D}'(\T^n)\to\mathcal{S}'(\Z^n)$ de la siguiente manera
	\begin{equation*}
		\angles{\mathcal{F}_{\T^n}u, \varphi} := \angles{u, \imath \circ \mathcal{F}_{\T^n}^{-1}\varphi},
	\end{equation*}
	donde $(\imath\circ\psi)(x) = \psi(-x)$. Esta definición es consistente cuando $u\in C^\infty(\T^n;X)$. En efecto, 
	\begin{align*}
		\angles{\widehat{u}, \varphi} &= \sum_{\xi\in\Z^n}\widehat{u}(\xi)\varphi(\xi)\\
		& = \sum_{\xi\in\Z^n} \varphi(\xi) \int_{\T^n}e^{-i2\pi x\cdot\xi}  u(x) \diff x \\
		& =\int_{\T^n}  \sum_{\xi\in\Z^n} \varphi(\xi) e^{i2\pi (-x)\cdot\xi}u(x)\diff x \\
		& = \int_{\T^n} (\mathcal{F}_{\T^n}^{-1}\varphi)(-x) u(x) \diff x = \angles{u, \imath\circ \mathcal{F}_{\T^n}^{-1}\varphi}.
	\end{align*}
	Lo que genera el resultado deseado.
\end{definition}
Al identificar $\T^n$ con $[0,1)^n\subset\R^n$, se puede utilizar la definición de espacios de Sobolev euclideana para el caso periodico.
\begin{definition}[Espacios de Sobolev]
	Sea $1\leq p \leq\infty$ y sea $k\in\N_0$. El \textit{espacio de Sobolev} $W^k_p(\T^n;X)$ consiste de todas las funciones $f\in L^p(\T^n;X)$ tales que para cualquier multi-índice $|\alpha|\leq k$ se tiene que $\partial^\alpha f$ existe (en el sentido de distribuciones) y pertenecen a $L^p(\T^n;X)$. Para tales funciones se define 
	\begin{equation*}
		\|f\|_{W^k_p} := \left( \sum_{|\alpha|\leq k} \|\partial^\alpha f\|_{L^p}^p
		\right)^{1/p},
	\end{equation*}
	para $1\leq p <\infty$. Para $p=\infty$ se define como 
	\begin{equation*}
		\|f\|_{W^k_\infty} := \max_{|\alpha|\leq k} \|\partial^\alpha f\|_{L^\infty}.
	\end{equation*}
\end{definition}
La demostración del hecho que los espacios de Sobolev son espacios de Banach es la realizada para el Teorema \ref{theo:Sobolev-Banach}. Además, se presenta el encaje de Sobolev para el caso toroidal, cuya prueba es analoga a la del Teorema \ref{theo:Sobolev-embedding-R}.
\begin{theorem}[Encaje de Sobolev]\label{theo:Sobolev-embedding-T}
	Sea $s\in\N$, tal que $s>k+n/2$. Entonces se tiene que el espacio de Sobolev $W^s_2(\T^n;X)$ está contenido en el espacio de H\"older $C^k(\T^n;X)$ y la inclusión es continua.
\end{theorem}
\section{Espacios de Hardy en $\R^n$ y $\T^n$}
En este apartado se presentan las bases de la teoría de interpolación compleja que permite extender propiedades de continuidad a espacios $L^p$ con $1<p<\infty$. Esto es posible gracias al clásico resultado de Fefferman \cite{fefferman-BMO}, que indica que el espacio de funciones de oscilación media acotada $\mathrm{BMO}$ es el dual del espacio de Hardy $H^1$. 
\begin{definition}[Espacio de Hardy $H^1$]
	Para $\Omega = \R^n, \T^n$. Se dice que $f\in L^1(\Omega;X)$ se encuentra en el \textit{espacio de Hardy} $H^1(\Omega;X)$ si existen $f_1,\ldots,f_n\in L^1(\Omega;X)$ que satisfacen
	\begin{equation*}
		\widehat{f_j} (\xi)= \frac{i\xi_j}{|\xi|}\widehat{f}(\xi).
	\end{equation*}
	Se escribe $f_j=:R_jf$, se le conoce como la transformada de Riesz, y se define la norma 
	\begin{equation*}
		\|f\|_{H^1} := \|f\|_{L^1} + \sum_{j=1}^n \|R_jf\|_{L^1}.
	\end{equation*}
\end{definition}
\begin{proposition}[Principio maximal]
	Sea $u$ una función real de clase $C^2$ en una región acotada $\Omega\subset\R^n$ y continua en $\overline{\Omega}$. Suponga que $\Delta u \geq 0$ en $\Omega$ y que $u\leq0$ en su frontera, entonces se tiene que $u\leq 0$ en todo $\Omega$. 
\end{proposition}
\begin{proof}
	Sin pérdida de generalidad, se puede suponer que $\nabla^2 u > 0$, en caso contrario se puede tomar $u+\varepsilon|x|^2 - \delta$ para $\varepsilon,\delta>0$ pequeños. Suponga que no vale $u\leq 0$ para todo $\Omega$, entonces existe algún máximo positivo en algún $x_0\in\Omega$. Como $\nabla u(x_0) >0$, se tiene que al menos algún $\partial^2_j u(x_0) > 0$. Además, como es máximo, se cumple $\partial_j u(x_0) = 0$. Usando el teorema de Taylor,
	\begin{equation*}
		u(x_0+he_j) - u(x_0) = \frac{1}{2} h^2\partial^2_j u(x_0) + o(h^3).
	\end{equation*}
	Por lo que no es un máximo y se obtiene la contradicción.
\end{proof}
\begin{lemma}\label{lem:subharmonic-gradient}
	Sea $F:=(u_0,\ldots,u_n)$, tal que satisface las ecuaciones de Cauchy-Riemann en $\Omega\times\R_+$, es decir 
	\begin{equation*}
		\sum_{j=0}^n \partial_{x_j}u_j = 0, \quad \text{ y } \quad \partial_{x_j}u_k = \partial_{x_k}u_j,
	\end{equation*}
	con $x_0=t$.  Suponga que $|F|>0$ en algún punto y sea $q\geq(n-1)/n$, entonces se tiene que $\Delta(|F|^q) \geq 0$. Particularmente, si $q>(n-1)/n$, 
	\begin{equation*}
		\Delta(|F|^q) \sim |F|^{q-2}|\nabla F|^2.
	\end{equation*}
	Note que las imagenes de las $u_j$ pueden encontrarse en un espacio de Hilbert abstracto.
\end{lemma}
\begin{proof}
	Primero, note que por las ecuaciones de Cauchy-Riemann
	\begin{equation*}
		\sum_{j=0}^n \partial_j^2 u_k = \partial_k\sum_{j=0}^n \partial_ju_j = 0,
	\end{equation*}
	y que $\Delta F=0$. Por la regla de Leibniz se tiene que
	\begin{equation*}
		\partial_j^2 |F|^q = q(q-2)|F|^{q-4}(\partial_j F\cdot F)^2 + q|F|^{q-2}[|\partial_jF]^2 + (\partial^2_jF\cdot F).
	\end{equation*}
	Al sumar respecto a $j$, se obtiene que
	\begin{equation*}
		\Delta|F|^q = q|F|^{q-4} \left[ (q-2)\sum (\partial_jF\cdot F)^2 + |F|^2\sum |\partial_jF|^2
		\right].
	\end{equation*}
	Ahora, si $q\geq2$, por la desigualdad de Schwarz se tiene que $\sum(\partial_jF\cdot F)^2 \leq |F|^2\sum|\partial_jF|^2$ y es claro que $\Delta(|F|^q)\geq 0$. Cuando $q<2$, note que se puede descomponer a las imagenes en sus componentes ortogonales y basta demostrarlo para un espacio de dimensión uno. Entonces, tome la matriz $M=(m_{jk}) := \partial_ju_k$, que es simétrica y por lo tanto diagonalizable a una matriz con diagonal $\lambda_0\geq\ldots\geq\lambda_n$. Como la traza es invariante respecto a conjugados, se tiene que
	\begin{equation*}
		\lambda_0 = -\sum_{j>0}\lambda_j,
	\end{equation*}
	y por la desigualdad de Schwarz,
	\begin{equation*}
		\lambda_0^2 \leq n \sum_{j>0} \lambda_j^2.
	\end{equation*}
	Ahora, la norma operador y la norma Frobenius de la matriz también son invariantes ortogonales y se obtiene que 
	\begin{equation*}
		\|M\|_{2}^2 = \lambda_0^2 \leq \frac{n}{n+1} \sum_{j}\lambda_j^2 = \frac{n}{n+1} \|M\|_{F}^2.
	\end{equation*}
	Esto implica que 
	\begin{equation*}
		\sum_j(\partial_jF\cdot F)^2 \leq |F|^2\|M\|_{2} \leq |F|^2\|M\|_F \leq  \frac{n}{n+1}  |F|^2\sum_j|\partial_jF|^2.
	\end{equation*}
	Lo que implica que $\Delta(|F|^q)$. Para obtener la comparación final deseada, basta con notar que 
	\begin{equation*}
		|F|^2\sum_j|\partial_jF|^2 = |F|^2|\nabla F|^2.
	\end{equation*}
	Lo que completa la prueba.
\end{proof}
Ahora, se presenta un operador bastante útil: la integral de Poisson. Para una discusión exhaustiva, vea Stein \cite{stein-singular}.
\begin{definition}
	Se define el \textit{kernel de Poisson}  $n$-dimensional como 
	\begin{equation*}
		P_t(x) := \frac{c_nt}{(t^2+|x|^2)^{(n+1)/2}},
	\end{equation*}
	donde $c_n$ es una constante para que su integral sea uno, y para $f\in L^1(\Omega;X)$, con $\Omega=\R^n,\T^n$, se define su \textit{integral o extensión de Poisson} como 
	\begin{equation*}
		(Pf)(x, t) := (P_t * f)(x).
	\end{equation*}
\end{definition}
Este kernel tiene una propiedad muy útil relacionada con la transformada de Fourier. Puede ser obtenida utilizando coordenadas polares y propiedades de las funciones de Besel, pero queda fuera del alcance de este trabajo.
\begin{proposition}
	El kernel de Poisson puede ser expresado como
	\begin{equation*}
		\mathcal{F}P_t(\xi) = e^{-2\pi t|\xi|}.
	\end{equation*}
\end{proposition}
 Además, tiene una relación interesante con el operador maximal de Hardy-Littlewood.
 \begin{theorem}
 	Para $f\in L^1(\Omega;X)$, con $\Omega=\R^n,\T^n$, se tiene que $\sup_{t>0}\|Pf(x, t)\|_X \lesssim \mathrm{M}f(x)$.
 \end{theorem}
 \begin{proof}
 	Note que 
 	\begin{equation*}
 		\int_{\varepsilon<|x-y|<R} \frac{t\|f(y)\|_X\diff y}{(t^2+|x-y|^2)^{(n+1)/2}} = \int_\varepsilon^R \int_{S^{n-1}}  P_t(r)\|f(x+rz)\|_Xr^{n-1}\diff \sigma(z)\diff r.
 	\end{equation*}
 	Al integrar por partes respecto a $r$, se obtiene que 
 	\begin{equation*}
 		\left. P_t(r)\int_{B(x, r)}\|f(y)\|_X\diff y\right\vert_{r=\varepsilon}^R - \int_\varepsilon^R \int_{B(x, r)}\|f(y)\|_X\diff y P_t'(r) \diff r,
 	\end{equation*} 
 	al tomar la norma y al hacer $\varepsilon\to0$ y $R\to\infty$, se puede estimar por
 	\begin{equation*}
 		\mathrm{M}f(x)\int_0^\infty r^n P_t'(r)\diff r .
 	\end{equation*}
 	La prueba se completa al notar que $P_t'(r) \sim \angles{r}^{-n-2}$.
 \end{proof}
Esto permite demostrar propiedades útiles de las extensiones de Poisson.
\begin{theorem}\label{theo:prop-Poisson}
	Sea $f\in L^1(\Omega;X)$, $g\in L^1(\Omega;\mathcal{B}(X, Y))$, con $\Omega=\R^n,\T^n$, entonces su integral de Poisson cumple que
	\begin{enumerate}
		\item $Pf$ es armónica, 
		\item cumple la condición de frontera $\lim_{t\to0}(Pf)(x, t) = f(x)$, para casi todo $x$,
		\item tiene propiedad de decaimiento $\lim_{t\to\infty}(Pf)(x, t) = 0$,
		\item se tiene que 
		\begin{equation*}
			\int_{\Omega} f(x)g(x)\diff x =2 \int_0^\infty\int_{\Omega} t[\nabla(Pf)(x, t)][\nabla(Pg)(x, t)]\diff x \diff t.
		\end{equation*}
	\end{enumerate}
\end{theorem}
\begin{proof}
	Para el primer inciso, se aprovecha su transformada de Fourier, para obtener que 
		\begin{equation*}
		\partial_t (P\varphi)(x, t) = - \mathcal{F}^{-1}\left\{2\pi|\xi| \widehat{f}(\xi) e^{-2\pi t |\xi|}  \right\}
	\end{equation*}
	\begin{equation*}
		\partial_{x_j} (P\varphi)(x, t) = \mathcal{F}^{-1}\left\{2\pi i\xi_j\widehat{f}(\xi)  \right\}.
	\end{equation*}
	Entonces, se tiene que 
	\begin{equation*}
		\nabla^2(Pf)(x, t) = \mathcal{F}^{-1}\left\{4\pi^2|\xi|^2 \widehat{f}(\xi) e^{-2\pi t |\xi|} \right\}- \sum_{j=1}^n 
		 \mathcal{F}^{-1}\left\{4\pi^2 \xi_j^2\widehat{f}(\xi) e^{-2\pi t |\xi|}\right\}
		 = 0.
	\end{equation*}
	Para el segundo y tercer inciso, se aprovecha que el operador $\sup_{t>0}\|Pf(x, t)\|_X$ es débil $(1, 1)$, entonces el Teorema \ref{theo:weak-closed} implica que basta demostrarlo para funciones en $\mathcal{S}(\Omega;X)$. En este caso es consecuencia de la continuidad de la transformada de Fourier inversa
	\begin{equation*}
		\lim_{t\to0} (Pf)(x, t) = \lim_{t\to0} \mathcal{F}^{-1}\left\{\widehat{f}(\xi) e^{-2\pi t |\xi|}\right\} = \mathcal{F}^{-1}\left\{\widehat{f}(\xi) \right\}  = f(x),
	\end{equation*}
	\begin{equation*}
		\lim_{t\to\infty} (Pf)(x, t) = \lim_{t\to\infty} \mathcal{F}^{-1}\left\{\widehat{f}(\xi) e^{-2\pi t |\xi|}\right\} =0.
	\end{equation*}
	Para el cuarto inciso, se procede descomponiendo
	\begin{equation*}
		 \int_0^\infty\int_{\Omega} t[\nabla(Pf)][\nabla(Pg)]\diff x \diff t = \int_0^\infty\int_{\Omega} t[\nabla_x(Pf)][\nabla_x(Pg)] + t\partial_t(Pf)\partial_t(Pg)\diff x \diff t =: I_1+I_2.
	\end{equation*}
	Por integración por partes en $I_2$, se tiene que
	\begin{align*}
		I_2& = \int_{\Omega} \left[ t(Pf)\partial_t(Pg) 
		\right]_{t=0}^\infty - \int_0^\infty (Pf)\partial_t(Pg) + t(Pf)\partial^2_t(Pg)\diff t \diff x \\  
		& =- \int_{\Omega}  \int_0^\infty (Pf)\partial_t(Pg) + t(Pf)\partial^2_t(Pg)\diff t \diff x\\
		&= \int_{\Omega}  \int_0^\infty   t(Pf)\nabla^2_x(Pg)-(Pf)\partial_t(Pg)\diff t \diff x,
	\end{align*}
	como $Pg$ es armónica. Ahora, integrando por partes respecto a $x$, y luego respecto a $t$, se tiene que
	\begin{align*}
		I_2 & = - \int_0^\infty \int_{\Omega}   t [\nabla_x(Pf)][\nabla_x(Pg)]+(Pf)\partial_t(Pg)\diff x \diff t \\
		& = -I_1 - \int_0^\infty \int_{\Omega} (Pf)\partial_t(Pg)\diff x \diff t 
	\end{align*}
	Por la simetría de $f$ y $g$, se puede reescribir como 
	\begin{align*}
		I_2 & =  - I_1 -\frac{1}{2}\int_{\Omega}  \int_0^\infty [\partial_t(Pf)](Pg) + (Pf)[\partial_t(Pg)]\diff t \diff x \\
		& =  - I_1 -\frac{1}{2}\int_{\Omega} [(Pf)(Pg)]_{t=0}^\infty \diff x \\
		&= -I_1 + \frac{1}{2}\int_{\Omega}fg\diff x.
	\end{align*}
	Lo que concluye la prueba.
\end{proof}
\begin{theorem}
	Para $f \in L^p(\Omega;X)$, con $\Omega=\R^n, \T^n$, se tiene que $\lim_{t\to0}(P_t*f) = f$ en norma $L^p$
\end{theorem}
\begin{proof}
	Primero note que por una pequeña manipulación
	\begin{equation*}
		P_t(y) = \frac{c_nt^{-n}}{(1+|y/t|^2)^{(n+1)/2}} = t^{-n}P_1(y/t)
	\end{equation*}
	Entonces, como $\int P_t \diff y = 1$, y por la desigualdad de Minkowski
	\begin{equation*}
		\|P_t*f - f\|_{L^p} \leq \int_{\Omega} \|f(\cdot-y)-f(\cdot)\|_{L^p} t^{-n}P_1(y/t) \diff y = \int_{\Omega} \|f(\cdot-ty)-f(\cdot)\|_{L^p} P_1(y) \diff y \to 0,
	\end{equation*}
	cuando $t\to0$. Esto se da como $\|f(\cdot-ty)-f(\cdot)\|_{L^p} \to 0$, por el Teorema de diferenciación de Lebesgue y se culmina argumentando convergencia dominada.
\end{proof}
Ahora, se presenta un caso especial de una desigualdad de Carleson para variedades demostrada por Hörmander \cite{hormander1967}.
\begin{lemma}\label{lem:Carleson}
	Sea $\mu$ una medida positiva sobre $\Omega\times\R_+$, con $\Omega=\T^n,\R^n$, tal que $\mu(T(x_0, h))\lesssim h^n$, donde $T(x_0, h) := \{(x, t): 0<t<h, \, |x-x_0| < h\}$. Entonces, se tiene que 
	\begin{equation*}
		\int_{\Omega\times\R_+} \|(Pf)(x, t)\|_X^p \diff \mu \lesssim \int_{\Omega} \|f(x)\|_X^p\diff x,
	\end{equation*}
	donde $f\in L^p(\R^n;X)$, y $1<p<\infty$.
\end{lemma}
\begin{proof}
	Primero se nota que 
	\begin{equation*}
		\|Pf(x, t)\|_X \lesssim \int_{\Omega} \frac{t\|f(y)\|_X\diff y}{(t + |x-y|)^{n+1}}.
	\end{equation*}
	Cuando $|x_0-y|<t$, se tiene que
	\begin{equation*}
		\int_{B(x_0, t)} \frac{t\|f(y)\|_X\diff y}{(t + |x-y|)^{n+1}} \lesssim t^{-n}  \int_{B(x_0, t)}  \|f(y)\|_X\diff y.
	\end{equation*}
	Además, cuando $2^{k-1}t \leq |x_0-y|\leq 2^kt$, se obtiene que
	\begin{align*}
		\int_{\{2^{k-1}t \leq |x_0-y|\leq 2^kt\}} \frac{t\|f(y)\|_X\diff y}{(t + |x-y|)^{n+1}}& \lesssim t^{-n}2^{n(k-1)}  \int_{B(x_0, 2^kt)}\| f(y)\|_X\diff y \\
		& \lesssim t^{-n}2^{n(k-1)} \frac{(2^kt)^{n}}{|B(x_0, 2^kt)|} \int_{B(x_0, 2^kt)}  \|f(y)\|_X\diff y.
	\end{align*}
	Sumando todas las partes se obtiene que 
	\begin{equation*}
		\|(Pf)(x, t)\|_X \lesssim \sup_{s\geq t}  \frac{1}{|B(x_0, s)|}\int_{B(x_0, s)}\|f(y)\|_X \diff y =: \tilde{f}(x, t).
	\end{equation*}
	Ahora, se define 
	\begin{equation*}
		E(\varepsilon, M) := \left\{ (x', t) : 0<t<M, \, \int_{B(x', t)} \|f\|_X\diff x > r(\varepsilon + t^n) \right\}.
	\end{equation*}
	Entonces, si existe una secuencia $(x_j, t_j) \subset E(\varepsilon, M)$, tales que las bolas $B(x_j, t_j)$ son disjuntas, se tiene que
	\begin{equation*}
		\sum_j r(\varepsilon + t_j^n) \leq \sum_j \int_{B(x_j, t_j)} \|f\|_X \diff x\leq \int_{\Omega} \|f\|_X \diff x.
	\end{equation*} 
	Por lo que dichas secuencias deben ser finitas. Entonces, sea $M_1 := \sup\{t:(x, t)\in E(\varepsilon, M)\}$ y se escoje $(x_1, t_1)\in E(\varepsilon, M)$, tal que $3t_1 \geq M_1$. Si ya se escogieron $j-1$ puntos, entonces se define $M_j$ como el supremo de los $t$ tales que $(x, t)\in E(\varepsilon, M)$ y que $B(x, t)$ es disjunta a las bolas $B(x_1, t_1),\ldots,B(x_{j-1}, t_{j-1})$. Luego se escoje $(x_j, t_j)$ tal que $3t_j \geq M_j$. Este proceso termina luego de un número de pasos. Entonces, para $(x, t)\in E(\varepsilon, M)$, se tiene que $B(x, t)$ intersecta a algún $B(x_j, t_j)$, particularmente, si $j$ es el menor índice para el que ocurre, se tiene que $t\leq M_j \leq 3t_j$, y que $B(x, t) \subset B(x_j, 3t_j)$. Ahora, se define
	\begin{equation*}
		E'(\varepsilon, M) := \{(x, t) : B(x, t) \subset B(x', t'), \, (x', t') \in E(\varepsilon, M)\} \subset \bigcup_j \{(x, t): B(x, t)\subset B(x_j, 3t_j)\}.
	\end{equation*}
	Por lo que se tiene que 
	\begin{align*}
		\mu (E'(\varepsilon, M) ) & \leq \sum_j \mu \{(x, t): B(x, t)\subset B(x_j, 3t_j)\} \\
		& \leq \sum_j \mu (T(x_j, 3t_j)) \\
		& \lesssim \sum_j t_j^n 
		 \leq \frac{1}{r} \int_{\R^n} \|f\|_X\diff x.
	\end{align*}
	Además, cuando $\varepsilon\to0$, y $M\to\infty$, se tiene que $E'(\varepsilon, M)$ crece al conjunto $\{|\tilde{f}|>r\} $. Finalmente, sea $\chi$ la función indicadora del conjunto $\{\|f\|_X<r\sigma\}$, para descomponer $f=f\chi + f(1-\chi)=:f_1+f_2$. Como $\tilde{f_1} \leq r\sigma$, entonces $\tilde{f_2} \geq r(1-\sigma)$ cuando $\tilde{f}>r$. Por lo que se tiene que
	\begin{align*}
		\int_0^\infty r^{p-1} \mu\{|\tilde{f}|>r\} \diff r & \lesssim \frac{1}{1-\sigma} \int_0^\infty \int_{\|f\|_X>r\sigma} r^{p-2} \|f\|_X \diff x \\
		& = \frac{1}{1-\sigma}\int_{\R^n}\int_{0<r<\|f\|_X/\sigma} r^{p-2}|f|\diff r \diff x \\
		& = \frac{\sigma^{1-p}}{(1-\sigma)(p-1)} \int_{\Omega}\|f\|_X^p \diff x \\
		\int_{\Omega\times\R_+} |\tilde{f}|^p \diff \mu & \lesssim \int_{\Omega}\|f\|_X^p \diff x,
	\end{align*}
	donde $\sigma<1$ puede ser escogido a conveniencia. Entonces, se concluye que 
	\begin{equation*}
		\int_{\Omega\times\R_+} \|Pf\|_X^p \diff \mu\lesssim \int_{\Omega\times\R_+} |\tilde{f}|^p \diff \mu \lesssim \int_{\Omega}\|f\|_X^p \diff x.
	\end{equation*}
	Lo que concluye la prueba.
\end{proof}
\begin{theorem}[Definición equivalente del espacio de Hardy $H^1$]\label{theo:H1-F}
	Para toda $f\in H^1(\Omega;X)$, con $\Omega =\R^n,\T^n$, existe una función $F:=(u_0, \ldots, u_n)$ tal que satisface las ecuaciones de Cauchy-Riemann en $\Omega\times\R_+$. Además, 
	\begin{equation*}
	\int_{\Omega} \sup_{t>0}  |F(x, t)| \diff x \lesssim \|f\|_{H^1},
	\end{equation*}
	 y $\lim_{t\to0}u_0(x, t) = f(x)$ casi en todas partes, y  en norma $L^1$.
\end{theorem}
\begin{proof}
	 Definase a $u_0 := Pf$, y $u_j := P(R_jf)$, que están bien definidas como $R_jf\in L^1$. Ahora, recuerde las derivadas de las integrales de Poisson
			\begin{equation*}
		\partial_t P(R_jf)(x, t) = -  \mathcal{F}^{-1}\left\{2\pi|\xi| \widehat{f}(\xi) \frac{i\xi_j}{|\xi|}e^{-2\pi t |\xi|} \right\},
	\end{equation*}
	\begin{equation*}
		\partial_{x_k} P(R_jf)(x, t) = \mathcal{F}^{-1}\left\{2\pi i\xi_k\widehat{f}(\xi) \frac{i\xi_j}{|\xi|}e^{-2\pi t |\xi|} \right\}.
	\end{equation*}
	Entonces, $\sum \partial_j u_j = 0$, y $\partial_j u_k = \partial_k u_j$. Además, $\lim_{t\to0}u_j(x, t) = R_jf(x)$ casi en todas partes y en norma $L^1$. La comparación de normas es resultado de esta convergencia.
\end{proof}
\begin{remark}
	En realidad la relación anterior es una doble implicación. Para cada $F$ cuyas componentes satsifagan las ecuaciones de Cauchy-Riemann, y que $\|F\|_{H^1}<\infty$, se puede encontrar una función $f(x):= \lim_{t\to0}u_0(x, t)$ en el espacio $H^1$. Esto implica demostrar que los límites no-tangenciales de estas funciones existen casi en todas partes y que las componentes $u_j$ son integrales de Poisson de medidas finitas. Entonces, se tiene una definición alternativa para el espacio de Hardy $H^1(\Omega;X)$ con normas equivalentes. Más aún, es posible definir mediante estas funciones $F$, los espacios $H^p$ para $p\leq1$, mediante la cuasi-norma
	\begin{equation*}
		\|F\|_{H^p} := \sup_{t>0} \left(\int_{\Omega} |F(x, t)|^p \diff x\right)^{1/p}.
	\end{equation*}
	Para un tratamiento más detallado, vea Stein \cite{stein-singular}.
\end{remark}
\begin{definition}[Espacio de funciones de oscilación media acotada $\mathrm{BMO}$]
	Para $f\in L^1_{loc}(\Omega; X)$, se define el \textit{operador $p$-maximal sharp} como
	\begin{equation*}
		\mathcal{M}_p^\#f(x) := \sup_{Q\ni x} \left( \frac{1}{|Q|}\int_Q \|f(y)-f_Q\|_X^p \diff y\right)^{1/p},
	\end{equation*}
	donde $f_Q$ es el valor promedio de $f$ sobre el cubo $Q$. Cuando $p=1$, se suele denotar $f^\#$. Se dice que $f$ pertenece al \textit{espacio de funciones de oscilación media acotada} $\mathrm{BMO}(\Omega; X)$ si se tiene que $f^\# \in{L^\infty}$. En ese caso se define la norma
	\begin{equation*}
		\|f\|_{\mathrm{BMO}} := \| f^\#\|_{L^\infty}.
	\end{equation*}
	En realidad, se toma el cociente respecto a funciones constantes para que la definición de norma sea adecuada.
\end{definition}
Ahora, se presenta una norma alternativa que no requiere del cálculo del valor promedio $f_Q$.
\begin{proposition}[Versión alternativa del operador maximal sharp]
	Para $f\in L^1_{loc}(\Omega;X)$, se tiene que 
	\begin{equation*}
		\frac{1}{2}\mathcal{M}_p^\# f(x) \leq \sup_{Q\ni x} \inf_{b\in X}\left(\frac{1}{|Q|} \int_Q \|f(x)-b\|_X^p\diff x\right)^{1/p} \leq \mathcal{M}_p^\# f(x). 
	\end{equation*}
\end{proposition}
\begin{proof}
	Para la cota superior basta tomar $b=f_Q$. Para la cota inferior, se tiene que
	\begin{align*}
		\left(\int_Q \|f(x) - f_Q\|_X^p\diff x\right)^{1/p} & \leq \left(\int_Q \|f(x) - b\|_X^p \diff x\right)^{1/p} + \left(\int_Q \|b- f_Q\|_X^p\diff x\right)^{1/p} \\
		& =  \left(\int_Q \|f(x) - b\|_X^p \diff x\right)^{1/p} + |Q|^{1/p}\|b-f_Q\|_X\\
		& \leq 2 \left(\int_Q \|f(x) - b\|_X^p \diff x\right)^{1/p}.
	\end{align*}
	Lo que completa la prueba.
\end{proof}
\begin{remark}
	El resultado anterior, permite entonces utilizar la definición equivalente 
	\begin{equation*}
		\mathcal{M}_p^\# f(x) := \sup_{Q\ni x} \inf_{b\in X}\left(\frac{1}{|Q|} \int_Q \|f(x)-b\|_X^p\diff x\right)^{1/p} 
	\end{equation*}
\end{remark}
El operador maximal sharp, tiene una relación interesante con el operador maximal de Hardy-Littlewood.
\begin{theorem}
	Para $f\in L^1_{loc}(\Omega;X)$, se tiene que $f^\# \leq 2\mathrm{M}f$.
\end{theorem}
\begin{proof}
	Basta con ver que 
	\begin{equation*}
		\frac{1}{|Q|}\int_Q\|f-f_Q\|_X \diff x \leq \frac{1}{|Q|} \int_Q \|f\|_X\diff x + \|f_Q\|_X \leq 2(\|f\|_X)_Q.
	\end{equation*}
	El resultado se obtien tomando el supremo.
\end{proof}
El resultado converso se tiene en norma y se debe a Fefferman y Stein \cite{fefferman-stein}.
\begin{theorem}[Fefferman-Stein]\label{theo:sharp-hardy-littlewood}
	Sea $f\in L^{p_0}(\Omega; X)$, y sea $1<p<\infty$, donde $1\leq p_0\leq p$. Suponga que $f^\#\in L^p(\Omega)$, entonces se tiene que 
	\begin{equation*}
		\|\mathrm{M}f\|_{L^p} \leq C_p \|f^\#\|_{L^p},
	\end{equation*}
	donde $\mathrm{M}$ es el operador maximal de Hardy-Littlewood, y $C_p$ solo depende de $p$.
\end{theorem}
\begin{proof}
	Se aplica la descomposición de Calderón-Zygmund (vea el Teorema \ref{theo:Calderon-Zygmund}) para obtener una suseción de cubos diádicos $\{Q_j^\lambda\}$ tales que $\|f(x)\|_X \leq \lambda$, para $x\notin\bigcup_j Q_j^\lambda$, y que 
	\begin{equation*}
		\lambda < \frac{1}{|Q_j^\lambda|} \int_{Q_j^\lambda} \|f(x)\|_X\diff x \leq 2^n\lambda.
	\end{equation*}
	Ahora, se puede ver que si $a_g(\lambda) := |\{|g|>\lambda\}|$ es la función densidad, entonces $a_{f}(\lambda) = |Q^\lambda| := \left|\bigcup_j Q_j^\lambda \right|$.Entonces, se busca probar
	\begin{equation}\label{eq:mu-lambda}
		a_{f}(\lambda) \leq a_{f^\#}(\lambda/A) + \frac{2}{A}a_{f}(2^{-n-1}\lambda),
	\end{equation}
	Para ello, se fija $Q_0 = Q_j^{\lambda2^{-n-1}}$. Si se tiene que $Q_0 \subset\{f^\#>\lambda/A\}$, entonces, es trivial que 
	\begin{equation*}
		\sum_{Q_j^\lambda \subset Q_0} |Q_j^\lambda| \leq |\{f^\# > \lambda/A\}\cap Q_0|.
	\end{equation*}
	En caso contrario, por definición se tiene que 
	\begin{equation*}
		\frac{1}{|Q_0|} \int_{Q_0} \|f - f_{Q_0}\|_X \diff x \leq \frac{\lambda}{A}.
	\end{equation*}
	Por otra parte, $\|f_{Q_0}\|_X \leq 2^n(2^{-n-1}\lambda) = \lambda/2$, y $(\|f\|_X)_{Q_j^\lambda} \leq \lambda$. Entonces, se tiene que 
	\begin{equation*}
		\int_{Q_j^\lambda} \|f - f_{Q_0}\|_X \diff x \geq \frac{\lambda}{2} |Q_j^\lambda|,
	\end{equation*}
	donde $Q_j^\lambda\subset Q_0$. Entonces, se tiene que 
	\begin{equation*}
		\sum_{Q_j^\lambda\subset Q_0} |Q_j^\lambda| \leq \frac{2}{A}|Q_0|.
	\end{equation*}
	Se obtiene (\ref{eq:mu-lambda}) al sumar sobre todos los cubos $Q_0$. Ahora, note que si $x\in Q_j^\lambda$, entonces 
	\begin{equation*}
		\mathrm{M}f(x) \geq \frac{1}{|Q_j^\lambda|} \int_{Q_j^\lambda} \|f\|_X\diff x > \lambda,
	\end{equation*}
	y $a_{f}(\lambda) \leq a_{\mathrm{M}f}(\lambda)$. Además, se define $2Q_j^\lambda$ como el cubo concentrico a $Q_j^\lambda$, tal que $\ell(2Q_j^\lambda) = 2\ell(Q_j^\lambda)$, y se fija $x\notin\bigcup 2Q_j^\lambda$. Sea $Q$ cualquier cubo que contenga a $x$, entonces
	\begin{equation*}
		\int_Q\|f(y)\|_X\diff y = \int_{Q\cap Q^\lambda} \|f(y)\|_X\diff y + \int_{Q\cap (\Omega\setminus Q^\lambda)} \|f(y)\|_X\diff y =: I_1 + I_2.
	\end{equation*}
	Note que $I_2 \leq \lambda |Q|$. Para $I_1$, vea que si $Q\cap Q_j^\lambda$ es no vacío, entonces $Q_j^\lambda \subset 4Q$, dado que $Q \not\subset 2Q_j^\lambda$ implica $\ell(Q) > \ell(Q_j^\lambda)$. Por lo que 
	\begin{equation*}
		I_1 \leq \sum_{Q_j^\lambda\subset4Q} \int_{Q_j^\lambda} \|f(y)\|_X\diff y \leq \sum_{Q_j^\lambda\subset4Q} 2^n\lambda |Q_j^\lambda| \leq 2^n4^n\lambda|Q|.
	\end{equation*}
	Entonces, se puede concluir que 
	\begin{equation}\label{eq:mu-density}
		2^{-n}a_{\mathrm{M}f}[(1+2^n4^n)\lambda] \leq a_{f}(\lambda) \leq a_{\mathrm{M}f}(\lambda).
	\end{equation}
	Por otra parte, se define 
	\begin{equation*}
		I_N := p\int_0^N\lambda^{p-1}a_{f}(\lambda)\diff \lambda \leq p\int_0^N\lambda^{p-1}a_{Mf}(\lambda)\diff \lambda.
	\end{equation*}
	También, se sabe que si $p_0>1$, entonces $I_N\leq \|Mf\|_{L^{p_0}}$, y que si $p_0=1$, entonces $a_{Mf}(\lambda)\lesssim \lambda^{-1}$ (vea el Teorema \ref{theo:Hardy-Littlewood}). En general, se tiene que $I_N<\infty$ como $f\in L^{p_0}$. Por (\ref{eq:mu-lambda}) se tiene que 
	\begin{align*}
		I_N & \leq p\int_0^N\lambda^{p-1}a_{f^\#}(\lambda/A)\diff\lambda + \frac{2}{A}p\int_0^N\lambda^{p-1}a_{f}(2^{-n-1}\lambda)\diff \lambda \\
		& \leq A^p p\int_0^N\lambda^{p-1}a_{f^\#}(\lambda)\diff\lambda +\frac{2}{A} p2^{(n+1)p} \int_0^{N2^{-n-1}}\lambda^{p-1} a_{f}(\lambda)\diff\lambda \\
		& \leq Ap\|f^\#\|_{L^p} + \frac{2}{A}2^{(n+1)p} I_N.
	\end{align*}
	Por lo que, si se escoge $A=4\cdot2^{(n+1)p}$, se tiene que $I_N \leq 4\cdot 2^{(n+1)p}\|f^\#\|_{L^p}$. Finalmente, se concluye que 
	\begin{align*}
		\|Mf\|_{L^p} & = p\int_0^N\lambda^{p-1}a_{Mf}(\lambda)\diff \lambda \\
		& \leq 2^n(1+2^n4^n)^{-p} p\int_0^\infty\lambda^{p-1}a_{f}(\lambda)\diff \lambda \\
		& =2^n(1+2^n4^n)^{-p}  \lim_{N\to\infty} I_N \leq C_p\|f^\#\|_{L^p},
	\end{align*}
	con $C_p = 2^n(1+2^n4^n)^{-p} 4\cdot 2^{(n+1)p}$. Se puede ver que la constante solo depende de $n$ y $p$, concluyendo la prueba.
\end{proof}
A continuación se presenta el resultado clásico de Fefferman \cite{fefferman-BMO}, y la demostración se guía de la presentada en su trabajo con Stein \cite{fefferman-stein}.
\begin{theorem}[Fefferman]\label{theo:H1-BMO}
	Suponga $X'$ satisface la propiedad de Radon-Nikodym. Entonces, el dual de $H^1(\Omega;X)$ es $\mathrm{BMO}(\Omega;X')$, con $\Omega=\R^n,\T^n$. Esto puede entenderse de la siguiente manera 
	\begin{enumerate}
		\item Para $\varphi \in \mathrm{BMO}$, el funcional $f\mapsto\int_{\Omega}{\varphi(x) f(x)}\diff x$ es acotado en $H^1$.
		\item Para cualquier funcional continuo en $H^1$, se comporta como el funcional en (1) para una única función $\varphi \in \mathrm{BMO}$.
	\end{enumerate}
\end{theorem}
Para ello, se enuncia este teorema auxiliar.
\begin{theorem}\label{theo:BMO-aux}
	Para $\Omega=\R^n,\T^n$, los siguientes enunciados son equivalentes
	\begin{enumerate}
		\item $\varphi$ pertenece a $\mathrm{BMO}(\Omega;X)$,
		\item $\varphi = \varphi_0 + \sum_{j=1}^n R_j\varphi_j$, donde $\varphi_0,\ldots,\varphi_n \in L^\infty(\Omega;X)$,
		\item se tiene que
		\begin{equation*}
			\int_{\Omega}\frac{\|\varphi(x)\|_{X}\diff x}{1+|x|^{n+1}} < \infty, 
		\end{equation*}
		y que 
		\begin{equation*}
			\sup_{x_0\in\Omega}\int_{T(x_0, h)} t\|\nabla (P\varphi)(x, t)\|_{X}^2\diff x \diff t \lesssim h^n,
		\end{equation*}
		donde $0<h<\infty$, $T(x_0, h) := \{(x, t): 0<t<h, \, |x-x_0| < h\}$, y $P\varphi$ su integral de Poisson.
	\end{enumerate}
\end{theorem}
Los dos teoremas anteriores se demuestran en conjunto de la siguiente manera: Teorema \ref{theo:H1-BMO} $\Rightarrow$ Teorema \ref{theo:BMO-aux}(2) $\Rightarrow$ (1) $\Rightarrow$ (3) $\Rightarrow$ Teorema \ref{theo:H1-BMO}.
\begin{proof}
	(Teorema \ref{theo:H1-BMO} $\Rightarrow$ 2) Sea $B:=\bigoplus^{n+1} L^1(X)$ el espacio de Banach con la norma dada por $\|(f_0,\ldots,f_n)\| := \sum\|f_j\|_{L^1}$. Entonces, sea $S$ el subespacio en el que $f_j = R_jf_0$, que claramente es cerrado y que es isométrico a $H^1$ mediante el mapa $f_0\mapsto(f_0, R_1f_0, \ldots, R_nf_0 )$. Entonces, por el Teorema Hahn-Banach, cada funcional continuo definido en $H^1$, puede ser extendido a un funcional continuo en $B$, cuyo dual es $\bigoplus^{n+1}L^\infty(X')$. Es decir, para cada funcional $T$ continuo en $H^1$, existen $\varphi_0,\ldots,\varphi_n\in L^\infty(X')$, tales que
	\begin{equation*}
		T(f)  = \int_{\Omega} {\varphi_0 f}\diff x + \sum_{j=1}^n \int_{\Omega} \varphi_j(R_jf) \diff x.
	\end{equation*}
	Por otra parte, se tiene que por la transformada de Fourier cuando $\Omega=\T^n$,
	\begin{align*}
		\int_{\T^n}\varphi_j(R_jf)\diff x & = \int_{\T^n}\varphi_j \left[\sum_{\xi\in\Z^n}\int_{\T^n}  \frac{i\xi_j}{|\xi|} f(y)e^{2\pi i (x-y)\cdot\xi} \diff y 
		\right]  \diff x \\
		& = \int_{\T^n}  \left[\sum_{\xi\in\Z^n}\int_{\T^n}  \frac{i\xi_j}{|\xi|}  \varphi_j(x)e^{2\pi i (x-y)\cdot\xi} \diff x 
		\right] f(y) \diff y \\
		& = - \int_{\T^n}  R_j\varphi_j(y)f(y) \diff y.
	\end{align*}
	El caso $\Omega=\R^n$ es análogo. Por lo que cada funcional continuo $T$ se puede escribir como 
	\begin{equation*}
		T(f) = \int_{\Omega}  \left[ \varphi_0 - \sum_{j=1}^n R_j\varphi_j
		\right] f \diff x.
	\end{equation*}
	(2 $\Rightarrow$ 1) En vista del inciso anterior, basta con demostrar que para cualquier $\varphi\in L^\infty$, se tiene que $R_j\varphi \in \mathrm{BMO}$. Para un cubo $Q$ de diametro $\sigma$, sea $\chi$, la función indicadora para puntos $|x-x_Q|\leq2\sigma$, para descomponer 
	\begin{equation*}
		\varphi = \varphi\chi + \varphi(1-\chi) =: \varphi_1 + \varphi_2. 
	\end{equation*}
	Entonces, se tiene que $\widehat{R_j\varphi_1}(\xi) = i\xi_j\widehat{\varphi_1}(\xi)/|\xi|$ y por la desigualdad de Hölder e identidad de Plancherel
	\begin{equation}\label{eq:Rj-L-infty}
		\frac{1}{|Q|}\int_Q \|R_j\varphi_1\|_X\diff x \leq |Q|^{-1/2} \|R_j\varphi_1\|_{L^2} \leq|Q|^{-1/2} \|\widehat{\varphi_1}\|_{L^2}  \lesssim \|\varphi\|_{L^\infty}.
	\end{equation}
	Por otra parte, se puede verificar que la transformada de Riesz, equivale a la siguiente convolución
	\begin{equation*}
		(R_jf)(x) = (K_j*f)(x) := \int_{\Omega} \frac{x_j-y_j}{|x-y|^{n+1}} f(y) \diff y.
	\end{equation*}
	Ahora, por el Teorema del valor medio, se tiene que 
	\begin{equation*}
		|K_j(x-y)-K_j(x_Q-y)| \leq |x-x_Q| \sup_z |\nabla K_j(z)|,
	\end{equation*}
	donde $z$ se encuentra entre $x-y$, y $x_Q - y$. Además, se tiene que $|\nabla K_j(z)| \lesssim |z|^{n+1}$. Ahora, para $y$ en el soporte de $\varphi_2$, y $x\in Q$, se tiene que 
	\begin{equation*}
		|y-x| \geq |y-x_Q| - |x_Q-x| \geq |x_Q - x|.
	\end{equation*}
	Por lo que $|z| \gtrsim |x_Q - y|$ y se tiene que 
	\begin{equation}\label{eq:Rj-diff-L-infty}
		\|(R_j\varphi_2)(x) - (R_j\varphi_2)(x_Q)\|_X \lesssim \sigma \int_{|x_Q-y|>2\sigma} \frac{\|\varphi\|_{L^\infty}}{|x_Q - y|^{n+1}}\diff y \lesssim \|\varphi\|_{L^\infty}. 
	\end{equation}
	Escogiendo $b=(R_j\varphi_2)(x_Q)$ en la norma equivalente en $\mathrm{BMO}$ se tiene que 
	\begin{equation*}
		\|R_j\varphi\|_{\mathrm{BMO}} \lesssim \sup_Q \frac{1}{|Q|} \int_Q \|(R_j\varphi_2)(x) - (R_j\varphi_2)(x_Q)\|_X + \|R_j\varphi_1(x)\|_X \diff x \lesssim \|\varphi\|_{L^\infty}.
	\end{equation*}
	(1 $\Rightarrow$ 3) Sea $Q$ el cubo de lado uno centrado en el origen, y $Q_k$ el cubo de lado $2^k$ con el mismo centro. Entonces, se tiene que
	\begin{equation*}
		\left\| \int_{Q_{k-1}}[\varphi(x) - \varphi_{Q_k}] \diff x
		\right\|_X \leq \int_{Q_k} \|\varphi(x) - \varphi_{Q_k}\|_X \diff x \leq 2^{nk}\|\varphi\|_{\mathrm{BMO}}.
	\end{equation*}
	Por lo que $\|\varphi_{Q_{k-1}} - \varphi_{Q_k}\|_X \leq 2^n\|\varphi\|_{\mathrm{BMO}}$, y se tiene que
	\begin{equation*}
		\int_{Q_k} \|\varphi(x)-\varphi_Q\|_X \diff x \leq 2^{nk}[1+2^nk] \|\varphi\|_{\mathrm{BMO}}.
	\end{equation*}
	Particularmente, se obtiene que
	\begin{align*}
		\int_{Q_{k+1}\setminus Q_k} \frac{\|\varphi(x)-\varphi_Q\|_X}{1+|x|^{n+1}} \diff x & \lesssim \int_{Q_{k+1}\setminus Q_k} \frac{\|\varphi(x)-\varphi_Q\|_X}{1+2^{k(n+1)}} \diff x \\
		& \leq \frac{2^{nk+n} [1+2^n(k+1)]}{1+2^{kn+k}}\|\varphi\|_{\mathrm{BMO}}.
	\end{align*}
	Por lo que sumar todas las expresiones de este estilo resulta en 
	\begin{equation*}
		\int_{\Omega} \frac{\|\varphi(x)-\varphi_Q\|_X}{1+|x|^{n+1}} \diff x \lesssim \|\varphi\|_{\mathrm{BMO}} < \infty.
	\end{equation*}
	Para el segundo estimativo, fije $x_0=0$, y sea $Q$ el cubo centrado en el origen, de lado $4h$. Sea $\chi$ su función característica y $\tilde{\chi}$ la de su complemento. Entonces, se descompone a $\varphi$ de la siguiente manera
	\begin{equation*}
		\varphi = \varphi_Q + (\varphi - \varphi_Q)\chi + (\varphi - \varphi_Q)\tilde{\chi} =: \varphi_1 + \varphi_2 + \varphi_3.
	\end{equation*}
	Primero, $\varphi_1$ es constante y no aporta a la norma del gradiente. Por la identidad de Plancherel se obtiene que 
	\begin{align*}
		\int_{T(0, h)} t\|\nabla (P\varphi_2)(x, t)\|_X^2 \diff x \diff t & \leq \int_0^\infty \int_{\Omega} t\|\nabla (P\varphi_2)(x, t)\|_X^2 \diff x \diff t  \\
		& = \int_0^\infty 8\left\|\widehat{\varphi_2}(\xi) \pi|\xi|  te^{-2\pi t|\xi|}\right\|_{L^2}^2 \diff t \\
		& = \frac{1}{2} \|\widehat{\varphi_2}\|_{L^2}^2 \\
		& = \int_Q \|\varphi(x) - \varphi_Q\|_X^2 \diff x \lesssim h^n \|\varphi\|_{\mathrm{BMO}}^2,
	\end{align*}
	por los cálculos realizados en el inciso anterior. Por otra parte, sigue que
	\begin{align*}
		\|\nabla(P\varphi_3)(x, t)\|_X & \leq \int_{\Omega} |\nabla P_t(x-y)|\|\varphi_3(y)\|_X\diff y \\
		& \lesssim \int_{\Omega\setminus Q} \left[ \frac{1}{t+|x-y|}
		\right]^{n+1} \|\varphi(y) - \varphi_Q\|_X\diff y.
	\end{align*}
	Pero, para $(x, t) \in T(0, h)$, $y \in \supp \varphi_3$ se tiene que
	\begin{equation*}
		\left[ \frac{1}{t+|x-y|}
		\right]^{n+1} \lesssim \frac{1}{h^{n+1} + |y|^{n+1}},
	\end{equation*}
	y que $\|\nabla (P\varphi_3)(x, t)\|_X \lesssim h^{-1}\|\varphi\|_{\mathrm{BMO}}$. Finalmente se concluye que en realidad
	\begin{equation*}
		\sup_{x_0\in\Omega}\int_{T(x_0, h)} t\|\nabla (P\varphi)(x, t)\|_X^2\diff x \diff t \lesssim h^n\|\varphi\|_{\mathrm{BMO}}.
	\end{equation*}
	(3 $\Rightarrow$ Teorema \ref{theo:H1-BMO}) En vista del Teorema \ref{theo:prop-Poisson}, basta demostrar que para $f\in H^1(X)$ apropiado se tiene que
	\begin{equation*}
		\left| \int_0^\infty \int_{\Omega} t[\nabla(P\varphi)][\nabla(Pf)]\diff x \diff t
		\right| \lesssim \|f\|_{H^1}.
	\end{equation*}
	Además, en vista del Teorema \ref{theo:H1-F} se tiene que existe $F=(u_0,\ldots, u_n)$, que satisfacen las ecuaciones de Cauchy-Riemann generalizadas, tal que $u_0(x, 0)=f(x)$, y que gracias a un argumento de densidad, decae rápidamente en el infinito. También, se puede requerir sin pérdida de generalidad que $|F|>0$, y que $\Delta(|F|) = O(|x|+t+1)^{-n-\delta}$ en su dominio. Entonces, el lado izquierdo de la expresión anterior puede ser acotado por 
	\begin{align*}
		\int_0^\infty \int_{\Omega} t\|\nabla(P\varphi)\|_{X'}\|\nabla(Pf)\|_X\diff x \diff t & \leq\int_0^\infty \int_{\Omega}  \|\nabla(P\varphi)\|_{X'}|\nabla F|\diff x\diff t
	\end{align*}
	\begin{equation*}
		\leq \left( \int_0^\infty \int_{\Omega} t\|\nabla(P\varphi)\|_{X'}^2|F| \diff x \diff t
		\right)^{1/2}\left(\int_0^\infty \int_{\Omega}  t|F|^{-1}|\nabla F|^2 \diff x \diff t
		\right)^{1/2}
	\end{equation*}
	Ahora, por el Lema \ref{lem:subharmonic-gradient} se tiene que
	\begin{align*}
		\int_0^\infty\int_{\Omega} t|F|^{-1}|\nabla F|^2 \diff x \diff t & \lesssim
		\int_0^\infty\int_{\Omega} t\Delta(|F(x, t)|) \diff x \diff t \\
		& = \int_{\Omega} |F(x, 0)| \diff x \lesssim \|f\|_{H^1},
	\end{align*}
	por el Teorema de Green. Por otra parte, se define $g(x):=|F(x, 0)|^q$, con $q=(n-1)/n$. Entonces, por el Lema \ref{lem:subharmonic-gradient}, se tiene que $\Delta(|F|^q - Pg)\geq 0$ y como $Pg = |F|^q$ en la frontera, el principio maximal implica $|F(x, t)| \leq [(Pg)(x, t)]^p$, donde $p=1/q>1$. En particular
	\begin{equation*}
		\|g\|_{L^p}^p \leq \int_{\Omega}|F(x, 0)|\diff x \lesssim \|f\|_{H^1}.
	\end{equation*}
	Entonces, el término restante puede ser estimado utilizando el Lema  \ref{lem:Carleson} con $\diff\mu := t\|\nabla(P\varphi)\|_{X'}^2 \diff x \diff t$. En efecto,
	\begin{equation*}
		\int_0^\infty \int_{\Omega} t\|\nabla(P\varphi)\|_{X'}^2|F| \diff x \diff t \leq \int_0^\infty \int_{\Omega} t\|\nabla(P\varphi)\|_{X'}^2|Pg|^p \diff x \diff t \lesssim \|g\|_{L^p}^p \lesssim \|f\|_{H^1}.
	\end{equation*}
	Lo que concluye la prueba.
\end{proof} 
\begin{remark}
	En esta demostración se demostró en (1 $\Rightarrow$ 3) una propiedad de la conocidafunción-$g$ de Littlewood-Paley, que se define como 
	\begin{equation*}
		g(f)(x) := \left(\int_0^\infty \|\nabla Pf(x, t)\|_X^2 \diff t\right)^{1/2}.
	\end{equation*}
	En particular, se demostro que $\|g\|_{L^2} \sim \|f\|_{L^2}$.
\end{remark}
Para concluir, se presenta una definición alternativa del espacio de Hardy $H^1$. Aquí, se adapta el caso $p$ de la prueba general presentada por Stein \cite{stein}.
\begin{definition}[Espacios de Hardy atómicos $H^{p,q}_{at}$]
	Sea $p\leq 1$. Se le llama $(p,q)$-\textit{átomo} a una función $a:\Omega\to X$, con $\Omega=\R^n,\T^n$, soportada en una bola $B$, que cumple que 
	\begin{equation*}
		\|a\|_{L^q} \leq |B|^{1/q-1/p}, \quad \text{ y } \quad \int_{\Omega}x^\beta a(x) \diff x = 0,
	\end{equation*}
	con $0\leq|\beta|\leq n(1/p-1)$.
	Además, se dice que $f\in H^{p,q}_{at}(\Omega;X)$ si existe una \textit{descomposición atómica}, es decir una sucesión de átomos $\{a_j\}$ que satisfaga
	\begin{equation*}
		f = \sum_j \lambda_j a_j, \quad \sum_j |\lambda_j|^p < \infty.
	\end{equation*}
	También se define la norma 
	\begin{equation*}
		\|f\|_{H^{p,q}_{at}} := \inf \left\{\left(\sum_j |\lambda_j|^p\right)^{1/p} : f=\sum_j\lambda_j a_j \right\}
	\end{equation*}
\end{definition}
\begin{theorem}
	Se tiene que $H^1(\Omega;X) = H^{1,\infty}_{at}(\Omega;X)$, con equivalencia de norma.
\end{theorem}
\begin{proof}
	($H^1 \hookrightarrow H^{1,\infty}_{at}$) Para esta dirección, basta con ver que los átomos tienen norma $H^1$ uniforme. Primero, por (\ref{eq:Rj-L-infty}), se tiene que para un $(1,\infty)$-átomo $a$ relacionado a una bola $B(z, \sigma)$
	\begin{equation*}
		\int_{B(z, 2\sigma)}\|R_j a(x) \|_X\diff x \lesssim |B(z, \sigma)|\|a\|_{L^\infty} \leq C.
	\end{equation*}
	Ahora, por (\ref{eq:Rj-diff-L-infty}), se tiene que para $|x-z|>2\sigma$, 
	\begin{equation*}
		\int_{\Omega\setminus B(z, 2\sigma)} |K_j(y-x) - K_j(z-x)|\diff x \leq C.
	\end{equation*}
	Además, por la propiedad de cancelación
	\begin{align*}
		\int_{\Omega\setminus B(z, 2\sigma)} \|R_ja\|_X\diff x & = \int_{\Omega\setminus B(z, 2\sigma)} \left\| \int_{B(z, \sigma)} a(y)  \left[ K_j(x - y)- K_j(x - z) \right] \diff y
		\right\|_X \diff x \\
		& \leq  \int_{B(z, \sigma)}\|a(y)\|_X \int_{\Omega\setminus B(z, 2\sigma)} |K_j(x - y)- K_j(x - z)| \diff x \diff y \\
		& \leq C \int_{B(z, \sigma)}\|a(y)\|_X \diff y \leq C.
	\end{align*}
	En conclusión $\|R_j a\|_{L^1} \leq C$, y esto permite asegurar la convergencia en $H^1$ de la serie de átomos, y que  para cualquier $\varepsilon>0$
	\begin{equation*}
		\|f\|_{H^1}  \leq \sum_j |\lambda_j| \|a_j\|_{H^1} \lesssim \sum_j |\lambda_j| \leq \|f\|_{H^1_{at}} + \varepsilon.
	\end{equation*}
	($H^{1,\infty}_{at}\hookrightarrow H^1$) Para la construcción de los átomos,  se aplica la descomposición de Calderón-Zygmund a $\mathcal{M}f:= \|f\|_X + \sum \|R_jf\|_X$, al nivel $\lambda=2^m$, para obtener una suseción de cubos diádicos $\{Q_k^m\}$ tales que 
	\begin{equation*}
		2^m < \frac{1}{|Q_k^m|}\int_{Q_k^m} \mathcal{M}f\diff x \leq 2^{n+m},
	\end{equation*}
	\begin{equation*}
		|Q^m| := \left|\bigcup_k Q_k^m\right| \lesssim 2^{-m}\|\mathcal{M}f\| _{L^1}= 2^{-m}\|f\|_{H^1},
	\end{equation*}
	y si $x\notin Q^m$, entonces $\mathcal{M}f(x) \leq 2^m$ casi en todas partes. Entonces, se definen 
	\begin{equation*}
		g^m := f(1-\chi_{Q^m}) + \sum_k f_{\frac{1}{2}Q_k^m}\chi_{\frac{1}{2}Q_k^m}, \quad b^m := \sum_k b_k^m := \sum_k (f-f_{\frac{1}{2}Q_k^m})\chi_{\frac{1}{2}Q_k^m}.
	\end{equation*}
	Por lo que, para $x\in Q_k^m$, se tiene que 
	\begin{align*}
		\|R_jb_k^m\|_X & \leq \left\| \int_{Q_k^m} K_j(x-y)f(y) \diff y \right\|_X +\left\| \int_{Q_k^m} K_j(x-y)f_{\frac{1}{2}Q_k^m} \diff y \right\|_X\\
		& \leq \|R_jf\|_X +\left\| \int_{Q_k^m} K_j(x-y)\frac{2^n}{|Q_k^m|}\int_{\frac{1}{2}Q_k^m}f(x)\diff x \diff y \right\|_X \\
		& \leq \|R_jf\|_X +\frac{2^n}{|Q_k^m|} \int_{Q_k^m} \|R_jf(y)\|_X \diff y  \\
		& \lesssim \|R_jf\|_X  + (\|R_jf\|_X)_{Q_k^m}| \lesssim \|R_jf\|_X + 2^{n+m} \lesssim \mathcal{M}f.
	\end{align*}
	Ahora, cuando $x\notin Q_k^m$, se aprovecha el hecho que $\int b_k^m = 0$, para tener que 
	\begin{align*}
		R_jb_k^m&=  \int_{\Omega} b_k^m(y)[K_j(x-y) - K_j(x-x_{Q_k^m} )] \diff y.
	\end{align*}
	Aquí, $x_{Q_k^m}$ es el centro del cubo. Por un argumento similar al de (\ref{eq:Rj-diff-L-infty}), se tiene que 
	\begin{equation*}
		| R_jb_k^m |\lesssim \int_{Q_k^m} \frac{\ell(Q_k^m)\|f(y) - f_{\frac{1}{2}Q_k^m}\|_X}{|x-x_{Q_k^m}|^{n+1}} \diff y \leq 2(\|f\|_X)_{\frac{1}{2}Q_k^m} \frac{\ell(Q_k^m)}{|x-x_{Q_k^m}|^{n+1}}|Q_k^m| \lesssim 2^{n+m}\frac{\ell(Q_k^m)^{n+1}}{|x-x_{Q_k^m}|^{n+1}}.
	\end{equation*}
	Por lo que 
	\begin{align*}
		\int_{\Omega}\mathcal{M}{b_k^m} \diff x &= \int_{Q_k^m}\mathcal{M}{b_k^m} \diff x + \int_{\Omega\setminus Q_k^m}\mathcal{M}{b_k^m} \diff x\\
		& \leq \int_{Q_k^m}\mathcal{M}f \diff x + \int_{|x-x_{Q_k^m}|>\ell(Q_k^m)} 2^{m}\frac{\ell(Q_k^m)^{n+1}}{|x-x_{Q_k^m}|^{n+1}} \diff x \\
		& \lesssim  \int_{Q_k^m}\mathcal{M}f \diff x + 2^m\ell(Q_k^m)^n\\
		 & \lesssim  \int_{Q_k^m}\mathcal{M}f \diff x + 2^m|Q_k^m| \lesssim  \int_{Q_k^m}\mathcal{M}f \diff x .
	\end{align*}
	Por lo que se tiene que 
	\begin{align*}
		\|b^m\|_{H^1} & \leq \sum_k	\int_{\Omega}\mathcal{M}{b_k^m} \diff x \leq \sum_k \int_{Q_k^m} \mathcal{M}f\diff x \leq \int_{Q^m}\mathcal{M}f\diff x \to 0,
	\end{align*}
	cuando se hace $m\to\infty$, dado que $Q^m = \{\mathcal{M}f>2^m\}$. Equivalentemente, $\|f-g^m\|_{H^1} \to 0$. Además, como $g^m\to0$ cuando $m\to-\infty$, se tiene que 
	\begin{equation*}
		f = \sum_{m\in\Z} (g^{m+1}-g^m) = \sum_{m\in\Z}\sum_k (g^{m+1}-g^m)\chi_{Q_k^m},
	\end{equation*}
	como $g^{m+1}-g^m = b^m - b^{m+1}$, que se encuentra soportado en $Q^m$. Ahora, se define $B_k^m$ como la menor bola que contiene al cubo $Q_k^m$, y como $\|g^{m+1}-g^{m}\|_X \leq C2^{m}$, se definen
	\begin{equation*}
		\lambda_k^m := C 2^m |B_k^m|, \quad a_k^m := (\lambda_k^m)^{-1} \left(b_k^m - \sum_\ell b_\ell^{m+1}\right),
	\end{equation*}
	donde los $\ell$ son aquellos tales que $Q_\ell^{m+1}\subset Q_k^m$. Esto está bien definido dado que son cubos diádicos, y que $\sum_k \lambda_k^ma_k^m = g^{m+1}-g^m$. Asimismo, se tiene que $\|a_k^m\|_{L^\infty} \leq |B_k^m|^{-1}$. Se puede ver que estos átomos también cumplen la propiedad de cancelación
	\begin{equation*}
		\int_{\Omega}a_k^m \diff x = \int_{\Omega} b_k^m \diff x -\sum_\ell  \int_{\Omega} b^{m+1}_\ell\diff x = 0.
	\end{equation*}
	Además, se tiene que los coeficientes cumplen que 
	\begin{equation*}
		\sum_{k,m} |\lambda_k^m|  \lesssim \sum_{k,m} 2^m|B_k^m| \lesssim \sum_m 2^m|\{\mathcal{M}f>2^m\}|  \lesssim \| \mathcal{M}f\|_{L^1} = \|f\|_{H^1}.
	\end{equation*}
	Por lo que, se obtiene el resultado deseado.
\end{proof}
\begin{theorem}
	Se tiene que $H^{p,q}_{at}(\Omega;X) = H^{p,r}_{at}(\Omega;X)$, incluso cuando $q\neq r$.
\end{theorem}
\begin{proof}
	Sin pérdida de generalidad suponga que $q<r$, entonces para cualquier $(p,r)$-átomo $a$, se tiene que por la desigualdad de Hölder
	\begin{equation*}
		\|a\|_{L^q}=\left(\int_B \|a\|_X^q \diff x\right)^{1/q} \leq \left\|\|a\|_X^q\right\|_{L^{r/q}}^{1/q} \|\chi_B\|_{L^{r/(r-q)}}^{1/q} = \|a\|_{L^r} |B|^{1/q-1/r} \leq |B|^{1/q-1/p}. 
	\end{equation*}
	Por lo mismo, se tiene que $H^{p,\infty}\hookrightarrow H^{p,r}\hookrightarrow H^{p,q}\hookrightarrow H^{p,1}$, y basta demostrar que $H^{p,1}\hookrightarrow H^{p,\infty}$. Entonces, tome un $(p,1)$-átomo $a$, soportado en $B$ y apliquele la descomposición de Calderón-Zigmund a $\|a\|_X$, en los niveles $\lambda=2^m$, para obtener, como en la prueba anterior, sucesiones de cubos $\{Q_k^m\}$, y de funciones
	\begin{equation*}
		g^m := a + \sum_k c^m_k\chi_{Q_k^m}, \quad b^m = \sum_k b_k^m := \sum_k [a-c_k^m]\chi_{Q_k^m},
	\end{equation*}
	 donde las $c_k^m$ son funciones tales que para la bola $B_k^m = B(z, \sigma)$, se tiene que
	\begin{equation*}
		c_k^m := \sum_{|\beta|\leq s} c_{k\beta}^m \frac{(x-z)^\beta}{\beta!}\chi_{B_k^m},
	\end{equation*}
	donde $s=n(1/p-1)$ y los $c_{k\beta}^m$ se escogen de tal manera que satisfaga las condiciones de momentos. Es decir 
	\begin{equation*}
		I_{k\alpha}^m := \int_{B_k^m} (x-z)^\alpha a \diff x = \sum_{|\beta|\leq s} c_{k\beta}^m\int_{B_k^m} (x-z)^\alpha \frac{(x-z)^\beta}{\beta!} \diff x,
	\end{equation*}
	para cualquier $|\alpha|\leq s$. Estas integrales se pueden estimar como
	\begin{equation*}
		\|I_{k\alpha}^m\|_X \leq  \left\|\int_{B_k^m} \sigma^{|\alpha|} a \diff x\right\|_X \leq 2^n 2^m \sigma^{|\alpha|}|{B_k^m}|.
	\end{equation*}
	Ahora, tome la matriz de de momentos para la bola $B_k^m$,
	\begin{equation*}
		M({B_k^m})_{\alpha\beta} := \int_{B_k^m} (x-z)^\alpha \frac{(x-z)^\beta}{\beta!} \diff x,
	\end{equation*}
	Se puede ver que esta puede expresarse como $\sigma^{|\alpha+\beta|+n}M_{\alpha\beta}$, en terminos de la matriz de momentos de la bola unitaria. En particular, como $D_\sigma MD_\sigma \sigma^{n}$, donde $D_\sigma$ es diagonal con entradas $(D_\sigma)_{\alpha\alpha} = \sigma^{|\alpha|}$. Entonces, al invertir, se obtiene que 
	\begin{equation*}
		M({B_k^m})^{-1} = D_{1/\sigma}M^{-1}D_{1/\sigma}\sigma^{-n}.
	\end{equation*}
	y se pueden estimar como 
	\begin{equation*}
		|(M({B_k^m})^{-1})_{\alpha\beta} | \leq |(M^{-1})_{\alpha\beta}| \sigma^{-|\alpha+\beta|-n} \leq C_{ns}\sigma^{-|\alpha+\beta|-n}.
	\end{equation*}
	Por lo que 
	\begin{equation*}
		\|c_{k\beta}^m\|_X\leq \sum_{|\alpha|\leq s} C_{ns}\sigma^{-|\alpha+\beta|-n} \cdot 3\cdot2^m\sigma^{|\alpha|}|{B_k^m}| \leq C_{ns}' 2^m\sigma^{-|\beta|}.
	\end{equation*}
	Esto implica que $\|b_k^m\|_{L^\infty} \leq C_{ns}'' 2^m$. Por lo que se definen 
	\begin{equation*}
		\tilde{a}_k^m := b_k^m - \sum_\ell b_\ell^{m+1}, 
	\end{equation*}
	donde los $\ell$ son tales que $Q_\ell^{m+1} \subset Q_k^m$, los cuales están bien definidos dado que son diádicos. Además, se tiene que 
	\begin{equation*}
		 \sum_k \tilde{a}_k^m = b^m - b^{m+1} = g^{m+1 }- g^m.
	\end{equation*} 
	Entonces, se tiene que $\|\tilde{a}_k^m\|_{L^\infty} \leq \|g^{m+1 }- g^m\|_{L^\infty} \leq C 2^m$ y se pueden definir los átomos como
	\begin{equation*}
		\lambda_k^m := C2^m|B_k^m|^{1/p}, \quad a_k^m := (\lambda_k^m)^{-1}\tilde{a}_k^m.
	\end{equation*} 
	Se tiene que los coeficientes cumplen 
	\begin{equation*}
		\sum_{k,m}|\lambda_k^m|^p \lesssim \sum_{k,m} 2^{mp}|Q_k^m| = \sum_m 2^{mp}|\{\|a\|_X > 2^m\}|\lesssim \int_B \|a\|_X^p \diff x.
	\end{equation*}
	Ahora, como $p\leq 1$, se tiene que por la desigualdad de Jensen
	\begin{equation*}
		\int_B \|a\|_X^p \diff x = |B|\int_B \|a\|_X^p \frac{\diff x}{|B|} \leq |B| \left(\int_B \|a\|_X \frac{\diff x}{|B|} \right)^p \leq 1.
	\end{equation*}
	Lo que completa la prueba.
\end{proof}
\begin{remark}
	Por lo tanto, se suele definir el espacio $H^p_{at}(\Omega;X)$ como el espacio $H^{p,q}(\Omega;X)$, para todo $1\leq q \leq \infty$. Generalmente se denotan como el espacio $H^p$, que también puede ser definido en términos de operadores maximales, vea Stein \cite{stein}.
\end{remark}
\section{Interpolación entre espacios}
Anteriormente, se han aprovechado resultados de interpolación, que permiten demostrar las propiedades para un par de espacios y como consecuencia obtener los resultados para los espacios 'intermedios'. Ahora, se considera un caso muy importante para el análisis armónico y se presenta la teoría de interpolación compleja.
\begin{definition}[Parejas de espacios]
	Sean $A_0$, y $A_1$ espacios de Banach. Se dice que estos son \textit{compatibles} si existe un espacio topológico de Hausdorff que los contenga como subespacios, y se denota a la \textit{pareja de espacios} como $\overline{A}:=(A_0, A_1)$. Ahora, se define el \textit{espacio intersección} $\Delta(\overline{A}):=A_0\cap A_1$, con la norma 
	\begin{equation*}
		\|a\|_{\Delta(\overline{A})} := \max \{\|a\|_{A_0}, \|a\|_{A_1}\}.
	\end{equation*} 
	Además, se define el \textit{espacio suma} $\Sigma(\overline{A}):=A_0+A_1$, con la norma
	\begin{equation*}
		\|a\|_{\Sigma(\overline{A})} := \inf_{a=a_0+a_1} (\|a_0\|_{A_0}+\|a_1\|_{A_1}).
	\end{equation*}
	No es complicado demostrar que estos espacios también son de Banach. Por otra parte, se define un mapa entre parejas $T:\overline{A}\to\overline{B}$ como un mapa lineal continuo $T:\Sigma(\overline{A})\to\Sigma(\overline{B})$, tales que sus restricciones  $T|_{A_j}:A_j\to B_j$ son mapas lineales continuos para $j=0,1$. En consecuencia se tiene que 
	\begin{equation*}
		\|T\|_{\Delta(\overline{A}), \Delta(\overline{B})} \leq \max\{\|T\|_{A_0, B_0}, \|T\|_{A_1, B_1}\},
	\end{equation*}
	\begin{equation*}
		\|T\|_{\Sigma(\overline{A}), \Sigma(\overline{B})} \leq \max\{\|T\|_{A_0, B_0}, \|T\|_{A_1, B_1}\}.
	\end{equation*}
\end{definition}
\begin{definition}[Espacios intermedios y de interpolación]
	Sea $\overline{A}:=(A_0, A_1)$ una pareja compatible, entonces $A$ es un \textit{espacio intermedio} respecto a $\overline{A}$ si $\Delta(\overline{A}) \subset A \subset \Sigma(\overline{A})$. Además, se dice que dos espacios intermedios $A, B$ respecto a $\overline{A},\overline{B}$ respectivamente, son \textit{espacio de interpolación}, si  $T:\overline{A}\to\overline{B}$ implica $T:A\to B$. Si además, se tiene que 
	\begin{equation*}
		\|T\|_{A, B} \leq C \max\{\|T\|_{A_0, B_0}, \|T\|_{A_1, B_1}\},
	\end{equation*}
	se dice que son \textit{espacios de interpolación uniformes}, y cuando $C=1$ se dice que son \textit{espacios de interpolación exacta}. Sea $0\leq\theta\leq1$, entonces se dice que $A, B$ son de \textit{exponente} $\theta$ si 
	\begin{equation*}
		\|T\|_{A, B} \leq C \|T\|_{A_0, B_0}^{1-\theta} \|T\|_{A_1, B_1}^{\theta}.
	\end{equation*}
	Además, si $C=1$, se dice que son \textit{de exponente exacto}.
\end{definition}
\begin{remark}
	Se puede ver que lo que en realidad se definió es una categoría cuyos objetos son las parejas compatibles y sus morfismos los mapas definidos anteriormente. A continuación se definen los functores de interpolación.
\end{remark}
\begin{definition}[Functor de interpolación]
	Se dice que $F$ es un \textit{functor de interpolación} si para parejas $\overline{A}, \overline{B}$, entonces $F(\overline{A}), F(\overline{B})$ son espacios de interpolación. Además, se tiene que $F(T)=T\vert_{F(\overline{A})}$ para $T:\overline{A}\to\overline{B}$.
\end{definition}
Ahora, se presentan las particularidades del método de interpolación compleja.
\begin{definition}
	Dada una pareja $\overline{A}$ de espacios de Banach, se define a $\mathfrak{F}(\overline{A})$ como el conjunto de funciones $f:\C\to\Sigma(\overline{A})$, tales que son continuas y acotadas en la banda $S:=\{z\in\C : 0\leq\mathrm{Re} z\leq1\}$, y que son analíticas en el interior de la misma. Además, se requiere que el mapa $t\mapsto f(j+it)$ sea continuo en $A_j$ y se desvanezcan cuando $t\to\infty$, para $j=1,0$. Además, se define la norma 
	\begin{equation*}
		\|f\|_{\mathfrak{F}} := \max\{\sup\|f(it)\|_{A_0}, \sup\|f(1+it)\|_{A_1}\}.
	\end{equation*}
	Se puede verificar que es un espacio de Banach. Por otra parte, se define el \textit{functor de interpolación} $\overline{A}_{[\theta]}=C_\theta(\overline{A})$ como el espacio de vectores $a\in\Sigma(\overline{A})$, tales que $a=f(\theta)$ para algún $f\in \mathfrak{F}(\overline{A})$. Además, se define la norma
	\begin{equation*}
		\|a\|_{[\theta]} := \inf \{\|f\|_{\mathfrak{F}} : f(\theta) = a\}.
	\end{equation*}
\end{definition}
\begin{theorem}
	El espacio $\overline{A}_{[\theta]}$ es un espacio de Banach y es un espacio de interpolación the exponente exacto $\theta$.
\end{theorem}
\begin{proof}
	Es claro que el mapa $f\mapsto f(\theta)$ es continuo desde $\mathfrak{F}(\overline{A})$ hacia $\Sigma(\overline{A})$, dado que $\|f(\theta)\|_{\Sigma} \leq \|f\|_{\mathfrak{F}}$. El kernel de este mapa es el conjunto $\mathcal{N}_\theta:=\{f: f(\theta)=0\}$. Entonces, $\overline{A}_{[\theta]}$ es isomorfo e isométrico al cociente $\mathfrak{F}/\mathcal{N}_\theta$. Además, como $\mathcal{N}_\theta$ es cerrado, se tiene que es un espacio de Banach. Ahora, se tiene que $\|a\|_\Sigma = \|f(\theta)\|_\Sigma \leq \|f\|_{\mathfrak{F}}$, que implica que $\overline{A}_{[\theta]} \subset \Sigma(\overline{A})$. Asimismo, sea 
	\begin{equation*}
		f(z) := e^{(z-\theta)^2}a,
	\end{equation*}
	que implica que $\Delta(\overline{A})\subset \overline{A}_{[\theta]}$. Entonces, es un espacio intermedio. Por otra parte, suponga $T_j$ mapea a $A_j$ en $B_j$ con norma $M_j$. Dado $a\in\overline{A}_{[\theta]}$, y dado $\varepsilon>0$, existe $f\in\mathfrak{F}(\overline{A})$ tal que $f(\theta)=a$ y que $\|f\|_{\mathfrak{F}}\leq \|a\|_{[\theta]}+\varepsilon$. Entonces, sea 
	\begin{equation*}
		g(z) := M_0^{z-1}M_1^{-z}T(f(z)),
	\end{equation*}
	que pertenece a $\mathfrak{F}(\overline{B})$. Además, $\|g\|_{\mathfrak{F}} \leq \|f\|_{\mathfrak{F}}$, pero $g(\theta)=M_0^{\theta-1}M_0^{-\theta} T(a)$ y se concluye que 
	\begin{equation*}
		\|Ta\|_{[\theta]} \leq M_0^{1-\theta}M_1^\theta \|g\|_{\mathfrak{F}} \leq M_0^{1-\theta}M_1^\theta \|f\|_{\mathfrak{F}} \leq M_0^{1-\theta}M_1^\theta (\|a\|_{[\theta]} + \varepsilon) .
	\end{equation*}
	Lo que concluye la prueba.
\end{proof}
Ahora, se puede expresar el Teorema de interpolación de Riesz en el lenguaje que se estableció anteriormente. 
\begin{theorem}
	Sean $1\leq p_0,p_1\leq \infty$, entonces para $0<\theta<1$ se tiene que 
	\begin{equation*}
		(L^{p_0}, L^{p_1})_{[\theta]} = L^p, \quad \text{ con } \quad \frac{1}{p} = \frac{1-\theta}{p_0} + \frac{\theta}{p_1}.
	\end{equation*}
\end{theorem}
Y este se extiende al caso de espacios pesados, vea Begh y Löfstrom \cite{bergh-lofstrom}
\begin{theorem}
	Suponga que $1\leq p_0,p_1<\infty$. Entonces, para $0<\theta<1$ se tiene que 
	\begin{equation*}
		(L^{p_0}(w_0), L^{p_1}(w_1))_{[\theta]} = L^p(w),
	\end{equation*}
	donde 
	\begin{equation*}
		\frac{1}{p} = \frac{1-\theta}{p_0} + \frac{\theta}{p_1} \quad \text{ y } \quad w = w_0^{\frac{p(1-\theta)}{p_0}} w_1^{\frac{p\theta}{p_1}}.
	\end{equation*}
	\label{theo:interpolation-Lpw}
\end{theorem}
Ahora, se presenta el resultado de dualidad de interpolación compleja que permite explotar el hecho que $\mathrm{BMO}$ es el dual de $H^1$.
\begin{theorem}
	Sea $\overline{A}:=(A_0, A_1)$ una pareja compatible de espacios de Banach tales que $\Delta(\overline{A})$ es denso en ambos espacios y al menos uno de los espacios es reflexivo. Entonces
	\begin{equation*}
		(A_0, A_1)_{[\theta]}' = (A_0', A_1')_{[\theta]},
	\end{equation*}
	con igualdad de normas.
\end{theorem}
Este resultado no se demuestra, ya que se encuentra fuera de los alcances de este trabajo, pero se recomienda revisar Bergh y Löfstrom \cite{bergh-lofstrom}. Por otra parte, se presenta el resultado que permite realizar interpolación utilizando el espacio $H^1$ y el espacio $\mathrm{BMO}$.
\begin{theorem}
	Sea $1<p\leq\infty$, entonces se tiene que para funciones $\Omega\to X$, con $\Omega=\R^n,\T^n$, 
	\begin{equation*}
		(\mathrm{BMO}, L^{p'})_{[\theta]} =L^{q'}, \quad \text{ con } \quad \frac{1}{q} = 1-\theta + \frac{\theta}{p}
	\end{equation*}
	si además, $X'$ cumple la propiedad Radon-Nikodym, se tiene que
	\begin{equation*}
		(H^1,L^p)_{[\theta]} = L^q.
	\end{equation*}
\end{theorem}
Sin embargo, para los intereses de este trabajo, se demuestran el siguiente corolario, cuya prueba directa contiene las ideas generales del resultado anterior.
\begin{corollary}
	Sea $z\mapsto T_z$ un mapa de la banda cerrada $0\leq\mathrm{Re} z\leq1$ a operadores acotados en $L^2(\Omega;X)$, con $\Omega=\R^n,\T^n$. Suponga que el mapa es continuo y acotado en la banda, y analítico en su interior. Además, fije $\frac{1}{p} = 1- \frac{\theta}{2}$, para $0<\theta<1$.
	\begin{enumerate}
		\item Suponga que $X'$ cumple la propiedad de Radon-Nikodym,
		\begin{align*}
			\sup_{y\in\R} \|T_{iy}f\|_{L^1} \leq M_0\|f\|_{H^1}, &\quad f\in L^2\cap H^1\\
			\sup_{y\in\R} \|T_{1+iy}f\|_{L^2} \leq M_1\|f\|_{L^2}, &\quad f\in L^2.
		\end{align*}
		Entonces,  se tiene que 
		\begin{equation*}
			\|T_\theta f\|_{L^p} \leq M_\theta \|f\|_{L^p}, \quad f \in L^2\cap L^p.
		\end{equation*}
		\item Suponga 
		\begin{align*}
			\sup_{y\in\R} \|T_{iy}f\|_{\mathrm{BMO}} \leq M_0\|f\|_{L^\infty}, &\quad f\in L^2\cap L^\infty \\
			\sup_{y\in\R} \|T_{1+iy}f\|_{L^2} \leq M_1\|f\|_{L^2}, &\quad f\in L^2.
		\end{align*}
		Entonces, se tiene que 
		\begin{equation*}
			\|T_\theta f\|_{L^{p'}} \leq M_\theta \|f\|_{L^{p'}}, \quad f \in L^2\cap L^{p'}.
		\end{equation*}
	\end{enumerate}
	Donde $M_\theta$ solo depende de $M_0$, $M_1$, y $\theta$.
\end{corollary}
\begin{proof}
	La prueba del primer inciso contiene incluida la prueba del segundo inciso, gracias a la propiedad de dualidad entre $H^1$ y $\mathrm{BMO}$. Así que suponga las hipótesis del inciso (1). Sea $S_z$ el dual (en $L^2$) del operador $T_{\overline{z}}$, es decir
	\begin{equation*}
		\int_{\Omega} gT_z(f) \diff x = \int_{\Omega} {S_{\overline{z}}(g)} f\diff x,
	\end{equation*}
	para $f\in L^2(\Omega;X)$, $g\in L^2(\Omega;X')$. Entonces, se tiene qu el mapa $z\mapsto S_z$ es continuo y acotado en la banda cerrada, y analítico en su interior. Ahora, para $g\in L^2\cap L^\infty$, y $f \in L^2\cap H^1$, se tiene que 
	\begin{equation*}
		\left| \int_{\Omega} S_{iy}(g)f \diff x 
		\right| = 	\left| \int_{\Omega} g{T_{-iy}(f)}  \diff x 
		\right| \leq \|T_{-iy}f\|_{L^1(X)}\|g\|_{L^\infty(X')} \leq M_0 \|f\|_{H^1(X)}\|g\|_{L^\infty(X')}.
	\end{equation*}
	Entonces, se tiene que $S_{iy}g$, es la restricción de un funcional acotado en $H^1(X)$, por lo que en vista del Teorema \ref{theo:H1-BMO}, se tiene que $\|S_{iy}g\|_{\mathrm{BMO}(X')}  \leq  C_0M_0\|g\|_{L^\infty(X')}$. La dualidad de $L^2$ permite un argumento similar para así concluir $\|S_{iy}g\|_{L^2(X')}  \leq  M_1\|g\|_{L^2(X')}$, y que $S_z$ cumple las condiciones del inciso (2) para $X'$. Así que se demuestra ese inciso, y las normas a partir de este momento son respecto a $X'$. Ahora, sea $x\mapsto Q(x)$ una función medible que mapea puntos a cubos que los contienen, y sea $\eta(x, y)$ una función medible en $\Omega\times\Omega$, tal que  $|\eta(x, y)|=1$. Además, se define el operador 
	\begin{equation*}
		(U_zg)(x) := \frac{1}{|Q(x)|} \int_{Q(x)} [F(y) - F_{Q(x)}]\eta(x, y) \diff y, 
	\end{equation*}
	con $F=S_zg$. Note que $\sup|U_zg(x)| = F^\#(x)$, si el supremo se toma sobre todas las funciones $Q$ y $\eta$. Además, como $\|F^\#\|_{L^2} \leq 2\|MF\|_{L^2} \leq C_1\|F\|_2$ (vea el Teorema \ref{theo:Hardy-Littlewood}), entonces se tiene que la función $z\mapsto \int U_z(g)f\diff x$ es acotada y continua en la banda cerrada, y analítica en su interior. También cumple que 
	\begin{equation*}
		\|U_{iy} g\|_{L^\infty} = \|F^\#\|_{L^\infty} = \|F\|_{\mathrm{BMO}} = \|S_{iy}g\|_{\mathrm{BMO}}  \leq C_0M_0 \|g\|_{L^\infty}, \quad g\in L^2\cap L^\infty,
	\end{equation*}
	\begin{equation*}
		\|U_{1+iy} g\|_{L^2} = \|F^\#\|_{L^2} \leq C_1\|F\|_{L^2}= C_1\|S_{1+iy}g\|_{L^2} \leq C_1M_1 \|g\|_{L^2}, \quad g\in L^2.
	\end{equation*}
	Por el Teorema de interpolación de Riesz-Thorin (Teorema \ref{theo:riesz-thorin}), se tiene que 
	\begin{equation*}
		\|U_\theta g\|_{L^{p'}} \leq (C_0M_0)^{1-\theta}(C_1M_1)^\theta \|g\|_{L^{p'}}, \quad \frac{1}{p'} = \frac{\theta}{2}.
	\end{equation*}
	Como esta cota no depende de las funciones $Q$ o $\eta$, entonces se puede tomar el supremo y obtener que 
	\begin{equation*}
		\|F^\#\|_{L^{p'}} =\| (S_\theta g)^\#\|_{L^{p'}} \leq (C_0M_0)^{1-\theta}(C_1M_1)^\theta \|g\|_{L^{p'}}.
	\end{equation*}
	Al aplicar el Teorema \ref{theo:sharp-hardy-littlewood}, se obtiene que 
	\begin{equation*}
		\|S_\theta g\|_{L^{p'}(X')} \leq \|M(S_\theta g)\|_{L^{p'}} \leq C_{p'} \|(S_\theta g)^\#\|_{L^{p'}} \leq C_{p'} (C_0M_0)^{1-\theta}(C_1M_1)^\theta \|g\|_{L^{p'}(X')}.
	\end{equation*}
	Por lo que se obtiene el resultado del inciso (2). El inciso (1) se completa mediante un argumento de dualidad, ambos con $M_\theta = C_{p'} (C_0M_0)^{1-\theta}(C_1M_1)^\theta$.
\end{proof}
\chapter{Operadores pseudo-diferenciales}
En esta sección se presenta la teoría de operadores pseudo-diferenciales con símbolos en las clases de Hörmander, como ha sido desarrollado por Ruzhansky y Turunen \cite{ruzhansky-turunen}.
\section{Definición y propiedades básicas en $\R^n$}
\begin{definition}[Clases de símbolos de Hörmander $S^m_{\rho,\delta}(\R^n\times\R^n)$]
	Sean $0\leq\delta,\rho\leq1$. Se dice que $a \in S^m_{\rho,\delta}(\R^n\times\R^n)$ si $a:=(x,\xi)$ es suave en $\R^n\times\R^n$ y cumple que
	\begin{equation*}
		|\partial^\beta_x\partial^\alpha_\xi a(x, \xi)| \lesssim_{\alpha\beta}\angles{\xi}^{m-\rho|\alpha|+\delta|\beta|},
	\end{equation*}
	para cualesquiera multi-índices $\alpha,\beta$. Se dice que estos \textit{símbolos} tienen orden $m\in\R$.
\end{definition}
\begin{definition}
	Sean $0\leq\delta,\rho\leq1$ y sea $a \in S^m_{\rho,\delta}(\R^n\times\R^n)$. El operador pseudo-diferencial con símbolo $a:=a(x,\xi)$ se define como
	\begin{equation*}
		T_af(x):= \int_{\R^n} e^{2\pi ix\cdot\xi} a(x, \xi)\widehat{f}(\xi)\diff \xi,
	\end{equation*}
	donde $f\in\mathcal{S}(\R^n)$. La clase de \textit{operadores pseudo-diferenciales} con símbolos en $S^m_{\rho,\delta}(\R^n\times\R^n)$ se denotan por $\Psi^m_{\rho,\delta}(\R^n\times\R^n)$.
\end{definition}
\begin{proposition}
	Para $a \in S^m_{\rho,\delta}(\R^n\times\R^n)$ y $f\in\mathcal{S}(\R^n)$ se tiene que $T_af\in\mathcal{S}(\R^n)$.
\end{proposition}
\begin{proof}
	Note que como $\widehat{f}\in\mathcal{S}(\R^n)$, se tiene que 
	\[|\partial^\beta_xa(x, \xi)\widehat{f}(\xi)|\lesssim \angles{\xi}^{m+\delta|\beta|}\angles{\xi}^{-N} ,\]
	para algún $N>0$ apropiado, por lo que todas sus derivadas respecto a $x$ son absolutamente convergentes y se tiene que $T_af\in C^\infty(\R^n)$. Ahora, se define el operador 
	\begin{equation*}
		L_\xi := (1 + 4\pi^2|x|^2)^{-1}(I-\mathcal{L}_\xi),
	\end{equation*}
	donde $\mathcal{L}_\xi$ es el laplaciano. Note que $ L_\xi(e^{2\pi ix\cdot\xi}) = e^{2\pi ix\cdot \xi} $ y por integracion por partes se tiene que
	\begin{equation*}
		T_af(x) = \int_{\R^n} e^{2\pi ix\cdot \xi}L_\xi^N [a(x,\xi)\widehat{f}(\xi)] \diff \xi.
	\end{equation*}
	Por lo que $|T_af(x)| \lesssim_N \angles{x}^{-2N}$ para cualquier $N$ y se concluye que $T_af$ decae rapidamente. Este argumento se puede aplicar para cualquiera de sus derivadas y se obtiene que $T_af\in\mathcal{S}(\R^n)$.
\end{proof}
\begin{example}[Operadores diferenciales]
	Sea $P:=\sum_{|\alpha|\leq m} a_\alpha(x)\partial^\alpha_x$ un operador de derivadas parciales. Entonces, al considerarle como un operador pseudo-diferencial se tiene que su símbolo es simplemente su polinomio característico $p(x,\xi) = \sum_{|\alpha|\leq m} a_\alpha(x)(2\pi i\xi)^\alpha$. Si las funciones coeficientes $a_\alpha$ son continuas, este símbolo pertenece a la clase de Hörmander de orden $m$.
\end{example}
\begin{remark}[Kernel de un operador pseudo-diferencial]
		Se puede reescribir la definición de operador pseudo-diferencial de la siguiente manera
		\begin{align*}
			T_af(x) & = \int_{\R^n} e^{2\pi ix\cdot\xi} a(x, \xi)\widehat{f}(\xi)\diff \xi\\
			& = \int_{\R^n} \int_{\R^n} e^{2\pi i (x-y)\cdot\xi}a(x, \xi)f(y) \diff y \diff \xi\\
			& =\int_{\R^n} k(x, y) f(y)\diff y,
		\end{align*}
		donde se define en el sentido de distribuciones al kernel de Schwartz del operador pseudo-diferencial como
		\begin{equation*}
			k(x, y) := \int_{\R^n} e^{2\pi i (x-y)\cdot\xi}a(x, \xi) \diff \xi 
		\end{equation*}
\end{remark}
\begin{theorem}[Composición de operadores pseudo-diferenciales]
	Sea $0\leq\delta<\rho\leq1$, sea $a \in S^{m_1}_{\rho,\delta}(\R^n\times\R^n) $ y sea $ b\in S^{m_2}_{\rho,\delta}(\R^n\times\R^n) $. Entonces, existe un símbolo $ c\in S^{m_1+m_2}_{\rho,\delta}(\R^n\times\R^n)$ tal que 
	$T_c = T_a \circ T_b$. Además, se tiene la fórmula asimptótica 
	\begin{equation*}
		c \sim \sum_\alpha \frac{(2\pi i)^{-|\alpha|}}{\alpha!}(\partial^\alpha_\xi a)(\partial^\alpha_x b).
	\end{equation*}
	Es decir, para cualquier $N>0$, se tiene que 
	\begin{equation*}
		c - \sum_{|\alpha|<N} \frac{(2\pi i)^{-|\alpha|}}{\alpha!}(\partial^\alpha_\xi a)(\partial^\alpha_x b) \in S^{m_1+m_2-(\rho-\delta) N}_{\rho,\delta}(\R^n\times\R^n).
	\end{equation*}
\end{theorem}
\begin{proof}
	Fije un $x_0\in\R^n$ y sea $\chi\in C^\infty_0(\R^n)$ tal que $\supp \chi\subset\{x\in\R^n:|x-x_0|\leq 2\}$ y tal que $\chi(x) = 1$ para $|x-x_0|\leq 1$. Realice la descomposición 
	\begin{equation*}
		b = \chi b + (1-\chi)b := b_1 + b_2.
	\end{equation*}
	Entonces, se tiene que 
	\begin{align*}
		(T_a\circ T_{b_1})f(x) &= \int_{\R^n}\int_{\R^n} e^{2\pi i (x-y)\cdot\eta} a(x, \eta)
		\int_{\R^n}\int_{\R^n} e^{2\pi i(y-z)\cdot\xi}b_1(y, \xi)f(z)\diff z \diff \xi \diff y \diff \eta\\
		& = \int_{\R^n}\int_{\R^n}  e^{2\pi i (x-z)\cdot\xi} \int_{\R^n}\int_{\R^n} 
		e^{2\pi i(x-y)\cdot(\eta-\xi)} a(x, \eta)b_1(y, \xi)\diff y \diff \eta f(z) \diff z \diff \xi,
	\end{align*}
	donde se aprovecho que $(x-y)\cdot(\eta-\xi) + (x-z)\cdot \xi= (x-y)\cdot\eta + (y-z)\cdot\xi$. Por lo que se define 
	\begin{align*}
		c(x, \xi) & :=  \int_{\R^n}\int_{\R^n} 
		e^{2\pi i(x-y)\cdot(\eta-\xi)} a(x, \eta)b_1(y, \xi)\diff y \diff \eta \\ 
		&=  \int_{\R^n}\
		e^{2\pi ix\cdot(\eta-\xi)} a(x, \eta)\widehat{b_1}(\eta-\xi, \xi) \diff \eta \\
		&=  \int_{\R^n}\
		e^{2\pi ix\cdot\eta} a(x, \eta+\xi)\widehat{b_1}(\eta, \xi) \diff \eta.
	\end{align*}
	Como $b_1$ tiene soporte compacto en $x$, se tiene que $\widehat{b_1}$ es de decaimiento rápido uniformemente en $\xi$ y que 
	\begin{equation*}
		|\widehat{b_1}(\eta,\xi)| \lesssim_M \angles{\eta}^{-M}\angles{\xi}^{m_2},
	\end{equation*}
	para todo $M\leq0$. La expansión de Taylor en la segunda variable de $a(x,\xi+\eta)$ resulta en 
	\begin{equation*}
		a(x, \xi + \eta) = \sum_{|\alpha|<N} \frac{1}{\alpha!} \partial^\alpha_\xi a(x, \xi)\eta^\alpha + R_N(x, \xi, \eta),
	\end{equation*}
	donde $R_N$ es un residuo que se discutirá más adelante. Al sustituir esta expresión en la fórmula para $c(x, \xi)$ se obtiene que 
	\begin{equation*}
		\int_{\R^n} e^{2\pi i x\cdot\eta}[\partial^\alpha_\xi a(x, \xi)\eta^\alpha]\widehat{b_1}(\eta, \xi) \diff \eta = (2\pi i)^{-|\alpha|} 
		\partial^\alpha_\xi a(x, \xi) \partial^\alpha_x b_1(x, \xi),
	\end{equation*}
	que corresponden a los términos de la expansión asimptótica. Ahora, el símbolo resultante del residuo es 
	\begin{equation*}
		\int_{\R^n} e^{2\pi i x\cdot\eta} R_N(x, \xi, \eta) \widehat{b_1}(\eta, \xi) \diff \eta.
	\end{equation*}
	Pero, se puede estimar mediante 
	\begin{equation*}
		|R_N(x, \xi, \eta)| \lesssim_N |\eta|^N \max\{|\partial^\alpha_\xi a(x, \zeta)| : |\alpha|=N,\, \zeta \text{ interpolación de } \eta \text{ y }\eta+\xi\}.
	\end{equation*}
	Note que si $|\eta|\leq|\xi|/2$, entonces cualquier $\zeta$ de la expresión anterior es proporcional a $\xi$, por lo que para este caso se puede estimar
	\begin{equation*}
		|R_N(x, \xi, \eta)| \lesssim_N |\eta|^N\angles{\xi}^{m_1-(\rho-\delta) N}.
	\end{equation*}
	Por otra parte, si $\rho N\geq m_1$, se tiene la siguiente cota para cualquier caso
	\begin{equation*}
		|R_N(x, \xi, \eta)| \lesssim_N |\eta|^N.
	\end{equation*}
	Combinando los estimativos y la expresión del residuo del símbolo se obtiene que 
	\begin{align*}
		&\left| \int_{\R^n} e^{2\pi i x\cdot\eta} R_N(x, \xi, \eta) \widehat{b_1}(\eta, \xi) \diff \eta \right| \\
		&\lesssim_{M,N} \angles{\xi}^{m_1+m_2-(\rho-\delta) N} \int_{|\eta|<|\xi|/2} \angles{\eta}^{-M}|\eta|^N\diff\eta + \angles{\xi}^{m_2} \int_{|\eta|\geq|\xi|/2}\angles{\eta}^{-M}|\eta|^N\diff\eta.
	\end{align*}
	Al escoger $M$ lo suficientemente grande, se puede estimar el residuo simbólico por $\angles{\xi}^{m_1+m_2-\rho N}$. Ahora, note que $\partial^\alpha_\xi\partial^\beta_x R_N(x, \xi, \eta)$ es el residuo de la expansión de $\partial^\alpha_\xi\partial^\beta_x a(x, \xi+\eta)$. Por lo que un argumento similar resulta en 
	\begin{equation*}
		\left|\int_{\R^n} e^{2\pi i x\cdot\eta} [\partial^\alpha_\xi\partial^\beta_x R_N(x, \xi, \eta)]\widehat{b_1}(\eta, \xi) \diff \eta\right| \lesssim_{\alpha\beta} \angles{\xi}^{m_1+m_2- \rho N -\rho|\alpha| + \delta|\beta|}.
	\end{equation*}
	Ahora, solo queda demostrar que $T_a\circ T_{b_2}$ tiene símbolo de orden $-\infty$ y no afecta la fórmula asimptótica. Para ello, se utiliza integración por partes para obtener propiedades de regularidad del símbolo restante. Considere el Laplaciano en $\eta$ 
	\begin{equation*}
		\Delta_\eta^{N_1} e^{2\pi i(x-y)\cdot(\eta-\xi)} = 
		(- 4\pi^2|x-y|^2)^{N_1}  e^{2\pi i(x-y)\cdot(\eta-\xi)}.
	\end{equation*}
	Y el Laplaciano en $y$,
	\begin{equation*}
		(1-\Delta_y)^{N_2} e^{2\pi i(x-y)\cdot(\eta-\xi)}  = 
		(1+4\pi^2|\xi-\eta|^2)^{N_2} e^{2\pi i(x-y)\cdot(\eta-\xi)}.
	\end{equation*}
	Además, se tiene que 
	\begin{equation*}
		\angles{\xi-\eta}\angles{\eta} = \sqrt{1 + |\xi-\eta|^2 + |\eta|^2 + |\xi-\eta|^2|\eta|^2} \geq \sqrt{1 + |\xi|^2} = \angles{\xi}.
	\end{equation*}
	Combinando estos estimativos y las desigualdades simbólicas se obtiene que
	\begin{align*}
		|c_2(x, \xi)| & = \left| \int_{\R^n}\int_{\R^n} 
		e^{2\pi i(x-y)\cdot(\eta-\xi)} a(x, \eta)b_2(y, \xi)\diff y \diff \eta \right| \\
		&  = \left| \int_{\R^n}\int_{\R^n} 
		e^{2\pi i(x-y)\cdot(\eta-\xi)} \frac{\Delta_\xi^{N_1}a(x, \eta)}{(-4\pi^2|x-y|^2)^{N_1}} \frac{(1-\Delta_y)^{N_2}b_2(y, \xi)}{(1+4\pi^2|\xi-\eta|^2)^{N_2}}\diff y \diff \eta \right|  \\
		& \lesssim \int_{\R^n}\int_{\R^n}  \frac{\angles{\eta}^{m_1-2\rho N_1}}{\angles{x-y}^{2N_1}} \frac{\angles{\xi}^{m_2 + 2\delta N_2}}{\angles{\xi-\eta}^{2N_2}} \diff y \diff \eta \\
		& \lesssim \int_{\R^n} \angles{\eta}^{m_1-2\rho N_1 + 2N_2} \angles{\xi}^{m_2-2(1-\delta) N_2 } \diff \eta \\
		& \lesssim \angles{\xi}^{m_2-2(1-\delta) N_2 }, \\
	\end{align*}
	donde se escoge $N_1$ tal que $-2\rho N_1 + 2N_2 + m_1 < -n$. Por lo que se puede escoger $N_2$ libremente para obtener la cota deseada. Un argumento análogo funciona para las derivadas de $c_2$, por lo que este pertenece a $S^{-\infty}(\R^n\times\R^n)$.
\end{proof}
\begin{definition}[Potencial de Bessel]
	Se define al \textit{potencial de Besse}l de orden $m\in\R$ al operador pseudo-diferencial con símbolo $\angles{\xi}^m$. Este se denota por $J^m$.
\end{definition}
Se puede ver que la composición con este operador no presenta la restricción $\delta<\rho$.
\begin{theorem}
	Sea $0\leq\delta<1$, sea $0<\rho\leq1$, sea $a:=a(x, \xi)\in S^m_{\rho,\delta}(\R^n\times\R^n)$, y sea $b:=b(\xi) \in S^s_{1,\delta}(\R^n\times\R^n)$. Entonces $T_aT_b$ y $T_bT_a$ pertenecen a $S^{m+s}_{\rho,\delta}(\R^n\times\R^n)$. Además, $T_aT_b$ tiene símbolo $a(x, \xi)b(\xi)$.
\end{theorem}
\begin{proof}
	Se tiene que para $T_c = T_aT_b$
	\begin{align*}
		c(x, \xi) & :=  \int_{\R^n}\int_{\R^n} 
		e^{2\pi i(x-y)\cdot(\eta-\xi)} a(x, \eta)b(\xi)\diff y \diff \eta \\ 
		& =  b(\xi)\int_{\R^n}e^{2\pi i(y-x)\cdot\xi} \widehat{a}(x, y-x)\diff y \\
		& = b(\xi)a(x, \xi).
	\end{align*}
	Para $T_c = T_bT_a$ se puede utilizar la fórmula asintótica de la fórmula de composición. Para manejar el hecho que en este caso se permite $\delta\geq\rho$ para el símbolo $a$, se aprovecha el hecho que 
	\begin{align*}
		|\partial^\alpha_\xi b(\xi) \partial^\alpha_x a(x, \xi)| \lesssim \angles{\xi}^{s-|\alpha|} \angles{\xi}^{m+\delta |\alpha|}.
	\end{align*}
	Por lo que se tiene el orden deseado en ambos casos.
\end{proof}
\section{Definición y propiedades básicas en $\T^n$}
El cálculo simbólico en el toro $\T^n$ presenta ciertas diferencias respecto al caso euclideano. Gran parte de ellas surgen del hecho que el dominio de frecuencias correspondiente es el retículo $\Z^n$. Por lo que es necesario definir herramientas análogas que funcionen en el caso discreto. 
\begin{definition}
	Sea $\varphi:\Z^n\to\C$, entonces se definen los \textit{operadores de diferencia }como 
	\begin{equation*}
		\Delta_{\xi_j} \varphi(\xi) := \varphi(\xi + \delta_j) - \varphi(\xi),
	\end{equation*}
	\begin{equation*}
		\overline{\Delta}_{\xi_j} \varphi(\xi) := \varphi(\xi) - \varphi(\xi - \delta_j).
	\end{equation*}
	Además, para un multi-índice $\alpha\in\N^n_0$, se define 
	\begin{equation*}
		\Delta^\alpha_\xi := \Delta^{\alpha_1}_{\xi_1} \cdots \Delta^{\alpha_n}_{\xi_n}, \quad \overline{\Delta}^\alpha_\xi := \overline{\Delta}^{\alpha_1}_{\xi_1} \cdots \overline{\Delta}^{\alpha_n}_{\xi_n}.
	\end{equation*}
\end{definition}
\begin{proposition}
	Sea $\varphi:\Z^n\to\C$, entonces 
	\begin{equation*}
		\Delta^\alpha_\xi \varphi(\xi) = \sum_{\beta\leq\alpha}(-1)^{|\alpha-\beta|} \binom{\alpha}{\beta}\varphi(\xi+\beta).
	\end{equation*}
\end{proposition}
\begin{proof}
	Se define el operador de traslación $E_j := (I+\Delta_{\xi_j})$, que actua de la siguiente manera
	\begin{equation*}
		E_j\varphi(\xi) = (I+\Delta_{\xi_j})\varphi(\xi) = \varphi(\xi +\delta_j).
	\end{equation*}
	Entonces, por el teorema del binomio se tiene que 
	\begin{align*}
		\Delta^\alpha_\xi\varphi(\xi) & = (E - I)^\alpha\varphi(\xi)\\
		& = \sum_{\beta\leq\alpha}(-1)^{|\alpha-\beta|}\binom{\alpha}{\beta}E^\beta\varphi(\xi) \\
		&  = \sum_{\beta\leq\alpha}(-1)^{|\alpha-\beta|} \binom{\alpha}{\beta}\varphi(\xi+\beta).
	\end{align*}
	Lo que completa la prueba.
\end{proof}
Note que este operador tiene propiedades análogas a las de la derivada en el caso continuo
\begin{proposition}
	Sean $\varphi, \psi:\Z^n\to\C$, sean $\alpha, \beta\in\N^n_0$, entonces
	\begin{enumerate}
		\item $\Delta^\alpha_\xi(s\varphi+t\psi)(\xi) = s\Delta^\alpha_\xi\varphi(\xi) + t\Delta^\alpha_\xi\psi(\xi)$,
		\item $\Delta^\alpha_\xi\Delta^\beta_\xi = \Delta^{\alpha+\beta}_\xi = \Delta^\beta_\xi\Delta^\alpha_\xi$, 
		\item $\Delta^\alpha_\xi(\varphi\psi)(\xi) = \sum_{\beta\leq\alpha}\binom{\alpha}{\beta}[\Delta^\beta_\xi\varphi(\xi)][\Delta^{\alpha-\beta}_\xi\psi(\xi+\beta)]$.
	\end{enumerate}
\end{proposition}
\begin{proof}
	Todas estas propiedades pueden ser demostradas mediante inducción, por lo que solo se demostrarán los casos base. La primera propiedad es equivalente a decir que los operadores de diferencia son lineales. Entonces, se tiene que 
	\begin{align*}
		\Delta_{\xi_j}(s\varphi+t\psi)(\xi) & = (s\varphi+t\psi)(\xi+\delta_j) - (s\varphi+t\psi)(\xi) \\
		& = s\varphi(\xi+\delta_j) + t\psi(\xi+\delta_j) - s\varphi(\xi)- t\psi(\xi) \\
		& = s\Delta_{\xi_j}\varphi(\xi) + t\Delta_{\xi_j}\psi(\xi).
	\end{align*}
	La segunda propiedad quiere decir que los operadores de diferencia conmutan, es decir
	\begin{align*}
		\Delta_{\xi_i}\Delta_{\xi_j}\varphi(\xi) &= \Delta_{\xi_i}[\varphi(\cdot +\delta_j) - \varphi(\cdot)](\xi) \\
		& =  \Delta_{\xi_i}[\varphi(\cdot+\delta_j)] (\xi)- \Delta_{\xi_i}\varphi(\xi) \\
		& = \varphi(\xi+\delta_j+\delta_i) - \varphi(\xi+\delta_j) - \varphi(\xi+\delta_i) + \varphi(\xi)\\
		& = \Delta_{\xi_j}[\varphi(\cdot+\delta_i)-\varphi(\cdot)](\xi) \\
		& = \Delta_{\xi_j}\Delta_{\xi_i}\varphi(\xi).
	\end{align*}
	La tercera propiedad es análoga a la regla de Leibniz, o regla del producto. En efecto, 
	\begin{align*}
		\Delta_{\xi_j}(\varphi\psi)(\xi) & = \varphi(\xi+\delta_j)\psi(\xi+\delta_j) - \varphi(\xi)\psi(\xi) \\
		& = \varphi(\xi+\delta_j)\psi(\xi+\delta_j) + \varphi(\xi)\psi(\xi+\delta_j) - \varphi(\xi)\psi(\xi+\delta_j) - \varphi(\xi)\psi(\xi)\\
		& = \varphi(\xi)[\psi(\xi+\delta_j) - \psi(\xi)] + [\varphi(\xi+\delta_j)-\varphi(\xi)]\psi(\xi+\delta_j)\\ 
		& = \varphi(\xi)[\Delta_{\xi_j}\psi(\xi)] + [\Delta_{\xi_j}\varphi(\xi)]\psi(\xi+\delta_j).
	\end{align*}
	Lo que concluye la prueba.
\end{proof}
\begin{proposition}[Suma por partes]
	Sean $\varphi,\psi:\Z^n\to\C$. Entonces, se tiene que 
	\begin{equation*}
		\sum_{\xi\in\Z^n}\varphi(\xi) [\Delta^\alpha_\xi\psi(\xi)] = (-1)^{|\alpha|}\sum_{\xi\in\Z^n} [\overline{\Delta}^\alpha_\xi\varphi(\xi)]\psi(\xi),
	\end{equation*}
	dado que ambas series sean absolutamente convergentes.
\end{proposition}
\begin{proof}
	Para probarlo para un multi-índice arbitrario basta con demostrarlo para los casos base y luego el resultado se obtiene por recursividad. Entonces, se tiene que 
	\begin{align*}
		\sum_{\xi\in\Z^n}\varphi(\xi) [\Delta_{\xi_j}\psi(\xi)] & = \sum_{\xi\in\Z^n}\varphi(\xi) [\psi(\xi+\delta_j) - \psi(\xi)] \\
		& = \sum_{\xi\in\Z^n}\varphi(\xi) \psi(\xi+\delta_j) - \sum_{\xi\in\Z^n}\varphi(\xi) \psi(\xi) \\
		& = \sum_{\xi\in\Z^n}\varphi(\xi-\delta_j) \psi(\xi) - \sum_{\xi\in\Z^n}\varphi(\xi) \psi(\xi) \\
		& = \sum_{\xi\in\Z^n}[\varphi(\xi-\delta_j) - \varphi(\xi)]\psi(\xi) \\
		& = -\sum_{\xi\in\Z^n} [\overline{\Delta}_{\xi_j}\varphi(\xi)]\psi(\xi).
	\end{align*}
	Lo que concluye la prueba.
\end{proof}
En el caso discreto, los polinomios tradicionales y los operadores de diferencia no se comportan exactamente como sus contrapartes continuas. Lo que inspira definiciones alternativas que se presentan a continuación.
\begin{definition}
	Para $\theta\in\Z^n$, se define su \textit{polinomio discreto} como $\theta^{(0)}_j := 1$, y
	\begin{equation*}
		\theta^{(k+1)}_j := \theta^{(k)}_j(\theta_j-k).
	\end{equation*}
	Además, para cualquier multi-índice $\alpha\in\N_0^n$, se define
	\begin{equation*}
		\theta^{(\alpha)} := \theta^{(\alpha_1)}_1 \cdots \theta^{(\alpha_n)}_n.
	\end{equation*}
	Similarmente, se define $D_{y_j}^{(0)} := I$, y 
	\begin{equation*}
		D_{y_j}^{(k+1)} := D_{y_j}^{(k)} \left(\frac{\partial_{y_j}}{i2\pi}-kI\right),
	\end{equation*}
	\begin{equation*}
		D^{(\alpha)}_y := D^{(\alpha_1)}_{y_1} \cdots D^{(\alpha_n)}_{y_n}.
	\end{equation*}
\end{definition}
\begin{proposition}
	Para $\alpha,\beta\in\N_0^n$, se tiene que 
	\begin{equation*}
		\Delta^\beta_\theta \theta^{(\alpha)} = \alpha^{(\beta)}\theta^{(\alpha-\beta)},
	\end{equation*}
	lo que concuerda con el caso continuo $\partial^\beta_\theta \theta^\alpha = \alpha^{(\gamma)}\theta^{\alpha-\beta}$.
\end{proposition}
\begin{proof}
	Basta con ver que 
	\begin{align*}
		\Delta_{\theta_j} \theta_j^{(\alpha_j)} &= (\theta_j+1)^{(\alpha_j)} - \theta_j^{(\alpha_j)} \\
		& = (\theta_j+1)\theta_j \cdots(\theta_j - \alpha_j+2) - \theta_j^{(\alpha_j)} \\
		& = (\theta_j+1)\theta_j^{(\alpha_j-1)} - \theta_j^{(\alpha_j-1)}(\theta_j - \alpha_j + 1) \\
		& = \alpha_j \theta_j^{(\alpha_j-1)}.
	\end{align*}
	El resultado sigue de recursividad.
\end{proof}
\begin{definition}[Integración discreta]
	Para $b\geq0$, se definen
	\begin{equation*}
		I^b_k := \sum_{0\leq k<b}, \quad \text{ y } \quad I^{-b}_k := -\sum_{-b\leq k < 0}.
	\end{equation*}
\end{definition}
\begin{theorem}['Teorema fundamental del cálculo' discreto 1D]
	Sea $\theta\in\Z$, y sea $\alpha\in\N_0$. Entonces, se tiene que
	\begin{equation*}
		I^\theta_{k_1}I^{k_1}_{k_2} \cdots I^{k_{\alpha-1}}_{k_\alpha} 1 = \frac{1}{\alpha!} \theta^{(\alpha)}.
	\end{equation*}
\end{theorem}
\begin{proof}
	Note que en general se tiene que
	\begin{equation*}
		I^b_k\Delta_k f(k) = \sum_{0\leq k<b} [f(k+1)- f(k)] = f(b) - f(0).
	\end{equation*}
	Particularmente, $I^b_k\Delta_k k^{(j)} = b^{(j)}$, que se puede combinar con el hecho que $\Delta_kk^{(j)}=ik^{(j-1)}$, y el resultado deseado sigue de una inducción.
\end{proof}
La extensión al caso multi-dimensional es inmediata. 
\begin{corollary}['Teorema fundamental del cálculo' discreto]
	Sea $\theta\in\Z^n$, y sea $\alpha\in\N_0^n$. Entonces, se tiene que
	\begin{equation*}
		\prod_{j=1}^n I^{\theta_j}_{k(j, 1)} I^{k(j, 1)}_{k(j, 2)} \cdots I^{k(j, \alpha_j-1)}_{k(j, \alpha_j)} 1 = \frac{1}{\alpha!}\theta^{(\alpha)}.
	\end{equation*}
\end{corollary}
Ahora, se presenta el análogo discreto de la expansión de Taylor.
\begin{theorem}[Expansión de Taylor discreta]
	Sea $p:\Z^n\to\C$. Entonces, se puede escribir como 
	\begin{equation*}
		p(\xi + \theta) = \sum_{|\alpha|<M} \frac{1}{\alpha!} \theta^{(\alpha)} \Delta^\alpha_\xi p(\xi) + r_M(\xi, \theta),
	\end{equation*}
	donde el residuo satisface
	\begin{equation*}
		|\Delta^\omega_\xi r_M(\xi, \theta)| \lesssim_M \max_{|\alpha|=M,\, \nu \in Q(\theta)} |\theta^{(\alpha)}\Delta^{\alpha+\omega}_\xi p(\xi+\nu)|,
	\end{equation*}
	donde $Q(\theta) := \{\nu\in\Z^n :|\nu_j|\leq|\theta_j|\,\, \text{ para todo } j=1,\ldots,n\}$.
\end{theorem}
\begin{proof}
	Primero, para $0\neq\alpha\in\N^n_0$, se denota $m_\alpha := \min\{j:\alpha_j\neq0\}$. Para $\theta\in\Z^n$, e $i\in\{1,\ldots,n\}$, se define $\nu(\theta,i,k)\in\Z^n$ como
	\begin{equation*}
		\nu(\theta,i,k)_j :=\begin{cases}
			\theta_j, & 1\leq j<i,\\
			k, & j=i,\\
			0, &i<j\leq n.
		\end{cases}
	\end{equation*}
	Se afirma que el residuo puede ser escrito como 
	\begin{equation*}
		r_M(\xi, \theta) = \sum_{|\alpha|=M}r_\alpha(\xi, \theta),
	\end{equation*}
	donde 
	\begin{equation*}
		r_\alpha(\xi, \theta) := \prod_{j=1}^n I^{\theta_j}_{k(j, 1)} I^{k(j, 1)}_{k(j, 2)} \cdots I^{k(j, \alpha_j-1)}_{k(j, \alpha_j)} \Delta^\alpha_\xi p(\xi + \nu(\theta, m_\alpha, k(m_\alpha, \alpha_{m_\alpha}))).
	\end{equation*}
	Se realiza una prueba por inducción, para $M=1$, se tiene que
	\begin{equation*}
		r_{\delta_i}(\xi, \theta) = I^{\theta_i}_k\Delta_{\xi_i} p(\xi + \nu(\theta, i, k)) = p(\xi + \nu(\theta, i, \theta_i)) - p(\xi + \nu(\theta, i, 0)).
	\end{equation*}
	Por lo que se cumple el resultado deseado
	\begin{equation*}
		\sum_{i=1}^n r_{\delta_i}(\xi, \theta)  = \sum_{i=1}^n [p(\xi + \nu(\theta, i+1, 0)) - p(\xi + \nu(\theta, i, 0))] = p(\xi+\theta) - p(\xi) = r_1(\xi, \theta).
	\end{equation*}
	Ahora, suponga el resultado es verdadero para $|\alpha|=M$, entonces se tiene que
	\begin{align*}
		r_{M+1}(\xi,\theta) & = r_M(\xi,\theta) - \sum_{|\alpha|=M}\frac{1}{\alpha!} \theta^{(\alpha)}\Delta^\alpha_\xi p(\xi) \\
		& = \sum_{|\alpha|=M} \left( r_\alpha(\xi,\theta) - \frac{1}{\alpha!} \theta^{(\alpha)}\Delta^\alpha_\xi p(\xi) 
		\right) \\
		& = \sum_{|\alpha|=M} \prod_{j=1}^n I^{\theta_j}_{k(j, 1)} I^{k(j, 1)}_{k(j, 2)} \cdots I^{k(j, \alpha_j-1)}_{k(j, \alpha_j)} \Delta^\alpha_\xi [p(\xi + \nu(\theta, m_\alpha, k(m_\alpha, \alpha_{m_\alpha}))) - p(\xi)].
	\end{align*}
	Por otra parte, se tiene la igualdad 
	\begin{align*}
		\sum_{i=1}^{m_\alpha} I^{\nu(\theta, m_\alpha, k)_i}_l \Delta_{\xi_i}p(\xi+\nu(\theta, i, l)) &=  \sum_{i=1}^{m_\alpha} [p(\xi+\nu(\theta, i, \nu(\theta, m_\alpha, k)_i)) - p(\xi+\nu(\theta, i, 0))]\\
		&=  \sum_{i=1}^{m_\alpha} [p(\xi+\nu(\nu(\theta, m_\alpha, k), i+1, 0)) - p(\xi+\nu(\theta, i, 0))]\\
		& =p(\xi+\nu(\theta,m_\alpha,k)) - p(\xi) .
	\end{align*}
	Por lo que se obtiene que
	\begin{align*}
		r_{M+1}(\xi, \theta) & = \sum_{|\alpha|=M} \prod_{j=1}^n I^{\theta_j}_{k(j, 1)} I^{k(j, 1)}_{k(j, 2)} \cdots I^{k(j, \alpha_j-1)}_{k(j, \alpha_j)} \sum_{i=1}^{m_\alpha} I^{\nu(\theta, m_\alpha, \alpha_{m_\alpha})_i}_{l_i} \Delta_{\xi}^{\alpha+\delta_i}p(\xi+\nu(\theta, i, l_i))\\
		& = \sum_{|\beta|=M+1} \prod_{j=1}^n I^{\theta_j}_{k(j, 1)} I^{k(j, 1)}_{k(j, 2)} \cdots I^{k(j, \beta_j-1)}_{k(j, \beta_j)} \Delta^\beta_\xi p(\xi + \nu(\theta, m_\beta, k(m_\beta, \beta_{m_\beta}))).
	\end{align*}
	Lo que completa la inducción. Para obtener los estimativos del residuo, se considera que 
	\begin{align*}
		|\Delta^\omega_\xi r_M(\xi, \theta)| & = \left|\sum_{|\alpha|=M}\Delta^{\omega}_\xi r_\alpha(\xi, \theta), 
		\right| \\
		& = \left| \sum_{|\alpha|=M}\prod_{j=1}^n I^{\theta_j}_{k(j, 1)} I^{k(j, 1)}_{k(j, 2)} \cdots I^{k(j, \alpha_j-1)}_{k(j, \alpha_j)} \Delta^{\alpha+\omega}_\xi p(\xi + \nu(\theta, m_\alpha, k(m_\alpha, \alpha_{m_\alpha})))
		\right| \\
		& \leq \sum_{|\alpha|=M} \frac{1}{\alpha!} |\theta|^{(\alpha)}  \max_{\nu\in Q(\theta)} |\Delta^{\alpha+\omega} p(\xi+\nu)|.
	\end{align*}
	Lo que completa la prueba.
\end{proof}
A continuación se presentan las definiciones toroidales para símbolos de clases de H\"ormander y sus respectivos operadores pseudo-diferenciales.
\begin{definition}[Clase de símbolos toroidales $S^m_{\rho, \delta}(\T^n\times\Z^n)$]
	Sea $m\in\R$, sean $0\leq\delta,\rho\leq1$. Entonces, la \textit{clase de símbolos toroidales} $S^m_{\rho, \delta}(\T^n\times\Z^n)$ consiste de las funciones $a:=a(x, \xi):\T^n\times\Z^n\to\C$ que son suaves en $x$ para todo $\xi$, y que satisfacen las desigualdades simbólicas
	\begin{equation*}
		|\Delta^\alpha_\xi\partial^\beta_x a(x, \xi)| \lesssim_{\alpha\beta} \angles{\xi}^{m-\rho|\alpha|+\delta|\beta|}, 
	\end{equation*}
	para cualesquiera multi-índices $\alpha,\beta\in\N_0^n$. 
\end{definition}
\begin{definition}[Operadores pseudo-diferenciales toroidales]
	Para $a\in S^m_{\rho, \delta}(\T^n\times\Z^n)$, se denota $T_a$ a su \textit{operador pseudo-diferencial toroidal} correspondiente, que se define como 
	\begin{equation*}
		T_af(x) := \sum_{\xi\in\Z^n} e^{i2\pi x\cdot\xi} a(x, \xi)\widehat{f}(\xi).
	\end{equation*}
	Además, se dice que $T_a\in\Psi^m_{\rho, \delta}(\T^n\times\Z^n)$
\end{definition}
\begin{proposition}
	Sea $f\in C^\infty(\T^n)$, entonces $T_af$ está bien definido y $T_af\in C^\infty(\T^n)$. Además, $T_a$ es un operador continuo de $C^\infty(\T^n)$ en sí mismo.
\end{proposition}
\begin{proof}
	Se tiene que $\widehat{f}\in\mathcal{S}(\Z^n)$, entonces la serie en la definición de $T_af$ converge absolutamente y $T_af\in C^\infty(\T^n)$. Por otra parte, se tiene que
	\begin{align*}
		T_af(x) &= \sum_{\xi\in\Z^n}e^{i2\pi x\cdot\xi} a(x, \xi) \widehat{f}(\xi) \\
		& = \sum_{\xi\in\Z^n}a(x, \xi) \int_{\T^n} e^{i2\pi (x-y)\cdot\xi} f(y) \diff y\\
		& = \sum_{\xi\in\Z^n}a(x, \xi)\angles{\xi}^{-2M} \int_{\T^n} e^{i2\pi (x-y)\cdot\xi} \left(I-\frac{\mathcal{L}_y}{4\pi^2}\right)^M f(y) \diff y,
	\end{align*}
	donde $\mathcal{L}_y$ es el Laplaciano respecto a $y$. Entonces, basta escoger $M$ lo suficientemente grande para obtener convergencia absoluta de la serie. Por lo que, para $f_j\to f$ en $C^\infty(\T^n)$, se puede intercambiar el límite con la serie y la integral mediante la convergencia dominada de Lebesgue para obtener $T_af_j\to T_af$ en $C^\infty(\T^n)$.
\end{proof}
\begin{remark}[Kernel de Schwartz para operadores pseudo-diferenciales toroidales]
	La definición de $T_af$ para un símbolo toroidal sugiere que puede ser reescrito (ignorando preguntas acerca de convergencia) como 
	\begin{equation*}
		T_af(x) = \int_{\T^n} \sum_{\xi\in\Z^n} e^{i2\pi (x-y)\cdot\xi} a(x, \xi) f(y) \diff y =\int_{\T^n} k(x, y)f(y)\diff y,
	\end{equation*}
	donde $k(x, y)$ es el \textit{kernel de Schwartz} que se expresa como 
	\begin{equation*}
		k(x, y) := \sum_{\xi\in\Z^n} e^{i2\pi (x-y)\cdot\xi} a(x, \xi),
	\end{equation*}
	y se entiende en el sentido de distribuciones. 
\end{remark}
\begin{theorem}
	Sean $0\leq\delta<\rho\leq1$, sea $a\in S^m_{\rho, \delta}(\T^n\times\Z^n)$, y sea $b\in S^l_{\rho, \delta}(\R^n\times\Z^n)$. Entonces, el símbolo del operador $T_aT_b$ pertenece a $S^{m+l}_{\rho, \delta}(\R^n\times\Z^n)$, y sigue la siguiente expansión asimptótica
	\begin{equation*}
		\sum_{\gamma\geq0} \frac{1}{\gamma!}[\Delta^\gamma_\xi a(x, \xi)][D^{(\gamma)}_xb(x, \xi)].
	\end{equation*}
\end{theorem}
\begin{proof}
	Se tiene que el símbolo de la composición esta dado por
	\begin{align*}
		c(x, \xi) & :=  \sum_{\eta\in\Z^n}\int_{\T^n} 
		e^{i2\pi (x-y)\cdot(\eta-\xi)} a(x, \eta)b(y, \xi)\diff y \\ 
		& = \sum_{\eta\in\Z^n} e^{i2\pi x\cdot(\eta-\xi)}  a(x, \eta) \widehat{b}(\eta-\xi, \xi)\\ 
		& = \sum_{\eta\in\Z^n} e^{i2\pi x\cdot\eta}  a(x, \eta+\xi) \widehat{b}(\eta, \xi).
	\end{align*}
	Ahora, se utiliza la expansión de Taylor discreta para obtener que
	\begin{align*}
		c(x, \xi) &=  \sum_{\eta\in\Z^n} e^{i2\pi x\cdot\eta} \left[\sum_{|\gamma|<N}\frac{1}{\gamma!}\Delta^\gamma_\xi a(x, \xi)\eta^{(\gamma)} + R_N(x, \xi, \eta)
		\right] \widehat{b}(\eta, \xi)\\
		& = \sum_{|\gamma|<N}\frac{1}{\gamma!}\left[ \Delta^\gamma_\xi a(x, \xi) \right] \sum_{\eta\in\Z^n} e^{i2\pi x\cdot\eta}\widehat{b}(\eta, \xi)\eta^{(\gamma)}  + \sum_{\eta\in\Z^n} e^{i2\pi x\cdot\eta}R_N(x, \xi, \eta)  \widehat{b}(\eta, \xi) \\
		& = \sum_{|\gamma|<N}\frac{1}{\gamma!}\left[ \Delta^\gamma_\xi a(x, \xi) \right] D^{(\gamma)}_x b(x, \xi) + E_N(x, \xi).
	\end{align*}
	Ahora, solo queda estimar el término correspondiente al error. En efecto, se tiene que
	\begin{align*}
		|\Delta^{\alpha'}_\xi\partial^{\beta'}_x [e^{i2\pi x\cdot\eta}\widehat{b}(\eta, \xi)]| & =  \left|\Delta^{\alpha'}_\xi\partial^{\beta'}_x  e^{i2\pi x\cdot\eta} \int_{\T^n} e^{-i2\pi y\cdot\eta} b(y, \xi) \diff y
		\right| \\
		& =  \left|\Delta^{\alpha'}_\xi\partial^{\beta'}_x  e^{i2\pi x\cdot\eta}\angles{\eta}^{-r} \int_{\T^n}\angles{\eta}^r e^{-i2\pi y\cdot\eta} b(y, \xi) \diff y
		\right|\\
		& =  \left|\partial^{\beta'}_x  e^{i2\pi x\cdot\eta}\angles{\eta}^{-r}  \int_{\T^n} e^{-i2\pi y\cdot\eta}\Delta^{\alpha'}_\xi\left(1-\frac{\mathcal{L}_y}{4\pi^2}\right)^{r/2} b(y, \xi) \diff y
		\right|\\
		& \lesssim \angles{\eta}^{|\beta'|-r} \angles{\xi}^{l-\rho|\alpha'|+\delta r}.
	\end{align*}
	Para el residuo de la serie de Taylor discreta se tiene que 
	\begin{align*}
		|\Delta^{\alpha''}_\xi\partial^{\beta''}_x R_N(x, \xi, \eta)| & \lesssim 
		\angles{\eta}^N \max_{|\omega|=N,\, \nu\in Q(\eta)} | \Delta^{\omega+\alpha''}_\xi\partial^{\beta''}_x a(x, \xi+\nu)|\\
		& \lesssim \angles{\eta}^N \max_{|\omega|=N,\, \nu\in Q(\eta)} \angles{\xi+\nu}^{m-\rho(N + |\alpha''|) + \delta|\beta''|}.
	\end{align*}
	Entonces, por la fórmula de Leibniz discreta y tomando $\alpha' + \alpha''=\alpha$, $\beta'+\beta''=\beta$, el término del error puede ser acotado por
	\begin{equation*}
		|\Delta^{\alpha}_\xi\partial^{\beta}_x E_N(x, \xi)| \lesssim 
		\angles{\xi}^{l-\rho|\alpha'|+\delta r} \sum_{\eta\in\Z^n} \angles{\eta}^{N+|\beta'|-r} \max_{\nu\in Q(\eta)}  \angles{\xi+\nu}^{m-\rho(N + |\alpha''|) + \delta|\beta''|}.
	\end{equation*}
	Ahora, suponga que $|\eta|\leq|\xi|/2$, entonces el término de error puede ser acotado por
	\begin{equation*}
		\angles{\xi}^{l-\rho|\alpha'|+\delta r+m-\rho(N + |\alpha''|) + \delta|\beta''|+n} \max_{|\eta|\leq|\xi|/2}\angles{\eta}^{N+|\beta'|-r},
	\end{equation*}
	y al escoger $r = N + |\beta'|$, se obtiene que
	\begin{equation*}
		\angles{\xi}^{m + l -(\rho-\delta)N -\rho|\alpha| + \delta|\beta| + n}.
	\end{equation*}
	Por lo que este término puede ser aproximado por cualquier $\angles{\xi}^{-M}$ escogiendo el $N$ apropiado en vista que $\rho>\delta$. Por otra parte, si $|\eta|>|\xi|/2$, entonces para $N$ lo suficientemente grande se puede estimar el término del error por 
	\begin{align*}
		\angles{\xi}^{l-\rho|\alpha'| + \delta r} \sum_{|\eta|>|\xi|/2} \angles{\eta}^{N+|\beta'|-r} \lesssim \angles{\xi}^{l-\rho|\alpha'| + \delta r+N+|\beta'|-r-1},
	\end{align*}
	cuando $r$ es más grande que $N$. Además, dado que $\delta<1$, se puede escoger $r$ para estimar este término por cualquier $\angles{\xi}^{-M}$ . Lo que completa la prueba.
\end{proof}
Ahora, ha sido expuesto que la herramienta correspondiente al trabajar en el espacio de frecuencias de símbolos toroidales es el cálculo de diferencias discretas. Sin embargo, puede ser de interés extender las técnicas utilizadas en el análisis de símbolos euclideanos para obtener resultados similares. Para ello, se realiza un proceso conocido como la \textit{extensión del símbolo toroidal}, que consiste en una interpolación suave de un símbolo definido en $\T^n\times\Z^n$ para obtener uno definido en $\T^n\times\R^n$. A continuación se presetnan los detalles de dicho proceso. 
\begin{lemma}\label{lem:aux-extension}
	Existen funciones $\phi_\alpha\in\mathcal{S}(\R^n)$, para cada $\alpha\in\N_0^n$, y una función $\theta\in\mathcal{S}(\R^n)$, tales que 
	\begin{equation*}
		\mathcal{P}\theta(x) := \sum_{k\in\Z^n}\theta(x+k) = 1,
	\end{equation*}
	\begin{equation*}
		(\mathcal{F}_{\R^n}\theta)|_{\Z^n}(\xi) = \delta_0(\xi) \quad \text{ y } \quad \partial^\alpha_\xi (\mathcal{F}_{\R^n}\theta)(\xi) = \overline{\Delta}^\alpha_\xi \phi_\alpha(\xi),
	\end{equation*}
	para todo $\xi\in\Z^n$.
\end{lemma}
\begin{proof}
	Primero, considere el caso unidimensional. Sea $\theta\in C^\infty(\R)$, tal que 
	\begin{equation*}
		\supp\theta \subset (-1, 1), \quad \theta(-x) = \theta(x), \quad \theta(1-y) + \theta(y) = 1,
	\end{equation*}
	para $x\in\R$ y $0\leq y\leq1$. Note entonces que $\theta\in\mathcal{S}(\R)$, y por lo tanto $\mathcal{F}_{\R}\theta\in\mathcal{S}(\R)$ también. En particular, para $\xi\in\Z$, se tiene que
	\begin{align*}
		(\mathcal{F}_{\R}\theta)(\xi) & = \int_\R \theta(x)e^{-i2\pi x\cdot\xi} \diff x\\
		& = \int_0^1 [\theta(x-1)+\theta(x)]e^{-i2\pi x\cdot\xi} \diff x\\
		& = \delta_0(\xi).
	\end{align*}
	Ahora, si la $\phi_\alpha\in\mathcal{S}(\R)$ deseada existe, entonces debe satisfacer que 
	\begin{align*}
		\int_\R e^{i2\pi x\cdot\xi} (\mathcal{F}_\R\theta)(\xi)  \diff \xi & = 
		\int_\R e^{i2\pi x\cdot\xi} \overline{\Delta}^\alpha_\xi \phi_\alpha(\xi) \diff \xi \\
		& = (1-e^{i2\pi x})^\alpha  \int_\R e^{i2\pi x\cdot\xi} 
		\phi_\alpha(\xi) \diff \xi.
	\end{align*}
	Por lo que se obtiene la formula 
	\begin{equation*}
		(-i2\pi x)^\alpha \theta(x) = (1-e^{i2\pi x})^\alpha (\mathcal{F}_\R^{-1}\phi_\alpha)(x).
	\end{equation*}
	Entonces, se puede definir $\phi_\alpha$ como 
	\begin{equation*}
		\mathcal{F}_\R^{-1}\phi_\alpha)(x) := \begin{cases}
			\left(\frac{-i2\pi x}{1-e^{i2\pi x}}\right)^\alpha\theta(x), & 0<|x|<1, \\
			1, & x=0,\\
			0, & |x| \geq 1.
		\end{cases}
	\end{equation*}
	Para el caso $n$-dimensional se puede definir el mapa $x\mapsto \theta(x_1)\cdots\theta(x_n)$, que cumple las mismas propiedades. 
\end{proof}
\begin{remark}
	La clase de símbolos $S^m_{\rho,\delta}(\T^n\times\R^n)$ se puede definir como el conjunto de símbolos en la clase $S^m_{\rho,\delta}(\R^n\times\R^n)$ que son 1-periódicos respecto a $x$. 
\end{remark}
\begin{theorem}\label{theo:equivalence-symbols}
	Sea $0\leq\delta\leq1$ y sea $0<\rho\leq1$. El símbolo $\tilde{a}\in S^m_{\rho,\delta}(\T^n\times\Z^n)$ es un símbolo toroidal si y solo si existe un símbolo euclideano $a\in S^m_{\rho,\delta}(\T^n\times\R^n)$ tal que $\tilde{a} = a|_{\T^n\times\Z^n}$. Además, esta extensión es única modulo $S^{-\infty}(\T^n\times\R^n)$.
\end{theorem}
\begin{proof}
	($\Leftarrow$) Por el Teorema de Valor Medio se tiene que para cualquier multi-índice $\alpha\in\N_0^n$ 
	\begin{align*}
		\Delta^\alpha_\xi \partial^\beta_x \tilde{a}(x, \xi) &= \Delta^\alpha_\xi \partial^\beta_x {a}(x, \xi) \\
		& = \partial^\alpha_\xi \partial^\beta_x {a}(x, \xi)|_{\xi=\eta},
	\end{align*}
	para algún $\eta\in Q:= [\xi_1, \xi_1 + \alpha_1] \times \cdots\times[\xi_n, \xi_n+\alpha_n]$. Por lo que, se obtiene que 
	\begin{align*}
		|\Delta^\alpha_\xi \partial^\beta_x \tilde{a}(x, \xi)| & = |\partial^\alpha_\xi \partial^\beta_x {a}(x, \xi)|_{\xi=\eta}| \\
		& \lesssim_{\alpha\beta} \angles{\eta}^{m-\rho|\alpha|+\delta|\beta|} \\
		& \lesssim_\alpha \angles{\xi}^{m-\rho|\alpha|+\delta|\beta|}.
	\end{align*}
	($\Rightarrow$)  Sea $\theta\in\mathcal{S}(\R^n)$ como en el Lema \ref{lem:aux-extension}, y se define 
	\begin{equation*}
		a(x, \xi) := \sum_{\eta\in\Z^n} (\mathcal{F}_{\R^n}\theta)(\xi - \eta)\tilde{a}(x, \eta).
	\end{equation*}
	Se puede ver que cuando $\xi\in\Z^n$, entonces $a(x, \xi) = \tilde{a}(x, \xi)$. Además, se tiene que
	\begin{align*}
		|\partial^\alpha_\xi\partial^\beta_x a(x, \xi) | & = \left| \sum_{\eta\in\Z^n} \partial^\alpha_\xi(\mathcal{F}_{\R^n}\theta)(\xi - \eta)\partial^\beta_x\tilde{a}(x, \eta) \right| \\
		& = \left| \sum_{\eta\in\Z^n} \overline{\Delta}^\alpha_\xi\phi_\alpha(\xi-\eta)\partial^\beta_x\tilde{a}(x, \eta) \right|  \\
		& = \left| \sum_{\eta\in\Z^n} \phi_\alpha(\xi-\eta)\Delta^\alpha_\eta\partial^\beta_x\tilde{a}(x, \eta) \right|\\
		& \lesssim_{\alpha\beta} \sum_{\eta\in\Z^n} |\phi_\alpha(\xi-\eta)| \angles{\eta}^{m-\rho|\alpha|+\delta|\beta|} \\
		& \lesssim_{\alpha} \sum_{\eta\in\Z^n} \angles{\eta}^{-M}  \angles{\xi-\eta}^{m-\rho|\alpha|+\delta|\beta|}  \\
		& \leq \angles{\xi}^{m-\rho|\alpha|+\delta|\beta|} \sum_{\eta\in\Z^n} \angles{\eta}^{|m-\rho|\alpha|+\delta|\beta|| - M} \\
		& \lesssim_{\alpha\beta} \angles{\xi}^{m-\rho|\alpha|+\delta|\beta|}.
	\end{align*}
	Aquí se utilizó el hecho que $\phi_\alpha\in\mathcal{S}(\Z^n)$ y que $\angles{\xi-\eta}^q \leq \angles{\xi}^q \angles{\eta}^{|q|}$. Lo que completa la prueba de la existencia del símbolo. Para demostrar la unicidad, sean $a,b\in S^m_{\rho,\delta}(\T^n\times\R^n)$, tales que $a|_{\T^n\times\Z^n} = b|_{\T^n\times\Z^n}$. Entonces, se define $c:=a-b$, y para $\xi\in\R^n\setminus\Z^n$ se escoge $\eta\in\Z^n$ uno de sus puntos más cercanos. Por lo que se tiene la siguiente expansión de Taylor de primer orden
	\begin{align*}
		c(x, \xi) & = c(x, \eta) + \sum_{|\alpha|=1} r_\alpha(x, \xi, \xi-\eta)(\xi-\eta)^\alpha \\
		& = \sum_{|\alpha|=1} r_\alpha(x, \xi, \xi-\eta)(\xi-\eta)^\alpha,
	\end{align*}
	donde
	\begin{equation*}
		r_\alpha(x, \xi, \theta) = \int_0^1(1-t)\partial^\alpha_\xi c(x, \xi+t\theta)\diff t.
	\end{equation*}
	Entonces, se tiene que $|c(x, \xi)| \lesssim \angles{\xi}^{m-\rho}$, y este proceso puede aplicarse recursivamente a $c$ y a sus derivadas para concluir que $c\in S^{-\infty}(\T^n\times\R^n)$.
\end{proof}
\begin{definition}[Periodización]
	La \textit{periodización} de una función $f\in\mathcal{S}(\R^n)$ se define como
	\begin{equation*}
		\mathcal{P}f(x):= \sum_{k\in\Z^n} f(x + k).
	\end{equation*}
\end{definition}
\begin{theorem}[Periodización de operadores]
	Sea $a:=a(x, \xi)\in S^m_{\rho, \delta}(\R^n\times\R^n)$, una función 1-periodica en $x$ para todo $\xi$. Sea $\tilde{a}=a|_{\T^n\times\Z^n}$, entonces 
	\begin{equation*}
		(\mathcal{P}\circ T_a) f = (T_{\tilde{a}}\circ \mathcal{P})f,
	\end{equation*}
	para toda $f\in\mathcal{S}(\R^n)$.
\end{theorem}
\begin{proof}
	Se tiene que 
	\begin{align*}
		\mathcal{P}(T_af)(x) & = \sum_{k\in\Z^n}(T_af)(x+k)\\
		& = \sum_{k\in\Z^n}\int_{\R^n} e^{i2\pi(x+k)\cdot\xi}a(x+k, \xi)(\mathcal{F}_{\R^n}f)(\xi) \diff \xi \\
		& = \int_{\R^n}\left(\sum_{k\in\Z^n}e^{i2\pi k\cdot\xi}\right) e^{i2\pi x\cdot\xi}a(x, \xi)(\mathcal{F}_{\R^n}f)(\xi) \diff \xi \\
		& = \int_{\R^n}\delta_{\Z^n}(\xi) e^{i2\pi x\cdot\xi}a(x, \xi)(\mathcal{F}_{\R^n}f)(\xi) \diff \xi \\
		& = \sum_{\xi\in\Z^n}e^{i2\pi x\cdot\xi}a(x, \xi)(\mathcal{F}_{\R^n}f)(\xi) \\
		& = \sum_{\xi\in\Z^n}e^{i2\pi x\cdot\xi}a(x, \xi)\mathcal{F}_{\T^n}(\mathcal{P}f)(\xi)  \\
		& = (T_{\tilde{a}}\circ\mathcal{P})f(x).
	\end{align*}
	Lo que completa la prueba.
\end{proof}
Al combinar este resultado con el Teorema \ref{theo:equivalence-symbols}, se obtiene que 
\begin{corollary}
	Sea $0\leq\delta\leq1$, y sea $0<\rho\leq1$. Entonces, se tiene que 
	\begin{equation*}
		\Psi^m_{\rho,\delta}(\T^n\times\R^n) = \Psi^m_{\rho,\delta}(\T^n\times\Z^n).
	\end{equation*}
\end{corollary}

\chapter{Continuidad de operadores pseudo-diferenciales}
En este capítulo se presentan resultados clásicos y originales, obtenidos con Cardona \cite{Cardona:Martinez, cardona-martinez-II, cardona-martinez-III}, acerca de la continuidad de los operadores pseudo-diferenciales discutidos anteriormente. Se incluyen ciertos resultados en $\R^n$, pero se presenta un mayor énfasis para los resultados en el caso del toro $\T^n$.
\section{Continuidad en espacios de Lebesgue}
\subsection{Continuidad de operadores pseudo-diferenciales en espacios de Lebesgue $L^p$}
\begin{theorem}\label{theo:kernel-estimates}
	Sea $a\in S^m_{\rho,\delta}(\R^n\times\R^n)$, entonces su kernel de Schwartz cumple que 
	\begin{equation*}
		|k(x, y)| \lesssim_N |x-y|^{-N},
	\end{equation*}
	para $x\neq y$, y para cualquier $N>(m+n)/\rho$. 
\end{theorem}
\begin{proof}
	Con el argumento de integración por partes se tiene que 
	\begin{align*}
		(2\pi i)^{|\gamma|} (x-y)^\gamma k(x, y) & = 
		\int_{\R^n} \partial^\gamma_\xi [e^{2\pi i (x-y)\cdot\xi}]a(x, \xi) \diff \xi \\
		& = (-1)^{|\gamma|} \int_{\R^n} e^{2\pi i (x-y)\cdot\xi} \partial^\gamma_\xi [a(x, \xi)] \diff \xi.
	\end{align*}
	Entonces, si se fija $|\gamma|=N$, se tiene que 
	\begin{equation*}
		|x-y|^{-N}|k(x, y)| \lesssim \int_{\R^n} \angles{\xi}^{m-\rho N} \diff \xi,
	\end{equation*}
	que es finito cuando $m-\rho N < -n$.
\end{proof}
\begin{theorem}\label{theo:L2-boundedness-S0}
	Sea $a\in S^0_{1,0}(\R^n\times\R^n)$, entonces $T_a$ extiende a un operador acotado de $L^2(\R^n)$ en sí mismo.
\end{theorem}
\begin{proof}
	Primero, suponga que $a:=a(x, \xi)$ tiene soporte compacto respecto a $x$. Además, es suficiente demostrar este enunciado para funciones $f\in\mathcal{S}(\R^n)$ debido a que este es secuencialmente denso en $L^2(\R^n)$ y a un simple argumento analítico. Entonces, se define a la transformada de Fourier de $a$ respecto a $x$ como 
	\begin{equation*}
		\widehat{a}(\lambda, \xi) := \mathcal{F}\{a(\cdot, \xi)\}(\lambda).
	\end{equation*}
	Ahora, note que para cualquier multi-índice $\alpha\in\N^n_0$ se tiene que 
	\begin{equation*}
		(2\pi i\lambda)^\alpha \widehat{a}(\lambda, \xi) = 
		\int_{\R^n} e^{-2\pi ix\cdot\lambda}\partial^\alpha_x a(x, \xi) \diff x.
	\end{equation*}
	Entonces, $|(2\pi i\lambda)^\alpha\widehat{a}(\lambda, \xi)|\leq C_\alpha$ uniformemente en $\xi$. Por otra parte, se tiene que
	\begin{align*}
		T_af(x) &= \int_{\R^n} e^{2\pi ix\cdot\xi} \int_{\R^n}e^{2\pi ix\cdot\lambda} 
		\widehat{a}(\lambda, \xi) \widehat{f}(\xi)\diff \lambda \diff \xi \\
		& = \int_{\R^n} e^{2\pi ix\cdot\lambda} T_{\widehat{a}} f(x) \diff \lambda.
	\end{align*}
	Entonces, gracias a la identidad de Plancherel 
	\begin{align*}
		\|T_{\widehat{a}}f\|_{L^2}  &= \|\mathcal{F}(T_{\widehat{a}}f)\|_{L^2} = 
		\|\widehat{a}(\lambda, \cdot)\widehat{f}\|_{L^2} \\
		& \leq \sup_{\xi\in\R^n}|\widehat{a}(\lambda, \xi)|\|\widehat{f}\|_{L^2} 
		\lesssim \angles{\lambda}^{-N}\|f\|_{L^2},
	\end{align*}
	para cualquier $N\geq0$. Por lo que
	\begin{align*}
		\|T_af\|_{L^2} & \leq \int_{\R^n}\|T_{\widehat{a}}f\|_{L^2} \diff \lambda \\
		& \lesssim \int_{\R^n}\angles{\lambda}^{-N}\|f\|_{L^2} \diff \lambda \lesssim \|f\|_{L^2},
	\end{align*}
	cuando se escoge $N>n$. Ahora, considere el caso en el que el símbolo no necesariamente tiene soporte compacto. Para ello, se fija $x_0=0$ y se descompone $f =f_1+f_2$, donde $f_1$ y $f_2$ son funciones suaves tales que $|f_1|\leq|f|$, $|f_2|\leq|f|$, y que $\supp f_1 \subset \{x\in\R^n:|x|\leq3\}$, $\supp f_2\subset \{x\in\R^n:|x|\geq2\}$. Fije $\eta\in C^\infty_0(\R^n)$ tal que sea igual a uno en la bola unitaria. Entonces 
	\begin{align*}
		\int_{\{|x|\leq1\}} |T_af_1(x)|^2 \diff x & = \int_{\R^n} |T_{\eta a}f_1(x)|^2 \diff x \\
		& \lesssim \int_{\R^n} |f_1(x)|^2 \diff x \\ 
		& \lesssim \int_{|x|\leq3}|f(x)|^2\diff x. 
	\end{align*}
	Ahora, por el Teorema \ref{theo:kernel-estimates}, se tiene que 
	\begin{equation*}
		|k(x, y)| \lesssim_N |x-y|^{-N} \lesssim_N |y|^{-N},
	\end{equation*}
	dado que $|y|\geq2$ y $|x|\leq1$, que implica que $|x-y|\geq1$. Entonces, se obtiene que
	\begin{align*}
		|T_af_2(x)| & \leq \int_{\{|y|\geq2\}} |k(x, y)||f_2(y)|\diff y \\
		& \lesssim  \int_{\{|y|\geq2\}} \frac{|f_2(y)|}{|y|^N}\diff y \\
		& \lesssim \int_{\R^n} \frac{|f(y)|}{\angles{y}^N}\diff y \\
		& \leq \left(\int_{\R^n} \frac{|f(y)|^2}{\angles{y}^N}\diff y\right)^{1/2}
		\left(\int_{\R^n} \frac{1}{\angles{y}^N}\diff y\right)^{1/2} \\
		&\lesssim \left(\int_{\R^n} \frac{|f(y)|^2}{\angles{y}^N}\diff y\right)^{1/2},
	\end{align*}
	cuando se elige $N>n$. Por lo que se tiene que
	\begin{equation*}
		\int_{\{|x|\leq1\}} |T_af_2(x)|^2 \diff x \lesssim \int_{\R^n} \frac{|f(y)|^2}{\angles{y}^N}\diff y.
	\end{equation*}
	Note que los estimativos de $f_1$ y $f_2$ solo dependen de la dimensión $n$ y las constantes de las desigualdades simbólicas de $a$. Es decir, no depende de $x_0$, y se puede escribir que 
	\begin{equation*}
		\int_{\{|x-x_0|\leq1\}} |T_af(x)|^2 \diff x \lesssim \int_{\R^n} \frac{|f(x)|^2\diff x}{\angles{x-x_0}^N}.
	\end{equation*}
	Por lo tanto, si $\chi_A$ es la función característica del conjunto $A$, entonces al integrar respecto a $x_0$ y cambiar el orden de integración se obtiene que
	\begin{align*}
		\int_{\R^n} \int_{\R^n} \chi_{\{|x-x_0|\leq1\}} |T_af(x)|^2\diff x \diff x_0 &\lesssim_N \int_{\R^n} \int_{\R^n}  \frac{|f(x)|^2\diff x}{\angles{x-x_0}^N} \diff x_0 \\
		|B(1)| \int_{\R^n} |T_af(x)|^2\diff x &\lesssim_N \int_{\R^n} |f(x)|^2\diff x.
	\end{align*}
	Lo que completa la prueba.
\end{proof}
\begin{remark}
	Los operadores $T\in\Psi^0_{1, 0}(\R^n\times\R^n)$ tienen propiedades que los hacen bastante especiales. En realidad, son ejemplos de los tipos de operadores que dieron inicio a la teoría de operadores integrales singulares de Calderón-Zygmund \cite{calderon-zygmund}. En particular, se puede utilizar esta teoría para concluir que los operadores en $\Psi^0_{1, 0}(\R^n\times\R^n)$ son continuos de $L^p$ en sí mismo, para $1<p<\infty$. Por lo que se enuncia este resultado sin demostración.
\end{remark}
\begin{theorem}\label{theo:Lp-boundedness-S0}
	Sea $T\in\Psi^0_{1, 0}(\R^n\times\R^n)$, entonces $T$ se extiende a un operador acotado de $L^p(\R^n)$ en sí mismo.
\end{theorem}
Ahora, se presenta un resultado de continuidad $L^2$ para el caso del toro. En este caso se relajan los requerimientos de regularidad del símbolo, en vista de la compacidad del toro.
\begin{theorem}\label{theo:L2-boundedness-T-C}
	Sea $k\in\N$, tal que $k>n/2$, y sea $a:=a(x,\xi):\T^n\times\Z^n\to\C$, tal que
	\begin{equation*}
		|\partial^\beta_x a(x, \xi)| \leq C,
	\end{equation*}
	para todo $x, \xi$, y $|\beta|\leq k$. Entonces $T_a$ extiende a un operador acotado en $L^2(\T^n)$.
\end{theorem}
\begin{proof}
	Primero, se define 
	\begin{equation*}
		A_yf(x) := \sum_{\xi\in\Z^n} \int_{\T^n} e^{i2\pi (x-z)\cdot\xi} a(y, \xi) f(z) \diff z, 
	\end{equation*}
	de tal manera que $A_xf(x) = T_af(x)$. Entonces se tiene que
	\begin{equation*}
		\|T_af\|_{L^2}^2 = \int_{\T^n} |T_af(x)|^2 \diff x \leq \int_{\T^n} \sup_{y\in\T^n} |A_yf(x)|^2 \diff x.
	\end{equation*}
	Al aplicar el Teorema de encaje de Sobolev, se obtiene que
	\begin{align*}
		\|T_af\|_{L^2}^2 & \lesssim \int_{\T^n} \sum_{|\alpha|\leq k} \int_{\T^n} |\partial^\alpha_yA_yf(x)|^2 \diff y \diff x \\
		& \leq \sum_{|\alpha|\leq k} \sup_{y\in\T^n}  \int_{\T^n} |\partial^\alpha_yA_yf(x)|^2 \diff x\\
		& =  \sum_{|\alpha|\leq k} \sup_{y\in\T^n} \|\partial^\alpha_yA_yf(x)\|_{L^2}^2 \\
		& \leq \sum_{|\alpha|\leq k} \sup_{y\in\T^n}  \sup_{\xi\in\Z^n} |\partial^\alpha_y a(y, \xi)|^2 \|f\|_{L^2}^2.
	\end{align*}
	Lo que completa la prueba.
\end{proof}
A continuación se presentan una serie de resultados de continuidad $L^p$ tanto en $\R^n$ como en $\T^n$. Como el enfoque de este trabajo es en el estudio de continuidad de operadores pseudo-diferenciales en el toro, entonces no se incluirán las demostraciones de los resultados en el caso euclideano. Sin embargo, las técnicas que se utilizarán en este caso pueden ser modificadas para recuperar las demostraciones de los casos euclideanos. El primer resultado que se presenta es uno clásico, demostrado por Fefferman \cite{fefferman-Lp} en el contexto euclideano. 
\begin{theorem}[Fefferman]
	Sean $0\leq\delta<1-\varepsilon<1$, y sea $T\in \Psi^m_{1-\varepsilon, \delta}(\R^n\times\R^n)$. Suponga que 
	\begin{equation*}
		m \leq - n\varepsilon\left|\frac{1}{p} - \frac{1}{2}\right|,
	\end{equation*}
	entonces el operador $T$ extiende a un operador continuo de $L^p(\R^n)$ en sí mismo para $1<p<\infty$.
\end{theorem} 
Este resultado fue extendido al caso toroidal para el rango $2\leq p<\infty$ por Delgado \cite{delgado}. Sin embargo, se relajan los requerimientos del símbolo como fue el caso en el Teorema \ref{theo:L2-boundedness-T-C}.
\begin{theorem}[Delgado]
	Sea $0<\varepsilon<1$, y sea $k\in\N$ tal que $k>n/2$. Sea $a:\T^n\times\Z^n\to\C$ un símbolo tal que 
	\begin{equation*}
		|\Delta^\alpha_\xi a(x, \xi)| \lesssim \angles{\xi}^{-n\varepsilon/2-(1-\varepsilon)|\alpha|}, \quad \quad |\partial^\beta_x a(x, \xi)| \lesssim \angles{\xi}^{-n\varepsilon/2},
	\end{equation*}
	para $|\alpha|,|\beta|\leq k$. Entonces el operador $T_a$ extiende a un operador acotado de $L^p(\T^n)$ en sí mismo para $2\leq p<\infty$.
\end{theorem}
\begin{remark}
	En el resultado de Delgado se puede entender a $a(x, \xi)$ como un símbolo con regularidad limitada, pero se puede considerar informalmente como un símbolo en una clase de H\"ormander con $\rho=1-\varepsilon$ y $\delta=0$.
\end{remark}
Note que en ambos casos se requiere $\delta < \rho$. Esta restricción puede ser salvada gracias al trabajo de \'Alvarez y Hounie \cite{alvarez-hounie}. A partir de este momento, se denota $\lambda := \max\{(\delta - \rho)/2, 0\}$.
\begin{theorem}[\'Alvarez y Hounie]\label{theo:Alvarez-Hounie}
	Sea $0\leq\delta<1$, sea $0<\rho\leq1$, y sea $T\in\Psi^m_{\rho, \delta}(\R^n\times\R^n)$. Suponga 
	\begin{equation*}
		m \leq -n \left[(1-\rho)\left|\frac{1}{p} - \frac{1}{2}\right|+\lambda\right],
	\end{equation*}
	entonces el operador $T$ extiende a un operador continuo de $L^p(\R^n)$ en sí mismo. 
\end{theorem}
El análogo toroidal de este resultado fue demostrado con Cardona en 
\cite{Cardona:Martinez}. Aquí se presenta la prueba del mismo siguiendo el mismo esquema que \'Alvarez y Hounie. Primero, se demuestran unos estimativos del kernel de Schwarz de operadores pseudo-diferenciales toroidales bastante útiles.
\begin{remark}
	Ahora, vale la pena indicar la razón por la que los resultados en el toro no son consecuencia de los del caso euclideano. Las clases $H^1$ y $\mathrm{BMO}$, no son estables bajo la multiplicación de funciones test, por lo que no es posible tratar al toro como una variedad mediante particiones de la unidad. Además, los operadores pseudo-diferenciales con símbolos en las clases de Hörmander no son estables bajo cambios de coordenadas cuando $\rho>1-\delta$. Esto justifica el hecho de que se considere al toro $\T^n$ como un caso distinto a $\R^n$ y se estudien los resultados por separado.
\end{remark}
\begin{theorem}\label{theo:kernel-properties}
	Sea $T \in \Psi^m_{\rho, \delta}(\mathbb{T}^n \times \mathbb{Z}^n) $, con $0 < \rho \leq 1$, $0 \leq \delta < 1$, con símbolo $p(x, \xi)$ y con kernel
	\begin{equation}
		k(x, y) := \sum_{\xi \in \mathbb{Z}^n} e^{i2\pi(x - y) \cdot \xi} p(x, \xi).
	\end{equation}
	\begin{enumerate}
		\item[(a)] (Propiedad pseudo-local) $k$ es suave fuera de la diagonal. Además, dados $\alpha, \beta \in \mathbb{N}^n_0$, entonces para cualquier $N > (m + n + |\alpha + \beta|)/\rho$ se tiene
		\begin{equation}
			\sup_{x \neq y} |x - y|^N |\partial^\alpha_x \partial^\beta_y k(x, y)| = C_{\alpha\beta N} < \infty.
			\label{eq:teo3-1a}
		\end{equation}
		\item[(b)] Suponiendo que $p$ tiene soporte compacto en $\xi$ uniformemente en $x$, entonces $k$ es suave y dados $\alpha, \beta \in \mathbb{N}^n_0$, se tiene que 
		\begin{equation}
			|\partial^\alpha_x \partial^\beta_y k(x, y)| \lesssim_{\alpha\beta} \langle x - y \rangle^{-N}.
		\end{equation}
		\item[(c)] Suponiendo $m + M + n < 0$ para algún $M \in \mathbb{Z}^+$, entonces $k$ es una función continua acotada con derivadas continuas acotadas hasta el orden $M$.\\
		\item[(d)] Suponiendo $m + M + n = 0$ para algún $M \in \mathbb{Z}^+$, entonces existe $ C >0$ tal que 
		\begin{equation}
			\sup_{|\alpha + \beta| = M} |\partial^\alpha_x \partial^\beta_y k(x, y)| \leq C | \log |x-y| | \; , \quad x \neq y.
		\end{equation} 
	\end{enumerate}
\end{theorem}
\begin{proof}
	Primero, se observa que existe $\tilde{p} \in S^m_{\rho, \delta}(\mathbb{T}^n \times \mathbb{R}^n)$ tal que $\tilde{p}|_{\mathbb{T}^n \times \mathbb{Z}^n} = p$ y con $T_{\tilde{p}} = T$. En consecuencia, el kernel de Schwarz puede considerarse como $k(x, y) = \int_{\mathbb{R}^n} e^{i2\pi(x-y)\cdot \xi} \tilde{p}(x, \xi)\diff\xi $. Ahora, las derivadas del kernel se ven de la siguiente manera:
	\begin{equation}
		\partial^\alpha_x \partial^\beta_y k(x, y) = \int_{\mathbb{R}^n} (-i2\pi\xi)^\beta e^{i2\pi(x-y)\cdot \xi} \sum_{\omega \leq \alpha} C_\omega (i2\pi\xi)^{\alpha - \omega} \partial^\omega_x \tilde{p}(x, \xi) \diff\xi,
		\label{eq:derivada-nucleo}
	\end{equation}
	que es el núcleo de un operador con símbolo de orden $m + |\alpha + \beta|$. Entonces, es suficiente probar los resultados cuando $|\alpha + \beta| = 0$.
	\begin{itemize}
		\item[(a)] La continuidad del kernel $k(x, y)$ se prueba en \cite[Teorema~4.3.6]{ruzhansky-turunen}. Por integración por partes se tiene que
		\begin{align*}
			(i2\pi)^{|\gamma|}(x - y)^\gamma k(x, y) =& \int_{\mathbb{R}^n} \partial^\gamma_\xi \left[e^{i2\pi(x-y)\cdot \xi} \right]  \tilde{p}(x, \xi) \diff\xi \\
			= & (-1)^{|\gamma|} \int_{\mathbb{R}^n} e^{i2\pi(x-y)\cdot \xi} \partial^\gamma_\xi \left[\tilde{p}(x, \xi) \right] \diff\xi .
		\end{align*}
		
		Por lo tanto, si se fija $|\gamma| = N$, se obtiene
		\begin{align*}
			|i2\pi|^N|x-y|^N| k(x, y)| \leq & \int_{\mathbb{R}^n} \langle\xi\rangle^{m-\rho N} \diff \xi.
		\end{align*}
		La última integral es finita cuando $N > (m + n)/\rho$, probando el resultado.\\
		\item[(b)] Se observa que el núcleo sería una suma finita de funciones continuas, probando la continuidad de $k(x, y)$. Además, $\tilde{p}$ tendría el mismo soporte de $p$. Por lo tanto, la última integral anterior sería finita sin ninguna restricción sobre $N$.\\
		\item[(c)] Sea $ m < -n$. Entonces, se tiene la serie finita
		\begin{align*}
			| k(x, y)| \lesssim &  \sum_{\xi \in \mathbb{Z}^n}\langle\xi\rangle^m,
		\end{align*}
		probando la acotación de $k(x, y)$.\\
		\item[(d)] Primero se observa que por (a), (b) es suficiente probar la estimación cuando $|x-y|<1$ y si $\tilde{p}(x, \xi)$ se anula para $|\xi|<1$ uniformemente en $x$. Sea $m + n = 0$ y sea $\varphi \in C_0^\infty(\mathbb{R})$ con soporte en $[0, 1]$, tal que $\int \varphi=1$, y se define 
		\begin{equation*}
			k(x, y, t) := \int_{\mathbb{R}^n} e^{i2\pi(x-y)\cdot \xi} \tilde{p}(x, \xi) \varphi(\langle\xi\rangle-t) \diff\xi,
		\end{equation*}
		de modo que
		\begin{equation*}
			k(x, y) = \int_1^\infty k(x, y, t)\diff t.
		\end{equation*}
		Luego, usando integración por partes, se tiene que
		\begin{align*}
			(i2\pi)^{|\gamma|}(x - y)^\gamma k(x, y, t) = & \int_{\mathbb{R}^n} \partial^\gamma_\xi\left[ e^{i2\pi(x-y)\cdot \xi} \right] \tilde{p}(x, \xi)\varphi(\langle\xi\rangle - t) \diff \xi \\
			= & (-1)^{|\gamma|} \int_{\mathbb{R}^n} e^{i2\pi(x-y)\cdot \xi} \sum_{\omega \leq \gamma}C_\omega \partial^\omega_\xi \tilde{p}(x, \xi) \partial^{\gamma - \omega}\varphi(\langle\xi\rangle-t) \diff \xi.
		\end{align*}
		Dado que $\langle\xi\rangle \sim t$ en el soporte de $\varphi(\cdot-t)$, que tiene volumen estimado por $Ct^{n-1}$, se obtiene para $|\gamma|=N$
		\begin{align*}
			|x-y|^N|k(x, y, t)|  & \lesssim \int_{\supp \varphi(\cdot - t)} \langle\xi\rangle^{m - \rho N} \diff \xi \\
			 & \lesssim \int_{\supp \varphi(\cdot - t)} t^{m-\rho N}\diff\xi \leq Ct^{m+n - \rho N-1}.
		\end{align*}
		Sumando las estimaciones para $N = 0, 1$, se obtiene 
		\begin{equation*}
			|k(x, y, t)| \lesssim \frac{t^{ -\rho - 1}}{t^{-\rho } + |x - y|}.
		\end{equation*}           
	\end{itemize} Entonces, el resultado de evaluar la integral anterior es la estimación deseada $|k(x, y)| \leq  C|\log |x-y|| $ para $x \neq y$.
\end{proof}
Ahora, se enuncia un resultado de continuidad en $L^2(\T^n)$, cuya demostración es muy similar a la realizada por Hounie en \cite{hounie}.
\begin{theorem}\label{theo:22boundT}
	Sea $\tilde{p}:\mathbb{T}^n \times \mathbb{R}^n \rightarrow \mathbb{C}$ un símbolo tal que para $0<\rho\leq1$, $0\leq \delta < 1$, $m\leq -n \lambda$ y $|\alpha|,|\beta| \leq \lceil n/2\rceil$ satisface: 
	\begin{equation}
		\left|\partial^\alpha_\xi \partial^\beta_x \tilde{p}(x, \xi)\right| \leq C_{\alpha\beta}\langle\xi\rangle^{m-\rho|\alpha| + \delta|\beta|}.
	\end{equation}
	Entonces $T_{\tilde{p}}$ es acotado de $L^2(\mathbb{T}^n)$ en sí mismo con  norma proporcional al las mejores cotas $C_{\alpha\beta}$.
\end{theorem}
\begin{theorem}
	Sea $T \in \Psi^m_{\rho, \delta}(\mathbb{T}^n \times \mathbb{Z}^n) $, con $0 < \rho \leq 1$, $0 \leq \delta < 1$, $m \leq - n [(1-\rho)/2 + \lambda] $, entonces $T$ es una aplicación continua
	\begin{itemize}
		\item[(a)]  de $L^2(\mathbb{T}^n)$ en $L^{2/\rho}(\mathbb{T}^n)$,
		\item[(b)] de $L^{2/(2-\rho)}(\mathbb{T}^n)$ en $L^{2}(\mathbb{T}^n)$.
	\end{itemize}
	\label{teo:cotas-L}
\end{theorem}

\begin{proof}
	Primero se observa que $J^{n(1-\rho)/2}T$ y $TJ^{n(1-\rho)/2}$ tienen orden $\leq -n\lambda$, por lo que son acotados en $L^2(\mathbb{T}^n)$ por el Teorema \ref{theo:22boundT}. Además, por la desigualdad de Hardy-Littlewood-Sobolev, se tiene que $J^{-n(1-\rho)/2}$ es una aplicación continua de $L^2(\mathbb{T}^n)$ en $L^{2/\rho}(\mathbb{T}^n)$ y de $L^{2/(2-\rho)}(\mathbb{T}^n)$ en $L^2(\mathbb{T}^n)$. Por lo tanto, 
	\begin{equation*}
		\| Tf \|_{L^{2/\rho}} = \| J^{-n(1-\rho)/2}J^{n(1-\rho)/2}Tf\|_{L^{2/\rho}} \lesssim \| J^{n(1-\rho)/2}Tf \|_{L^2} \lesssim \| f \|_{L^2},
	\end{equation*}
	y 
	\begin{equation*}
		\| Tf \|_{L^2} = \| TJ^{n(1-\rho)/2}J^{-n(1-\rho)/2}f\|_{L^2} \lesssim \| J^{-n(1-\rho)/2}f \|_{L^2} \lesssim \|f\|_{L^{2/(2-\rho)}}.
	\end{equation*}
	Por lo tanto, se prueba el resultado deseado.
\end{proof}
\begin{theorem}	\label{theo:kernel-estimate-sigma}
	Sea $T \in \Psi^m_{\rho, \delta}(\mathbb{T}^n \times \mathbb{Z}^n) $, con $0 < \rho \leq 1$, $0 \leq \delta < 1$, con símbolo $p:=p(x, \xi)$ y con núcleo $k:=k(x, y)$. 
	\begin{itemize}
		\item[(a)] Para cualquier $z \in \mathbb{T}^n$ fijo, y $\sigma \geq \varepsilon > 0$, se tienen las desigualdades para el kernel:
		\begin{equation}
			\sup_{|y-z| \leq \sigma} \int_{|x-z| > 2\sigma}|k(x, y) - k(x, z)|\diff x \leq C_\varepsilon,
		\end{equation}
		\begin{equation}
			\sup_{|y-z| \leq \sigma} \int_{|x-z| > 2\sigma}|k(y, x) - k(z, x)|\diff x \leq C_\varepsilon.
		\end{equation}
		\item[(b)] Si $m \leq -n [(1-\rho)/2 + \lambda] $, y $\sigma < 1$, se tiene para cualquier $z \in \mathbb{T}^n$ fijo, 
		\begin{equation}
			\sup_{|y-z| \leq \sigma} \int_{|x-z| > 2\sigma^\rho}|k(x, y) - k(x, z)|\diff x \leq C.
		\end{equation}
		\item[(c)] Si $m \leq -n (1 - \rho) / 2$, y $\sigma < 1$, se tiene para cualquier $z \in \mathbb{T}^n$ fijo,
		\begin{equation}
			\sup_{|y-z| \leq \sigma} \int_{|x-z| > 2\sigma^\rho}|k(y, x) - k(z, x)|\diff x \leq C.
		\end{equation} 
	\end{itemize}
\end{theorem}
\begin{proof}
	\begin{itemize}
		\item[(a)] Primero, se observa que $|x-y| \geq |x - z| - |z - y| > \sigma$ en el dominio de evaluación. Luego, por la desigualdad triangular y (\ref{eq:teo3-1a}) se tiene
		\begin{equation*}
			\sup_{|y-z| \leq \sigma} \int_{|x-z| > 2\sigma}|k(x, y) - k(x, z)|\diff x  
		\end{equation*}
		\begin{equation*}
			\leq \sup_{|y-z| \leq \sigma} \int_{|x-z| > 2\sigma}|k(x, y)|\diff x + \sup_{|y-z| \leq \sigma} \int_{|x-z| > 2\sigma}|k(x, z)|\diff x
		\end{equation*}
		\begin{equation*}
			\lesssim  \int_{|x-y|>\sigma} |x-y|^{-N} \diff x + \int_{|x-z|>\sigma} |x-z|^{-N} \diff x
		\end{equation*}
		\begin{equation*}
			\leq  \int_{\mathbb{T}^n} \sigma^{-N} \diff x + \int_{\mathbb{T}^n} \sigma^{-N} \diff x \leq C_\varepsilon.
		\end{equation*}
		
		\item[(b)] Como antes, sea $\tilde{p}$ el símbolo correspondiente en $\mathbb{T}^n \times \mathbb{R}^n$.  Sea $\varphi \in C_0^\infty(\mathbb{R})$ con soporte contenido en $[1/2, 1]$, tal que 
		\begin{equation*}
			\int_0^\infty\varphi(1/t)/t \diff t =\int_1^2\varphi(1/t)/t \diff t = 1.
		\end{equation*} 
		Se define 
		\begin{equation*}
			k(x, y, t) := \int_{\mathbb{R}^n} e^{i2\pi(x - y) \cdot \xi} \tilde{p}(x, \xi) \varphi(\langle\xi\rangle/t) \diff \xi,
		\end{equation*}
		de modo que  
		\begin{equation*}
			k(x, y) = \int_0^\infty k(x, y, t) \diff t = \int_1^\infty k(x, y, t) \diff t.
		\end{equation*}
		Sea $N > n/2$ un entero, entonces se obtienen las estimaciones por Cauchy-Schwarz
		\begin{equation*}
			\int_{|x-z|>2\sigma^\rho} |k(x, y, t) - k(x, z, t)| \diff x
		\end{equation*}
		\begin{equation*}
			\leq  \left[ \int_{\mathbb{T}^n} (1+t^{2\rho}|x-z|^2)^N |k(x, y, t) - k(x, z, t)| \diff x  \right]^{1/2} \left[ \int_{\mathbb{T}^n} (1+t^{2\rho}|x-z|^2)^{-N} \diff x \right]^{1/2} 
		\end{equation*}
		\begin{equation}
			\lesssim \left[ \int_{\mathbb{T}^n} (1+t^{2\rho}|x-z|^2)^N |k(x, y, t) - k(x, z, t)| \diff x  \right]^{1/2}t^{-\rho n/2}.
			\label{eq:primera-cota-b}
		\end{equation}
		Así, para $|\alpha| \leq N$ se tiene que
		\begin{equation*}
			t^{\rho|\alpha|}(x-z)^\alpha \int_{\mathbb{R}^n} \left[ e^{i2\pi(x - y) \cdot \xi} - e^{i2\pi(x - z) \cdot \xi} \right]\tilde{p}(x, \xi)\varphi(\langle\xi\rangle/t)\diff \xi 
		\end{equation*}
		\begin{equation*}
			=t^{\rho|\alpha|}(x-z)^\alpha \int_{\mathbb{R}^n} e^{i2\pi(x - z) \cdot \xi} \left[ e^{i2\pi(z - y) \cdot \xi} - 1 \right]\tilde{p}(x, \xi)\varphi(\langle\xi\rangle/t)\diff \xi 
		\end{equation*}
		\begin{equation*}
			=\sum_{\beta\leq\alpha} C_{\alpha\beta} t^{\rho|\alpha|} \int_{\mathbb{R}^n} e^{i2\pi(x - z) \cdot \xi} \partial^\beta_\xi \left[ 
			\left(e^{i2\pi(z - y) \cdot \xi} - 1 \right) \tilde{p}(x, \xi) \right] \partial^{\alpha-\beta}_\xi \varphi(\langle\xi\rangle/t)\diff \xi.
		\end{equation*}
		Ahora, $|e^{i2\pi(x - z) \cdot \xi}-1| \leq |x-z||\xi|\leq t\sigma$ en el soporte de $\varphi(\langle\xi\rangle/t)$. Por otro lado, 
		\begin{equation*}
			|\partial^\gamma_\xi e^{i2\pi(x - z) \cdot \xi}| \lesssim_\gamma|y-z|^{|\gamma|} \lesssim_\gamma \sigma ^{|\gamma|} \lesssim \langle\xi\rangle^{-|\gamma|}(t\sigma)^{|\gamma|}.
		\end{equation*}
		Asuma que $t\sigma < 1$, entonces para cualquier $\chi \in C_0^\infty(\mathbb{R}^n)$, que es igual a uno en el soporte de $\varphi(\langle\xi\rangle/t)$ el conjunto
		\begin{equation*}
			\Sigma_{\alpha\beta}=\left\{ \langle\xi\rangle^{n(1 - \rho)/2 + \rho|\beta|} \partial^\beta_\xi \left[\left( e^{i2\pi(x - z) \cdot \xi}-1 \right) 
			\tilde{p}(x, \xi) \right] \chi(\langle\xi\rangle/t) : |y-z| < \sigma,\; z \in \mathbb{T}^n \right\}
		\end{equation*}
		tiene medida acotada por $\langle\xi\rangle^{n(1 - \rho)/2 + \rho|\beta|} (t\sigma)\langle\xi\rangle^{m-\rho |\beta|} \leq t\sigma \langle\xi\rangle^{-n\lambda} $. Así, se puede considerar $\tilde{p}(x, \xi)$ como un símbolo en $\mathbb{R}^n\times\mathbb{R}^n$ y por el Teorema \ref{theo:Alvarez-Hounie} cada uno de los operadores correspondientes con símbolos $$\langle\xi\rangle^{n(1 - \rho)/2 + \rho|\beta|} \partial^\beta_\xi \left[\left( e^{i2\pi(x - z) \cdot \xi}-1 \right) 
		\tilde{p}(x, \xi) \right] \chi(\langle\xi\rangle/t),$$ en el conjunto $\Sigma_{\alpha\beta}$ son acotados en $L^2(\mathbb{R}^n)$ con norma estimada por $t\sigma$.
		Por lo tanto, (\ref{eq:primera-cota-b}) puede estimarse usando la identidad de Plancherel mediante 
		\begin{equation*}
			t\sigma\sum_{\beta\leq\alpha,\;|\alpha|\leq N} C_{\alpha\beta}t^{\rho|\alpha|}\left\|  
			\langle\xi\rangle^{-n(1-\rho)/2 - \rho|\beta|} t^{-|\alpha-\beta|} \partial^{\alpha-\beta}_\xi \varphi(\langle\xi\rangle/t) \right\|_{L^2(\mathbb{R}^n)}t^{-\rho n/2} 
		\end{equation*}
		\begin{equation}
			\leq t\sigma t^{\rho|\alpha|} t^{-n(1-\rho)/2 - \rho|\beta|}t^{-|\alpha-\beta|}t^{-\rho n/2} \leq Ct\sigma, 
			\label{eq:tsigma<1-b}
		\end{equation}
		como $\angles{\xi}\sim t$ en el soporte de la función. Ahora, se elimina la restricción $t\sigma < 1$. Para $|\alpha| = N$ se tiene que
		\begin{equation*}
			\int_{|x-z|>2\sigma^\rho} |k(x, y, t)| \diff x 
		\end{equation*}
		\begin{equation*}
			\leq \left[ \int_{\mathbb{T}^n} \left( t^{2\rho} |x-y|^2 \right)^N |k(x, y, t)|^2 \diff x \right]^{1/2} \left[ \int_{|x-y|>\sigma^\rho} \left(|t^{2\rho}|x-y|^2\right)^{-N}\diff x \right]^{1/2}
		\end{equation*}
		\begin{equation}
			\lesssim \left[ \int_{\mathbb{T}^n} \left( t^{2\rho}|x-y|^2 \right)^N|k(x,y,t)|^2 \diff x \right]^{1/2}t^{-\rho N}\sigma^{\rho(n/2 - N)}.
			\label{eq:segunda-cota-b}
		\end{equation} 
		Sea $|\alpha|=N$, entonces 
		\begin{equation*}
			t^{\rho|\alpha|} (x-y)^\alpha \int_{\mathbb{R}^n} e^{i2\pi(x - y) \cdot \xi} \tilde{p}(x, \xi)\varphi(\langle\xi\rangle/t)\diff\xi 
		\end{equation*}
		\begin{equation*}
			= \sum_{\beta\leq\alpha} C_{\alpha\beta}t^{\rho|\alpha|} \int_{\mathbb{R}^n} e^{i2\pi(x - y) \cdot \xi} \partial^\beta_\xi \tilde{p}(x, \xi)t^{-|\alpha-\beta|} \partial^{\alpha-\beta}_\xi \varphi(\langle\xi\rangle/t)\diff \xi,
		\end{equation*}
		y para cada $\beta\leq\alpha$, la norma $L^2(\mathbb{T}^n)$ como función de $x$ de 
		\begin{equation*}
			\int_{\mathbb{R}^n} e^{i2\pi(x - y) \cdot \xi} \partial^\beta_\xi \tilde{p}(x, \xi)t^{-|\alpha-\beta|} \partial^{\alpha-\beta}_\xi \varphi(\langle\xi\rangle/t)\diff \xi
		\end{equation*}
		es igual a la norma $L^2(\mathbb{T}^n)$ como función de $x$ de 
		\begin{equation*}
			\int_{\mathbb{R}^n} e^{i2\pi x \cdot \xi} \partial^\beta_\xi \tilde{p}(x+y, \xi)t^{-|\alpha-\beta|} \partial^{\alpha-\beta}_\xi \varphi(\langle\xi\rangle/t)\diff \xi.
		\end{equation*}
		Por otro lado, el conjunto $\{ \langle\xi\rangle^{n(1-\rho)/2 + \rho|\beta|} \partial^\beta_\xi\tilde{p}(x + y, \xi) : y \in \mathbb{T}^n  \}$ tiene medida acotada por $\langle\xi\rangle^{n(1-\rho)/2 + \rho|\beta|} \langle\xi\rangle^{m- \rho|\beta|} = \langle\xi\rangle^{n(1-\rho)/2 +m}   $, así que sus respectivos operadores son acotados en $L^2(\mathbb{R}^n)$ por el Teorema \ref{theo:Alvarez-Hounie}. Por lo tanto, (\ref{eq:segunda-cota-b}) puede estimarse mediante  
		\begin{equation*}
			\sum_{\beta\leq\alpha,\; |\alpha|=N} t^{\rho|\alpha|}t^{-n(1-\rho)/2 - \rho|\beta|}t^{-\rho|\alpha-\beta|}t^{n/2}t^{\rho N} \sigma^{\rho(n/2 - N)} \lesssim (t\sigma)^{\rho (n/2 - N)}.
		\end{equation*}
		De manera similar, se puede estimar 
		\begin{equation*}
			\int_{|x-z|>2\sigma^\rho} |k(x, z, t)| \diff x \lesssim (t\sigma)^{\rho(n/2 - N)}.
		\end{equation*}
		Usando estas estimaciones y (\ref{eq:tsigma<1-b}) se obtiene el resultado de la siguiente expresión:
		\begin{equation}
			\int_{|x-z| > 2\sigma^\rho}|k(x, y) - k(x, z)|\diff x \lesssim \int_1^{1/\sigma} t\sigma/t \diff t + \int_{1/\sigma}^\infty(t\sigma)^{\rho(n/2-N)}/t\diff t \leq C
			\label{eq:ultima-cota-b}.
		\end{equation} 
		Completando la demostración.\\
		
		\item[(c)] Primero, se observa que
		\begin{equation*}
			k(y, x, t) - k(z, x, t) 
		\end{equation*}
		\begin{equation*}
			=\int_{\mathbb{R}^n} e^{-i2\pi (x-y) \cdot \xi}[\tilde{p}(y, \xi) - \tilde{p}(z, \xi)]\varphi(\langle\xi\rangle/t) \diff \xi
		\end{equation*}
		\begin{equation*}
			+  
			\int_{\mathbb{R}^n} e^{-i2\pi x \cdot \xi} \left[ e^{i2\pi y \cdot \xi} -e^{i2\pi z \cdot \xi}  \right]\tilde{p}(z, \xi)\varphi(\langle\xi\rangle/t) \diff \xi
		\end{equation*}
		\begin{equation*}
			=f(x-y, y, z, t) + g(x, y, z, t).
		\end{equation*}
		Entonces, se obtiene 
		\begin{align*}
			\int_{\mathbb{T}^n}|g(x, y, z, t)|\diff x  & \lesssim t^{-\rho n/2} \left[ \int_{\mathbb{T}^n} |g(x, y, z, t)|^2(1 + t^{2\rho}|x|^2)^N \diff x  \right]^{1/2}\\
			 & \lesssim t^{-\rho n/2} \sum_{|\alpha|\leq N}  \left[ \int_{\mathbb{T}^n} |(t^\rho x)^\alpha g(x, y, z, t)|^2\diff x  \right]^{1/2}.
		\end{align*}
		Se observa que $g$ es la transformada de Fourier en $\mathbb{R}^n$ en la primera variable de la función $G(\xi, y, z, t) = \left[ e^{i2\pi y \cdot \xi} -e^{i2\pi z \cdot \xi}  \right]\tilde{p}(z, \xi)\varphi(\langle\xi\rangle/t)$. Además, como antes, $|\partial^\gamma_\xi( e^{i2\pi y \cdot \xi} -e^{i2\pi z \cdot \xi}  )| \leq Ct\sigma$ cuando $t\sigma<1$. Por lo tanto, asumiendo $t\sigma < 1$ y usando la identidad de Plancherel, se obtiene
		\begin{align*}
			\int_{\mathbb{T}^n}|g(x, y, z, t)|\diff x  &\lesssim t^{-\rho n/2} \sum_{|\alpha|\leq N} \left\| \partial^\alpha_\xi G(\xi, y, z ,t) \right\|_{L^2(\mathbb{R}^n)}\\ 
			 & \leq t^{-\rho n/2} \sum_{\beta\leq\alpha,\;|\alpha|\leq N} \left\| \partial^\beta\xi \left[ (e^{i2\pi y \cdot \xi} -e^{i2\pi z \cdot \xi})  \tilde{p}(z, \xi)\right]  \partial^{\alpha-\beta}_\xi\varphi(\langle\xi\rangle/t)  \right\|_{L^2(\mathbb{R}^n)} \\
			 & \lesssim t^{-\rho n/2} \sum_{\beta\leq\alpha,\;|\alpha|\leq N} t\sigma t^{m-\rho|\beta|}t^{-|\alpha-\beta|}t^{n/2} \\
			& \leq t\sigma t^{n(1-\rho)/2 + m} \leq t\sigma.
		\end{align*}
		Ahora, se elimina la restricción $t\sigma<1$ y se observa que
		\begin{equation*}
			g(x, y, z, t) = \int_{\mathbb{R}^n} e^{-i2\pi (x-z) \cdot \xi} \left[ e^{i2\pi (y-z) \cdot \xi} -1 \right]\tilde{p}(z, \xi)\varphi(\langle\xi\rangle/t) \diff \xi.
		\end{equation*}
		Así, se tiene que
		\begin{equation*}
			\int_{\mathbb{T}^n} |g(x, y, z, t)|\diff x \lesssim t^{-\rho N}\sigma^{\rho(n/2 - N)} \left[ \int_{\mathbb{T}^n} (t^{2\rho}|x-z|^2)^N|g(x, y, z, t)|^2\diff x \right].
		\end{equation*}
		Se puede usar la identidad de Plancherel como antes, para obtener
		\begin{align*}
			\int_{\mathbb{T}^n} |g(x, y, z, t)|\diff x  & \lesssim t^{-\rho N}\sigma^{\rho(n/2 - N)} \sum_{\beta\leq\alpha,\;|\alpha|=N} \left\| \partial^\beta_\xi \left[ (e^{i2\pi (y-z) \cdot \xi} -1)\tilde{p}(z, \xi) \right] \partial^{\alpha-\beta}_\xi\varphi(\langle\xi\rangle/t) \right\|_{L^2(\mathbb{R}^n)} \\
			&\lesssim t^{-\rho N}\sigma^{\rho(n/2 - N)} \sum_{\beta\leq\alpha,\;|\alpha|=N}t^{-|\alpha-\beta|}t^{n/2} \\
			 & \lesssim (t\sigma)^{\rho(n/2-N)}.
		\end{align*}
	\end{itemize}
	Se pueden usar los mismos procedimientos para encontrar estas cotas para $f(x-y, y,z, t)$ al observar que $\int_{|x-z|>2\sigma^\rho} |f(x-y, y,z, t)|\diff x \leq \int_{|x|>\sigma^\rho} |f(x, y,z, t)|\diff x$. Entonces la cota deseada proviene del cálculo en (\ref{eq:ultima-cota-b}).
\end{proof}
\begin{theorem}\label{theo:L1-H1-L-infty-BMO}
	Sea $T \in \Psi^m_{\rho, \delta}(\mathbb{T}^n \times \mathbb{Z}^n) $, con $0 < \rho \leq 1$, $0 \leq \delta < 1$, $m \leq - n [(1-\rho)/2 + \lambda] $, entonces $T$ y su adjunto $T^*$ son aplicaciones continuas
	\begin{itemize}
		\item[(a)] del espacio de Hardy $H^1(\mathbb{T}^n)$ en $L^1(\mathbb{T}^n)$,
		\item[(b)] de $L^\infty(\mathbb{T}^n)$ en $\text{BMO}(\mathbb{T}^n)$.
	\end{itemize}
\end{theorem}
\begin{proof}
	\begin{itemize}
		\item[(a)] Sea $a$ un átomo de $H^1(\mathbb{T}^n)$ con soporte en $B(z, \sigma)$, que satisface $\|a\|_{L^\infty}\leq |B|^{-1}$ y la condición de cancelación. Si $\sigma < 1$, se define $B'=B(z, 2\sigma^\rho)$ y $A = \mathbb{T}^n\setminus B'$. Entonces 
		\begin{equation*}
			\int_{\mathbb{T}^n}|Ta(x)|\diff x \leq \int_{B'}|Ta(x)|\diff x  + \int_{A}|Ta(x)|\diff x =: I_1 + I_2.
		\end{equation*}
		Ahora, usando la desigualdad de Cauchy-Schwarz y el Teorema \ref{teo:cotas-L} (b) se obtiene 
		\begin{align*}
			I_1  &\leq \|\chi_{B'}\|_{L^2}\|Ta\|_{{L^2}}\lesssim \sigma^{\rho n/2}\|a\|_{L^{2/(2-\rho)}} \\ 
			 &\leq  \sigma^{\rho n/2} \left[ \int_B |B|^{-2/(2-\rho)}\diff x\right]^{(2-\rho)/2} 
			\leq  \sigma^{\rho n/2}|B|^{-\rho/2} \leq C,
		\end{align*}
		y usando el Teorema \ref{theo:kernel-estimate-sigma} (b) se tiene la estimación,
		\begin{equation*}
			I_2 \leq \int_A\int_B |k(x, y) - k(x, z)||a(y)|\diff y \diff x \leq \sup_{|y-z|<\sigma}\int_A|k(x,y)-k(x, z)|\diff x \leq C.
		\end{equation*}
		Cuando $\sigma \geq 1$, se define $B'=B(z, 2\sigma)$ y $A = \mathbb{T}^n\setminus B'$. Entonces se divide la norma $L^1(\mathbb{T}^n)$ en $I_1 + I_2$ como anteriormente. Ahora se usa el Teorema \ref{teo:cotas-L} (a) para obtener
		\begin{align*}
			I_1  &\leq \|\chi_{B'}\|_{L^2}\|Ta\|_{{L^2}}\lesssim\sigma^{ n/2}\|a\|_{L^{2}} \\ 
			 & \lesssim \sigma^{ n/2} \left[ \int_B |B|^{-2}\diff x\right]^{1/2}  \leq \sigma^{n/2}|B|^{-1/2}\leq C,
		\end{align*}
		y se usa el Teorema \ref{theo:kernel-estimate-sigma} (a) para estimar $I_2$.\\
		
		\item[(b)] Sea $B = B(z, \sigma)$. Si $\sigma < 1$, se define $B'=B(z, 2\sigma^\rho)$ y $A = \mathbb{T}^n\setminus B'$ de modo que se obtiene $f = f\chi_{B'} + f\chi_A =: f_1 + f_2$. Ahora, se define $b = Tf_2(z)$, que está bien definido, ya que $Tf_2$ es suave en $B'$. Entonces 
		\begin{equation*}
			\frac{1}{|B|}\int_B|Tf(x) - b| \diff x \leq \frac{1}{|B|}\int_B|Tf_1(x)|\diff x + \frac{1}{|B|}\int_B|Tf_2(x) -b|\diff x =: I_1 + I_2.
		\end{equation*}
		Por la desigualdad de H\"older y el Teorema \ref{teo:cotas-L} (a) se tienen las desigualdades
		\begin{equation*}
			I_1 \leq \frac{1}{|B|}\left\|\chi_{B'}\right\|_{L^{2/(2-\rho)}}\|Tf_1\|_{L^{2/\rho}}
			\lesssim |B|^{-\rho/2}\|f_1\|_{L^2} \leq |B|^{-\rho/2}\left[\int_{B'}\|f\|^2_{L^\infty}\diff x\right]^{1/2} \lesssim  \|f\|_{L^\infty},
		\end{equation*}
		y usando el Teorema \ref{theo:kernel-estimate-sigma} (c) se obtiene 
		\begin{equation*}
			I_2 \leq \frac{1}{|B|}\int_B\int_A|k(x, y) - k(z, y)||f(y)|\diff y \diff x \leq \sup_{|y-z|<\sigma}\int_A|k(x, y) - k(z, y)|\|f\|_{L^\infty} \diff y \lesssim \|f\|_{L^\infty}.
		\end{equation*}
		Cuando $\sigma \geq 1$, se define $B'=B(z, 2\sigma)$ y $A = \mathbb{T}^n\setminus B'$. Entonces se divide la norma BMO en $I_1 + I_2$ como anteriormente. Ahora se usa la desigualdad de Cauchy-Schwarz y el Teorema  \ref{teo:cotas-L} (b) para obtener
		\begin{equation*}
			I_1 \leq\frac{1}{|B|} \left\|\chi_B\right\|_{L^2}\left\|Tf_1\right\|_{L^2} = |B|^{-1/2}\|Tf_1\|_{L^2} \lesssim |B|^{-1/2}\left[\int_{B'}\|f\|^2_{L^\infty}\diff x\right]^{1/2} \lesssim  \|f\|_{L^\infty},
		\end{equation*}
		y se usa el Teorema \ref{theo:kernel-estimate-sigma} (a) para estimar $I_2$. Así, se obtiene la desigualdad
		\begin{equation*}
			\|Tf\|_{\mathrm{BMO}(\mathbb{T}^n)} = \sup_B \inf_{b\in \mathbb{C}} \frac{1}{|B|} \int_B |Tf(x) - b|\diff x \lesssim \|f\|_{L^\infty(\mathbb{T}^n)}.
		\end{equation*}
		Por lo tanto, se completa la demostración para $T$. 
	\end{itemize}  
	Ahora, se observa que por un argumento de dualidad y en vista del Teorema \ref{theo:H1-BMO}, se tiene que también es válido para el adjunto $T^*$.
\end{proof}
\begin{theorem}
	Sea $T \in \Psi^m_{\rho, \delta}(\mathbb{T}^n \times \mathbb{Z}^n) $, con $0 < \rho \leq 1$, $0 \leq \delta < 1$ y  
	\begin{equation}
		m \leq - n \left[(1-\rho)\left|\frac{1}{p} - \frac{1}{2}\right| + \lambda\right] .
		\label{eq:restriccion-Lp}
	\end{equation} 
	Entonces $T$ es una aplicación continua de $L^p(\mathbb{T}^n)$ en sí mismo.
	\label{teo:cotas-Lp}
\end{theorem}
\begin{proof}
	Se puede usar el argumento de interpolación compleja entre $(H^1(\mathbb{T}^n),\; L^1(\mathbb{T}^n))$ y  $(L^2(\mathbb{T}^n),\; L^2(\mathbb{T}^n))$ para $1 < p < 2$, y entre $(L^2(\mathbb{T}^n),\; L^2(\mathbb{T}^n))$ y el par $(L^1(\mathbb{T}^n),\; \mathrm{BMO}(\mathbb{T}^n))$ para $2 < p<\infty$. De hecho, $T$ es acotado en $L^2(\mathbb{T}^n)$ si $m\leq -n\lambda$ por el Teorema \ref{theo:22boundT}. Por otro lado, si $m\leq -n[(1-\rho)/2 + \lambda]$, entonces $T$ será acotado de $H^1(\mathbb{T}^n)$ en $L^1(\mathbb{T}^n)$ y de $L^\infty(\mathbb{T}^n)$ en $\mathrm{BMO}(\mathbb{T}^n)$. Entonces $T$ será acotado en $L^p(\mathbb{T}^n)$, por el argumento de interpolación de Fefferman-Stein, para
	\begin{equation*}
		\frac{1}{p} = \frac{1-\theta}{q} + \frac{\theta}{2},
	\end{equation*}
	y $0<\theta<1$, con $q=1$ o $q=\infty$. Lo cual es equivalente a la restricción 
	\begin{equation*}
		m \leq -n\lambda\theta - n\left[(1-\rho)/2 +\lambda\right](1 - \theta).
	\end{equation*}
	Es decir, que $m$ satisface (\ref{eq:restriccion-Lp}), completando la demostración.
\end{proof}
Se pueden usar las propiedades de los operadores de potencial de Bessel para extender el resultado de acotación en $L^p(\mathbb{T}^n)$ al caso $L^p(\mathbb{T}^n)$-$L^q(\mathbb{T}^n)$. En el caso del toro, el siguiente teorema extiende el resultado de acotación $L^p(\mathbb{T}^n)$-$L^q(\mathbb{T}^n)$ en  \cite{cardona-delgado-kumar} al rango completo $0 < \rho \leq 1$, $0 \leq \delta < 1$.
\begin{theorem}
	Sea $T \in \Psi^m_{\rho, \delta}(\mathbb{T}^n \times \mathbb{Z}^n) $, con $0 < \rho \leq 1$, $0 \leq \delta < 1$, entonces $T$ es una aplicación continua de $L^p(\mathbb{T}^n)$ en $L^q(\mathbb{T}^n)$ para $1<p\leq q<\infty,$ si 
	\begin{itemize}
		\item[(a)]  $1<p\leq 2\leq q$ y 
		\begin{equation}
			m\leq -n \left( \frac{1}{p} - \frac{1}{q} + \lambda \right),
			\label{eq:cotas-Lp-Lq-a}
		\end{equation}
		\item[(b)]  $2\leq p \leq q$ y 
		\begin{equation}
			m \leq -n \left[ \frac{1}{p} - \frac{1}{q} + (1-\rho)\left(\frac{1}{2} - \frac{1}{p}\right) + \lambda \right],
			\label{eq:cotas-Lp-Lq-b}
		\end{equation}
		\item[(c)]  $p\leq q\leq 2$ y  
		\begin{equation}
			m \leq -n \left[ \frac{1}{p} - \frac{1}{q} + (1-\rho)\left(\frac{1}{q} - \frac{1}{2}\right) + \lambda \right].
			\label{eq:cotas-Lp-Lq-c}
		\end{equation}
	\end{itemize}
	\label{teo:cotas-Lp-Lq}
\end{theorem}
\begin{proof}
	\begin{itemize}
		\item[(a)] Sean $m_1 = -n(1/p - 1/2)$ y $m_2=-n(1/2 - 1/q)$. Entonces $m\leq m_1+m_2-n\lambda$ y $J^{-m_2}TJ^{-m_1}$ es acotado en $L^2(\mathbb{T}^n)$ por el Teorema \ref{theo:22boundT}. Además, por la desigualdad de Hardy-Littlewood-Sobolev, se obtiene que $J^{m_1}$ es acotado de $L^p(\mathbb{T}^n)$ en $L^2(\mathbb{T}^n)$ y $J^{m_2}$ es acotado de $L^q(\mathbb{T}^n)$ en $L^2(\mathbb{T}^n)$. Por lo tanto, 
		\begin{equation*}
			\| J^{m_2}(J^{-m_2}TJ^{-m_1})J^{m_1}f \|_{L^q} \lesssim \|(J^{-m_2}TJ^{-m_1})J^{m_1}f  \|_{L^2} \lesssim \| J^{m_1}f \|_{L^2} \lesssim  \| f\|_{L^p}.
		\end{equation*}
		Así, se prueba el resultado.
		\item[(b)] Sea $m' = -n(1/p - 1/q)$, de modo que $J^{m'}$ es continuo de  $L^p(\mathbb{T}^n)$ en $L^q(\mathbb{T}^n)$ y $J^{-m'}T$ tiene orden $m-m' \leq -n[(1-\rho)(1/2 - 1/q) + \lambda]$ y es una aplicación continua de $L^p(\mathbb{T}^n)$ en sí mismo. Así, se tiene que
		\begin{equation*}
			\| J^{m'}(J^{-m'}T)f \|_{L^q} \lesssim \|(J^{-m'}T)f\|_{L^p} \lesssim \|f\|_{L^p}.
		\end{equation*}
		Por lo tanto, se obtiene la estimación deseada.

		\item[(c)] Como anteriormente, se define  $m' = -n(1/p - 1/q)$ de modo que $TJ^{-m'}$ aplica continuamente  $L^q(\mathbb{T}^n)$ en sí mismo y
		\begin{equation*}
			\|(TJ^{-m'})J^{m'}f\|_{L^q} \lesssim \|J^{m'}f\|_{L^q} \lesssim \|f\|_{L^p}.
		\end{equation*}
		Por lo tanto, se completa la demostración.
	\end{itemize}
\end{proof}
Además, se tiene que estos operadores pseudo-diferenciales toroidales son de tipo débil $(1,1)$. Para ello se demuestra el caso vectorial adaptado del caso euclideano de Álvarez y Milman \cite{alvarez-milman}, que fue extendido al caso toroidal con Cardona en \cite{Cardona:Martinez}.
\begin{theorem}
	Sea $T$ un operador con kernel valuado en operadores, como en la Definición \ref{def:operator-kernel}, que se extiende a un operador acotado de $L^2(\mathbb{T}^n; X)$ en $L^2(\mathbb{T}^n; Y)$ y de $L^q(\mathbb{T}^n; X)$ en $L^2(\mathbb{T}^n; Y)$ tal que para algunos $\alpha$ y $\beta$:
	\begin{equation}
		\frac{1}{q} = \frac{1}{2} + \frac{\beta}{n} ; \quad (1-\alpha)\frac{n}{2} \leq \beta < \frac{n}{2}.
		\label{eq:alpha_condition}
	\end{equation}
	También se asume que su núcleo $k(x, y)$ satisface la siguiente condición cuando $|y-z|<\sigma$
	\begin{equation}
		\int_{|x-z|>c\sigma^\alpha}\|k(x, y)-k(x,z)\|_{\mathcal{B}(X, Y)}\diff x\leq C, \quad 0<\sigma < 1,
		\label{eq:hyp-sigma<1}
	\end{equation}
	\begin{equation}
		\int{|x-z|>c\sigma}\|k(x, y)-k(x,z)\|_{\mathcal{B}(X, Y)}\diff x \leq C , \quad 1 \leq \sigma.
		\label{eq:hyp-sgima>1}
	\end{equation}
	Entonces el operador $T$ se extiende a un operador de tipo débil $(1,1)$.
	\label{theo:L1-weak}
\end{theorem}
\begin{proof}
	Sea $f \in L^1(\mathbb{T}^n; X)$, sea $\lambda > 0$ y considere la descomposición de Calderón-Zygmund a nivel $\lambda$. Así
	\begin{equation*}
		\Omega = \{x \in \mathbb{T}^n : \mathrm{M}f(x) > \lambda\} = \bigcup_{j=1}^\infty Q_j,
	\end{equation*}
	donde $\mathrm{M}$ es el operador maximal de Hardy-Littlewood. De modo que se puede definir $f = g + b$, aquí $f_{Q_j}$ es el valor medio de $f$ sobre $Q_j$ y
	\begin{equation*}
		g = f\chi_{\mathbb{T}^n \setminus \Omega} + \sum_{j=1}^\infty f_{Q_j}\chi_{Q_j},
	\end{equation*}
	\begin{equation*}
		b = \sum_{j=1}^\infty(f - f_{Q_j})\chi_{Q_j} = \sum_{j=1}^\infty b_j.
	\end{equation*}
	Además,
	\begin{equation}
		\|g(x)\|_X \leq C\lambda \quad \text{y} \quad \|g\|_{L^1(\mathbb{T}^n; X)} \leq \|f\|_{L^1(\mathbb{T}^n; X)},
		\label{eq:1norm-X}
	\end{equation}
	\begin{equation}
		\int_{Q_j} \|b_j(x)\|_X \diff x \leq C|Q_j|\lambda, \quad  \quad \quad \int_{\mathbb{T}^n} b_j \diff x = 0,
		\label{eq:cz-condition}
	\end{equation}
	\begin{equation*}
		|\Omega| \leq \frac{C}{\lambda}\|f\|_{L^1(\mathbb{T}^n; X)}.
	\end{equation*}
	De (\ref{eq:1norm-X}) se tiene que $g \in L^2(\mathbb{T}^n; X)$, y también la desigualdad $\|g\|_{L^2(\mathbb{T}^n; X)}^2 \lesssim \lambda \|f\|_{L^1(\mathbb{T}^n; X)}$. Entonces, usando la desigualdad de Chebyshev y la acotación $L^2$ de $T$ se obtiene
	\begin{equation*}
		\lambda^2 |\{x \in \mathbb{T}^n : \|Tg(x)\|_Y > \lambda/2\}| \lesssim \|Tg\|_{L^2(\mathbb{T}^n; Y)}^2 \lesssim \|g\|_{L^2(\mathbb{T}^n; X)}^2 \lesssim \lambda\|f\|_{L^1(\mathbb{T}^n; X)}.
	\end{equation*}
	Por otro lado, si se establece $c\Omega$ como la unión de los cubos $cQ_j$ con el mismo centro y longitud de lado escalada, se obtiene
	\begin{equation*}
		|\{ x \in c\Omega : \|Tb(x)\|_Y > \lambda/2  \}| \leq |c\Omega| \lesssim \frac{1}{\lambda}\|f\|_{L^1(\mathbb{T}^n; X)}.
	\end{equation*}
	Sea $\sigma_j$ el diámetro de $Q_j$ y
	\begin{equation*}
		F = \sum_{\sigma_j < 1} b_j\;, \quad  \quad G = \sum_{\sigma_j \geq 1}b_j,
	\end{equation*}
	entonces para un $c>0$ adecuado y $Q_j$, centrados en $z_j$, se obtiene $|x - z_j| > 4\sigma_j$, cuando $x \in \mathbb{T}^n \setminus c\Omega$. Usando la desigualdad de Chebyshev y (\ref{eq:cz-condition}) se obtiene por (\ref{eq:hyp-sgima>1}) las siguientes estimaciones
	\begin{align*}
		 & \lambda|\{ x \in \mathbb{T}^n \setminus c\Omega : \|TG(x)\|_Y > \lambda/4 \}| \\
		 &\lesssim \int_{\mathbb{T}^n \setminus c\Omega} \|TG(x)\|_Y \diff x \\
		 &\lesssim \sum_{\sigma_j \geq 1} \int_{|x-z_j|>4\sigma_j} \int_{Q_j} \|k(x, y) - k(x, z_j)\|_{\mathcal{B}}\|b_j(y)\|_X \diff y \diff x \\
		 &\lesssim \sum_{\sigma_j \geq 1} \int_{Q_j}\|b_j(y)\|_X \diff y \lesssim \|f\|_{L^1(\mathbb{T}^n; X)}.
	\end{align*}
	Ahora, sea $\varphi$ una función test soportada en $\{x \in \mathbb{T}^n : |x| \leq 1/c\}$ y tal que $\int\varphi(x) \diff x = 1$ con $\varphi \geq 0$. Defínase
	\begin{equation*}
		\varphi_j(x) = \frac{1}{\sigma_j^{n/\alpha}} \varphi\left( \frac{x}{\sigma_j^{1/\alpha}} \right).
	\end{equation*}
	Se escribe
	\begin{equation*}
		F = \sum_{\sigma_j < 1}b_j *\varphi_j + \sum_{\sigma_j<1}(b_j - b_j * \varphi_j) = F' + F'' .
	\end{equation*}
	Entonces, para $x \in\mathbb{T}^n \setminus c\Omega $, se tiene que
	\begin{align*}
		T(b_j - b_j * \varphi_j)(x) = & \int_{\mathbb{T}^n}k(x, y)b_j(y)\diff y - \int_{\mathbb{T}^n}k(x, w) \int_{\mathbb{T}^n} \varphi_j(w-y)b_j(y)\diff y \diff w \\
		= & \int_{\mathbb{T}^n}\left[\int_{\mathbb{T}^n}[k(x, y) - k(x, w)]\varphi_j(w-y)\diff w\right] b_j(y)\diff y.
	\end{align*}
	Por lo tanto, usando la desigualdad de Chebyshev se obtiene
	\begin{equation}
		\lambda |\{x \in \mathbb{T}^n \setminus c\Omega : \|TF''(x)\|_Y > \lambda/8\}|
		\label{eq:TF''}
	\end{equation}
	\begin{equation*}
		\lesssim \int_{\mathbb{T}^n \setminus c\Omega}\|TF''(x)\|_Y \diff x
	\end{equation*}
	\begin{equation*}
		\lesssim \sum_{\sigma_j< 1}\int_{\mathbb{T}^n}\left[\int_{\mathbb{T}^n}\|k(x, y) - k(x, w)\|_{\mathcal{B}(X,Y)}\varphi_j(w-y)\diff w\right] \|b_j(y)\|_X\diff y.
	\end{equation*}
	Cuando $x \in \mathbb{T}^n \setminus c\Omega$ y $|y - w| < \sigma_j^{1/\alpha}$ se tiene que
	\begin{equation*}
		|x-w| \geq |x-z_j| - |z_j - y| - |y - w| > 4\sigma_j - \sigma_j - \sigma_j^{1/\alpha} > 2\sigma_j.
	\end{equation*}
	Así que se puede estimar (\ref{eq:TF''}) por
	\begin{equation*}
		\sum_{\sigma_j< 1}\int_{Q_j}\left[\int_{|w-y|<\sigma_j^{1/\alpha}}\int_{|x-w|>2\sigma_j}\|k(x, y) - k(x, w)\|_{\mathcal{B}(X,Y)}\diff x \diff w\right] \|b_j(y)\|_X\diff y
	\end{equation*}
	\begin{equation*}
		\lesssim \sum_{\sigma_j<1} \int_{Q_j}\|b_j(y)\|_X\diff y  \leq C\|f\|_{L^1(\mathbb{T}^n; X)}.
	\end{equation*}
	Ahora, solo queda probar la desigualdad
	\begin{equation*}
		\lambda |\{ x \in \mathbb{T}^n \setminus c\Omega : \|TF'(x)\|_Y > \lambda/8 \}| \leq C\|f\|_{L^1(\mathbb{T}^n; X)},
	\end{equation*}
	lo cual se probaría si se obtiene la estimación
	\begin{equation}
		\|J^{-\beta} F'\|_{L^2(\mathbb{T}^n; X)}^2 \leq A\lambda \|f\|_{L^1(\mathbb{T}^n; X)},
	\end{equation}
	donde $J$ es el potencial de Bessel de orden uno. Dado que $\beta$ satisface (\ref{eq:alpha_condition}) se tiene que $TJ^{\beta}$ es acotado en $L^2(\mathbb{T}^n;Y)$ y
	\begin{equation*}
		\|T F'\|_{L^2(\mathbb{T}^n; Y)}^2 =\|TJ^{\beta}J^{-\beta }F'\|_{L^2(\mathbb{T}^n; Y)}^2 \lesssim \|J^{-\beta} F'\|_{L^2(\mathbb{T}^n; X)}^2 \lesssim \lambda \|f\|_{L^1(\mathbb{T}^n; X)}.
	\end{equation*}
	Por lo tanto, se puede usar la desigualdad de Chebyshev para obtener
	\begin{equation*}
		\lambda^2 |\{ x \in \mathbb{T}^n \setminus c\Omega : \|TF'(x)\|_Y > \lambda/8 \}| \lesssim  \int_{\mathbb{T}^n \setminus c\Omega} \|TF'(x)\|_Y^2\diff x  \lesssim \lambda\|f\|_{L^1(\mathbb{T}^n; X)}.
	\end{equation*}
	Recuerde que
	\begin{equation*}
		F' = \sum_{\sigma_j < 1}f\chi_{Q_j} * \varphi_j - \sum_{\sigma_j < 1}f_{Q_j}\chi_{Q_j}*\varphi_j.
	\end{equation*}
	Se define $x \sim Q_j$ si $x$ pertenece a la clausura de cualquier cubo $Q_k$ adyacente a $Q_j$. Así,
	\begin{equation*}
		J^{-\beta}\sum_{\sigma_j < 1}f\chi_{Q_j} * \varphi_j (x) = \sum_{x\sim Q_j}J^{-\beta} (f\chi_{Q_j} * \varphi_j )(x) + \sum_{x\nsim Q_j}J^{-\beta} (f\chi_{Q_j} * \varphi_j )(x)  = F_1(x) + F_2(x).
	\end{equation*}
	De modo que
	\begin{equation*}
		(J^{-\beta} F')(x) = F_1(x) + F_2(x) - \sum_{\sigma_j < 1}f_{Q_j}J^{-\beta} (\chi_{Q_j}*\varphi_j)(x).
	\end{equation*}
	Usando la prueba de Fefferman adaptada a grupos de Lie compactos \cite{cardona-ruzhansky}, se tiene que para $x \nsim Q_j$
	\begin{equation*}
		\left\|J^{-\beta} (f\chi_{Q_j} *\varphi_j)(x)\right\|_X \lesssim J^{-\beta} \|f_{Q_j}\|_X (\chi_{Q_j}*\varphi_j)(x).
	\end{equation*}
	Por (\ref{eq:cz-condition}) y el hecho de que $J^{-\beta}$ mapea funciones positivas en funciones positivas se tiene que
	\begin{align*}
		\left\| F_2(x) - \sum_{\sigma_j<1} f_{Q_j}J^{-\beta}(\chi_{Q_j} * \varphi_j)(x) \right\|_X = & \left\| \sum_{\sigma_j<1} J^{-\beta}[(f\chi_{Q_j} - f_{Q_j}\chi_{Q_j}) * \varphi_j](x) \right\|_X \\
		 &\leq \left\| \sum_{\sigma_j<1} J^{-\beta}(b_j * \varphi_j)(x) \right\|_X \\
		 & \lesssim\lambda \sum_{\sigma_j<1} J^{-\beta} (\chi_{Q_j}*\varphi_j)(x)\\
		 & \lesssim \lambda \|J^{-\beta}\|_1\left\| \sum_{\sigma_j <1}\chi_{Q_j} *\varphi_j \right\|_{L^\infty(\mathbb{T}^n; X)} \lesssim \lambda ,
	\end{align*}
	ya que los soportes de $ \chi_{Q_j} * \varphi_j $ tienen superposición finita. Por otro lado,
	\begin{align*}
		\left\| F_2 - \sum_{\sigma_j < 1} J^{-\beta} f_{Q_j}\chi_{Q_j} * \varphi_j \right\|_{L^1(\mathbb{T}^n; X)} \leq & \sum_{\sigma_j<1} \left\|J^{-\beta}[\varphi_j * (f - f_{Q_j})\chi_{Q_j}]\right\|_{L^1(\mathbb{T}^n; X)} \\
		 & \lesssim \sum_{\sigma_j < 1}\int_{Q_j}\|f(y)\|_X \diff y \lesssim \|f\|_{L^1(\mathbb{T}^n; X)}.
	\end{align*}
	Combinando estas dos estimaciones se obtiene
	\begin{equation*}
		\left\| F_2 - \sum_{\sigma_j < 1} J^{-\beta} f_{Q_j}\chi_{Q_j} * \varphi_j \right\|_{L^2(\mathbb{T}^n; X)}^2 \leq C\lambda\|f\|_{L^1(\mathbb{T}^n; X)} .
	\end{equation*}
	Ahora solo queda probar la desigualdad $\|F_1\|_{L^2(\mathbb{T}^n; X)}^2 \lesssim \lambda \|f\|_{L^1(\mathbb{T}^n; X)}$, lo cual puede hacerse como en la prueba de Fefferman, \cite{cardona-ruzhansky}. Para un $x \in \mathbb{T}^n$ fijo, sea
	\begin{equation*}
		F_1^j(x) = \begin{cases}
			J^{-\beta} f\chi_{Q_j}*\varphi_j(x) & \text{si }x \nsim Q_j \\ 0 & \text{si no}.
		\end{cases}
	\end{equation*}
	Así, $F_1(x)=\sum_{\sigma_j <1} F_1^j(x)$ y $F_1^j(x) \neq 0$ para a lo sumo $N$ valores. Entonces,
	\begin{align*}
		\|F_1(x)\|_Y^2 \leq & \left( \sum_{j=1}^N\left\|F_1^{j(x)}(x)\right\|_Y \right)^2 \\
		\leq & \sum_{j,h=1}^N\left\|F_1^{j(x)}(x)\right\|_Y \left\|F_1^{h(x)}(x)\right\|_Y \\
		\leq & 2\left( \sum_{j,h=1}^N \left\|F_1^{j(x)}(x)\right\|_Y^2 + \left\|F_1^{h(x)}(x)\right\|_Y^2 \right) \\
		\leq & 4N\sum_{\sigma_j < 1}\left\|F_1^j(x)\right\|_Y^2.
	\end{align*}
	Por lo tanto, usando la desigualdad de Hausdorff-Young se tiene que
	\begin{equation*}
		\|F_1\|_{L^2(\mathbb{T}^n; X)}^2 \leq 4N\sum_{\sigma_j<1} \left\| F_1^j \right\|_{L^2(\mathbb{T}^n; X)}^2 \leq 4N\sum_{\sigma_j<1} \left\| J^{-\beta} \varphi_j \right\|_{L^2(\mathbb{T}^n; X)}^2\left\| f\chi_{Q_j} \right\|_{L^1(\mathbb{T}^n; X)}^2.
	\end{equation*}
	Por otro lado, por la identidad de Plancherel y dado que $-2\beta < -n$ se obtiene
	\begin{equation*}
		\left\| J^{-\beta} \varphi_j \right\|_{L^2(\mathbb{T}^n; X)}^2 = \sum_{\xi \in \mathbb{Z}^n } \langle\xi\rangle^{-2\beta} \left|(\mathcal{F}_{\mathbb{T}^n} \varphi)\left(\sigma_j^{1/\alpha}\xi\right)\right|^2 \lesssim   \frac{1}{|Q_j|},
	\end{equation*}
	ya que $|Q_j| \leq 1$. Finalmente
	\begin{equation*}
		\|F_1\|_{L^2(\mathbb{T}^n; X)}^2 \lesssim \sum_{\sigma_j<1}\frac{1}{|Q_j|} \|f\chi_{Q_j}\|_{L^1(\mathbb{T}^n; X)}^2 \lesssim \lambda \|f\|_{L^1(\mathbb{T}^n; X)}.
	\end{equation*}
	Así, se completa la prueba.
\end{proof}
\begin{theorem}
	Sea $T \in \Psi^m_{\rho, \delta}(\mathbb{T}^n \times \mathbb{Z}^n) $, con $0 < \rho \leq 1$, $0 \leq \delta < 1$, $m \leq - n [(1-\rho)/2 + \lambda] $, entonces $T$ es del tipo débil $(1, 1)$.
\end{theorem}
\begin{proof}
	Note que por el Teorema \ref{theo:22boundT} se puede tomar $q$ como $2/(2-\rho)$ en la hipótesis del Teorema \ref{theo:L1-weak}. Además, tomar $\alpha$ como $\rho$ y $\beta$ como $n(1-\rho)/2$ satisface la condición  (\ref{eq:alpha_condition}). Entonces, en vista del Teorema \ref{theo:kernel-estimate-sigma} se obtienen las condiciones necesarias para el resultado
\end{proof}
\subsection{Continuidad de operadores pseudo-diferenciales en espacios de Lebesgue pesados $L^p(w)$}
Ahora, se presentan los resultados de continuidad de operadores pseudo-difernciales toroidales para espacios de Lebesgue pesados $L^p(w)$, obtenidos con Cardona en \cite{cardona-martinez-III}. Estos extienden el caso euclideano demostrado por Park y Tomita \cite{park-tomita}.

Sea $\varphi$ una función de Schwartz definida en $\mathbb{R}^n$ tal que su transformada de Fourier $\widehat{\varphi}$ es igual a uno en la bola unitaria centrada en el origen y tiene soporte en la bola concéntrica de radio 2. Sea $\psi$ otra función test tal que $\widehat{\psi}(\xi) := \widehat{\varphi}(\xi) - \widehat{\varphi}(2\xi)$ para $\xi \in \mathbb{R}^n$. Para cada $k \in \mathbb{N}$, se define
\begin{equation*}
	\psi_k(x):=2^{kn}\psi(2^kx).
\end{equation*}
Entonces se tiene una partición de la unidad de Littlewood-Paley no homogénea formada por $\varphi$ y $\psi_k$, con $k\in\mathbb{N}$. Además, note que 
\begin{equation*}
	\supp \widehat{\psi_k} \subset \left\{\xi \in \mathbb{R}^n : 2^{k-1} \leq |\xi|\leq 2^{k+1}\right\}
\end{equation*}
y que
\begin{equation*}
	\widehat{\varphi}(\xi) + \sum_{k\in\mathbb{N}} \widehat{\psi_k}(\xi) = 1.
\end{equation*}
Así, se puede descomponer cualquier $\sigma \in S^m_{\rho,\rho}(\mathbb{T}^n \times \mathbb{Z}^n)$ como 
\begin{equation*}
	\sigma(x,\xi) = \sigma(x,\xi)\widehat{\varphi}(\xi) + \sum_{k\in \mathbb{N}} \sigma(x,\xi)\widehat{\psi_k}(\xi) =: \sigma_0(x,\xi) + \sum_{k\in \mathbb{N}} \sigma_k(x,\xi),
\end{equation*}
por lo que se puede escribir 
\begin{equation*}
	T_\sigma = \sum_{k\in \mathbb{N}_0} T_{\sigma_k},
\end{equation*}
donde $T_{\sigma_k}$ son los operadores pseudo-diferenciales toroidales asociados con $\sigma_k \in S^m_{\rho,\rho}(\mathbb{T}^n \times \mathbb{Z}^n)$. Ahora, se escriben sus kernels correspondientes (en el sentido de distribuciones) como
\begin{equation}
	K_k(y, u) := \sum_{\xi \in \mathbb{Z}^n} \sigma_k(y, \xi)e^{i2\pi u \cdot \xi}.
	\label{eq:kernel-Kk}
\end{equation}
Además, sea $\tilde{\sigma}_k$ la extensión del símbolo $\sigma_k$ definida en $\mathbb{T}^n \times \mathbb{R}^n$. Primero, se demuestran las siguientes estimaciones del kernel.
\begin{lemma}
	Sea $0 < \rho \leq 1$ y $m\in \mathbb{R}$. Supóngase que $\sigma \in S^m_{\rho,\rho}(\mathbb{T}^n \times \mathbb{Z}^n) $ y sea $K_k$ definido como en (\ref{eq:kernel-Kk}). Para $N\geq 0$ arbitrario, y $1\leq r\leq 2$, se tiene que 
	\begin{equation*}
		\left\| (1+2^{k\rho}|u|)^N K_k(y, u)
		\right\|_{L^{r'}} \lesssim_N 2^{k(m+n/r)},
	\end{equation*}
	\begin{equation*}
		\left\| (1+2^{k\rho}|u|)^N \nabla_y K_k(y, u)
		\right\|_{L^{r'}} \lesssim_N 2^{k(\rho+m+n/r)},
	\end{equation*}
	\begin{equation*}
		\left\| (1+2^{k\rho}|u|)^N \nabla_u K_k(y, u)
		\right\|_{L^{r'}} \lesssim_N 2^{k(1+m+n/r)},
	\end{equation*}
	como funciones de $u\in\mathbb{R}^n$, uniformemente en $y\in\mathbb{R}^n$.
	\label{lem:kernel-estimates}
\end{lemma}
\begin{proof}
	Se sigue el argumento de Park y Tomita \cite{park-tomita}. Dado que $\tilde{\sigma}_k \in S^m_{\rho,\rho}(\mathbb{T}^n \times \mathbb{R}^n) $, entonces para cualquier multi-índice $\beta\in \mathbb{N}_0^n$ se tiene que
	\begin{equation}
		\left|\partial_\xi^\beta \tilde{\sigma}_k(y, \xi)\right| \lesssim 2^{k(m-\rho|\beta|)},
		\label{eq:partial-sigma}
	\end{equation}
	\begin{equation}
		\left|\partial_\xi^\beta \nabla_y \tilde{\sigma}_k(y, \xi)\right| \lesssim 2^{k(m+\rho-\rho|\beta|)},
		\label{eq:nabla-sigma}
	\end{equation}
	\begin{equation}
		\left|\partial_\xi^\beta [\xi  \tilde{\sigma}_k(y, \xi)]\right| \leq 
		\left|\xi\cdot\partial_\xi^\beta \tilde{\sigma}_k(y,\xi)\right| + \sum_{j=1}^n \left| \partial_\xi^{\beta-j} \tilde{\sigma}_k(y,\xi) \right|
		\lesssim
		2^{k(m+1-\rho|\beta|)},
		\label{eq:xi-sigma}
	\end{equation}
	
	ya que $\langle\xi\rangle \sim 2^k$. Ahora, se usa la desigualdad de Hausdorff-Young y (\ref{eq:partial-sigma})para obtener que
	\begin{align*}
		\left\| (2^{k\rho}u)^\beta K_k(y, u)
		\right\|_{L^{r'}} & \lesssim2^{k\rho|\beta|} \left\| 
		\partial_\xi^\beta \tilde{\sigma}_k(x, \xi)
		\right\|_{L^{r}} \\
		& = 2^{k\rho|\beta|} 2^{k(m-\rho|\beta|)} 2^{kn/r} \\
		& =  2^{k(m+n/r)},
	\end{align*}
	ya que el volumen del soporte de $\tilde{\sigma}_k$ es comparable con $2^{kn}$. Además, aquí la norma $L^{r'}$ se toma con respecto a $u \in \mathbb{R}^n$ y la norma $L^{r}$ se toma con respecto a $\xi\in\mathbb{R}^n$. Esto concluye la demostración de la primera estimación, las estimaciones restantes pueden probarse usando el mismo procedimiento y las estimaciones (\ref{eq:nabla-sigma}) y (\ref{eq:xi-sigma}) respectivamente.
\end{proof}
Ahora, se demuestran algunos lemas de acotamiento útiles.
\begin{lemma}
	Sea $1 < r \leq 2$, sea $0<\rho\leq r/2$, sea $0<\rho<1$, y sea $m = -n(1-\rho)/r$. Entonces para todo $\sigma \in S^m_{\rho,\rho}(\mathbb{T}^n \times \mathbb{Z}^n) $, su operador correspondiente $T_\sigma$ es continuo de $L^r$ en $L^{r/\rho}$.
	\label{lem:Lr-Lr/rho-boundedness}
\end{lemma}
\begin{proof}
	Note que se tiene $r\leq 2 \leq r/\rho$ y $m$ satisface las condiciones del Teorema \ref{teo:cotas-Lp-Lq}.
\end{proof}
Para manejar el caso $\rho > r/2$ que no se considera en el lema anterior, se demuestra lo siguiente.
\begin{lemma}
	Sea $1<r<2$, sea $r/2\leq\rho<1$, y sea $m=-n(1-\rho)/r$. Suponga que $k\in\mathbb{N}_0$ y 
	\begin{equation*}
		\frac{2\rho-r}{2-r} < \lambda < \rho.
	\end{equation*}
	Entonces todo $\sigma \in S^m_{\rho,\rho}(\mathbb{T}^n \times \mathbb{Z}^n) $ satisface 
	\begin{equation*}
		\|T_{\sigma_k}f\|_{L^{\frac{r(1-\lambda)}{\rho-\lambda}} } \lesssim 2^{\lambda nk \frac{1-\rho}{r(1-\lambda)}} \|f\|_{L^r}
	\end{equation*}
	para $f\in C^\infty(\mathbb{T}^n)$.
	\label{lem:Lr-lambda-boundedness}
\end{lemma}

\begin{proof}
	Se usa la estrategia de Park y Tomita \cite{park-tomita}. Se define    
	\begin{equation*}
		a_k(x, \xi) := \tilde{\sigma}_k(2^{-\lambda k}x, 2^{\lambda k} \xi).
	\end{equation*}
	Entonces para cualquier multi-índices $\alpha, \beta \in \mathbb{N}_0^n$, se obtiene que
	\begin{align*}
		|\partial_x^\alpha\partial_\xi^\beta a_k(x, \xi)| & = 2^{\lambda k(|\beta|-|\alpha|)} |\partial_x^\alpha\partial_\xi^\beta \tilde{\sigma}_k(2^{-\lambda k}x, 2^{\lambda k} \xi)| \\
		& \lesssim 2^{\lambda k(|\beta|-|\alpha|)} 2^{k(m - \rho(|\beta|-|\alpha|))} \chi_{ \{ \langle\xi\rangle \sim 2^{(1-\lambda)k} \} } \\
		& =  2^{ k[m - (\rho-\lambda)(|\beta|-|\alpha|)] } \chi_{ \{ \langle\xi\rangle \sim 2^{(1-\lambda)k} \} } \\
		& \lesssim  \langle\xi\rangle^{ \frac{m}{1-\lambda}  -  \frac{\rho-\lambda}{1-\lambda}(|\beta|-|\alpha|) },
	\end{align*}
	donde la constante es independiente de $k$. Por lo tanto, se tiene que 
	\begin{equation*}
		a_k \in S^{\frac{m}{1-\lambda}}_{ \frac{\rho-\lambda}{1-\lambda} , \frac{\rho-\lambda}{1-\lambda} } (\mathbb{T}^n\times\mathbb{R}^n)
	\end{equation*}
	uniformemente en $k$. Note que 
	\begin{equation*}
		0 < \frac{\rho-\lambda}{1-\lambda} < \frac{r}{2} \quad \text{ y } \quad \frac{m}{1-\lambda} = -\frac{n}{r} \left(  1 - \frac{\rho-\lambda}{1-\lambda}
		\right),
	\end{equation*}
	lo que permite emplear el Lema \ref{lem:Lr-Lr/rho-boundedness}, y obtener la continuidad $L^r$-$L^\frac{r(1-\lambda)}{\rho-\lambda}$ para los operadores $T_{a_k}$ uniformemente en $k$. Dado que se tiene que 
	\begin{equation*}
		T_{\sigma_k}f(x) = T_{a_k}(f(2^{-\lambda k}\cdot ))(2^{
			\lambda k
		}x),
	\end{equation*}
	se puede obtener que 
	\begin{align*}
		\|T_{\sigma_k}f\|_{L^\frac{r(1-\lambda)}{\rho-\lambda}} &= 2^{ -\lambda nk \frac{\rho-\lambda}{r(1-\lambda)} } \|T_{a_k}(f(2^{-\lambda k} \cdot))\|_{L^\frac{r(1-\lambda)}{\rho-\lambda}} \\
		& \lesssim 2^{ -\lambda nk \frac{\rho-\lambda}{r(1-\lambda)} } \|f(2^{-\lambda k} \cdot)\|_{L^r} \\
		& =  2^{ -\lambda nk \frac{\rho-\lambda}{r(1-\lambda)} } 2^{\lambda nk/r} \|f\|_{L^r} \\
		& =  2^{ \lambda nk \frac{1-\rho}{r(1-\lambda)} } \|f\|_{L^r}.
	\end{align*}
	Así, completando la demostración.
\end{proof}

Ahora, sea $P$ una dilatación concéntrica de $Q$ con $\ell(P)\geq 10\sqrt{n}\ell(Q)$ y sea $\chi_P$ su función característica. Ahora, se considera 
\begin{equation}
	f = f_{\chi_P} + f_{\chi_{\mathbb{T}^n\setminus P}} =: f_0 + f_1,
	\label{eq:decomposition}
\end{equation}
de modo que
\begin{equation*}
	T_{\sigma_k}f = T_{\sigma_k}f_0 + T_{\sigma_k}f_1.
\end{equation*}

\begin{proposition}
	Sea $0<\rho<1$, sea $1<r\leq2$, y sea $m=-n(1-\rho)/r$. Suponga que $x\in Q$.
	
	\begin{enumerate}
		\item Sea $0<\rho<r/2$, y sea $k\in\mathbb{N}_0$. Entonces todo símbolo $\sigma \in S^m_{\rho,\rho}(\mathbb{T}^n \times \mathbb{Z}^n) $ satisface
		\begin{equation}
			\left( \frac{1}{|Q|}\int_Q |T_{\sigma_k} f_0(y)|^r\diff y  
			\right)^{1/r} \lesssim \left[ \frac{\ell(P)}{\ell(Q)^\rho} 
			\right]^{n/r} \mathrm{M}_rf(x).
			\label{eq:fo-estimate-rho-less}
		\end{equation}
		\item Sea $r/2\leq \rho <1$, sea $ \frac{2\rho - r}{2-r} < \lambda < \rho$, y sea $k\in\mathbb{N}_0$. Entonces todo símbolo $\sigma \in S^m_{\rho,\rho}(\mathbb{T}^n \times \mathbb{Z}^n) $ satisface
		\begin{equation}
			\left( \frac{1}{|Q|}\int_Q|T_{\sigma_k}f_0(y)|^r \diff y
			\right)^{1/r} \lesssim [2^k\ell(Q)]^{\lambda n \frac{1-\rho}{r(1-\lambda)}} \left[ \frac{\ell(P)}{\ell(Q)^\rho}
			\right]^{n/r} \mathrm{M}_rf(x).
			\label{eq:f0-estimate-rho-greater}
		\end{equation}
	\end{enumerate}
	
	
	
	
\end{proposition}
\begin{proof}
	Primero, se considera el caso $0<\rho<r/2$. Por la desigualdad de H\"older y el Lema \ref{lem:Lr-Lr/rho-boundedness} se tiene que 
	\begin{align*}
		\left(
		\frac{1}{|Q|} \int_Q |T_{\sigma_k}f_0(y)|\diff y
		\right)^{1/r} &\leq \frac{1}{|Q|^{\rho/r}} \left\| T_{\sigma_k}f_0  \right\|_{L^{r/\rho}} \\
		& \lesssim \frac{1}{\ell(Q)^{n\rho/r}} \|f \chi_P \|_{L^r} \\
		& \lesssim \left[ \frac{\ell(P)}{\ell(Q)^\rho} 
		\right]^{n/r} \mathrm{M}_rf(x),
	\end{align*}
	ya que $x \in Q \subset P$. Por lo tanto, probando (\ref{eq:fo-estimate-rho-less}). Ahora, se asume que $r/2\leq \rho<1$. Por la desigualdad de H\"older y (\ref{lem:Lr-lambda-boundedness}) se obtiene que 
	\begin{align*}
		\left(
		\frac{1}{|Q|} \int_Q |T_{\sigma_k}f_0(y)|\diff y
		\right)^{1/r} &\leq {|Q|^{-\frac{\rho-\lambda}{r(1-\lambda)}}} \|T_{\sigma_k}f_0\|_{L^\frac{r(1-\lambda)}{\rho-\lambda}} \\
		&\lesssim {\ell(Q)^{-n\frac{\rho-\lambda}{r(1-\lambda)}}} 2^{\lambda nk\frac{1-\rho}{r(1-\lambda)}} \|f\chi_P\|_{L^r}  \\
		& \lesssim 2^{\lambda nk\frac{1-\rho}{r(1-\lambda)}}\ell(Q)^{-n\frac{\rho-\lambda}{r(1-\lambda)}} \ell(P)^{n/r} \mathrm{M}_rf(x)\\
		&\lesssim[2^k\ell(Q)]^{\lambda n\frac{1-\rho}{n(1-\lambda)}} 
		\left[ \frac{\ell(P)}{\ell(Q)^\rho}
		\right]^{n/r} \mathrm{M}_rf(x).
	\end{align*}
	Así, completando la demostración.
\end{proof}
Ahora se demuestran estimaciones para la segunda parte de la descomposición como en (\ref{eq:decomposition}).
\begin{proposition}
	Sea $0\leq \rho<1$, sea $1\leq r\leq2$, sea $m=-n(1-\rho)/r$, y $k\in\mathbb{N}_0$. Supóngase que $x, y \in Q$. Entonces todo $\sigma \in S^m_{\rho,\rho}(\mathbb{T}^n \times \mathbb{Z}^n) $ satisface
	\begin{equation}
		|T_{\sigma_k}f_1(y)| \lesssim_N [2^{k\rho}\ell(P)]^{-(N-n/r)} \mathrm{M}_rf(x),
		\label{eq:f1-estimate}
	\end{equation}
	y
	\begin{equation}
		|T_{\sigma_k}f_1(y) - T_{\sigma_k}f_1(x)| \lesssim_N 2^k\ell(Q) [2^{k\rho}\ell(P)]^{-(N-n/r)} \mathrm{M}_rf(x),
		\label{eq:diff-f1}
	\end{equation}
	para cualquier $N>n/r$.
\end{proposition}
\begin{proof}
	Primero, se considera (\ref{eq:f1-estimate}). Sea $N>n/r$, y sea $y\in Q$. Por la desigualdad de H\"older se obtiene que
	\begin{equation*}
		|T_{\sigma_k}f_1(y)| \leq \int_{\mathbb{T}^n\setminus P} |K_k(y, y-u)||f(u)|\diff u 
	\end{equation*}
	\begin{equation*}
		\leq \left\| \ell(P)^{n/r} \left[ \frac{|\cdot|}{\ell(P)} \right]^N K_k(y, \cdot) 
		\right\|_{L^{r'}} \left\| \ell(P)^{-n/r} \left[ \frac{|y-\cdot|}{\ell(P)} \right]^{-N} f\chi_{\mathbb{T}^n\setminus P} 
		\right\|_{L^r}.
	\end{equation*}
	La norma $L^{r'}$ puede estimarse usando Lema \ref{lem:kernel-estimates} por 
	\begin{align*}
		\left\| \ell(P)^{n/r} \left[ \frac{|\cdot|}{\ell(P)} \right]^N K_k(y, \cdot) 
		\right\|_{L^{r'}} & = \ell(P)^{-(N-n/r)} \left\| |\cdot|^N K_k(y, \cdot)  \right\|_{L^{r'}} \\
		& \leq \ell(P)^{-(N-n/r)} 2^{-k\rho N} \left\| (1+2^{k\rho} |\cdot|)^N K_k(y, \cdot)  \right\|_{L^{r'}} \\
		& \lesssim \ell(P)^{-(N-n/r)} 2^{-k\rho N} 2^{k(m+n/r)}\\
		& = [2^{k\rho}\ell(P)]^{-(N-n/r)}.
	\end{align*}
	Por otro lado, la norma $L^r$ puede estimarse notando que para $x, y \in Q$, y para $u \in \mathbb{T}^n\setminus P$, se tiene que
	\begin{equation*}
		|y-u|\geq |x-u|-|x-y| \geq C_n[\ell(P)+|x-u|].
	\end{equation*}
	Así, para $N>n/r$, se obtiene que
\begin{align*}
	\left\| \ell(P)^{-n/r} \left[ \frac{|y-\cdot|}{\ell(P)} \right]^{-N}f \chi_{\mathbb{T}^n\setminus P}
	\right\|_{L^r}& \lesssim\left( \int_{\mathbb{T}^n} \ell(P)^{-n} \left[ 1+\frac{|x-u|}{\ell(P)}
	\right]^{-rN} |f(u)|^r \diff u
	\right)^{1/r} \\
	& \lesssim\mathrm{M}_rf(x).
\end{align*}
Así, se combinan ambas estimaciones para obtener (\ref{eq:f1-estimate}). Ahora, se considera (\ref{eq:diff-f1}) usando una demostración similar al caso anterior, definiendo ahora 
\begin{equation*}
	H_k(y, x, u) := K_k(y, y-u) - K_k(x, x-u).
\end{equation*}
Asúmase $x, y \in Q$ y $N>n/r$. Usando la desigualdad de H\"older, se tiene que
\begin{equation*}
	|T_{\sigma_k}f_1(y) - T_{\sigma_k}f_1(x)| \leq \int_{\mathbb{T}^n\setminus P} |H_k(y, x, u)||f(y)|\diff u
\end{equation*}
\begin{equation*}
	\leq \left\| \ell(P)^{n/r} \left[ \frac{|y-\cdot|}{\ell(P)}
	\right]^N H_k(y, x, \cdot) \chi_{\mathbb{T}^n\setminus P}
	\right\|_{L^{r'}} \left\| \ell(P)^{-n/r} \left[ \frac{|y-\cdot|}{\ell(P)}
	\right]^{-N} f\chi_{\mathbb{T}^n\setminus P}
	\right\|_{L^r}.
\end{equation*}
Se estima el primer factor usando el Lema \ref{lem:kernel-estimates} por
\begin{equation*}
	\left\| \ell(P)^{n/r} \left[ \frac{|y-\cdot|}{\ell(P)}
	\right]^N H_k(y, x, \cdot) \chi_{\mathbb{T}^n\setminus P}
	\right\|_{L^{r'}}
\end{equation*}
\begin{equation*}
	\lesssim \left\| \ell(Q)\ell(P)^{n/r} \left[ \frac{|y-\cdot|}{\ell(P)}
	\right]^N \int_0^1|\nabla K_k(y(t), y(t) - \cdot)|\diff t \chi_{\mathbb{T}^n\setminus P}
	\right\|_{L^{r'}}
\end{equation*}
\begin{equation*}
	\lesssim \ell(Q)\ell(P)^{-(N-n/r)} \int_0^1\left\| |y(t) - \cdot|^N |\nabla K_k(y(t), y(t) - \cdot)|
	\right\|_{L^{r'}} \diff t
\end{equation*}
\begin{equation*}
	\lesssim \ell(Q)\ell(P)^{-(N-n/r)} 2^{-k\rho N} \int_0^1 \left\| (1+2^{k\rho}|\cdot|)^N |\nabla K_k(y(t), \cdot)| 
	\right\|_{L^{r'}} \diff t
\end{equation*}
\begin{equation*}
	\lesssim \ell(Q)\ell(P)^{-(N-n/r)} 2^{-k\rho N} 2^{k(1+m+n/r)} = C 2^k \ell(Q)[2^{k\rho}\ell(P)]^{-(N-n/r)},
\end{equation*}
donde $y(t) := ty + (1-t)x \in Q$ de modo que $|y-x|\lesssim|y(t)-u|$ para $u \in \mathbb{T}^n\setminus P$. Para el segundo factor, se usa la misma estimación que en el caso anterior, completando la demostración.
\end{proof}
Se procede a demostrar el teorema principal de esta sección, del cual se seguirá la continuidad deseada.
\begin{theorem}
Sea $1<r\leq 2$, y sea $0<\rho<1$. Suponga que $m\leq-n(1-\rho)/r$ y $\sigma \in S^m_{\rho,\rho}(\mathbb{T}^n \times \mathbb{Z}^n)  $. Entonces se tiene que 
\begin{equation*}
	\mathcal{M}^\#_r(T_\sigma f)(x) \lesssim\mathrm{M}_r f(x)
\end{equation*}
para $f \in C^\infty(\mathbb{T}^n)$.
\label{theo:sharp-maximal}
\end{theorem}
\begin{proof}
Note que es suficiente probar 
\begin{equation}
	\inf_{c_Q\in \mathbb{C}} \left( 
	\frac{1}{|Q|} \int_Q|T_\sigma f(y) - c_Q|^r\diff y
	\right)^{1/r} \lesssim \mathrm{M}_rf(x) ,
	\label{eq:main-goal}
\end{equation}
uniformemente en $Q$ y $x\in \mathbb{T}^n$. Ahora, sea $P_\rho$ una dilatación concéntrica de $Q$ tal que $\ell(P_\rho) = 10\sqrt{n}\ell(Q)^\rho$ y se descompone $f$ como en (\ref{eq:decomposition}). Primero se considerará el caso cuando $0<\rho < r/2$. Entonces, el lado izquierdo de (\ref{eq:main-goal}) es menor que la suma de 
\begin{equation*}
	\mathcal{I}_0 := \left( 
	\frac{1}{|Q|} \int_Q|T_\sigma f_0(y)|^r\diff y
	\right)^{1/r}
\end{equation*}
y 
\begin{equation*}
	\mathcal{I}_1 := \inf_{c_Q \in \mathbb{C}} \left(  \frac{1}{|Q|}\int_Q|T_\sigma f_1 - c_Q|^r
	\right)^{1/r}.
\end{equation*}
Usando la desigualdad de H\"older y el Lema \ref{lem:Lr-Lr/rho-boundedness} se puede verificar que 
\begin{equation*}
	\mathcal{I}_0 \leq \frac{1}{|Q|^{\rho/r}} \|T_\sigma f_0\|_{L^{r/\rho}} \lesssim\frac{1}{\ell(Q)^{n\rho/r}} \|f\chi_{P_\rho}\|_{L^r} \lesssim\mathrm{M}_rf(x). 
\end{equation*}
Para estimar $\mathcal{I}_1$, se establece 
\begin{equation}
	c_Q := \sum_{k:2^k\ell(Q)<1} T_{\sigma_k} f_1(x), 
	\label{eq:cQ}
\end{equation}
por lo que se tiene que
\begin{equation*}
	|T_\sigma f_1(y) - c_Q| \leq \sum_{k:2^k\ell(Q) \geq 1} |T_{\sigma_k}f_1(y)| + \sum_{k:2^k\ell(Q)<1}|T_{\sigma_k} f_1(y) - T_{\sigma_k} f_1(x)|  .
\end{equation*}
Primero, se estima la primera suma usando (\ref{eq:f1-estimate}) por
\begin{equation*}
	\sum_{k:2^k\ell(Q) \geq 1} |T_{\sigma_k}f_1(y)| \lesssim \sum_{k:2^k\ell(Q) \geq 1} [2^k\ell(Q)]^{-\rho(N-n/r)} \mathrm{M}_rf(x) \lesssim \mathrm{M}_rf(x),
\end{equation*}
ya que $\rho > 0$ y $N > n/r$. Luego, se aplica (\ref{eq:diff-f1}) para estimar los términos restantes por 
\begin{equation*}
	\sum_{k:2^k\ell(Q)<1}|T_{\sigma_k} f_1(y) - T_{\sigma_k} f_1(x)| \lesssim_N\sum_{k:2^k\ell(Q)<1} [2^k\ell(Q)]^{1-\rho(N-n/r)}\mathrm{M}_r f(x) = C\mathrm{M}_rf(x),
\end{equation*}
cuando se elige $N<n/r + 1/\rho$. Así, completando la demostración cuando $0<\rho<r/2 $. Ahora, se considera el caso $r/2\leq \rho<1$. Se puede estimar (\ref{eq:main-goal}) por la suma de 
\begin{equation*}
	\mathcal{J}_1 := \inf_{c_Q \in \mathbb{C}} \left( \frac{1}{|Q|}\int_Q \left| \sum_{k:2^k\ell(Q)<1} T_{\sigma_k} f(y) - c_Q \right|^r \diff y
	\right)^{1/r},
\end{equation*}
\begin{equation*}
	\mathcal{J}_2 := \sum_{\substack{k:2^k\ell(Q)\geq 1, \\ 2^{\rho k}\ell(Q) <1}} \left(\frac{1}{|Q|} \int_Q |T_{\sigma_k}f(y)|^r \diff y 
	\right)^{1/r},
\end{equation*}
\begin{equation*}
	\mathcal{J}_3 := \sum_{k:2^{\rho k} \ell(Q) \geq 1} \left( \frac{1}{|Q|} \int_Q |T_{\sigma_k}f(y)|^r \diff y 
	\right)^{1/r}.
\end{equation*}
Primero, se establece $\frac{2\rho - r}{2-r} < \lambda < \rho$ para el resto de esta demostración. Luego, se usa la misma descomposición por $P_\rho$ como arriba y se usa $c_Q$ como se define en (\ref{eq:cQ}) para estimar $\mathcal{J}_1$ por
\begin{equation*}
	\sum_{k:2^k\ell(Q)<1} \left[
	\left(  \frac{1}{|Q|} \int_Q |T_{\sigma_k} f_0 (y)|^r \diff y
	\right)^{1/r} + \left( \frac{1}{|Q|}\int_Q|T_{\sigma_k} f_1(y) -T_{\sigma_k}f_1(x)|^r \diff y
	\right)^{1/r}
	\right].
\end{equation*}
Usando (\ref{eq:f0-estimate-rho-greater}) se obtiene que 
\begin{equation*}
	\left( 
	\frac{1}{|Q|} \int_Q |T_{\sigma_k} f_0 (y)|^r \diff y
	\right)^{1/r} \lesssim [2^k\ell(Q)]^{\lambda n \frac{1-\rho}{r(1-\lambda)}} \mathrm{M}_rf(x).
\end{equation*}
Además, se emplea (\ref{eq:diff-f1}) para obtener que
\begin{equation*}
	|T_{\sigma_k} f_1(y) - T_{\sigma_k} f_1(x)| \lesssim [2^k\ell(Q)]^{1-\rho(N-n/r)} \mathrm{M}_rf(x).
\end{equation*}
De nuevo, se elige $N < n/r + 1/\rho$, para obtener una cota uniforme y completar la demostración para el primer término. Para el término $\mathcal{J}_2$, se elige un número positivo $\varepsilon$ tal que 
\begin{equation*}
	\lambda \left( \frac{1-\rho}{1-\lambda} \right) < \varepsilon < \rho,
\end{equation*}
que se sabe existe ya que $\lambda < \rho$. Ahora, se define $P_{\varepsilon, k}$ como la dilatación concéntrica de $Q$  con $\ell(P_{\varepsilon, k}) = 10 \sqrt{n}\ell(Q)^\rho [2^k\ell(Q)]^{-\varepsilon} $. Nótese que 
\begin{equation*}
	10\sqrt{n} \ell(Q) \leq \ell(P_{\varepsilon, k}),
\end{equation*}
ya que es equivalente a 
\begin{equation*}
	2^{\rho k} \ell(Q)^{\frac{\rho(1-\rho + \varepsilon)}{\varepsilon} } \leq 1, 
\end{equation*}
lo cual es cierto ya que 
\begin{align*}
	\rho(1-\rho) & \geq \varepsilon(1-\rho) \\
	\frac{\rho(1-\rho + \varepsilon)}{\varepsilon} & \geq 1,
\end{align*}
y $2^{\rho k}\ell(Q) < 1$. Entonces, se considera 
\begin{equation*}
	f = f\chi_{P_{\varepsilon, k}} + f\chi_{ \mathbb{T}^n\setminus P_{\varepsilon, k}} =: f_{0,k} + f_{1,k}.
\end{equation*}
Por lo tanto, por (\ref{eq:f0-estimate-rho-greater}) se tiene que 
\begin{align*}
	\left( \frac{1}{|Q|}\int_Q |T_{\sigma_k} f_{0, k}(y)|^r \diff y 
	\right)^{1/r} & \lesssim [2^k\ell(Q)]^{\lambda n \frac{1-\rho}{r(1-\lambda)}} \left[ \frac{\ell(P_{\varepsilon, k}) }{\ell(Q)^\rho} \right]^{n/r} \mathrm{M}_rf(x)\\
	& =  [2^k\ell(Q)]^{-\frac{n}{r}\left(\varepsilon - \lambda \frac{1-\rho}{1-\lambda}\right)} \mathrm{M}_rf(x).
\end{align*}
Además, por (\ref{eq:f1-estimate}) se obtiene que
\begin{align*}
	|T_{\sigma_k}f_{1, k}(y)| & \lesssim [2^{k\rho}\ell(P_{\varepsilon,k})]\mathrm{M}_rf(x)\\
	& \lesssim\left( 2^{k\rho}\ell(Q)^\rho [2^k\ell(Q)]^{-\varepsilon} 
	\right)^{-(N-n/r)} \mathrm{M}_rf(x) \\
	& \leq [2^k\ell(Q)]^{-(\rho-\varepsilon)(N-n/r)}\mathrm{M}_rf(x),        
\end{align*}
donde $N>n/r$. Por lo tanto, se tiene una cota uniforme para el segundo término. Ahora, se considera $\mathcal{J}_3$. Para este caso se usará $P$ como la dilatación concéntrica de $Q$ tal que $\ell(P)=10\sqrt{n}\ell(Q)$, y se usa la misma descomposición que en (\ref{eq:decomposition}). Note que dado que $\langle\xi\rangle \sim 2^k$ en el soporte de $\sigma_k$, entonces se tiene que
\begin{equation*}
	2^{\frac{kn}{2}(1-\rho)}\sigma_k \in S_{\rho, \rho}^{-n(1-\rho)(1/r-1/2)} (\mathbb{T}^n \times \mathbb{Z}^n),
\end{equation*}
uniformemente en $k$. Por lo tanto, el Teorema \ref{teo:cotas-Lp} implica la acotación $L^r$ para cada $T_{\sigma_k}$, y se obtiene que
\begin{align*}
	\left( \frac{1}{|Q|}\int_Q |T_{\sigma_k}f_0(y)|^r \diff y
	\right)^{1/r} & \leq \frac{1}{|Q|^{1/r}} \|T_{\sigma_k}f_0(y)\|_{L^r} \\
	& \lesssim 2^{-\frac{kn}{2}(1-\rho)} \frac{1}{\ell(Q)^{n/r}} \|f\chi_P\|_{L^r} \\
	& \lesssim2^{-\frac{kn}{2}(1-\rho)} \mathrm{M}_rf(x).
\end{align*}
Por otro lado, se emplea (\ref{eq:f1-estimate}) para obtener que
\begin{equation*}
	|T_{\sigma_k}f_1(y)| \lesssim[2^{k\rho}\ell(Q)]^{-(N-n/r)} \mathrm{M}_rf(x),
\end{equation*}
donde $N>n/r$. Por lo tanto se concluye que $\mathcal{J}_3 \lesssim\mathrm{M}_rf(x)$, y se completa la demostración.
\end{proof}
Ahora, se aprovecha el Corolario \ref{cor:muckenhoupt} y el Teorema \ref{theo:sharp-maximal} para obtener el siguiente resultado de continuidad.
\begin{corollary}
Sea $0\leq \delta \leq \rho < 1$, sea $0<\rho<1$, sea $1<r\leq 2$, y sea $r\leq p < \infty$. Suponga que $m\leq -n(1-\rho)/r$ y $\sigma\in S^m_{\rho,\delta}(\mathbb{T}^n\times\mathbb{Z}^n)$. Si $w \in A_{p/r}$, entonces 
\begin{equation*}
	\|T_\sigma f\|_{L^p(w)} \lesssim\|f\|_{L^p(w)},
\end{equation*}
para cualquier $f\in C^\infty(\mathbb{T}^n)$.
\end{corollary}
\begin{proof}
Primero, se considera el caso $p>r$. Por lo tanto, se obtiene que
\begin{equation*}
	\|T_\sigma f\|_{L^p(w)} \lesssim\| \mathcal{M}^\#_r(T_\sigma f)\|_{L^p(w)} \lesssim\|\mathrm{M}_rf\|_{L^p(w)} \lesssim\|f\|_{L^p(w)}.
\end{equation*}
Ahora, se considera el caso $p=r$. Entonces, en vista del Corolario \ref{cor:inverse-Holder}, para $w\in A_1$ se sabe que existe $\varepsilon>0$ tal que 
\begin{equation*}
	w^{1+\varepsilon} \in A_1.
\end{equation*}
Para cualquier $r<q_0<\infty$, la teoría de encaje para pesos $A_p$ implica que $w^{1+\varepsilon}\in A_{q_0/r}$, y por un argumento similar al anterior se concluye que 
\begin{equation*}
	\|T_\sigma f\|_{L^{q_0}(w^{1+\varepsilon})} \lesssim\|f\|_{L^{q_0}(w^{1+\varepsilon})}.
\end{equation*}
Además, del Teorema \ref{teo:cotas-Lp} se tiene que 
\begin{equation*}
	\|T_\sigma f\|_{L^{q_1}} \lesssim\|f\|_{L^{q_1}},
\end{equation*}
para cualquier $1<q_1<r$, porque $m \leq -n(1-\rho)(1/q_1 - 1/2)$. El resultado deseado se sigue del argumento de interpolación como en el Teorema \ref{theo:interpolation-Lpw}.
\end{proof}

\section{Continuidad en espacios de Sobolev}
Primero, se revisita la definición de espacios de Sobolev utilizando los operadores pseudo-diferenciales que han sido definidos en apartados anteriores.
\begin{definition}[Espacios de Sobolev]
	Sea $s\in\R$, entonces se dice que $f$ pertenece al \textit{espacio de Sobolev} $W^s_p(\R^n)$, si $J^sf\in L^p(\R^n)$, donde $1\leq p\leq\infty$ y $J^s$ es el potencial de Bessel de orden $s$. Además, se define la norma
	\begin{equation*}
		\|f\|_{W^s_p} := \|J^sf\|_{L^p}.
	\end{equation*} 
\end{definition}
Ahora, se extiende el resultado del Teorema \ref{theo:Lp-boundedness-S0} para estos espacios. 
\begin{theorem}\label{theo:Sobolev-S0}
	Sea $T\in\Psi^m_{1,0}(\R^n\times\R^n)$ sea un operador pesudo-diferencial de orden $m\in\R$. Entonces, el operador $T$ se extiende a un operador continuo desde el espacio de Sobolev $W^{s-m}_p(\R^n)$ hacia $W^s_p(\R^n)$, con $1<p<\infty$.
\end{theorem}
\begin{proof}
	Note que $J^{s-m}TJ^{-s}$ es un operador de orden cero, por lo que es continuo en $L^p$, en vista del Teorema \ref{theo:Lp-boundedness-S0}. Por lo que se tiene que 
	\begin{equation*}
		\|f\|_{W^{s-m}_p} = \|J^{s-m}Tf\|_{L^p} = \|J^{s-m}TJ^{-s}J^sf\|_{L^p} \lesssim \|J^sf\|_{L^p} = \|f\|_{W^s_p}.
	\end{equation*}
	Completando la prueba.
\end{proof}
Ahora, se demuestra que la definición revisitada de espacios de Sobolev coincide con la definición para regularidad entera que se dio anteriormente. 
\begin{theorem}
	El espacio $W^s_p(\R^n)$ coincide con el espacio $W^k_p(\R^n)$, para $1<p<\infty$, cuando $s=k$ es entero, con equivalencia de normas.
\end{theorem}
\begin{proof}
	Se utilizará $k$ para simplificar la notación. Para $|\alpha|\leq k$, se tiene que $\partial^\alpha_xJ^{-k}$ es un operador pseudo-diferencial con símbolo $(2\pi i\xi)^\alpha\angles{\xi}^{-k},$ y orden $|\alpha|-k\leq0$. Entonces, por el Teorema \ref{theo:Lp-boundedness-S0} se puede concluir que son acotados en $L^p$, y que
	\begin{equation*}
		\sum_{|\alpha|\leq k} \|\partial^\alpha_xf\|_{L^p} = \sum_{|\alpha|\leq k} \|\partial^\alpha_xJ^{-k}J^kf\|_{L^p} \lesssim \|J^kf\|_{L^p} = \|f\|_{W^k_p}.
	\end{equation*}
	Por otra parte, se tiene que el operador con símbolo
	\begin{equation*}
		\frac{\angles{\xi}^k}{p_k(\xi)}:= \angles{\xi}^k \left(\sum_{|\alpha|\leq k} \xi^\alpha
		\right)^{-1} \leq C <\infty,
	\end{equation*}
	es acotado en $L^p$. Por lo que  
	\begin{align*}
		\|f\|_{W^k_p} =\|J^kf\|_{L^p} & = \|J^kT_{p_k}^{-1}T_{p_k} f\|_{L^p} \\
		& \lesssim \|T_{p_k} f\|_{L^p} \\ 
		& = \left\|\mathcal{F}^{-1}\left(\sum\xi^\alpha \widehat{f}(\xi)\right)\right\|_{L^p}  \\
		& \leq \sum \|\mathcal{F}^{-1}(\xi^\alpha\widehat{f}(\xi))\|_{L^p} \\
		& \lesssim \sum_{|\alpha|\leq k} \|\partial^\alpha_xf\|_{L^p}
	\end{align*}
	Lo que completa la prueba de equivalencia de definiciones.
\end{proof}
La definición de espacios de Sobolev también puede realizarse mediante el potencial de Bessel en el caso del toro.
\begin{definition}[Espacios de Sobolev]
Sea $s\in\R$, entonces se dice que $f$ pertenece al \textit{espacio de Sobolev} $W^s_p(\T^n)$, si $J^sf\in L^p(\T^n)$, donde $1\leq p\leq\infty$ y $J^s$ es el potencial de Bessel de orden $s$. Además, se define la norma
\begin{equation*}
	\|f\|_{W^s_p} := \|J^sf\|_{L^p}.
\end{equation*} 
\end{definition}
Ahora, se extiende el Teorema \ref{teo:cotas-Lp-Lq} a espacios de Sobolev.
\begin{theorem}
	Sean $0\leq\delta<1$, $0<\rho\leq 1$, $m\in\mathbb{R}$, y $T\in \Psi^m_{\rho,\delta}(\mathbb{T}^n\times\mathbb{Z}^n)$. Entonces, $T$ se extiende a un operador acotado de $W^{s}_p(\mathbb{T}^n)$ en $W^{s-\mu}_q (\mathbb{T}^n)$ donde $1<p\leq q<\infty$, para cualquier $s\in\mathbb{R}$, cuando 
	\begin{enumerate}
		\item $1<p\leq 2 \leq q$ y 
		\begin{equation*}
			\mu \geq m + n \left( \frac{1}{p} - \frac{1}{q} + \lambda
			\right),
		\end{equation*}
		\item si $2 \leq p \leq q$ y 
		\begin{equation*}
			\mu \geq m + n \left[ \frac{1}{p} - \frac{1}{q} + (1-\rho) \left( \frac{1}{2} - \frac{1}{p}
			\right)
			+ \lambda
			\right],
		\end{equation*}
		\item si $p\leq q \leq 2$ y 
		\begin{equation*}
			\mu \geq m + n \left[ \frac{1}{p} - \frac{1}{q} + (1-\rho) \left( \frac{1}{q} - \frac{1}{2}
			\right)
			+ \lambda
			\right],
		\end{equation*}
	\end{enumerate}
	donde $\lambda := \max\{ 0, (\delta-\rho)/2 \}$. 
\end{theorem}
\begin{proof}
	Se observa que $J^{s-\mu}T J^{-s}$ tiene orden $m-\mu$, que satisface los requisitos del Teorema \ref{teo:cotas-Lp-Lq}, implicando su continuidad $L^p$-$L^q$. Por lo tanto, se obtiene que
	\begin{equation*}
		\|T f\|_{W^{s-\mu}_q} = \|T J^{-s}J^sf\|_{W^{s-\mu}_q} 
		= \|J^{s-\mu}T J^{-s}J^sf\|_{L^q}
		\lesssim \|J^sf\|_{L^p} = \|f\|_{W^s_p}. 
	\end{equation*}
	Así, se completa la demostración.
\end{proof}

\section{Continuidad en espacios de Hardy}
Aquí se presentan los resultados de continuidad de operadores pseudo-differenciales toroidales desde espacios de Lebesgue $L^p$ hacia espacios de Hardy $H^p$, y de espacios de Hardy $H^p$ en sí mismos, ambos para $p\leq 1$. Estos fueron obtenidos con Cardona \cite{cardona-martinez-II}, y extienden el caso eculideano demostrado por \'Alvarez y Hounie \cite{alvarez-hounie}. En el análisis posterior, se define la siguiente descomposición 'diádica' en los anillos
\begin{equation}
	A_j(z, \sigma) = \{ x \in \mathbb{T}^n : 2^j\sigma < |x-z|<2^{j+1}\sigma \} \; , \quad j=1, 2, 3, ...
\end{equation}
\begin{remark}
	Nótese que $\mathbb{T}^n$ está contenido en cualquier bola con radio mayor que $\sqrt{n}/2$. Por lo tanto, para cualquier $\sigma>0$ dado, existe $N_\sigma\in \mathbb{Z}^+$ tal que 
	\begin{equation}
		\frac{\sqrt{n}}{2\sigma} < 2^{N_\sigma} \leq \frac{\sqrt{n}}{\sigma}.
	\end{equation}
	Así, esta descomposición "diádica" es finita en el caso del toro y $2^{N_\sigma} \sim \sigma^{-1}$.
	\label{rem:N-sigma}
\end{remark}
\subsection{Continuidad de operadores pseudo-diferenciales de $H^p$ en $L^p$}
\label{section:Hp-Lp}
Primero, se considera el caso general para operadores con núcleo de valor operador. 
\begin{theorem}
	\label{theo:operator-kernel-2j}
	Sea $T$ un operador con kernel de valuado en operadores $k:=k(x,y)$ que satisface para algún $0<\omega\leq1$ las estimaciones
	\begin{equation}
		\int_{A_j(z, \sigma)} \| k(x, y) - k(x, z)\|_{\mathcal{B}(X, Y)} \diff x \leq C2^{-j\omega}, \quad \text{si} \quad \sigma\geq1;
	\end{equation}
	\begin{equation}
		\int_{A_j(z, \sigma^\gamma)} \| k(x, y) - k(x, z)\|_{\mathcal{B}(X,Y)} \diff x \leq C2^{-j\omega/\alpha}\sigma^{\omega(1-\gamma/\alpha)}, \quad \text{si} \quad \sigma<1,
	\end{equation}
	para $|y-z|<\sigma$ y cualquier $0<\gamma\leq\alpha\leq1$. Además, suponga que el operador $T$ se extiende a un operador acotado de $L^2(\mathbb{T}^n; X)$ en $L^2(\mathbb{T}^n; Y)$ y de $L^q(\mathbb{T}^n; X)$ en $L^2(\mathbb{T}^n; Y)$, donde 
	\begin{equation}
		\frac{1}{q} = \frac{1}{2} + \frac{\beta}{n}, \quad \text{para algún} \quad (1-\alpha)\frac{n}{2} \leq \beta < \frac{n}{2}.
	\end{equation}
	Entonces, el operador $T$ es acotado de $H^p(\mathbb{T}^n; X)$ en $L^p(\mathbb{T}^n;Y)$, para $1\geq p \geq p_0$ cuando $\alpha<1$, donde
	\begin{equation*}
		\frac{1}{p_0} = \frac{1}{2}+\frac{\beta(\omega/\alpha + n/2)}{n(\omega/\alpha-\omega+\beta)},
	\end{equation*}  
	y para $1\geq p > p_0=n/(n+\omega)$ cuando $\alpha=1$.
\end{theorem}
\begin{proof}
	Se fija $0< p \leq1$ y sea $a$ un $(p,\infty)$-átomo soportado en la bola $B(z,\sigma)$. Primero, se asume que $\sigma\geq1$, entonces se emplea la descomposición diádica y la propiedad de cancelación de $a$ para obtener
	\begin{align*}
		\int_{\mathbb{T}^n}\|Ta(x)\|^p_Y \diff x \leq & \int_{B(z,\sigma)}\|Ta(x)\|^p_Y \diff x \\ 
		& + \sum_{j=1}^{N_\sigma} \int_{A_j(z,\sigma)}\left( \int_{B(z,\sigma)} \|k(x, y) - k(x, z)\|_\mathcal{B} \|a(y)\|_X 
		\diff y\right)^p \diff x \\
		= & I_1 + I_2
	\end{align*}
	Usando la desigualdad de H\"older con exponente $2/p$, y la continuidad $L^2$ de $T$ se obtiene
	\begin{align*}
		I_1  &\leq \left( \int_{\mathbb{T}^n} \|Ta(x)\|_Y^2\diff x \right)^{p/2}\left( \int_{\mathbb{T}^n} \chi_{B(z,\sigma)}(x)\diff x \right)^{(2-p)/2} \\
		 & =  \|Ta\|_{L^2(\mathbb{T}^n;Y)}^p|B|^{(2-p)/2} \\
		 &\lesssim \|a\|_{L^2(\mathbb{T}^n;Y)}^p |B|^{(2-p)/2} \\
		 & \lesssim |B|^{(p-2)/2} |B|^{(2-p)/2}\leq C.
	\end{align*}
	Para $I_2$, se usa la desigualdad de H\"older con exponente $1/p$ y las estimaciones del núcleo de la hipótesis para obtener
	\begin{align*}
		I_2  &\leq \sum_{j=1}^{N_\sigma}\int_{A_j(z,\sigma)}\left( \int_{B(z,\sigma)} \|k(x, y) - k(x, z)\|_\mathcal{B} |B|^{-1/p} 
		\diff y\right)^p \diff x \\
		 & \leq\sum_{j=1}^{N_\sigma} \left(\int_{B(z,\sigma)} \int_{A_j(z,\sigma)} \|k(x, y) - k(x, z)\|_\mathcal{B} \diff x\diff y \right)^p \left( \int_{A_j(z,\sigma)} |B|^{-1/(1-p)} \diff x \right)^{1-p} \\
		 &\lesssim \sum_{j=1}^{N_\sigma}  2^{-j\omega p}|B|^p \cdot |B|^{-1}|A_j(z, \sigma)|^{1-p} \\
		 &\lesssim \sum_{j=1}^{N_\sigma} 2^{-j\omega p}\sigma^{np} \cdot \sigma^{-n}2^{jn(1-p)}\sigma^{n(1-p)}\\
		 & =\sum_{j=1}^{N_\sigma} 2^{j[n-(n+\omega)p]},
	\end{align*}
	que puede acotarse por una constante $C>0$ siempre que 
	\begin{equation}
		p > \frac{n}{n+\omega} \geq \frac{n}{n+\omega/\alpha}.
		\label{eq:base-p0}
	\end{equation}
	Ahora, se considera el caso $\sigma<1$. Entonces 
	\begin{align*}
		\int_{\mathbb{T}^n}\|Ta(x)\|^p_Y \diff x \leq & \int_{B(z,2\sigma^\gamma)}\|Ta(x)\|^p_Y \diff x \\ 
		& + \sum_{j=1}^{N_\sigma} \int_{A_j(z,\sigma^\gamma)}\left( \int_{B(z,\sigma)} \|k(x, y) - k(x, z)\|_\mathcal{B} \|a(y)\|_X 
		\diff y\right)^p \diff x \\
		= & I_1 + I_2,
	\end{align*}
	con $\gamma$ por elegir más tarde.  El primer término puede estimarse usando la desigualdad de H\"older con exponente $2/p$ y la acotación $L^q$-$L^2$ de $T$ para obtener
	\begin{align*}
		I_1 &\leq \left( \int_{\mathbb{T}^n} \|Ta(x)\|_Y^2\diff x \right)^{p/2}\left( \int_{\mathbb{T}^n} \chi_{B(z,2\sigma^\gamma)}(x)\diff x \right)^{(2-p)/2} \\
		&= \|Ta\|_{L^2(\mathbb{T}^n;Y)}^p|B(z,2\sigma^\gamma)|^{(2-p)/2} \\
		 & \lesssim\|a\|_{L^q(\mathbb{T}^n;X)}^p\sigma^{n\gamma(2-p)/2} \\
		 & \lesssim\left( \int_{B(z, \sigma)}|B|^{-q/p} \right)^{p/q}\sigma^{n\gamma(2-p)/2}  \\
		 & \lesssim\sigma^{-n}\sigma^{np/q}\sigma^{n\gamma(2-p)/2} = \sigma^{n[\gamma(1-p/2) +p/q-1]}.
	\end{align*}
	Así, dado que $1/q=1/2+\beta/n$ se puede concluir que $I_1$ estará acotado siempre que 
	\begin{equation}
		\gamma \geq \frac{2n - p(n + 2\beta)}{n(2-p)}, 
		\label{eq:gamma-lowe-bound}
	\end{equation}
	que es una función decreciente de $p$ y alcanza $\gamma\geq1-2\beta/n$ cuando $p=1$. De ahí el requisito $\beta \geq (1-\alpha) n/2$.
	Por otro lado, usando la desigualdad de H\"older con exponente $1/p$ y la estimación del kernel de la hipótesis, se obtiene 
	\begin{align*}
		I_2  &\leq \sum_{j=1}^{N_{\sigma^\gamma}}\int_{A_j(z,2\sigma^\gamma)}\left( \int_{B(z,\sigma)} \|k(x, y) - k(x, z)\|_\mathcal{B} |B|^{-1/p} 
		\diff y\right)^p \diff x \\
		 &\leq \sum_{j=1}^{N_{\sigma^\gamma}} \left(\int_{B(z,\sigma)} \int_{A_j(z,2\sigma^\gamma)} \|k(x, y) - k(x, z)\|_\mathcal{B} \diff x\diff y \right)^p \left( \int_{A_j(z,2\sigma^\gamma)} |B|^{-1/(1-p)} \diff x \right)^{1-p} \\
		 &\leq \sum_{j=1}^{N_{\sigma^\gamma}}  2^{-j\omega p/\alpha}\sigma^{\omega p(1-\gamma/\alpha)}|B|^p \cdot |B|^{-1}|A_j(z, 2\sigma^\gamma)|^{1-p} \\
		 &\leq \sum_{j=1}^{N_{\sigma^\gamma}} 2^{-j\omega p/\alpha}\sigma^{\omega p(1-\gamma/\alpha)}\sigma^{np} \cdot \sigma^{-n}2^{jn(1-p)}2^{n(1-p)}\sigma^{n\gamma(1-p)}\\
		 & = \sigma^{-\gamma[p(n+\omega/\alpha)-n] +p(n+\omega)-n} \sum_{j=1}^{N_{\sigma^\gamma}} 2^{j[n-(n+\omega/\alpha)p]}
	\end{align*}
	Ahora, dado que $p$ debe satisfacer (\ref{eq:base-p0}), se tiene que $n - (n+\omega/\alpha)p < 0$. Por lo tanto, se puede acotar la suma geométrica por una constante y obtener que
	\begin{align*}
		I_2  & \lesssim \sigma^{-\gamma[p(n+\omega/\alpha)-n] +p(n+\omega)-n},
	\end{align*}
	que puede estimarse por una constante siempre que 
	\begin{equation}
		\gamma \leq \frac{p(n+\omega)-n}{p(n+\omega/\alpha)-n},
		\label{eq:gamma-upper-bound}
	\end{equation}
	que es una función creciente de $p$ que alcanza $\gamma \leq \alpha$ cuando $p=1$. Por lo tanto, el $p_0$ crítico ocurre cuando se igualan los lados derechos de (\ref{eq:gamma-lowe-bound}) y (\ref{eq:gamma-upper-bound}), completando la demostración.
\end{proof} 

Ahora, se demuestra que bajo ciertas condiciones, los operadores pseudo-diferenciales toroidales satisfacen las estimaciones del núcleo necesarias para usar el teorema anterior. 
\begin{theorem}
	Sea $T\in \Psi^m_{\rho,\delta}(\mathbb{T}^n\times\mathbb{Z}^n)$, $0<\rho\leq1$, $0\leq\delta<1$ con núcleo $k:=k(x, y)$. Entonces, 
	\begin{itemize}
		\item[a)] Si $\sigma \geq \varepsilon>0$, y $j=1,2,3,...$,
		\begin{equation}
			\sup_{|y-z|<\sigma}\int_{A_j(z, \sigma)} |k(y, x) - k(z, x)|\diff x \leq C_\varepsilon2^{-j},
		\end{equation}
		\begin{equation}
			\sup_{|y-z|<\sigma}\int_{A_j(z, \sigma)} |k(x, y) - k(x, z)|\diff x \leq C_\varepsilon2^{-j},
		\end{equation}
		donde $C_\varepsilon$ no depende de $\sigma$, $j$, o $z$.
		\item[b)] Si $m\leq -n[(1-\rho)/2 +\lambda]$, $0<\gamma\leq1$, $\sigma<1$, y $j=1,2,3,...$, 
		\begin{equation}
			\sup_{|y-z|<\sigma}\int_{A_j(z, \sigma^\gamma)} |k(x, y) - k(x, z)|\diff x \leq C2^{-j/\rho}\sigma^{1-\gamma/\rho}.
		\end{equation}
		
		\item[c)] Si $m\leq -n(1-\rho)/2 $, $0<\gamma\leq1$, $\sigma<1$, and $j=1,2,3,...$, 
		\begin{equation}
			\sup_{|y-z|<\sigma}\int_{A_j(z, \sigma^\gamma)} |k(y, x) - k(z, x)|\diff x \leq C2^{-j/\rho}\sigma^{1-\gamma/\rho}.
			\label{eq:kernel-estimate-c}
		\end{equation}
		
	\end{itemize}
	\label{theo:pdo-kernel}
\end{theorem}
\begin{proof}
	\begin{itemize}
		\item[a)] Por el Teorema \ref{theo:kernel-properties} y la desigualdad triangular se tiene que 
		\begin{align*}
			\int_{A_j(z, \sigma)} |k(y, x) - k(z, x)|\diff x  &\leq  \int_{A_j(z, \sigma)} |k(y, x)| \diff x +\int_{A_j(z, \sigma)} |k(z, x)|\diff x\\
			 & \lesssim \int_{A_j(z, \sigma)} |x-y|^{-N} \diff x + \int_{A_j(z, \sigma)} |x-z|^{-N} \diff x,
		\end{align*}
		para algún $N\geq (m+n)/\rho$. Ahora, se tiene que $|x-y| \geq |x-z| - |z-y|>2^j\sigma -\sigma \geq 2^{j-1}\sigma$. Así, dado que el toro tiene volumen uno, se obtiene que 
		\begin{align*}
			\int_{A_j(z, \sigma)} |k(y, x) - k(z, x)|\diff x  & \lesssim (2^{j-1}\sigma)^{-N}  +  (2^{j}\sigma)^{-N}  \\
			 & \leq (2^{j} \varepsilon)^{-N} \leq C_\varepsilon 2^{-j}.
		\end{align*}
		
		\item[b)] Sea $\tilde{p}:=\tilde{p}(x,\xi)$ el símbolo correspondiente de $T$ definido en $\mathbb{T}^n \times \mathbb{R}^n$, ver el Teorema \ref{theo:equivalence-symbols}.  Sea $\varphi \in C_0^\infty(\mathbb{R})$ soportada en $[1/2, 1]$ tal que  
		\begin{equation*}
			\int_0^\infty \varphi(1/t)/t\diff t = \int_1^2 \varphi(1/t)/t\diff t=1.
		\end{equation*}
		Se define 
		\begin{equation*}
			k(x, y, t) = \int_{\mathbb{R}^n} e^{i2\pi (x-y) \cdot \xi} \tilde{p}(x, \xi) \varphi(\langle\xi\rangle/t)\diff t,
		\end{equation*}
		de modo que 
		\begin{equation*}
			k(x, y) = \int_0^\infty k(x, y, t) \diff t = \int_1^\infty k(x, y, t) \diff t.
		\end{equation*}
		Para $0<\gamma\leq1$, se tiene que 
		\begin{equation*}
			\int_{A_j(z, \sigma^\gamma)} |k(x, y, t) - k(x, z, t)| \diff x \leq
		\end{equation*}
		\begin{equation*}
			\left[ \int_{\mathbb{T}^n} \left(1 + t^{2\rho} |x-z|^2\right)^N 
			|k(x, y, t) - k(x, z, t)|^2 \diff x
			\right]^{1/2}  \left[ \int_{A_j(z, \sigma^\gamma)} \left(1 + t^{2\rho} |x-z|^2\right)^{-N} \diff x  \right]^{1/2},
		\end{equation*}
		donde $N>n/2$ es un número natural por determinar. En el Teorema  \ref{theo:kernel-estimate-sigma}  se demuestra que el lado izquierdo está dominado por 
		\begin{equation*}
			\sigma t \cdot t^{\rho n/2} \quad \text{si} \quad \sigma t\leq 1. 
		\end{equation*}
		De aquí viene la restricción de orden. Para estimar el segundo factor, se define 
		\begin{equation*}
			F(r) = \left[ \int_r^{2r} (1+s^2)^{-N} s^{n-1} \diff s  \right]^{1/2}, \quad 0<r<\infty. 
		\end{equation*}
		Nótese que $F$ es una función suave, tal que $F(r) \sim r^{n/2}$ cuando $r\rightarrow 0$ y $F(r)\sim r^{n/2-N}$ cuando $r \rightarrow \infty$. Por lo tanto, se obtiene que
		\begin{equation*}
			\left[ \int_{A_j(z, \sigma^\gamma)} \left(1 + t^{2\rho} |x-z|^2\right)^{-N}  \right]^{1/2}
		\end{equation*}
		\begin{equation*}
			\leq \left[ \int_{A_j(z, \sigma^\gamma)} \left(1 + t^{2\rho} |x-z|^2\right)^{-N} (t^\rho|x-z|)^{n-1} (t^\rho 2^j \sigma^\gamma)^{1-n} \diff x  \right]^{1/2}
		\end{equation*}
		\begin{equation*}
			\lesssim t^{-\rho n/2} F(t^\rho2^j\sigma^\gamma).
		\end{equation*}
		Así, se tiene que 
		\begin{equation*}
			\int_{A_j(z, \sigma^\gamma)} |k(x, y, t) - k(x, z, t)| \diff x \lesssim t\sigma F(t^\rho 2^j\sigma^\gamma), \quad t\sigma \leq 1.
		\end{equation*}
		Ahora, se considera el caso $t\sigma>1$. El cálculo hecho en el Teorema \ref{theo:kernel-estimate-sigma} , muestra que 
		\begin{equation*}
			\int_{A_j(z, \sigma^\gamma)} |k(x, y, t)| + |k(x, z, t)| \diff x \lesssim (t^\rho 2^j \sigma^\gamma )^{n/2 - N}.
		\end{equation*} 
		Combinando las dos últimas estimaciones se obtiene que 
		\begin{equation*}
			I_{j}(y, z, t) := \int_{A_j(z, \sigma^\gamma)} |k(x, y, t)-k(x, z, t)| \diff x 
		\end{equation*}
		\begin{equation}
			\leq 
			C \left[  \int_1^{1/\sigma} t\sigma F(t^\rho 2^j \sigma^\gamma )/t\diff t + \int_{1/\sigma}^\infty (t^\rho 2^j \sigma^\gamma )^{n/2 - N}/t \diff t \right].
			\label{eq:kernel-estimates-sum}
		\end{equation}
		Ahora, se elige $N$ de modo que $\rho(N-n/2)>1$, lo que implica que $\smallint F(t^\rho)\diff t < \infty$. Además, de (\ref{eq:kernel-estimates-sum}) se obtiene que la integral $I_j(y, z, t)$ puede estimarse por
		\begin{equation*}
			  2^{-j/\rho} \sigma^{1-\gamma/\rho} + 2^{j(n/2 -N)} \sigma^{(1-\gamma/\rho)\rho(N-n/2)} 
			  \leq 2^{-j/\rho} \sigma^{1-\gamma/\rho}, \quad 0<\sigma<1.
		\end{equation*}
		Así, completando la demostración de este caso. 
		\item[c)] Se puede usar el mismo método que en el caso anterior para estimar (\ref{eq:kernel-estimate-c}) cuando $t\sigma\leq1$. Además, inspeccionando la demostración del Teorema \ref{theo:kernel-estimate-sigma} se puede ver que se puede estimar como sigue 
		\begin{equation*}
			\int_{A_j(z, \sigma^\gamma)} |k(y, x) - k(z, x)|\diff x  
		\end{equation*}
		\begin{equation*}
			\leq C \left[  \int_1^{1/\sigma} t\sigma F(t^\rho 2^j \sigma^\gamma )/t\diff t + \int_{1/\sigma}^\infty (t^\rho2^j\sigma^\gamma)^{n/2 - N} \diff t   \right].
		\end{equation*}        
	\end{itemize}
	Obteniendo el resultado deseado.
\end{proof}
Se procede a usar estas estimaciones y Teorema \ref{theo:operator-kernel-2j} para obtener la acotación $H^p$-$L^p$ para operadores pseudo-diferenciales toroidales. 
\begin{theorem}
	Sea $T\in \Psi^m_{\rho,\delta}(\mathbb{T}^n\times\mathbb{Z}^n)$, $0<\rho\leq1$, $0\leq\delta<1$. Supóngase que 
	\begin{equation*}
		m\leq-\beta-n\lambda \quad  \text{para algún} \quad (1-\rho)\frac{n}{2}\leq\beta< \frac{n}{2}.
	\end{equation*}
	Entonces, el operador $T$ es una aplicación continua de $H^p(\mathbb{T}^n)$ en $L^p(\mathbb{T}^n)$ para $1 \geq p \geq p_0$ cuando $\rho<1$, donde 
	\begin{equation}
		\frac{1}{p_0} = \frac{1}{2} + \frac{\beta(1/\rho + n/2)}{n(1/\rho-1+\beta)},
	\end{equation}
	y para $1\geq p > p_0=n/(n+1)$ cuando $\rho=1$.
	\label{theo:Hp-Lp}
\end{theorem}
\begin{proof}
	Nótese que por Teorema \ref{theo:pdo-kernel}, el operador $T$ satisface las estimaciones del núcleo con $\alpha=\rho$ y $\omega=1$. La acotación $L^2$ se demostró en el Teorema \ref{theo:22boundT}, dado que $m\leq -n\lambda$. Así, el único requisito del Teorema \ref{theo:operator-kernel-2j} que se necesita probar es la acotación $L^q$-$L^2$. Nótese que $J^{-\beta}$ es acotado de $L^q(\mathbb{T}^n)$ en $L^2(\mathbb{T}^n)$ por la desigualdad de Hardy-Littlewood-Sobolev. Por lo tanto, dado que $J^\beta T$ tiene orden $m+\beta \leq -n\lambda$, se obtiene que 
	\begin{equation}
		\|Tf\|_{L^q} =    \|J^{-\beta}(J^\beta T)f\|_{L^q} \lesssim \|(J^\beta T)f\|_{L^2} \lesssim \|f\|_{L^2}.
		\label{eq:q-2-boundedness}
	\end{equation}
	Así, completando los requisitos para obtener el resultado deseado del Teorema \ref{theo:operator-kernel-2j}.
\end{proof}
\begin{remark}
	Nótese que para $p=1$ y $\beta=n(1-\rho)/2$ se obtiene la continuidad $H^1$-$L^1$ demostrada en el Teorema \ref{theo:L1-H1-L-infty-BMO}.
	\end{remark}
\subsection{Continuidad de operadores pseudo-diferenciales en $H^p$}
Ahora, se introduce un objeto con propiedades similares a los átomos en $H^p$.
\begin{definition}[Molécula]\label{def:molecule}
	Para un espacio de Banach $Y$, se dice que $M:\mathbb{T}^n\rightarrow Y$ es una \textit{$(p, \theta, \mu)$-molécula} relacionada con la bola $B(z, \sigma) \subset \mathbb{T}^n$ si satisface lo siguiente:
	\begin{itemize}
		\item Si $\sigma \geq 1$:
		\begin{itemize}
			\item[($\mathbf{M_1}$)] \begin{equation*}
				\int \|M(x)\|_Y^2\diff x \lesssim \sigma^{n(1-2/p)}.
			\end{equation*}
			\item[($\mathbf{M_2}$)] Para algún $2n/p-n < \mu < n + (2\omega/\alpha)$, se tiene que
			\begin{equation*}
				\int \|M(x)\|_Y^2|x-z|^\mu \diff x \lesssim \sigma^{\mu + n(1-2/p)}.
			\end{equation*}
		\end{itemize}
		\item Si $\sigma < 1$:
		\begin{itemize}
			\item[($\mathbf{M_1'}$)] \begin{equation*}
				\int \|M(x)\|_Y^2\diff x \lesssim \sigma^{n(1/q-2/p)}.
			\end{equation*}
			\item[($\mathbf{M_2'}$)] Para algún $2n/p-n < \mu < (2\beta/(1-\theta)) \leq n + (2\omega/\alpha)$, se tiene que
			\begin{equation*}
				\int \|M(x)\|_Y^2|x-z|^\mu \diff x \lesssim \sigma^{\theta\mu + n(1/q-2/p)}, 
			\end{equation*}
			donde 
			\begin{equation*}
				\theta = \frac{n/2 + \omega - \beta}{n/2+\omega/\alpha}\leq \alpha \quad \text{ y } \quad \frac{1}{q} = \frac{1}{2} + \frac{\beta}{n}.
			\end{equation*}
		\end{itemize}
	\end{itemize}
	Además, debe satisfacer la propiedad de cancelación, a saber, que $\smallint M(x)\diff x = 0$.
\end{definition}
Se demostrará que la imagen de un átomo $(p, 2)$ es una molécula adecuada. Por lo tanto, para dar sentido a la propiedad de cancelación de una molécula, se demuestra el siguiente lema.
\begin{lemma}
	Cualquier $(p,\theta,\mu)$-molécula $M:=M(x)$ relacionada con una bola $B(z, \sigma)$ es una función absolutamente integrable.
\end{lemma}
\begin{proof}
	Primero, se asume que $\sigma\geq1$. Entonces, usando la desigualdad de H\"older y ($\mathbf{M}_1$) se tiene que
	\begin{align*}
		\int_{B(z,\sigma)}\|M(x)\|_Y \diff x  &\leq  \|M(x)\|_{L^2(\mathbb{T}^n;Y)} \|\chi_{B(z,\sigma)}\|_{L^2} \\
		 & \lesssim \sigma^{n(1/2-1/p)}\cdot \sigma^{n/2} \lesssim |B|^{1-1/p}.
	\end{align*}
	Por otro lado, por la desigualdad de H\"older y ($\mathbf{M}_2$)
	\begin{align*}
		\int_{\mathbb{T}^n\setminus B(z,\sigma)}\|M(x)\|_Y\diff x  &\leq  \left\|M(x)|x-z|^{\mu/2} \right\|_{L^2(\mathbb{T}^n;Y)} \left\| |x-z|^{-\mu/2} \chi_{\mathbb{T}^n\setminus B(z,\sigma)}(x) \right\|_2 \\
		 & \lesssim \sigma^{\mu/2 + n(1/2-1/p)} \cdot\sigma^{(n-\mu)/2} \\
		 & \lesssim \sigma^{n(1-1/p)} \lesssim |B|^{1-1/p}.
	\end{align*}
	Por lo tanto, $M(x) \in L^1$ cuando $\sigma \geq 1$. Ahora, se asume que $\sigma < 1$. Usando ($\mathbf{M}_1'$) y la desigualdad de H\"older se obtiene que
	\begin{align*}
		\int_{B(z,\sigma)}\|M(x)\|_Y \diff x  &\leq 
		\|M(x)\|_{L^2(\mathbb{T}^n;Y)}\|\chi_{B(z,\sigma)}\|_{L^2} \\
		 & \lesssim \sigma^{n(1/q - 1/p)} \cdot \sigma^{n/2}\\  
		 & \lesssim |B|^{1/q-1/p+1/2} = |B|^{\beta/n+1-1/p},
	\end{align*}
	donde se usó el hecho de que $1/q=1/2+\beta/n$. Usando ($\mathbf{M}_2'$) se obtiene que
	\begin{align*}
		\int_{\mathbb{T}^n\setminus B(z,\sigma)}\|M(x)\|_Y\diff x  &\leq  \left\|M(x)|x-z|^{\mu/2} \right\|_{L^2(\mathbb{T}^n;Y)} \left\| |x-z|^{-\mu/2} \chi_{\mathbb{T}^n\setminus B(z,\sigma)}(x) \right\|_{L^2} \\
		 & \lesssim \sigma^{\theta\mu/2+n(1/q-1/p)} \cdot\sigma^{(n-\mu)/2} \\
		 & \sim |B|^{\beta/n+1-1/p-(1-\theta)\mu/2n}.
	\end{align*}
	Así, $M(x) \in L^1$ también cuando $\sigma<1$, completando la demostración.
\end{proof}
Ahora, se demuestra que la norma $H^p$ de una molécula solo depende de las constantes relacionadas con ella. Esto será útil al demostrar que la norma $H^p$ de la imagen de un $(p, 2)$-átomo  bajo un cierto operador es uniforme.
\begin{lemma}
	Sea $M:=M(x)$ una molécula $(p, \theta, \mu)$ relacionada con $B(z, \sigma)$. Entonces, 
	\begin{equation*}
		M(x)=\sum_{j=0}^{N_\sigma} \lambda_j a_j,
	\end{equation*}
	donde $a_j$ es un $(p, 2)$-átomo soportado en $B(z, 2^{j+1}\sigma)$. Además $\|M\|_{H^p(\mathbb{T}^n; Y)}$ solo depende de las constantes en la Definición \ref{def:molecule}.
	\label{lem:M-in-Hp}
\end{lemma}

\begin{proof}
	Sea $B_j = B(z, 2^{j+1}\sigma)$ y sea $M_j$ el valor promedio de $M(x)$ en $A_j(z, \sigma)$. Nótese que $A_j = B_j \setminus B_{j-1}$. Se define 
	\begin{equation*}
		\psi_j(x) = [M(x)-M_j]\chi_{A_j}(x),
	\end{equation*}
	que está soportada en $B_j$ y tiene valor promedio cero. Además, 
	\begin{equation*}
		\|\psi_j\|_{L^2(\mathbb{T}^n;Y)}^2 \leq 2^2 \int_{A_j(z,\sigma)}\|M(x)\|_Y^2 \diff x.
	\end{equation*}
	Ahora, se asume $\sigma\geq1$ y se usa ($\mathbf{M}_2$) para obtener
	\begin{align*}
		\|\psi_j\|_{L^2(\mathbb{T}^n;Y)}^2  & \lesssim \int_{A_j(z,\sigma)}\|M(x)\|_Y^2|x-z|^{\mu} |x-z|^{-\mu} \diff x
		\\
		 & \lesssim \sigma^{\mu+n(1-2/p)}\cdot2^{-j\mu} \sigma^{-\mu}
		\\
		 & \lesssim 2^{-j[\mu + n(1-2/p)]} |B_j|^{1-2/p}.
	\end{align*}
	Por lo tanto, se tiene que 
	\begin{equation}
		\|\psi_j\|_{L^2(\mathbb{T}^n;Y)} \lesssim 2^{-j[\mu/2 + n(1/2-1/p)]} |B_j|^{1/2-1/p}.
		\label{eq:psi-j}
	\end{equation}
	Por otro lado, se tiene que
	\begin{equation*}
		M(x)\chi_{B_m}(x) - \sum_{j=0}^m\psi_j(x) = \sum_{j=0}^mM_j\chi_{B_j}(x),
	\end{equation*}
	donde el lado izquierdo converge en la norma $L^2$ a $M(x) -\sum_{j=0}^\infty\psi_j(x)$. Para estimar el lado izquierdo, se define una sucesión $\{\nu_j\}$, $j=-1, 0, 1, 2,...,N_\sigma$ por
	\begin{equation*}
		\nu_{-1} = \int M(x)\diff x = 0, \quad \nu_j = \int_{\mathbb{T}^n\setminus B_j} M(x)\diff x.
	\end{equation*}
	Entonces se tiene que
	\begin{equation*}
		\sum_{j=0}^{N_\sigma}M_j \chi_{A_j} =\sum_{j=0}^{N_\sigma}(\nu_{j-1}-\nu_j)|A_j|^{-1}\chi_{A_j} = \sum_{j=0}^{N_\sigma-1}\phi_j - \nu_{N_\sigma}|A_{N_\sigma}|^{-1}\chi_{A_{N_\sigma}},
	\end{equation*}
	donde $\phi_j=\nu_j\left( |A_{j+1}|^{-1}\chi_{A_{j+1}} - |A_{j}|^{-1}\chi_{A_{j}}  \right)$. Se puede notar que cada $\phi_j$ está soportada en $B_{j+1}$ y tiene valor promedio cero. Además, se puede usar el método usado para $\psi_j$ para obtener
	\begin{equation}
		\|\phi_j\|_{L^2(\mathbb{T}^n;Y)} \lesssim 2^{-j[\mu/2 + n(1/2-1/p)]} |B_{j+1}|^{1/2-1/p}.
		\label{eq:phi-j}
	\end{equation}
	Ahora, se tiene que $\nu_{N_\sigma}=0$ gracias a que $B_{N_\sigma}$ contiene el toro, ver la Nota \ref{rem:N-sigma}. Por lo tanto se tiene que 
	\begin{equation}
		M(x)= \psi_0(x)+ \sum_{j=1}^{N_\sigma} \left[\psi_j(x) + \phi_{j-1}(x) \right],
		\label{eq:psi-phi}
	\end{equation}
	donde cada término puede reescribirse como un$(p,2)$-átomo por (\ref{eq:psi-j}) y (\ref{eq:phi-j}). Ahora, se asume $\sigma<1$ y se usa ($\mathbf{M}_2'$) para obtener
	\begin{align*}
		\|\psi_j\|_{L^2(\mathbb{T}^n;Y)}^2  &\lesssim \int_{A_j(z,\sigma)}\|M(x)\|_Y^2|x-z|^{\mu} |x-z|^{-\mu} \diff x
		\\
		 & \lesssim \sigma^{\theta\mu+2n(1/q-1/p)}\cdot2^{-j\mu} \sigma^{-\mu}
		\\
		 &\lesssim 2^{-j[\mu + 2n(1/2-1/p)]} \sigma^{\mu(\theta-1) + 2n(1/q-1/2)} |B_j|^{2(1/2-1/p)}.
	\end{align*}
	Por lo tanto, se tiene que 
	\begin{align*}
		\|\psi_j\|_{L^2(\mathbb{T}^n;Y)}  \lesssim 2^{-j[\mu/2 + n(1/2-1/p)]} \sigma^{\mu(\theta-1)/2+n(1/q-1/2)} |B_j|^{1/2-1/p}.
	\end{align*}
	El exponente de $\sigma$ se convierte en $\mu(\theta-1)/2 + \beta$, de ahí la restricción $\mu \leq 2\beta/(1-\theta)$. Por otro lado, $\|\phi_j\|_2$ puede estimarse de manera similar para obtener la expresión (\ref{eq:psi-phi}) para $\sigma<1$. Ahora, se puede considerar la norma $H^p$ de $M$. En este caso, se tiene que
	\begin{equation*}
		|\lambda_j|^p \leq C 2^{j[ n(1-p/2) -\mu p/2 ]},
	\end{equation*}
	Por lo tanto $\|M\|_{H^p(\mathbb{T}^n;Y)} \leq C$ siempre que $\mu > np/2 -n$. Así, completando la demostración.
\end{proof}
Se procede a definir una condición para operadores con kernel valuado en operadores que será útil para demostrar propiedades de continuidad para estos operadores.
\begin{definition}[Condición $D_{r,\alpha}$] Sea $1\leq r\leq\infty$ y $0<\alpha\leq1$. Se dice que un operador $T:C^\infty(\mathbb{T}^n;X)\rightarrow C^\infty(\mathbb{T}^n;Y)$ satisface la condición $D_{r,\alpha}$ si su kernel valuado en operadores asociado $k:=k(x,y)$ es continuo fuera de la diagonal de $\mathbb{T}^n\times\mathbb{T}^n$ y existe una sucesión $\{d_j\}\in \ell^1$, tal que para todo $\sigma>0$ se tiene que 
	\begin{equation*}
		\left( \int_{A_j(z,\sigma^\alpha)} \|k(x, y) - k(x, z)\|^r_\mathcal{B} \diff x \right)^{1/r} \leq d_j|A_j(z,\sigma^\alpha)|^{-1/r'}, \quad j=1,2,... ,
	\end{equation*}
	siempre que $|y-z|<\sigma$. Además, también se requiere que $\tilde{k}(x, y) := k(y, x)$ satisfaga estas estimaciones.
	\label{def:D-r-alpha}
\end{definition}
Un caso límite de esta condición se enuncia de la siguiente manera.
\begin{remark}[Condición $D_\alpha$]
	Se dice que el operador $T$ de la Definición \ref{def:D-r-alpha} satisface la condición $D_\alpha$ si para algún $0<\omega\leq1$, $0<\alpha\leq1$ se tiene que
	\begin{equation*}
		\|k(x,y)-k(x, z)\|_\mathcal{B}+\|k(y, x)-k(z, x)\|_\mathcal{B} \leq C \frac{|y-z|^\omega}{|x-z|^{n+\omega/\alpha}},
	\end{equation*}
	cuando $2|y-z|^\alpha \leq |x - z|$ para todo $x, y, z\in \mathbb{T}^n$. 
	\label{rem:D-alpha-condition}
\end{remark}
Ahora, se demuestra un resultado auxiliar que relaciona la función maximal $p$ de Hardy-Littlewood y el operador maximal sharp.
\begin{theorem}
	Sea $T$ un operador que satisface la condición $D_{r,\alpha}$ tal que para $1<p<q\leq\infty$ y $p/q\leq\alpha$ se tiene que
	\begin{equation}
		\left( \frac{1}{|B(z, \sigma)|} \int_{B(z, \sigma)} \|Tf(x)\|_Y^q \diff x  \right)^{1/q} \lesssim \left( \frac{1}{|B(z, \sigma^\alpha)|} \int \|f(x)\|_X^p \diff x  \right)^{1/p}, 
		\label{eq:c-p-q-alpha}
	\end{equation}
	para todo $0<\sigma<1$ y 
	\begin{equation*}
		\|Tf\|_{L^p(\mathbb{T}^n;Y)} \lesssim \|f\|_{L^p(\mathbb{T}^n;X)},
	\end{equation*}
	cuando $\sigma\geq1$, para alguna constante absoluta $C>0$. Entonces, para $s=\max \{ p, r' \}$, se tiene que 
	\begin{equation}
		(Tf)^\#(x) \lesssim \mathrm{M}_sf(x), \quad f\in L^\infty (\mathbb{T}^n; X).
		\label{eq:Tf-Msf}
	\end{equation}
	\label{theo:Tf-Msf}
\end{theorem}
\begin{proof}
	Se fija una bola $B(z, \sigma)$ y se escribe $f=f_1+f_2$, donde 
	\begin{equation*}
		f_1 = f\chi_{B(z, 2\sigma^\alpha)}.
	\end{equation*}
	Sea
	\begin{equation*}
		c = \int k(z, y)f_2(y)\diff y,
	\end{equation*}
	entonces se tiene que
	\begin{align*}
		\int_{B(z, \sigma)} \|Tf(x)-c\|_Y \diff x \leq & \int_{B(z, \sigma)} \|Tf_1(x)\|_Y \diff x \\
		& + \sum_{j=1}^{N_{\sigma^\alpha}} \int_{B(z, \sigma)}\int_{A_j(z, \sigma^\alpha)} \|k(x,y)-k(z,y)\|_\mathcal{B}\|f(y)\|_X \diff y \diff x .
	\end{align*}
	Primero, usando la desigualdad de H\"older y la condición $D_{r, \alpha}$ se obtiene que para $j=1,2,...,N_{\sigma^\alpha}$
	\begin{equation*}
		\int_{A_j(z, \sigma^\alpha)} \|k(x,y)-k(z,y)\|_\mathcal{B}\|f(y)\|_X \diff y 
	\end{equation*}
	\begin{equation*}
		\leq\left( \int_{A_j(z, \sigma^\alpha)} \|k(x,y)-k(z,y)\|_\mathcal{B}^r \diff y \right)^{1/r}\left(  \int_{A_j(z,\sigma^\alpha)}  \|f(y)\|_X^{r'} \diff y \right)^{1/r'}
	\end{equation*}
	\begin{equation*}
		\leq  d_j |A_j(z,\sigma^\alpha)|^{-1/r'}\left(  \int_{A_j(z,\sigma^\alpha)}  \|f(y)\|_X^{r'} \diff y \right)^{1/r'}.
	\end{equation*}
	\begin{equation*}
		\leq d_j \mathrm{M}_{r'}f(z).
	\end{equation*}
	Por lo tanto, se tiene que 
	\begin{equation*}
		\frac{1}{|B(z,\sigma)|}\sum_{j=1}^{N_{\sigma^\alpha}} \int_{B(z,\sigma)}\int_{A_j(z, \sigma^\alpha)} \|k(x,y)-k(z,y)\|_\mathcal{B}\|f(y)\|_X \diff y \diff x
	\end{equation*}
	\begin{equation*}
		\lesssim \mathrm{M}_{r'}f(z).
	\end{equation*}
	Por otro lado, se asume $\sigma<1$. Entonces, usando la desigualdad de H\"older y (\ref{eq:c-p-q-alpha}), se obtiene que
	\begin{equation*}
		\frac{1}{|B(z,\sigma)|}\int_{B(z, \sigma)}\|Tf_1(x)\|_Y \diff x 
	\end{equation*}
	\begin{equation*}
		\leq \left( 
		\frac{1}{|B(z,\sigma)|}\int_{B(z,\sigma)} \|Tf_1(x)\|^q_Y \diff x 
		\right)^{1/q} \left( \int_{B(z,\sigma)} \frac{\diff x}{|B(z,\sigma)|}  
		\right)^{1/q'}
	\end{equation*}
	\begin{equation*}
		\lesssim \left( \frac{1}{|B(z, \sigma^\alpha)|} \int_{B(z,2\sigma^\alpha)} \|f(x)\|_X^p \diff x  \right)^{1/p} 
	\end{equation*}
	\begin{equation*}
		\lesssim \mathrm{M}_pf(z).
	\end{equation*}
	Para el caso $\sigma \geq 1$ se usa la desigualdad de H\"older y la acotación $L^p$ de $T$ para obtener
	\begin{align*}
		\frac{1}{|B(z,\sigma)|}\int_{B(z, \sigma)}\|Tf_1(x)\|_Y \diff x   &\leq 
		|B(z,\sigma)|^{-1/p} \|Tf_1\|_{L^p(\mathbb{T}^n;Y)} \\
		 & \lesssim |B(z,\sigma)|^{-1/p} \|f_1\|_{L^p(\mathbb{T}^n;X)} \\
		 & \lesssim \mathrm{M}_pf(z).
	\end{align*}
	Combinando las estimaciones anteriores se puede concluir que 
	\begin{equation*}
		(Tf)^\#(z) \lesssim \mathrm{M}_sf(z),
	\end{equation*}
	terminando la demostración.
\end{proof}
Además, cuando el operador satisface la condición $D_{1,\alpha}$ se obtiene el siguiente corolario.
\begin{corollary}
	Sea $T$ un operador como en el Teorema \ref{theo:Tf-Msf}, pero que satisface la condición $D_{1,\alpha}$. Entonces $T$ es un operador continuo de $L^\infty(\mathbb{T}^n;X)$ en $\mathrm{BMO}(\mathbb{T}^n;Y)$.
	\label{cor:L-inf-BMO}
\end{corollary}
\begin{proof}
	Note que $r=1$ implica que $s=\infty$ y la estimación en (\ref{eq:Tf-Msf}) se convierte en
	\begin{equation*}
		(Tf)^\#(x) \lesssim \|f\|_{L^\infty(\mathbb{T}^n;X)},
	\end{equation*}
	probando el resultado.
\end{proof}
En la hipótesis del siguiente lema se tiene la condición $T^*(\overline{e})$, una contraparte vectorial de la condición $T(1)$ famosamente establecida por David y Journ\'e, ver \cite{david-journe}.
\begin{lemma}
	Sean $X, Y$ espacios de Banach reflexivos, sean $T$ y  $T^*$ operadores con núcleos que satisfacen la condición $D_\alpha$. También, supóngase que pueden extenderse a operadores acotados de $L^2(\mathbb{T}^n;X)$ en $L^2(\mathbb{T}^n;Y)$ y de $L^q(\mathbb{T}^n;X)$ en $L^2(\mathbb{T}^n;Y)$ de modo que 
	\begin{equation}
		\frac{1}{q} = \frac{1}{2} + \frac{\beta}{n} \quad \text{ para algún } \quad (1-\alpha)\frac{n}{2}\leq \beta<\frac{n}{2}.
	\end{equation}
	Además, supóngase que $T^*(\overline{e})=0$ para todo ${e}\in Y'$ y $\overline{e}:\mathbb{T}^n\rightarrow Y'$ dado por $\overline{e}(x)=e$. Sea $a:=a(x)$ un$(p,2)$-átomo  soportado en $B(z,\sigma)$. Entonces, $M(x):=Ta(x)$ es una molécula $(p, \theta, \mu)$ con constantes que dependen solo de $T$ y sus propiedades de continuidad.
	\label{lem:Ta-is-molecule}
\end{lemma}

\begin{proof}
	Primero, se asume $\sigma \geq 1$. Entonces, por la acotación $L^2$ se obtiene que
	\begin{equation*}
		\int\|M(x)\|_Y^2\diff x = \|Ta\|_{L^2(\mathbb{T}^n;Y)}^2 \lesssim \|a\|_{L^2(\mathbb{T}^n;Y)}^2 \lesssim |B|^{2(1/2-1/p)} \lesssim\sigma^{n(1-2/p)},
	\end{equation*}
	probando $(\mathbf{M_1})$. Además, se tiene que 
	\begin{align*}
		\int \|M(x)\|_Y^2|x-z|^\mu \diff x  &\leq
		\int_{B(z, 2\sigma^\alpha)} \|M(x)\|_Y^2|x-z|^\mu \diff x\\
		& \quad + \sum_{j=1}^{N_{\sigma^\alpha}} \int_{A_j(z, \sigma^\alpha)} \|M(x)\|_Y^2|x-z|^\mu \diff x \\
		 &= I_1 + I_2.
	\end{align*}
	Por $(\mathbf{M_1})$, se obtiene que
	\begin{equation*}
		I_1 \lesssim\int\|M(x)\|_Y^2 \sigma^{\alpha\mu} \diff x\lesssim  \sigma^{\mu +n(1-2/p)}.
	\end{equation*}
	Además, asumiendo $x\in A_j(z, \sigma^\alpha)$, y por la propiedad de cancelación de $a$ se obtiene que
	\begin{align*}
		\|M(x)\|_Y^2  &\leq  \left[ \int_{B(z, \sigma)} \|k(x, y) - k(x,z)\|_\mathcal{B} \|a(y)\|_X \diff y \right]^2 \\
		 & \lesssim \frac{|y-z|^{2\omega}}{|x-z|^{2(n+\omega/\alpha)}} 
		\left[ \int_{B(z, \sigma)} \|a(y)\|_X \diff y \right]^2 \\
		 & \lesssim \frac{\sigma^{2\omega} \cdot \sigma^{2n(1-1/p)}}{|x-z|^{2(n+\omega/\alpha)}}.
	\end{align*}
	Ahora, dado que $\mu < n + 2\omega/\alpha$ se tiene que
	\begin{align*}
		I_2  &\leq  \sum_{j=1}^{N_{\sigma^\alpha}} \sigma^{2\omega + 2n(1-1/p)} \cdot (2^j\sigma^\alpha)^{\mu - 2\omega/\alpha - n} \\
		 &  \lesssim \sigma^{\alpha\mu + 2n(1-1/p) - n\alpha} \\ 
		 & \lesssim \sigma^{\mu + 2n(1-1/p)},
	\end{align*}
	dado que $\mu > n$ y $\alpha \leq 1$, probando $(\mathbf{M_2})$. Ahora, se supone que $\sigma < 1$. Primero, por la estimación $L^q$-$L^2$ de $T$, se obtiene que 
	\begin{equation*}
		\int\|M(x)\|_Y^2\diff x = \|Ta\|_{L^2(\mathbb{T}^n;Y)}^2 \lesssim \|a\|_{L^q(\mathbb{T}^n;Y)}^2 \leq |B|^{2(1/q-1/p)} \lesssim \sigma^{2n(1/q-1/p)},
	\end{equation*}
	probando $(\mathbf{M_1'})$. Por otro lado, se tiene
	\begin{align*}
		\int \|M(x)\|_Y^2|x-z|^\mu \diff x &\leq 
		\int_{B(z, 2\sigma^\theta)} \|M(x)\|_Y^2|x-z|^\mu \diff x\\
		&  \quad + \sum_{j=1}^{N_{\sigma^\theta}} \int_{A_j(z, \sigma^\theta)} \|M(x)\|_Y^2|x-z|^\mu \diff x \\
		 & = I_1 + I_2.
	\end{align*}
	Como en el caso anterior, se puede usar $(\mathbf{M_1'})$ para estimar 
	\begin{equation*}
		I_1 \lesssim \sigma^{\theta\mu + 2n(1/q - 1/p)}.
	\end{equation*}
	Además, cuando $x \in A_j(z, \sigma^\theta)$ se tiene una estimación similar a la anterior, a saber que
	\begin{align*}
		I_2  & \lesssim \sum_{j=1}^{N_{\sigma^\theta}} \sigma^{2\omega + 2n(1 - 1/p) } (2^j\sigma^\theta)^{\mu - n - 2\omega/\alpha} \\
		 &\lesssim \sigma^{ 2\omega + 2n(1-1/p) + \theta\mu -\theta(n + 2\omega/\alpha) } \\
		 & =\sigma^{ \theta\mu + 2n(1/q - 1/p) },
	\end{align*}
	dado que 
	\begin{equation*}
		\theta = \frac{n(1-1/q)+\omega}{ n/2 + \omega/\alpha }.
	\end{equation*}
	Esto completa la demostración de $(\mathbf{M_2'})$. Solo queda probar que $\int M(x)\diff x=0$. Nótese que $|B|^{-1+1/p}a$  es un $(1, 2)$-átomo y dado $e \in Y'$ se tiene que
	\begin{align*}
		|B|^{-1+1/p}\left\langle e, \int M(x)\diff x \right\rangle & = \int \left\langle \overline{e}(x) , T\left(  |B|^{-1+1/p}a\right)(x)\right\rangle \diff x \\
		 &= \int \left\langle T^*\overline{e}(x) ,   |B|^{-1+1/p}a(x)\right\rangle \diff x = 0.
	\end{align*}
	Donde se ha usado el hecho de que $T^*$ es continuo en $L^2$ para emplear el Corolario \ref{cor:L-inf-BMO} y concluir que $T^*$ mapea $L^\infty(\mathbb{T}^n, Y')$ en $\mathrm{BMO}(\mathbb{T}^n; X')$, que es el dual de $H^1(\mathbb{T}^n;X)$, dado que $X$ es reflexivo. Por lo tanto, se demostró que $M(x)$ es una molécula $(p, \theta,\mu)$.
\end{proof}
Nótese que las constantes relacionadas de la molécula resultante solo dependen del operador $T$. Además, la norma $H^p$ de la molécula solo depende de dichas constantes. Por lo tanto, se procede a combinar los resultados anteriores de esta sección para obtener la acotación $H^p$ para operadores con núcleos de valor operador bajo ciertas condiciones. 
\begin{theorem}
	Sean $X, Y$ espacios de Banach reflexivos, sean $T$ y  $T^*$ operadores que satisfacen la condición $D_\alpha$. También, supóngase que pueden extenderse a operadores acotados de $L^2(\mathbb{T}^n;X)$ en $L^2(\mathbb{T}^n;Y)$ y de $L^q(\mathbb{T}^n;X)$ en $L^2(\mathbb{T}^n;Y)$ de modo que 
	\begin{equation}
		\frac{1}{q} = \frac{1}{2} + \frac{\beta}{n} \quad \text{ para algún } \quad (1-\alpha)\frac{n}{2}\leq \beta<\frac{n}{2}.
	\end{equation}
	Además, supóngase que $T^*(\overline{e})=0$ para todo ${e}\in Y'$ y $\overline{e}:\mathbb{T}^n\rightarrow Y'$ dado por $\overline{e}(x)=e$. Sea 
	\begin{equation}
		\frac{1}{p_0}=\frac{1}{2} + \frac{\beta(\omega/\alpha+n/2)}{n(\omega/\alpha - \omega +\beta)},
	\end{equation}
	entonces para $p_0<p\leq1$, el operador $T$ es acotado de $H^p(\mathbb{T}^n;X)$ en $H^p(\mathbb{T}^n;Y)$.
	\label{theo:Hp-operator}
\end{theorem}

\begin{proof}
	Nótese que por el Lema \ref{lem:Ta-is-molecule}, la imagen de todo átomo $H^{p,2}(\mathbb{T}^n;X)$ es una molécula $(p, \theta,\mu)$ con constantes que dependen solo de $T$, y por el Lema \ref{lem:M-in-Hp} se tiene que dichas moléculas tienen normas $H^p$ que dependen solo de dichas constantes. Por lo tanto se tiene una cota uniforme $\|Ta\|_{H^p(\mathbb{T}^n;Y)}\leq C$ para todo átomo $(p,2)$-$a$. Inspeccionando la Definición \ref{def:molecule} se puede ver que esto ocurre cuando 
	\begin{equation*}
		2n/p -n < \frac{2\beta}{1-\theta}, \quad \text{ donde } \quad \theta = \frac{n/2+\omega-\beta}{n/2+\omega/\alpha}.
	\end{equation*}
	Así, se puede concluir la expresión para $p_0$ establecida anteriormente.
\end{proof}
Ahora, se aplica este resultado en el contexto específico de operadores pseudo-diferenciales toroidales.
\begin{theorem}
	Sea $T\in \Psi^m_{\rho,\delta}(\mathbb{T}^n\times\mathbb{Z}^n)$, $0<\rho\leq1$, $0\leq\delta<1$. Supóngase que
	\begin{equation}
		m\leq-\beta-n\lambda \quad  \text{para algún} \quad (1-\rho)\frac{n}{2}\leq\beta\leq \frac{n}{2}, 
	\end{equation}
	y que $T^*(1)=0$ en el sentido de BMO. Entonces el operador $T$ es una aplicación continua de $H^p(\mathbb{T}^n)$ en sí mismo para $p_0<p\leq1$ donde 
	\begin{equation}
		\frac{1}{p_0} = \frac{1}{2}+\frac{\beta(1/\rho + n/2)}{n(1/\rho-1+\beta)}.
	\end{equation}
\end{theorem}
\begin{proof}
	Se puede elegir $\gamma =\rho$ en el Teorema \ref{theo:pdo-kernel} para obtener la condición $D_{1,\rho}$ para $T$ y $T^*$. Por lo tanto estos operadores satisfacen la condición $D_\rho$  con $\omega=1$ (ver la Nota \ref{rem:D-alpha-condition}). Además, $T$ es acotado en $L^2$ en vista del Teorema \ref{teo:cotas-Lp} y la acotación $L^q$-$L^2$ de $T$ se sigue de( \ref{eq:q-2-boundedness}) en la demostración del Teorema \ref{theo:Hp-Lp}. Así se ha demostrado que $T$ satisface todas las condiciones para aplicar el Teorema \ref{theo:Hp-operator}. Lo que completa la demostración.
\end{proof}

